%
% Copyright � 2025 Peeter Joot.  All Rights Reserved.
% Licenced as described in the file LICENSE under the root directory of this GIT repository.
%
%{
\input{../latex/blogpost.tex}
\renewcommand{\basename}{cos4}
%\renewcommand{\dirname}{notes/phy1520/}
\renewcommand{\dirname}{notes/ece1228-electromagnetic-theory/}
%\newcommand{\dateintitle}{}
%\newcommand{\keywords}{}

\input{../latex/peeter_prologue_print2.tex}

\usepackage{peeters_layout_exercise}
\usepackage{peeters_braket}
\usepackage{peeters_figures}
\usepackage{siunitx}
\usepackage{verbatim}
%\usepackage{macros_cal} % \LL
%\usepackage{amsthm} % proof
%\usepackage{mhchem} % \ce{}
%\usepackage{macros_bm} % \bcM
%\usepackage{macros_qed} % \qedmarker
%\usepackage{txfonts} % \ointclockwise

\beginArtNoToc

\generatetitle{XXX}
%\chapter{XXX}
%\label{chap:cos4}

Find:
\begin{equation}\label{eqn:cos4:20}
I = \int_0^{\pi/4} \frac{\cos\lr{4x}}{\cos^4 x} dx.
\end{equation}

First reduce the \( \cos 4 x \)
\begin{equation}\label{eqn:cos4:40}
\begin{aligned}
\cos \lr{ 4 x }
&= \Real\lr{ \cos x + i \sin x }^4 \\
&= \Real\lr{ \cos^4 x + 4 i \cos^3 x \sin x + 6 i^2 \cos^2 x \sin^2 x + 4 i^3 \cos x \sin^3 x + i^4 \sin^4 x } \\
&= \cos^4 x - 6 \cos^2 \sin^2 x + \sin^4 x \\
&= \cos^4 x - 6 \cos^2 \lr{ 1 - \cos^2 x } + \lr{ 1 - \cos^2 x }^2 \\
&= 8 \cos^4 x - 8 \cos^2 + 1.
\end{aligned}
\end{equation}

We can now rewrite the integral as
\begin{equation}\label{eqn:cos4:60}
I = 8 \frac{\pi}{4} - 8 \int_0^{\pi/4} \frac{dx}{\cos^2 x} + \int_0^{\pi/4} \frac{dx}{\cos^4 x}.
\end{equation}

With a bit of trial and error, we find
\begin{equation}\label{eqn:cos4:80}
\begin{aligned}
\lr{\tan x}' &= \frac{1}{\cos^2 x} \\
\lr{\tan^3 x}' &= \frac{3}{\cos^4 x} - \frac{3}{\cos^2 x},
\end{aligned}
\end{equation}
so
%\begin{equation}\label{eqn:cos4:100}
%\frac{1}{\cos^4 x} = \frac{1}{\cos^2 x} + \inv{3} \lr{\tan^3 x}',
%\end{equation}
%and
\begin{equation}\label{eqn:cos4:120}
\begin{aligned}
-\frac{8}{\cos^2 x} + \frac{1}{\cos^4 x}
&= -\frac{7}{\cos^2 x} + \inv{3} \lr{\tan^3 x}' \\
&= -7 \lr{ \tan x}' + \inv{3} \lr{\tan^3 x}' \\
\end{aligned}
\end{equation}
We are left with
\begin{equation}\label{eqn:cos4:140}
\boxed{
I = 2 \pi -7 + \inv{3} = 2 \pi - \frac{20}{3}.
}
\end{equation}

%}
%\EndArticle
\EndNoBibArticle
