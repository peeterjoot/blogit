%
% Copyright � 2023 Peeter Joot.  All Rights Reserved.
% Licenced as described in the file LICENSE under the root directory of this GIT repository.
%
%{
\input{../latex/blogpost.tex}
\renewcommand{\basename}{bicomplexGA20}
%\renewcommand{\dirname}{notes/phy1520/}
\renewcommand{\dirname}{notes/ece1228-electromagnetic-theory/}
%\newcommand{\dateintitle}{}
%\newcommand{\keywords}{}

\input{../latex/peeter_prologue_print2.tex}

\usepackage{peeters_layout_exercise}
\usepackage{peeters_braket}
\usepackage{peeters_figures}
\usepackage{siunitx}
\usepackage{verbatim}
%\usepackage{mhchem} % \ce{}
%\usepackage{macros_bm} % \bcM
%\usepackage{macros_qed} % \qedmarker
%\usepackage{txfonts} % \ointclockwise

\beginArtNoToc

\generatetitle{A complex-pair representation of GA(2,0).}
%\chapter{A complex-pair representation of GA(2,0)}
%\label{chap:bicomplexGA20}
\section{Motivation.}
Suppose that we want to represent GA(2,0) (Euclidean) multivectors as a pair of complex numbers, with a structure like
\begin{equation}\label{eqn:bicomplexGA20:20}
M = (m_1, m_2),
\end{equation}
where
\begin{equation}\label{eqn:bicomplexGA20:40}
\begin{aligned}
   \gpgrade{M}{0,2} &\sim m_1 \\
   \gpgrade{M}{1} &\sim m_2.
\end{aligned}
\end{equation}
Specifically
\begin{equation}\label{eqn:bicomplexGA20:60}
\begin{aligned}
   \gpgrade{M}{0} &= \Real(m_1) \\
   \gpgrade{M}{1} \cdot \Be_1 &= \Real(m_2) \\
   \gpgrade{M}{1} \cdot \Be_2 &= \Imag(m_2) \\
   \gpgrade{M}{2} i^{-1} &= \Imag(m_1),
\end{aligned}
\end{equation}
where \( i \sim \Be_1 \Be_2 \).

\section{Multivector product representation.}
Let's figure out how we can represent the various GA products, starting with the geometric product.  Let
\begin{equation}\label{eqn:bicomplexGA20:80}
\begin{aligned}
   M &= \gpgrade{M}{0,2} + \gpgrade{M}{1} = (m_1, m_2) = (m_{11} + m_{12} i, m_{21} + m_{22} i) \\
   N &= \gpgrade{N}{0,2} + \gpgrade{N}{1} = (n_1, n_2) = (n_{11} + n_{12} i, n_{21} + n_{22} i),
\end{aligned}
\end{equation}
so
\begin{equation}\label{eqn:bicomplexGA20:200}
\begin{aligned}
   M N
   &= \gpgrade{M}{0,2} \gpgrade{N}{0,2} + \gpgrade{M}{1} \gpgrade{N}{1}  \\
   &\quad+ \gpgrade{M}{1} \gpgrade{N}{0,2} + \gpgrade{M}{0,2} \gpgrade{N}{1}
\end{aligned}
\end{equation}

The first two terms have only even grades, and the second two terms are vectors.  The complete representation of the even grade components of this multivector product is
\begin{equation}\label{eqn:bicomplexGA20:240}
   \gpgrade{M N}{0,2} \sim \lr{ m_1 n_1 + \Real(m_2 n_2^\conj) - i \Imag(m_2 n_2^\conj), 0 },
\end{equation}
or
\begin{equation}\label{eqn:bicomplexGA20:260}
\begin{aligned}
   \gpgrade{M N}{0} &= \Real\lr{ m_1 n_1 + m_2 n_2^\conj } \\
   \gpgrade{M N}{2} i^{-1} &= \Imag\lr{ m_1 n_1 - m_2 n_2^\conj }.
\end{aligned}
\end{equation}

For the vector components we have
\begin{equation}\label{eqn:bicomplexGA20:280}
\begin{aligned}
   \gpgrade{M N}{1}
   &=
   \gpgrade{M}{1} \gpgrade{N}{0} + \gpgrade{M}{0} \gpgrade{N}{1}
   +
   \gpgrade{M}{1} \gpgrade{N}{2} + \gpgrade{M}{2} \gpgrade{N}{1} \\
   &= \gpgrade{M}{1} n_{11} + m_{11} \gpgrade{N}{1} + \gpgrade{M}{1} i n_{12} + i m_{12} \gpgrade{N}{1}.
\end{aligned}
\end{equation}
For these,
\begin{equation}\label{eqn:bicomplexGA20:300}
\begin{aligned}
   \gpgrade{M}{1} i
   &= \lr{ m_{21} \Be_1 + m_{22} \Be_2 } \Be_{12}
   &= - m_{22} \Be_1 + m_{21} \Be_2,
\end{aligned}
\end{equation}
and
\begin{equation}\label{eqn:bicomplexGA20:320}
\begin{aligned}
   i \gpgrade{N}{1}
   &= \Be_{12} \lr{ n_{21} \Be_1 + n_{22} \Be_2 }
   &=
   n_{22} \Be_1 - n_{21} \Be_2.
\end{aligned}
\end{equation}
Comparing to
\begin{equation}\label{eqn:bicomplexGA20:340}
   i (a + i b)
   = -b + i a,
\end{equation}
we see that
\begin{equation}\label{eqn:bicomplexGA20:360}
   \gpgrade{M N}{1}
   \sim
   \lr{ 0, n_{11} m_2 + m_{11} n_2 + n_{12} i m_2 - m_{12} i n_2 }.
\end{equation}
If we want the vector coordinates, those are
\begin{equation}\label{eqn:bicomplexGA20:380}
\begin{aligned}
   \gpgrade{M N}{1} \cdot \Be_1 &= \Real \lr{ n_{11} m_2 + m_{11} n_2 + n_{12} i m_2 - m_{12} i n_2 } \\
   \gpgrade{M N}{1} \cdot \Be_2 &= \Imag \lr{ n_{11} m_2 + m_{11} n_2 + n_{12} i m_2 - m_{12} i n_2 }.
\end{aligned}
\end{equation}

\section{Summary.}
\begin{equation}\label{eqn:bicomplexGA20:400}
M N \sim
   \lr{ m_1 n_1 + \Real(m_2 n_2^\conj) - i \Imag(m_2 n_2^\conj), n_{11} m_2 + m_{11} n_2 + n_{12} i m_2 - m_{12} i n_2 }.
\end{equation}

A \href{https://github.com/peeterjoot/gapauli/blob/master/Cl20.m}{sample Mathematica implementation} is available, as well as \href{https://github.com/peeterjoot/gapauli/blob/master/testCl20.nb}{an example notebook} (that also doubles as a test case.)

\section{Clarification.}
I skipped a step above, showing the correspondances to the dot and wedge product.

Let \(z = a + bi\), and \(w = c + di\).  Then:
\begin{equation}\label{eqn:bicomplexGA20:420}
\begin{aligned}
   z w^\conj
   &=
   \lr{ a + b i } \lr{ c - d i } \\
   &= a c + b d -i \lr{ a d - b c }.
\end{aligned}
\end{equation}
Compare that to the geometric product of two vectors \( \Bx = a \Be_1 + b \Be_2 \), and \( \By = c \Be_1 + d \Be_2 \), where we have
\begin{equation}\label{eqn:bicomplexGA20:440}
\begin{aligned}
   \Bx \By &= \Bx \cdot \By + \Bx \wedge \By \\
   &= \lr{ a \Be_1 + b \Be_2 } \lr{ c \Be_1 + d \Be_2 } \\
   &= a c + b d + \Be_1 \Be_2 \lr{ a d - b c }.
\end{aligned}
\end{equation}
So we have
\begin{equation}\label{eqn:bicomplexGA20:460}
\begin{aligned}
   a b + cd &= \Bx \cdot \By = \Real \lr{ z w^\conj } \\
   a d - b c &= \lr{ \Bx \wedge \By } \Be_{12}^{-1} = - \Imag \lr{ z w^\conj }.
\end{aligned}
\end{equation}
We see that \( \lr{z w^\conj}^\conj = z^\conj w \) can be used as a representation of the geometric product (setting \( i = \Be_1 \Be_2 \) as usual.)

\section{Another simplification.}
We have sums of the form
\begin{equation}\label{eqn:bicomplexGA20:480}
   \Real(z) w \pm \Imag(z) i w
\end{equation}
above.  Let's see if those can be simplified.  For the positive case we have
\begin{equation}\label{eqn:bicomplexGA20:500}
\begin{aligned}
   \Real(z) w + \Imag(z) i w
   &=
   \inv{2} \lr{ z + z^\conj } w + \inv{2} \lr{ z - z^\conj } w \\
   &=
   z w,
\end{aligned}
\end{equation}
and for the negative case, we have
\begin{equation}\label{eqn:bicomplexGA20:520}
\begin{aligned}
   \Real(z) w - \Imag(z) i w
   &=
   \inv{2} \lr{ z + z^\conj } w - \inv{2} \lr{ z - z^\conj } w \\
   &=
   z^\conj w.
\end{aligned}
\end{equation}
This, with the vector-vector product simplification above, means that we can represent the full multivector product in this representation as just
\begin{equation}\label{eqn:bicomplexGA20:540}
M N \sim
   \lr{ m_1 n_1 + m_2^\conj n_2, m_2 n_1 + m_1^\conj n_2 }.
\end{equation}

%}
%\EndArticle
\EndNoBibArticle
