%
% Copyright � 2022 Peeter Joot.  All Rights Reserved.
% Licenced as described in the file LICENSE under the root directory of this GIT repository.
%
%{
\input{../latex/blogpost.tex}
\renewcommand{\basename}{reciprocalAndTangentspace}
%\renewcommand{\dirname}{notes/phy1520/}
\renewcommand{\dirname}{notes/ece1228-electromagnetic-theory/}
%\newcommand{\dateintitle}{}
%\newcommand{\keywords}{}

\input{../latex/peeter_prologue_print2.tex}

\usepackage{peeters_layout_exercise}
\usepackage{peeters_braket}
\usepackage{peeters_figures}
\usepackage{siunitx}
\usepackage{verbatim}
%\usepackage{mhchem} % \ce{}
%\usepackage{macros_bm} % \bcM
%\usepackage{macros_qed} % \qedmarker
%\usepackage{txfonts} % \ointclockwise

\beginArtNoToc

\generatetitle{Curvilinear coordinates and reciprocal frames.}
%\chapter{XXX}
%\label{chap:reciprocalAndTangentspace}

\section{Curvilear coordinates.}
Let's start by considering a two parameter surface specified by \( \Bx = \Bx(a,b) \).  This defines a surface, for which the partials are both tangent to at each point of the surface.  We write
\begin{equation}\label{eqn:reciprocalAndTangentspace:20}
\begin{aligned}
   \Bx_a &= \PD{a}{\Bx} \\
   \Bx_b &= \PD{b}{\Bx}.
\end{aligned}
\end{equation}
We call \( \Span{ \Bx_a, \Bx_b } \) the tangent space of the surface at the parameter values \( a,b \).  One important role of the curvilinear vectors \( \Bx_a, \Bx_b \) is to describe the area element for the subspace
\begin{equation}\label{eqn:reciprocalAndTangentspace:40}
   d\Bx_a \wedge d\Bx_b
   =
   \lr{ \Bx_a \wedge \Bx_b } da db.
\end{equation}
Observe that for a two dimensional space, this has the form
\begin{equation}\label{eqn:reciprocalAndTangentspace:60}
d\Bx_a \wedge d\Bx_b =
\begin{vmatrix}
   \Bx_a & \Bx_b
\end{vmatrix} \Bi\, da db,
\end{equation}
where \( \Bi \) is the pseudoscalar for the space.  The reader may be familiar with the determinant here, which is the Jacobian encountered in a change of variable context.
We may generalize this idea of tangent space to more variables an obvious fashion.  For example, given
\begin{equation}\label{eqn:reciprocalAndTangentspace:80}
   \Bx = \Bx(a^1, a^2, \cdots, a^M),
\end{equation}
we write
\begin{equation}\label{eqn:reciprocalAndTangentspace:100}
   \Bx_{a^i} = \PD{a^i}{\Bx}.
\end{equation}

Let's look at some examples, starting with circular coordinates in a plane
\begin{equation}\label{eqn:reciprocalAndTangentspace:120}
   \Bx(r, \theta) = r \Be_1 e^{i\theta},
\end{equation}
where \( i = \Be_1 \Be_2 \).  Our tangent space vectors are
\begin{equation}\label{eqn:reciprocalAndTangentspace:140}
   \Bx_r = \Be_1 e^{i\theta},
\end{equation}
and
\begin{dmath}\label{eqn:reciprocalAndTangentspace:160}
   \Bx_\theta
   = r \Be_1 i e^{i\theta}
   = r \Be_2 e^{i\theta}.
\end{dmath}
The area element in this case is
\begin{dmath}\label{eqn:reciprocalAndTangentspace:180}
   d\Bx_r \wedge d\Bx_\theta
   =
   \Bx_r \wedge \Bx_\theta dr d\theta
   =
   \gpgradetwo{
   \Be_1 e^{i\theta}
r \Be_2 e^{i\theta} }
dr d\theta
   = i r dr d\theta.
\end{dmath}
Integration over a circular region gives
\begin{dmath}\label{eqn:reciprocalAndTangentspace:200}
   \int_{r = 0}^R \int_{\theta=0}^{2\pi} d\Bx_r \wedge d\Bx_\theta
   =
   i \int_{r = 0}^R
   r dr
   \int_{\theta=0}^{2\pi}
   d\theta
   =
   i \frac{R^2}{2} 2 \pi
   = i \pi R^2.
\end{dmath}
This is the area of the circle, scaled by the unit bivector that represents the orientation of the plane in this two dimensional subspace.

As another example, consider a spherical parameterization, as illustrated in
\cref{fig:sphericalCoordinates:sphericalCoordinatesFig2}.
\begin{dmath}\label{eqn:reciprocalAndTangentspace:220}
   \Bx(r, \theta, \phi) = r \Be_1 e^{i\phi} \sin\theta + r \Be_3 \cos\theta.
\end{dmath}
\imageFigure{../figures/blogit/sphericalCoordinatesFig2}{Spherical coordinates.}{fig:sphericalCoordinates:sphericalCoordinatesFig2}{0.3}
Our curvilinear vectors in this case are
\begin{dmath}\label{eqn:reciprocalAndTangentspace:240}
   \Bx_r = \Be_1 e^{i\phi} \sin\theta + \Be_3 \cos\theta,
\end{dmath}
\begin{dmath}\label{eqn:reciprocalAndTangentspace:260}
   \Bx_\theta = r \Be_1 e^{i\phi} \cos\theta - r \Be_3 \sin\theta,
\end{dmath}
\begin{dmath}\label{eqn:reciprocalAndTangentspace:280}
   \Bx_\phi = r \Be_2 e^{i\phi} \sin\theta.
\end{dmath}
In this case our (pseudoscalar) volume element is
\begin{dmath}\label{eqn:reciprocalAndTangentspace:300}
   d\Bx_r  \wedge
   d\Bx_\theta \wedge
   d\Bx_\phi
   =
   r^2 \sin\theta \gpgradethree{
      \lr{ \Be_1 e^{i\phi} \sin\theta + \Be_3 \cos\theta }
      \lr{ \Be_1 e^{i\phi} \cos\theta - \Be_3 \sin\theta }
   \Be_2 e^{i\phi}
}
\, dr d\theta d\phi
   =
   r^2 \sin\theta \gpgradethree{
      \lr{ \Be_1 e^{i\phi} \sin\theta + \Be_3 \cos\theta }
      \lr{ \Be_1 \cos\theta - \Be_3 e^{-i\phi} \sin\theta }
   \Be_2
   } \, dr d\theta d\phi
   =
   r^2 \sin\theta \gpgradethree{
      \lr{ -\Be_1 \Be_3 \sin^2\theta
      +
      \Be_3 \Be_1 \cos^2\theta
   }
   \Be_2
   } \, dr d\theta d\phi
   =
   \Be_3 \Be_1 \Be_2 r^2 \sin\theta
   \, dr d\theta d\phi
   =
   I r^2 \sin \theta
   \, dr d\theta d\phi.
\end{dmath}
This is just the standard spherical volume element, but scaled with the pseudoscalar.  We would find \( \int_{r=0}^R \int_{\phi=0}^{2 \pi} \int_{\theta=0}^\pi d\Bx_r  \wedge d\Bx_\theta \wedge d\Bx_\phi = I (4/3) \pi R^3 \), the volume of the sphere, again weighted by the pseudoscalar for the space.

As a final example, let's pick the coordinates associated a relativistic boost and scale parameterization in spacetime, illustrated in
\cref{fig:boost:boostFig3}, with \( r = 1 \).
\imageFigure{../figures/blogit/boostFig3}{Boost worldline.}{fig:boost:boostFig3}{0.3}
\begin{dmath}\label{eqn:reciprocalAndTangentspace:360}
x = r \gamma_0 e^{\gamma_0 \gamma_1 \alpha}.
\end{dmath}
For this surface we have
\begin{equation}\label{eqn:reciprocalAndTangentspace:380}
\begin{aligned}
   x_r &= \gamma_0 e^{\gamma_0 \gamma_1 \alpha} \\
   x_\alpha &= r \gamma_1 e^{\gamma_0 \gamma_1 \alpha}.
\end{aligned}
\end{equation}
In this case the volume element is
\begin{dmath}\label{eqn:reciprocalAndTangentspace:400}
   dx_r \wedge dx_\alpha
   = r dr d\alpha \gpgradetwo{
      \gamma_0 e^{\gamma_{01} \alpha}
      \gamma_1 e^{\gamma_{01} \alpha}
   }
   = r dr d\alpha \gpgradetwo{
      \gamma_0
      \gamma_1
      e^{-\gamma_{01} \alpha}
      e^{\gamma_{01} \alpha}
   }
   = \gamma_{01} r dr d\alpha.
\end{dmath}
This is cosmetically similar to the circular area element above, also weighted by a pseudoscalar, but in this case, \( \alpha \) is not restricted to a bounded interval.  We also see that the basic ideas here work for both Euclidean and non-Euclidean vector spaces.

\section{Reciprocal frame vectors.}
Returning to a two dimensional surface, with tangent plane \( \Span{ \Bx_a, \Bx_b } \), any vector in that plane has the form
\begin{equation}\label{eqn:reciprocalAndTangentspace:320}
   \By = y^a \Bx_a + y^b \Bx_b.
\end{equation}
This is illustrated in \cref{fig:tangentPlane:tangentPlaneFig1}.
\imageFigure{../figures/blogit/tangentPlaneFig1}{Tangent plane for two parameter surface.}{fig:tangentPlane:tangentPlaneFig1}{0.3}

\subsection{Coordinates.}
We call \( y^a, y^b \) the coordinates of the vector \( \By \) with respect to the basis for the tangent space \( \Span{ \Bx_a, \Bx_b } \).  The computation of these coordinates is facilitated by finding the reciprocal frame \( \Bx^a, \Bx_b \) for the tangent space that satisfies both \( \Bx^a, \Bx^b \in \Span {\Bx_a, \Bx_b } \), and
\begin{equation}\label{eqn:reciprocalAndTangentspace:340}
   \Bx^\mu \cdot \Bx_\nu = {\delta^\mu}_\nu,
\end{equation}
for all \( \mu \in \setlr{a,b} \).

We may demonstrate that this works by example, dotting with each of our reciprocal frame vectors
\begin{equation}\label{eqn:reciprocalAndTangentspace:420}
\begin{aligned}
   y \cdot \Bx^a
   &=
   \lr{ y^a \Bx_a + y^b \Bx_b } \cdot \Bx^a \\
   &=
   y^a \lr{ \Bx_a \cdot \Bx^a } + y^b \lr{\Bx_b \cdot \Bx^a } \\
   &= y^a,
\end{aligned}
\end{equation}
and similarly
\begin{equation}\label{eqn:reciprocalAndTangentspace:440}
\begin{aligned}
   y \cdot \Bx^b
   &=
   \lr{ y^a \Bx_a + y^b \Bx_b } \cdot \Bx^b \\
   &=
   y^a \lr{ \Bx_a \cdot \Bx^b } + y^b \lr{\Bx_b \cdot \Bx^b } \\
   &= y^b.
\end{aligned}
\end{equation}
Provided we can find these reciprocal vectors, they provide the projections along each of the respective directions, allowing us to formulate the coordinate decomposition with respect to either the curvilinear or the reciprocal basis
\begin{equation}\label{eqn:reciprocalAndTangentspace:460}
   \By =
   \lr{ \By \cdot \Bx^a } \Bx_a
   +
   \lr{ \By \cdot \Bx^b } \Bx_b
   =
   \lr{ \By \cdot \Bx_a } \Bx^a
   +
   \lr{ \By \cdot \Bx_b } \Bx^b.
\end{equation}
This is a generalization of
\begin{equation}\label{eqn:reciprocalAndTangentspace:480}
   \Bx = \sum_i \lr{ \Bx \cdot \Be_i } \Be_i,
\end{equation}
a sum of projections, that we have for Euclidean orthonormal bases.  The reciprocal frame allows us to find the coordinates with respect to a oblique (non-orthonormal) basis,
also not imposing a requirement for the space to be Euclidean.

\subsection{Orthogonal curvilinear coordinates.}
When our tangent plane vectors are orthogonal, computation of the reciprocal frame just requires scaling.  That scaling, perhaps not suprisingly, given the name reciprocal, just requires a vector inverse.  For our two parameter case, that is just
\begin{equation}\label{eqn:reciprocalAndTangentspace:500}
   \Bx^a = \inv{\Bx_a} = \frac{\Bx_a}{\Bx_a \cdot \Bx_a}
   , \quad
   \Bx^b = \inv{\Bx_b} = \frac{\Bx_b}{\Bx_b \cdot \Bx_b}.
\end{equation}
The reader can readily verify that
\( \Bx^a \cdot \Bx_a = \Bx^b \cdot \Bx_b = 1 \), and
\( \Bx^a \cdot \Bx_b = \Bx^b \cdot \Bx_a = 0 \).

As an example, using the circular frame considered above, where we had
\begin{equation}\label{eqn:reciprocalAndTangentspace:520}
\begin{aligned}
   \Bx_r &= \Be_1 e^{i\theta} \\
   \Bx_\theta &= r \Be_2 e^{i\theta},
\end{aligned}
\end{equation}
the reciprocals are just
\begin{equation}\label{eqn:reciprocalAndTangentspace:540}
\begin{aligned}
   \Bx^r &= \Be_1 e^{i\theta} \\
\Bx^\theta &= \inv{r} \Be_2 e^{i\theta}.
\end{aligned}
\end{equation}
In this specific case, the reader can also readily verify that
\( \Bx^r \cdot \Bx_r = \Bx^\theta \cdot \Bx_\theta = 1 \), and
\( \Bx^r \cdot \Bx_\theta = \Bx^\theta \cdot \Bx_r = 0 \).

Similarly, for the spherical frame basis (\cref{eqn:reciprocalAndTangentspace:240}, ...), we have
\begin{equation}\label{eqn:reciprocalAndTangentspace:560}
   \Bx_r^2 = \Abs{e^{i\phi} \sin\theta}^2 + \cos^2\theta = 1,
\end{equation}
\begin{equation}\label{eqn:reciprocalAndTangentspace:580}
   \Bx_\theta^2 = r^2 \lr{ \Abs{e^{i\phi} \cos\theta}^2 + \sin^2\theta } = r^2,
\end{equation}
and
\begin{equation}\label{eqn:reciprocalAndTangentspace:600}
   \Bx_\phi^2 = r^2 \sin^2\theta,
\end{equation}
so the spherical reciprocals are just
\begin{equation}\label{eqn:reciprocalAndTangentspace:620}
   \Bx^r = \Be_1 e^{i\phi} \sin\theta + \Be_3 \cos\theta,
\end{equation}
\begin{equation}\label{eqn:reciprocalAndTangentspace:640}
   \Bx^\theta = \inv{r} \lr{ \Be_1 e^{i\phi} \cos\theta - \Be_3 \sin\theta},
\end{equation}
\begin{equation}\label{eqn:reciprocalAndTangentspace:660}
   \Bx^\phi = \inv{r \sin\theta} \Be_2 e^{i\phi}.
\end{equation}

Using straight inversion to compute the reciprocal frame vectors even works for non-Euclidean spaces.  Consider the following example, using the relativisitic (Dirac) basis
\begin{equation}\label{eqn:reciprocalAndTangentspace:680}
   x(a,b) = a \lr{ \gamma_1 + \gamma_2 } + b \gamma_3,
\end{equation}
for which we have
\begin{equation}\label{eqn:reciprocalAndTangentspace:700}
   x_a = \gamma_1 + \gamma_2,
\end{equation}
and
\begin{equation}\label{eqn:reciprocalAndTangentspace:720}
   x_b = \gamma_3.
\end{equation}
We have to be a bit more careful to compute the squares for this mixed metric space, but if we do that, we find
\begin{equation}\label{eqn:reciprocalAndTangentspace:740}
   x_a^2 =
   \gamma_1^2 + \gamma_2^2
= -2,
\end{equation}
and
\begin{equation}\label{eqn:reciprocalAndTangentspace:760}
   x_b^2 = -1,
\end{equation}
so
\begin{equation}\label{eqn:reciprocalAndTangentspace:780}
   x^a = -\inv{2} \lr{ \gamma_1 + \gamma_2} ,
\end{equation}
and
\begin{equation}\label{eqn:reciprocalAndTangentspace:800}
   x^b = -\gamma_3.
\end{equation}
However, other than the fact that our vectors may square to either positive or negative values, the reciprocals are still trivial to calculate.

This example also serves to point out the importance of the span constraint \( x^a, x^b \in \Span \setlr{ x_a, x_b } \).  For example, suppose we altered one of the reciprocal frames with a vector component that is orthogonal to either of the original \( x_a, x_b \) vectors, such as
\begin{equation}\label{eqn:reciprocalAndTangentspace:820}
   x^b = -\gamma_3 + 2 \gamma_0.
\end{equation}
We still have \( x^a \cdot x_a = x^b \cdot x_b = 1 \), and \( x^a \cdot x_b = x^b \cdot x_a = 0 \), but can no longer write \( y = \lr{ y \cdot x_a } x^a + \lr{ y \cdot x_b } x^b \) for any vector \( y \in \Span \setlr{ x_a, x_b } \), since this would now introduce a contribution in space that no longer lies in the tangent plane.

Another gotcha to consider for non-Euclidean spaces is that we will need some other way to compute the reciprocals if we have lightlike vectors (with zero square) as in the following parameterization
\begin{equation}\label{eqn:reciprocalAndTangentspace:840}
   x(a,b) =
   a \lr{ \gamma_0 + \gamma_1 } +
   b \lr{ \gamma_0 - \gamma_1 }.
\end{equation}
Here both of the tangent space vectors
\begin{equation}\label{eqn:reciprocalAndTangentspace:860}
\begin{aligned}
   x_a &= \gamma_0 + \gamma_1 \\
   x_b &= \gamma_0 - \gamma_1,
\end{aligned}
\end{equation}
are lightlike.  This basis spans the \(ct,x\) spacetime plane (\(\Span \setlr{ \gamma_0, \gamma_1 } \)), so we can reach any points on that plane.  Clearly it must be possible to find the coordinates of vectors on that plane with respect to this basis, but we will have to figure out how to do so.  We also do not know how to find the coordinates of vectors that lie in the tangent planes with curvilinear basis vectors that are not-orthogonal.

\subsection{Reciprocal frame for non-orthogonal coordinates.}
Now let's figure out how to compute the reciprocal vectors for the more general case where the tangent space vectors are not orthogonal.
Doing so for a two parameter surface will be sufficient, as the generalization to higher degree surfaces will be clear.

%}
%\EndArticle
\EndNoBibArticle
