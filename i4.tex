%
% Copyright � 2021 Peeter Joot.  All Rights Reserved.
% Licenced as described in the file LICENSE under the root directory of this GIT repository.
%
%{
\input{../latex/blogpost.tex}
\renewcommand{\basename}{i4}
%\renewcommand{\dirname}{notes/phy1520/}
\renewcommand{\dirname}{notes/ece1228-electromagnetic-theory/}
%\newcommand{\dateintitle}{}
%\newcommand{\keywords}{}

\input{../latex/peeter_prologue_print2.tex}

\usepackage{peeters_layout_exercise}
\usepackage{peeters_braket}
\usepackage{peeters_figures}
\usepackage{siunitx}
\usepackage{verbatim}
%\usepackage{mhchem} % \ce{}
%\usepackage{macros_bm} % \bcM
%\usepackage{macros_qed} % \qedmarker
%\usepackage{txfonts} % \ointclockwise

\beginArtNoToc

\generatetitle{XXX}
This was probably stated with the implicit assumption that 4D referred to an underlying space-time metric.  In general, we have
\begin{equation*}
\begin{aligned}
\lr{ e_1 e_2 e_3 e_4 }^2 
   &= e_1 e_2 e_3 e_4 e_1 e_2 e_3 e_4   \\
   &= -e_1^2 e_2 e_3 e_4 e_2 e_3 e_4   \\
   &= -e_1^2 e_2^2 e_3 e_4 e_3 e_4   \\
   &= +e_1^2 e_2^2 e_3^2 e_4^2.
\end{aligned}
\end{equation*}
However, for special relativistic physics, we have two choices of metric, one is:
\begin{equation*}
   e_1^2 = e_2^2 = e_3^2 = -1 = -e_4^2,
\end{equation*}
or
\begin{equation*}
   e_1^2 = e_2^2 = e_3^2 = 1 = -e_4^2,
\end{equation*}
In either case, we have \( e_1^2 e_2^2 e_3^2 e_4^2 = -1.\)

%}
%\EndArticle
\EndNoBibArticle
