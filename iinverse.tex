%
% Copyright � 2019 Peeter Joot.  All Rights Reserved.
% Licenced as described in the file LICENSE under the root directory of this GIT repository.
%
%{
\input{../latex/blogpost.tex}
\renewcommand{\basename}{iinverse}
%\renewcommand{\dirname}{notes/phy1520/}
\renewcommand{\dirname}{notes/ece1228-electromagnetic-theory/}
%\newcommand{\dateintitle}{}
%\newcommand{\keywords}{}

\input{../latex/peeter_prologue_print2.tex}

\usepackage{peeters_layout_exercise}
\usepackage{peeters_braket}
\usepackage{peeters_figures}
\usepackage{siunitx}
\usepackage{verbatim}
%\usepackage{mhchem} % \ce{}
%\usepackage{macros_bm} % \bcM
%\usepackage{macros_qed} % \qedmarker
%\usepackage{txfonts} % \ointclockwise

\beginArtNoToc

\generatetitle{XXX}
%\chapter{XXX}
%\label{chap:iinverse}
% \citep{sakurai2014modern} pr X.Y
% \citep{pozar2009microwave}
% \citep{qftLectureNotes}
% \citep{doran2003gap}
% \citep{jackson1975cew}
% \citep{griffiths1999introduction}

You can always factor a unit-bivector \( \Bi = \Bu \Bv \) (at least in 2-3 dimensions), where \( \Bu, \Bv \) are orthogonal vectors (which can be orthonormal if you choose), then
\begin{equation}\label{eqn:iinverse:n}
   \Bi^{-1} = \inv{\Bv} \inv{\Bu} = \frac{\Bv \Bu}{\Bu^2 \Bv^2}.
\end{equation}

The simplest examples are those where the bivector is some scalar multiple of two clearly orthonormal factors, such as 

\begin{equation}\label{eqn:iinverse:n}
(\Be_1 \Be_2)^{-1} = \inv{\Be_2} \inv{\Be_1} = \Be_2 \Be_1,
\end{equation}

A less trivial example is \( \Bi = \inv{\sqrt{3}}\lr{ \Be_1 \Be_2 + \Be_2 \Be_3 + \Be_3 \Be_1 } \), where possible factorizations include
\begin{equation}\label{eqn:iinverse:n}
\begin{aligned}
   \Bi 
   &= \inv{\sqrt{3}} 
   \lr{ \Be_1 + \Be_2 - 2 \Be_3 } \frac{ \Be_2 - \Be_1 }{2}
   &= \inv{\sqrt{3}} 
   \frac{ \Be_3 - \Be_2 }{2} \lr{ 2 \Be_1 - \Be_2 - \Be_3 },
\end{aligned}
\end{equation}
but once you find such factors, you can easily compute $\mathbf{i}^{-1} = -\mathbf{i}$ using these factors, for example:

\begin{equation}\label{eqn:iinverse:n}
\begin{aligned}
\Bi^{-1} 
&= 2 \sqrt{3} \frac{ 2 \Be_1 - \Be_2 - \Be_3 }{6} \frac{ \Be_3 - \Be_2 }{2 } \\
&= \inv{2 \sqrt{3}} \lr{ 2 \Be_1 - \Be_2 - \Be_3 } \lr{ \Be_3 - \Be_2 } \\
&= \inv{\sqrt{3}} \lr{ \Be_1 \Be_3 + \Be_2 \Be_1 + \Be_3 \Be_2 }
&= -\Bi.
\end{aligned}
\end{equation}

%}
\EndArticle
%\EndNoBibArticle
