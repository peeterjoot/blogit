%
% Copyright � 2025 Peeter Joot.  All Rights Reserved.
% Licenced as described in the file LICENSE under the root directory of this GIT repository.
%
%{
\input{../latex/blogpost.tex}
\renewcommand{\basename}{sqLogQuot2}
%\renewcommand{\dirname}{notes/phy1520/}
\renewcommand{\dirname}{notes/ece1228-electromagnetic-theory/}
%\newcommand{\dateintitle}{}
%\newcommand{\keywords}{}

\input{../latex/peeter_prologue_print2.tex}

\usepackage{peeters_layout_exercise}
\usepackage{peeters_braket}
\usepackage{peeters_figures}
\usepackage{siunitx}
\usepackage{verbatim}
%\usepackage{macros_cal} % \LL
%\usepackage{amsthm} % proof
%\usepackage{mhchem} % \ce{}
%\usepackage{macros_bm} % \bcM
%\usepackage{macros_qed} % \qedmarker
%\usepackage{txfonts} % \ointclockwise

\beginArtNoToc

\generatetitle{XXX}
%\chapter{XXX}
%\label{chap:sqLogQuot2}

Solve
\begin{equation}\label{eqn:sqLogQuot2:20}
I = \int \frac{ \lr{ \ln x - 1 }^2 }{ \lr{ 1 + \lr{ \ln x }^2 }^2} dx.
\end{equation}
Let \( u = \ln x \), or \( x = e^u \), so \( dx = e^u du \), and
\begin{equation}\label{eqn:sqLogQuot2:40}
I = \int \frac{ \lr{ u - 1 }^2 }{ \lr{ 1 + u^2 }^2 } e^u du.
\end{equation}
Guess
\begin{equation}\label{eqn:sqLogQuot2:60}
I = \lr{ A + \frac{B}{1 + u^2} } e^u,
\end{equation}
or
\begin{equation}\label{eqn:sqLogQuot2:80}
I' = \lr{ - \frac{ 2 B u }{\lr{1 + u^2}^2 } + A + \frac{B}{1 + u^2} } e^u.
\end{equation}
This requires
\begin{equation}\label{eqn:sqLogQuot2:100}
- 2 B u + A \lr{ 1 + u^2 }^2 + B \lr{ 1 + u^2 } = 1 - 2 u + u^2,
\end{equation}
so \( A = 0, B = 1 \).  That is
\begin{equation}\label{eqn:sqLogQuot2:120}
\begin{aligned}
I &= \frac{e^u}{1 + u^2} + C \\
&= \frac{ x }{1 + \lr{ \ln x }^2 } + C.
\end{aligned}
\end{equation}

%}
%\EndArticle
\EndNoBibArticle
