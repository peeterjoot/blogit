
\section{OLD.}
\subsection{Coordinates.}
\subsection{Vector length operation.}
\subsection{Vector dot product.}
In engineering, vectors are usually represented as \( N \times 1 \) column matrixes\footnote{Matrices are like meth or crack to engineers, who will attempt to cast everything as a matrix problem.}.
A vector \( \Bx \) with coordinates \((x,y,z)\) would have the representation
\begin{equation}\label{eqn:multivector:60}
\Bx =
\begin{bmatrix}
x \\
y \\
z
\end{bmatrix}.
\end{equation}
With a coordinate representation, we can add and subtract vectors by simply adding and subtracting their coordinates.
For example, with
\(
\Bx =
{
\begin{bmatrix}
x_1 &
x_2 &
x_3
\end{bmatrix}
}^\T,
\By =
{\begin{bmatrix}
y_1 &
y_2 &
y_3
\end{bmatrix}}^\T \), addition and scaling operations are trivial
\begin{equation}\label{eqn:multivector:140}
a \Bx + b \By
=
a
\begin{bmatrix}
x_1 \\
x_2 \\
x_3
\end{bmatrix}
+
b
\begin{bmatrix}
y_1 \\
y_2 \\
y_3
\end{bmatrix}
=
\begin{bmatrix}
a x_1 + b y_1 \\
a x_2 + b y_2 \\
a x_3 + b y_3
\end{bmatrix}.
\end{equation}

We may decompose a vector into components in each of the ``unit directions'' as follows
\begin{equation}\label{eqn:multivector:160}
\Bx =
\begin{bmatrix}
x \\
y \\
z
\end{bmatrix}
+
x
\begin{bmatrix}
1 \\
0 \\
0 \\
\end{bmatrix}
+
y
\begin{bmatrix}
0 \\
1 \\
0 \\
\end{bmatrix}
+
z
\begin{bmatrix}
0 \\
0 \\
1 \\
\end{bmatrix}.
\end{equation}
The vectors with a single one value in each of the directions may be interpretted as unit vectors, vectors with length one in each of the respective directions.
If we write
\begin{equation}\label{eqn:prerequisites:20}
\Be_1 =
\begin{bmatrix}
1 \\
0 \\
0 \\
\end{bmatrix},\quad
\Be_2 =
\begin{bmatrix}
0 \\
1 \\
0 \\
\end{bmatrix},\quad
\Be_3 =
\begin{bmatrix}
0 \\
0 \\
1 \\
\end{bmatrix}.
\end{equation}
then the matrix representation of the vector takes the form
\begin{equation}\label{eqn:prerequisites:40}
\Bx = x \Be_1 + y \Be_2 + z \Be_3.
\end{equation}
Here we have assumed that the set of vectors
\( \setlr{ \Be_1, \Be_2, \Be_3 } \) were all mutually perpendicular.
With that assumption, we call such an ordered set the standard basis.
This is a loose definition, since a proper definition of basis depends on the concept of linear independence\footnote{It is assumed that basis, linear dependence and linear independence are all familiar concepts, but I won't take the time to define them systematically yet.}.
%independent of whether the underlying representation of the unit vectors themselves are tuple, row, column, or anything else.
%
%The simplest interpretation of each of these
%There is a built in ambguity in a coordinate vector representation.
%, what can be described as the problem of implied basis.
%For example a vector \( \Bx \) with coordinates \( x, y, z \) is
%
%\subsection{Unit vectors: include?}
%In particular, the unit vectors in the x, y, z directions are
%I'll use the symbol \( \Be_i \) to designate the unit vector in the ith direction%
%\footnote{
%The notation for the unit vectors themselves varies by author.
%For example, it's not uncommon in engineering texts to use \( \hat{a}_x, \hat{a}_y, \hat{a}_z \) instead of \( \Be_1, \Be_2, \Be_3 \).  I would guess that the \(\hat{a}\) notation evolved to avoid the overloading of the symbol \(e = 2.718\cdots\).}.
%This symbolic designation allows any vector to be encoded in a
%representation agnostic fashion.
%

\subsection{Metric and dot product.}
Our niave description of a vector is a quantity with direction and magnitude.
The
coordinate representation of a vector encodes the direction, at least assuming that a vector is anchored at the zero vector (the origin), and the arrow head sits at the location specified by the coordinates.
However, we must also supplement the coordinate representation with an operation that provides the length of the vector.
That operation is the dot product, which has the following general form
\begin{equation}\label{eqn:multivector:101}
\Bx \cdot \By = \Bx^\T G \By,
\end{equation}
where \( G \) is a symmetric matrix, called the metric.
When \( G = I \), the identity matrix, then the dot product provides a compact encoding of the Euclidean length
\begin{equation}\label{eqn:multivector:80}
\Norm{\Bx} = \sqrt{ \Bx \cdot \Bx } = \sqrt{x^2 + y^2 + z^2}.
\end{equation}
The Euclidean dot product (i.e. \( G = I \)) has an absolute maximum value when the two vectors are colinear, and is zero when the two vectors are perpendicular.

The idea of a metric may appear to be useless abstraction, but it has applications in both physics and computer graphics.  In particular, the metric for the four-vectors of special relativity is
\begin{equation}\label{eqn:multivector:120}
G =
\pm
\begin{bmatrix}
-1 & 0 & 0 & 0 \\
0 & 1 & 0 & 0 \\
0 & 0 & 1 & 0 \\
0 & 0 & 0 & 1
\end{bmatrix}.
\end{equation}
Vectors can be characterized as either
``timelike'' or ``spacelike'', depending on the sign of their
``squared-length''.
This special relativistic dot product does not have the usual \( \Bx \cdot \Bx \ge 0 \) property that is usually included in an axiomatic definition of a dot product.

An engineering student may ask, ``Why would I care about special relativity?''.
Unlike some fields of physics (like classical and quantum mechanics), electromagnetism is relativistically correct to start with, requiring no corrections to generalize it.  This means that the relativisitic toolbox can be used to solve some problems with less effort.  This is one of the reasons why we care to setup our algebraic toolbox in a way that will work for both relativistic problems and the everyday 3D Euclidean geometry problems that surround us.

Non-identity metrics are also of interest in some computer graphics applications.  The interested reader is referred to the
conformal and projective geometric algebra literature for details, as the are beyond the scope of this work\footnote{The basic idea is that extra dimensions are introduced to represent the origin and to represent a point-at-infinity.}.
If all this seems too complicated, one of the saving graces is that the metric that is used in all the
geometric algebra applications that I'm aware of is always a
diagonal matrix with diagonal values that have values \( 0, \pm 1 \).
