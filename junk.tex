%
% Copyright � 2025 Peeter Joot.  All Rights Reserved.
% Licenced as described in the file LICENSE under the root directory of this GIT repository.
%
%{
\input{../latex/blogpost.tex}
\renewcommand{\basename}{junk}
%\renewcommand{\dirname}{notes/phy1520/}
\renewcommand{\dirname}{notes/ece1228-electromagnetic-theory/}
%\newcommand{\dateintitle}{}
%\newcommand{\keywords}{}

\input{../latex/peeter_prologue_print2.tex}

\usepackage{peeters_layout_exercise}
\usepackage{peeters_braket}
\usepackage{peeters_figures}
\usepackage{siunitx}
\usepackage{verbatim}
%\usepackage{macros_cal} % \LL
%\usepackage{amsthm} % proof
%\usepackage{mhchem} % \ce{}
%\usepackage{macros_bm} % \bcM
%\usepackage{macros_qed} % \qedmarker
%\usepackage{txfonts} % \ointclockwise

\beginArtNoToc

\generatetitle{XXX}
%\chapter{XXX}
%\label{chap:junk}

\begin{equation*}
\int f(x)\, dx\, g(x)
\end{equation*}

\begin{equation*}
\begin{aligned}
i \sqrt{ i \sqrt{ i^{22} }} 
&=
i \sqrt{ i \sqrt{ \lr{i^4}^5 i^2 }} \\
&=
i \sqrt{ i \sqrt{ {1}^5 \lr{ -1 } }} \\
&=
i \sqrt{ i^2 } \\
&=
i \sqrt{ -1 } \\
&=
i^2 \\
&=
-1.
\end{aligned}
\end{equation*}

\begin{equation*}
6! = \lr{3 \cdot 2 } \cdot 5! = 3! 5!
\end{equation*}

%}
%\EndArticle
\EndNoBibArticle
