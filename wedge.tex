%
% Copyright � 2020 Peeter Joot.  All Rights Reserved.
% Licenced as described in the file LICENSE under the root directory of this GIT repository.
%
%{
\input{../latex/blogpost.tex}
\renewcommand{\basename}{wedge}
%\renewcommand{\dirname}{notes/phy1520/}
\renewcommand{\dirname}{notes/ece1228-electromagnetic-theory/}
%\newcommand{\dateintitle}{}
%\newcommand{\keywords}{}

\input{../latex/peeter_prologue_print2.tex}

\usepackage{peeters_layout_exercise}
\usepackage{peeters_braket}
\usepackage{peeters_figures}
\usepackage{siunitx}
\usepackage{verbatim}
%\usepackage{mhchem} % \ce{}
%\usepackage{macros_bm} % \bcM
%\usepackage{macros_qed} % \qedmarker
%\usepackage{txfonts} % \ointclockwise

\beginArtNoToc

\generatetitle{XXX}
%\chapter{XXX}
%\label{chap:wedge}
% \citep{sakurai2014modern} pr X.Y
% \citep{pozar2009microwave}
% \citep{qftLectureNotes}
% \citep{doran2003gap}
% \citep{jackson1975cew}
% \citep{griffiths1999introduction}

\begin{equation*}
\begin{aligned}
e_1 \wedge(e_2 - \frac{1}{2}e_4) \wedge (-2 e_1 +e_4)  
&=
(-1)^2 (e_2 - \frac{1}{2}e_4) \wedge (-2 e_1 +e_4)  \wedge e_1 \\
&=
       (e_2 - \frac{1}{2}e_4) \wedge \lr{ (-2 e_1 +e_4)  \wedge e_1 } \\
&=
       (e_2 - \frac{1}{2}e_4) \wedge \lr{ e_4 \wedge e_1 } \\
&=
       e_2 \wedge e_4 \wedge e_1.
\end{aligned}
\end{equation*}

%}
\EndArticle
%\EndNoBibArticle
