%
% Copyright � 2023 Peeter Joot.  All Rights Reserved.
% Licenced as described in the file LICENSE under the root directory of this GIT repository.
%
%{
\input{../latex/blogpost.tex}
\renewcommand{\basename}{solarellipse}
%\renewcommand{\dirname}{notes/phy1520/}
\renewcommand{\dirname}{notes/ece1228-electromagnetic-theory/}
%\newcommand{\dateintitle}{}
%\newcommand{\keywords}{}

\input{../latex/peeter_prologue_print2.tex}

\usepackage{peeters_layout_exercise}
\usepackage{peeters_braket}
\usepackage{peeters_figures}
\usepackage{siunitx}
\usepackage{verbatim}
%\usepackage{mhchem} % \ce{}
%\usepackage{macros_bm} % \bcM
%\usepackage{macros_qed} % \qedmarker
%\usepackage{txfonts} % \ointclockwise

\beginArtNoToc

\generatetitle{Elliptical motion from Newton's law of gravitation.}
\chapter{Elliptical motion from Newton's law of gravitation.}
\label{chap:solarellipse}
In the last video we found that
\begin{equation}\label{eqn:solarellipse:n}
\rcap' = \inv{r} \rcap \lr{ \rcap \wedge \Bx' }.
\end{equation}

%}
%\EndArticle
\EndNoBibArticle
