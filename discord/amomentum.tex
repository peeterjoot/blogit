%
% Copyright � 2022 Peeter Joot.  All Rights Reserved.
% Licenced as described in the file LICENSE under the root directory of this GIT repository.
%
%{
\input{../latex/blogpost.tex}
\renewcommand{\basename}{amomentum}
%\renewcommand{\dirname}{notes/phy1520/}
\renewcommand{\dirname}{notes/ece1228-electromagnetic-theory/}
%\newcommand{\dateintitle}{}
%\newcommand{\keywords}{}

\input{../latex/peeter_prologue_print2.tex}

\usepackage{peeters_layout_exercise}
\usepackage{peeters_braket}
\usepackage{peeters_figures}
\usepackage{siunitx}
\usepackage{verbatim}
%\usepackage{mhchem} % \ce{}
%\usepackage{macros_bm} % \bcM
%\usepackage{macros_qed} % \qedmarker
%\usepackage{txfonts} % \ointclockwise

\beginArtNoToc

\generatetitle{Angular momentum bivector in cylindrical and spherical bases.}
%\chapter{Angular momentum bivector in cylindrical and spherical bases.}
%\label{chap:amomentum}

\section{Motivation}

In \href{https://discord.com/channels/607264339480674324/895669553227513896/1020022965737369650}{A discord thread} on the bivector group (a geometric algebra group chat), MoneyKills posts about trouble he has calculating the correct expression for the angular momentum bivector or it's dual.
% https://discord.com/channels/607264339480674324/895669553227513896/1020031923902488699

This blog post is a more long winded answer than my bivector response and includes this calculation using both cylindrical and spherical coordinates.

\section{Cylindrical coordinates.}
The position vector for any point on a plane can be expressed as
\begin{equation}\label{eqn:amomentum:20}
\Br = r \rcap,
\end{equation}
where \( \rcap = \rcap(\phi) \) encodes all the angular dependence of the position vector, and \( r \) is the length along that direction to our point, as illustrated in \cref{fig:amomentum:amomentumFig1}.
\imageFigure{../figures/blogit/amomentumFig1}{Cylindrical coordinates position vector.}{fig:amomentum:amomentumFig1}{0.3}
The radial unit vector has a compact GA representation
\begin{equation}\label{eqn:amomentum:40}
\rcap = \Be_1 e^{i\phi},
\end{equation}
where \( i = \Be_1 \Be_2 \).

The velocity (or momentum) will have both \( \rcap \) and \( \phicap \) dependence.  By chain rule, that velocity is
\begin{equation}\label{eqn:amomentum:60}
\Bv = \dot{r} \rcap + r \dot{\rcap},
\end{equation}
where
\begin{equation}\label{eqn:amomentum:80}
\begin{aligned}
\dot{\rcap}
&= \Be_1 i e^{i\phi} \dot{\phi} \\
&= \Be_2 e^{i\phi} \dot{\phi} \\
&= \phicap \dot{\phi}.
\end{aligned}
\end{equation}
It is left to the reader to show that the vector designated \( \phicap \), is a unit vector and perpendicular to \( \rcap \) (Hint: compute the grade-0 selection of the product of the two to show that they are perpendicular.)

We can now compute the momentum, which is
\begin{equation}\label{eqn:amomentum:100}
\Bp = m \Bv = m \lr{ \dot{r} \rcap + r \dot{\phi} \phicap },
\end{equation}
and the angular momentum bivector
\begin{equation}\label{eqn:amomentum:120}
\begin{aligned}
L
&= \Br \wedge \Bp \\
&= m \lr{ r \rcap } \wedge \lr{ \dot{r} \rcap + r \dot{\phi} \phicap } \\
&= m r^2 \dot{\phi} \rcap \phicap.
\end{aligned}
\end{equation}

This has the \( m r^2 \dot{\phi} \) magnitude that the OP was seeking.

\section{Spherical coordinates.}

In spherical coordinates, our position vector is
\begin{equation}\label{eqn:amomentum:140}
\Br = r \lr{ \Be_1 \sin\theta \cos\phi + \Be_2 \sin\theta \sin\phi + \Be_3 \cos\theta },
\end{equation}
as sketched in \cref{fig:amomentum:amomentumFig2}.
\imageFigure{../figures/blogit/amomentumFig2}{Spherical coordinates.}{fig:amomentum:amomentumFig2}{0.3}

We can factor this into a more compact representation
\begin{equation}\label{eqn:amomentum:160}
\begin{aligned}
\Br
&= r \lr{ \sin\theta \Be_1 (\cos\phi + \Be_{12} \sin\phi ) + \Be_3 \cos\theta } \\
&= r \lr{ \sin\theta \Be_1 e^{\Be_{12} \phi } + \Be_3 \cos\theta } \\
&= r \Be_3 \lr{ \cos\theta + \sin\theta \Be_3 \Be_1 e^{\Be_{12} \phi } }.
\end{aligned}
\end{equation}

It is useful to name two of the bivector terms above, first, we write \( i \) for the azimuthal plane bivector sketched in \cref{fig:amomentum:amomentumFig3}.
\imageFigure{../figures/blogit/amomentumFig3}{Spherical coordinates, azimuthal plane.}{fig:amomentum:amomentumFig3}{0.3}
\begin{equation}\label{eqn:amomentum:180}
i = \Be_{12},
\end{equation}
and introduce a bivector \( j \) that encodes the \( \Be_3, \rcap \) plane as sketched in \cref{fig:amomentum:amomentumFig4}.
\imageFigure{../figures/blogit/amomentumFig4}{Spherical coordinates, ``j-plane''.}{fig:amomentum:amomentumFig4}{0.3}
\begin{equation}\label{eqn:amomentum:200}
j = \Be_{31} e^{i \phi}.
\end{equation}

Having done so, we now have a compact representation for our position vector
\begin{equation}\label{eqn:amomentum:220}
\begin{aligned}
   \Br
&= r \Be_3 \lr{ \cos\theta + j \sin\theta } \\
   &= r \Be_3 e^{j \theta}.
\end{aligned}
\end{equation}

This provides us with a nice compact representation of the radial unit vector
\begin{equation}\label{eqn:amomentum:240}
\rcap = \Be_3 e^{j \theta}.
\end{equation}

%It's possible to show that the other unit vectors are
%\begin{equation}\label{eqn:amomentum:260}
%   \thetacap = \Be_1 e^{i\phi} e^{i\theta},
%\end{equation}
%and
Just as was the case in cylindrical coordinates, our azimuthal plane unit vector is
\begin{equation}\label{eqn:amomentum:280}
   \phicap = \Be_2 e^{i\phi}.
\end{equation}

Now we want to compute the velocity vector.  As was the case in cylindrical coordinates, we have
\begin{equation}\label{eqn:amomentum:300}
   \Bv = \dot{r} \rcap + r \dot{\rcap},
\end{equation}
but now we need the spherical representation for the \( \rcap \) derivative, which is
\begin{equation}\label{eqn:amomentum:320}
\begin{aligned}
\dot{\rcap}
&=
\PD{\theta}{\rcap} \dot{\theta} + \PD{\phi}{\rcap} \dot{\phi} \\
&=
\Be_3 e^{j\theta} j \dot{\theta} + \Be_3 \sin\theta \PD{\phi}{j} \dot{\phi} \\
&=
\rcap j \dot{\theta} + \Be_3 \sin\theta j i \dot{\phi}.
\end{aligned}
\end{equation}
We can reduce the second multivector term without too much work
\begin{equation}\label{eqn:amomentum:340}
\begin{aligned}
   \Be_3 j i
   &=
   \Be_3 \Be_{31} e^{i\phi} i \\
   &=
   \Be_3 \Be_{31} i e^{i\phi} \\
   &=
   \Be_{33112} e^{i\phi} \\
   &=
   \Be_{2} e^{i\phi} \\
   &= \phicap,
\end{aligned}
\end{equation}
so we have
\begin{equation}\label{eqn:amomentum:360}
\dot{\rcap}
=
\rcap j \dot{\theta} + \sin\theta \phicap \dot{\phi}.
\end{equation}

The velocity is
\begin{equation}\label{eqn:amomentum:380}
   \Bv = \dot{r} \rcap + r \lr{ \rcap j \dot{\theta} + \sin\theta \phicap \dot{\phi} }.
\end{equation}

Now we can finally compute the angular momentum bivector, which is
\begin{equation}\label{eqn:amomentum:400}
\begin{aligned}
   L &=
   \Br \wedge \Bp \\
   &=
   m r \rcap \wedge \lr{ \dot{r} \rcap + r \lr{ \rcap j \dot{\theta} + \sin\theta \phicap \dot{\phi} } } \\
   &=
   m r^2 \rcap \wedge \lr{ \rcap j \dot{\theta} + \sin\theta \phicap \dot{\phi} } \\
   &=
   m r^2 \gpgradetwo{ \rcap \lr{ \rcap j \dot{\theta} + \sin\theta \phicap \dot{\phi} } },
\end{aligned}
\end{equation}
which is just
\begin{equation}\label{eqn:amomentum:420}
   L =
   m r^2 \lr{ j \dot{\theta} + \sin\theta \rcap \phicap \dot{\phi} }.
\end{equation}

I was slightly surprised by this result, as I naively expected the cylindrical coordinate result.  We have a \( m r^2 \rcap \phicap \dot{\phi} \) term, as was the case in cylindrical coordinates, but scaled down with a \( \sin\theta \) factor.  However, this result does make sense.  Consider for example, some fixed circular motion with \( \theta = \mathrm{constant} \), as sketched in \cref{fig:amomentum:amomentumFig5}.  The radius of this circle is actually \( r \sin\theta \), so the total angular momentum for that motion is scaled down to \( m r^2 \sin\theta \dot{\phi} \), smaller than the maximum circular angular momentum of \( m r^2 \dot{\phi} \) which occurs in the \( \theta = \pi/2 \) azimuthal plane.  Similarly, if we have circular motion in the ``j-plane'', sketched in \cref{fig:amomentum:amomentumFig6}.
where \( \phi = \mathrm{constant} \), then our angular momentum is \( L = m r^2 j \dot{\theta} \).
\imageFigure{../figures/blogit/amomentumFig5}{Circular motion for constant \( \theta \).}{fig:amomentum:amomentumFig5}{0.3}
\imageFigure{../figures/blogit/amomentumFig6}{Circular motion for constant \( \phi \).}{fig:amomentum:amomentumFig6}{0.3}

%}
%\EndArticle
\EndNoBibArticle
