%
% Copyright � 2022 Peeter Joot.  All Rights Reserved.
% Licenced as described in the file LICENSE under the root directory of this GIT repository.
%
%{
\input{../latex/blogpost.tex}
\renewcommand{\basename}{trivectorAntisymmetrization}
%\renewcommand{\dirname}{notes/phy1520/}
\renewcommand{\dirname}{notes/ece1228-electromagnetic-theory/}
%\newcommand{\dateintitle}{}
%\newcommand{\keywords}{}

\input{../latex/peeter_prologue_print2.tex}

\usepackage{peeters_layout_exercise}
\usepackage{peeters_braket}
\usepackage{peeters_figures}
\usepackage{siunitx}
\usepackage{verbatim}
%\usepackage{mhchem} % \ce{}
%\usepackage{macros_bm} % \bcM
%\usepackage{macros_qed} % \qedmarker
%\usepackage{txfonts} % \ointclockwise

\beginArtNoToc

\generatetitle{XXX}
%\chapter{XXX}
%\label{chap:trivectorAntisymmetrization}

You aren't going wrong, but there's a slightly easier approach.  Expand

\begin{equation*}
\begin{aligned}
a \wedge \lr{ b \wedge c }
&=
\inv{2} \lr{ 
a \lr{ b \wedge c }
+
\lr{ b \wedge c } a 
} \\
&=
\inv{2} \lr{ 
a \lr{ b c - b \cdot c }
+
\lr{ b c - b \cdot c } a 
} \\
&=
\inv{2} \lr{ a b c + b c a } - a \lr{ b \cdot c } \\
&=
\inv{2} \lr{ a b c + b c a } - \inv{2} a \lr{ b c + c b } \\
&=
\inv{2} \lr{ b c a - a c b }.
\end{aligned}
\end{equation*}

Now note that we can toggle any pair of vectors using \( y x = 2 x \cdot y - x y \), so
\begin{equation*}
\begin{aligned}
b c a - a c b
&=
b \lr{ 2 c \cdot a - a c } - \lr{ 2 a \cdot c - c a } b \\
&=
c a b - b a c,
\end{aligned}
\end{equation*}
or
\begin{equation*}
\begin{aligned}
b c a - a c b
&=
\lr{ 2 b \cdot c - c b } a - a \lr{ 2 c \cdot b - b c } \\
&=
a b c - c b a.
\end{aligned}
\end{equation*}
That is
\begin{equation*}
b c a - a c b = c a b - b a c = a b c - c b a.
\end{equation*}

We can plug this into our first expansion of the wedge, by writing

\begin{equation*}
\begin{aligned}
a \wedge \lr{ b \wedge c }
&=
\inv{2} \lr{ b c a - a c b } \\
&=
\inv{3 \times 2} 3 \lr{ b c a - a c b } \\
&=
\inv{3!} 
\lr{
   \lr{ b c a - a c b }
   +
   \lr{ c a b - b a c }
   +
   \lr{ a b c - c b a }
} \\
&=
\inv{3!}\lr{ a b c + b c a + c a b - a c b - b a c - c b a }.
\end{aligned}
\end{equation*}
Observe that we have all the permutations of the products in this sum, each weighted by the sign of the permutation.

%}
%\EndArticle
\EndNoBibArticle
