%
% Copyright � 2022 Peeter Joot.  All Rights Reserved.
% Licenced as described in the file LICENSE under the root directory of this GIT repository.
%
%{
\input{../latex/blogpost.tex}
\renewcommand{\basename}{associativity}
%\renewcommand{\dirname}{notes/phy1520/}
\renewcommand{\dirname}{notes/ece1228-electromagnetic-theory/}
%\newcommand{\dateintitle}{}
%\newcommand{\keywords}{}

\input{../latex/peeter_prologue_print2.tex}

\usepackage{peeters_layout_exercise}
\usepackage{peeters_braket}
\usepackage{peeters_figures}
\usepackage{siunitx}
\usepackage{verbatim}
%\usepackage{mhchem} % \ce{}
%\usepackage{macros_bm} % \bcM
%\usepackage{macros_qed} % \qedmarker
%\usepackage{txfonts} % \ointclockwise

\beginArtNoToc

\generatetitle{XXX}
%\chapter{XXX}
%\label{chap:associativity}

% not mine:
% https://math.stackexchange.com/questions/4516657/apparent-inconsistency-in-geometric-product-associativity

\section{Question}

With a little bit of work, I have proven to myself that the geometric product between three vectors is associative ($a$, $b$, and $c$ are 1-vectors):
\begin{equation*}
\begin{aligned}
(ab)c
&= a(bc)  \\
&= (b \cdot c) a - (a \cdot c) b + (a \cdot b) c + a \wedge b \wedge c.
\end{aligned}
\end{equation*}

And, by extension, that it is consistent for four vectors ($a$, $b$, $c$, $d$) as:
\begin{equation*}
\begin{aligned}
   (abc)d
   &= a(bcd) \\
   &= (b \cdot c)(a \cdot d) - (a \cdot c)(b \cdot d) + (a \cdot b)(c \cdot d) \\
   &\quad+ (b \cdot c)a \wedge d - (a \cdot c)b \wedge d + (a \cdot b)c \wedge d \\
   &\quad+ b \wedge c(a \cdot d) - a \wedge c(b \cdot d) + a \wedge b(c \cdot d) \\
   &\quad+ a \wedge b \wedge c \wedge d
\end{aligned}
\end{equation*}
However, the following, which I would expect to match the rules for associativity, is quite different:
\begin{equation*}
\begin{aligned}
   (ab)(cd) &= (a \cdot b + a \wedge b)(c \cdot d + c \wedge d) \\
&= (a \cdot b)(c \cdot d) + (a \cdot b)(c \wedge d) + (a \wedge b)(c \cdot d) + (a \wedge b)(c \wedge d) \\
&= (a \cdot b)(c \cdot d) + (a \cdot b)(c \wedge d) + (a \wedge b)(c \cdot d) + (a \wedge b).(c \wedge d) + a \wedge b \wedge c \wedge d \\
&= (a \cdot b)(c \cdot d) + (a \cdot b)(c \wedge d) + (a \wedge b)(c \cdot d) + a \cdot (b \cdot (c \wedge d)) + a \wedge b \wedge c \wedge d \\
   &= (a \cdot b)(c \cdot d) + (a \cdot b)(c \wedge d) + (a \wedge b)(c \cdot d) \\
      &\quad + (a \cdot d)(b \cdot c) - (a \cdot c)(b \cdot d) \\
      &\quad + a \wedge b \wedge c \wedge d \\
&= (b \cdot c)(a \cdot d) - (a \cdot c)(b \cdot d) + (a \cdot b)(c \cdot d) \\
&\quad + (a \cdot b)(c \wedge d) + (a \wedge b)(c \cdot d) \\
&\quad + a \wedge b \wedge c \wedge d.
\end{aligned}
\end{equation*}
Which, when compared with $a(bcd)$ and $(abc)d$ has all pure scalars and the quad-vector, but is missing four of the six scaled bi-vectors.

Is it that my understanding of associativity is incorrect in that $(ab)(cd)$ is not the same as $a(bcd)$ or $(abc)d$, did I make a mistake in my algebra, or is there something else going on?

My digging into this comes from the desire to understand $e_{12} e_{23}$ which is short-hand for $(e_1 \wedge e_2)(e_2 \wedge e_3)$ (note, geometric product for $()()$) where $e_1$, $e_2$, $e_3$ are orthonormal basis vectors and I have been unable to determine whether the result should be $0$ or $e_{13}$. Also, that $e_1 e_2 e_2 e_3$ comes up in my working out of geometric products when $e_1 e_2$ is used as shorthand for $e_1 \wedge e_2$.


\section{Answer}

It looks like you are missing a factor in your expansion of \( \lr{a \wedge b} \lr{c \wedge d} \).  In general, a product of bivectors \( A B \), should have scalar, bivector, and quadvector components:
\begin{equation*}
A B = A \cdot B + \gpgradetwo{ A B } + A \wedge B,
\end{equation*}
so for \( A = a \wedge b, B = c \wedge d \), we have
\begin{equation*}
   \lr{ a \wedge b }
   \lr{ c \wedge d }
   =
   \lr{ a \wedge b }
   \cdot
   \lr{ c \wedge d }
   +
   \gpgradetwo{
   \lr{ a \wedge b }
   \lr{ c \wedge d }
}
   +
   \lr{ a \wedge b }
   \wedge
   \lr{ c \wedge d }.
\end{equation*}
Let's expand out that grade-two selection:
\begin{equation*}
\begin{aligned}
   \gpgradetwo{
   \lr{ a \wedge b }
   \lr{ c \wedge d }
}
&=
   \gpgradetwo{
   \lr{ a b - a \cdot b }
   \lr{ c \wedge d }
} \\
&=
   \gpgradetwo{
   \lr{ a b \lr{ c \wedge d } } }
   - \lr{ a \cdot b } \lr{ c \wedge d } \\
&=
   a \wedge \lr{ b \cdot \lr{ c \wedge d } }
   +
   a \cdot \lr{ b \wedge \lr{ c \wedge d } }
   - \lr{ a \cdot b } \lr{ c \wedge d } \\
\end{aligned}
\end{equation*}
The first term expands as
\begin{equation*}
   a \wedge \lr{ b \cdot \lr{ c \wedge d } }
   =
   \lr{ b \cdot c } \lr{ a \wedge d }
   -
   \lr{ b \cdot d } \lr{ a \wedge c },
\end{equation*}
and the second term as
\begin{equation*}
   a \cdot \lr{ b \wedge \lr{ c \wedge d } }
   =
\lr{ a \cdot b } \lr{ c \wedge d }
-
\lr{ a \cdot c } \lr{ b \wedge d }
+
\lr{ a \cdot d } \lr{ b \wedge c }.
\end{equation*}
Also, as you pointed out,
\begin{equation*}
\begin{aligned}
   \lr{ a \wedge b }
   \cdot
   \lr{ c \wedge d }
   &=
   \gpgradezero{
   \lr{ a \wedge b }
   \lr{ c \wedge d }
} \\
   &=
   \gpgradezero{
   \lr{ a b - a \cdot b }
   \lr{ c \wedge d }
} \\
   &=
   \gpgradezero{
   a b \lr{ c \wedge d }
} \\
&=
a \cdot \lr{ b \cdot \lr{ c \wedge d } } \\
&=
\lr{ b \cdot c } \lr{ a \cdot d }
-
\lr{ b \cdot d } \lr{ a \cdot c }.
\end{aligned}
\end{equation*}

Putting all the pieces together, we have
\begin{equation*}
\begin{aligned}
\lr{ a \wedge b } \lr{ c \wedge d }
&= \lr{ a \cdot b } \lr{ c \cdot d } + \lr{ b \cdot c } \lr{ a \cdot d } - \lr{ b \cdot d } \lr{ a \cdot c } \\
&\quad+ \lr{ b \cdot c } \lr{ a \wedge d } - \lr{ b \cdot d } \lr{ a \wedge c } \\
&\quad+ \lr{ a \cdot b } \lr{ c \wedge d } - \lr{ a \cdot c } \lr{ b \wedge d } + \lr{ a \cdot d } \lr{ b \wedge c } \\
&\quad- \lr{ a \cdot b } \lr{ c \wedge d } \\
&\quad+ a \wedge b \wedge c \wedge d
.
\end{aligned}
\end{equation*}

\section{motivation}

Because \( e_k \wedge e_j = e_k e_j \) whenever \( k \ne j \), we have
\begin{equation*}
\begin{aligned}
\lr{e_1 \wedge e_2} \lr{ e_2 \wedge e_3 }
&=
\lr{e_1 e_2} \lr{ e_2 e_3 } \\
&=
e_1 e_2 e_2 e_3  \\
&=
e_1 \lr{ e_2 e_2 } e_3  \\
&=
e_1 e_3.
\end{aligned}
\end{equation*}
As the dot and wedges of a pair of bivectors are the grade-0 and grade-4 terms respectively, you can immediately conclude that
\begin{equation*}
\lr{e_1 \wedge e_2} \cdot \lr{ e_2 \wedge e_3 } = 0,
\end{equation*}
\begin{equation*}
\lr{e_1 \wedge e_2} \wedge \lr{ e_2 \wedge e_3 } = 0,
\end{equation*}
and
\begin{equation*}
   \gpgradetwo{\lr{e_1 \wedge e_2} \lr{ e_2 \wedge e_3 }} = e_1 e_3.
\end{equation*}

\section{One expansion.}

Here's one way to expand the product of four vectors, grouping the first and last pairs.
\begin{equation*}
\begin{aligned}
a b c d
&=
\lr{ a \cdot b + a \wedge b } \lr{ c \cdot d + c \wedge d } \\
&=
\lr{ a \cdot b } \lr{ c \cdot d } \\
&+\quad
\lr{ a \cdot b } \lr{ c \wedge d }
+
\lr{ c \cdot d } \lr{ a \wedge b }  \\
&+\quad
\lr{ a \wedge b }\lr{ c \wedge d }.
\end{aligned}
\end{equation*}
The first term of this bivector-bivector product can be expanded with application of \( a b = a \cdot b + a \wedge b \) in reverse
\begin{equation*}
\begin{aligned}
\lr{ a \wedge b }\lr{ c \wedge d }
&=
\lr{ a b - a \cdot b }\lr{ c \wedge d } \\
&=
a b
\lr{ c \wedge d }
- \lr{ a \cdot b }
\lr{ c \wedge d }
\end{aligned}
\end{equation*}
The product of \( b \) with \( c \wedge d \) is
\begin{equation*}
\begin{aligned}
b
\lr{ c \wedge d }
&=
b \cdot
\lr{ c \wedge d }
+
b \wedge
\lr{ c \wedge d }  \\
&=
\lr{ b \cdot c } d
- \lr{ b \cdot d } c
+ b \wedge c \wedge d.
\end{aligned}
\end{equation*}
Multiplying this by \( a \) on the left, we have a scalar grade
\begin{equation*}
   \gpgradezero{
   a b \lr{ c \wedge d }
}
= \lr{ b \cdot c } \lr{ a \cdot d } - \lr{ b \cdot d } \lr{ a \cdot c },
\end{equation*}
a bivector grade
\begin{equation*}
\begin{aligned}
\gpgradetwo{
   a b \lr{ c \wedge d }
}
&=
\lr{ b \cdot c } \lr{ a \wedge d }
- \lr{ b \cdot d } \lr{ a \wedge c }
+ a \cdot \lr{ b \wedge c \wedge d } \\
&=
\lr{ b \cdot c } \lr{ a \wedge d }
- \lr{ b \cdot d } \lr{ a \wedge c } \\
&\quad + \lr{ a \cdot b } \lr{ c \wedge d }
- \lr{ a \cdot c } \lr{ b \wedge d }
+ \lr{ a \cdot d } \lr{ b \wedge c },
\end{aligned}
\end{equation*}
and a grade-four component
\begin{equation*}
\gpgradefour{
   a b \lr{ c \wedge d }
}
=
   a \wedge b \wedge c \wedge d.
\end{equation*}
Putting all the pieces together, here are all the grades of the \( a b c d \) product
\begin{equation*}
\begin{aligned}
   \gpgradezero{ a b c d }
   &=
\lr{ a \cdot b } \lr{ c \cdot d }
+
\lr{ b \cdot c } \lr{ a \cdot d } - \lr{ b \cdot d } \lr{ a \cdot c } \\
   \gpgradetwo{ a b c d }
   &=
  \lr{ a \cdot b } \lr{ c \wedge d }
+ \lr{ c \cdot d } \lr{ a \wedge b }
- \lr{ a \cdot b } \lr{ c \wedge d } \\
&\quad
+ \lr{ b \cdot c } \lr{ a \wedge d }
- \lr{ b \cdot d } \lr{ a \wedge c }
+ \lr{ a \cdot b } \lr{ c \wedge d } \\
&\quad
- \lr{ a \cdot c } \lr{ b \wedge d }
+ \lr{ a \cdot d } \lr{ b \wedge c } \\
   &=
  \lr{ a \cdot b } \lr{ c \wedge d }
+ \lr{ c \cdot d } \lr{ a \wedge b }
+ \lr{ b \cdot c } \lr{ a \wedge d } \\
&\quad - \lr{ b \cdot d } \lr{ a \wedge c }
- \lr{ a \cdot c } \lr{ b \wedge d }
+ \lr{ a \cdot d } \lr{ b \wedge c } \\
   \gpgradefour{ a b c d }
   &=
   a \wedge b \wedge c \wedge d.
\end{aligned}
\end{equation*}

%}
\EndArticle
%\EndNoBibArticle
