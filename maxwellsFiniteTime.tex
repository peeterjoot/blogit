%
% Copyright � 2022 Peeter Joot.  All Rights Reserved.
% Licenced as described in the file LICENSE under the root directory of this GIT repository.
%
%{
\input{../latex/blogpost.tex}
\renewcommand{\basename}{maxwellsFiniteTime}
%\renewcommand{\dirname}{notes/phy1520/}
\renewcommand{\dirname}{notes/ece1228-electromagnetic-theory/}
%\newcommand{\dateintitle}{}
%\newcommand{\keywords}{}

\input{../latex/peeter_prologue_print2.tex}

\usepackage{peeters_layout_exercise}
\usepackage{peeters_braket}
\usepackage{peeters_figures}
\usepackage{siunitx}
\usepackage{verbatim}

\beginArtNoToc

\generatetitle{Maxwell's equation: finite time differences.}
%\chapter{XXX}
%\label{chap:maxwellsFiniteTime}

Starting point is \citep{pjootGAEE}, writing Maxwell's equations as

\begin{subequations}
\begin{equation}\label{eqn:maxwellsFiniteTime:20}
\lr{\spacegrad + \inv{c} \PD{t}{}} F = J
\end{equation}
\begin{equation}\label{eqn:maxwellsFiniteTime:40}
F = \BE + I \eta \BH \quad(= \BE + I c \BB)
\end{equation}
\begin{equation}\label{eqn:maxwellsFiniteTime:60}
   J = \eta \lr{ c \rho - \BJ } + I \lr{ c \rho_\txtm - \BM }.
\end{equation}
\end{subequations}

Solving for just the time variation, we have
\begin{equation}\label{eqn:maxwellsFiniteTime:80}
\PD{t}{F} = c J - c \spacegrad F,
\end{equation}
but since the LHS has just vector and bitvector terms, we can apply a grade-1,2 selection operator to both sides to find
\begin{equation}\label{eqn:maxwellsFiniteTime:100}
\begin{aligned}
\PD{t}{F}
&=
\gpgrade{
c J - c \spacegrad F}{1,2} \\
&=
- c \eta \BJ - c I \BM
- c
\gpgrade{ \spacegrad F}{1,2} \\
&=
- \frac{\BJ}{\epsilon} - c I \BM
- c
\gpgrade{ \spacegrad F}{1,2}.
\end{aligned}
\end{equation}
% c \eta = \sqrt{ \mu/\epsilon    x    1/\mu\epsilon}
% c \eta = 1/\epsilon

Writing \( F(t_k) = F_k, t_k = t_0 + k \Delta t \), we may form a finite time approximation
as follows
\begin{equation}\label{eqn:maxwellsFiniteTime:120}
   \PD{t}{F} \approx \frac{F_{k+1} - F_k}{\Delta t}.
\end{equation}
Here, I've arbitarily opted for forward time differences, but backwards differences could be used if desired.  Plugging this into Maxwell's equation, we have
\begin{equation}\label{eqn:maxwellsFiniteTime:140}
F_{k+1} =
F_k - \Delta t \lr{
  \frac{\BJ}{\epsilon} + c I \BM
+ c
\gpgrade{ \spacegrad F_k}{1,2}
},
\end{equation}
where \( \BJ, \BM \) are also evaluated at \( t = t_k \).

\section{Component form.}
In your PDF you expanded everything in terms of the field components.  We can do that here too, selecting the vector and bivector components of \cref{eqn:maxwellsFiniteTime:100} to find
%%\begin{subequations}
%\begin{equation}\label{eqn:freespace:3399}
%\begin{aligned}
%\spacegrad \cross \BE &= - \BM - \PD{t}{\BB} \\
%c^2 \spacegrad \cross \BB &= \inv{\epsilon} \BJ + \PD{t}{\BE},
%\end{aligned}
%\end{equation}
%or
%\PD{t}{\BB} &= -\BM - \spacegrad \cross \BE
%\PD{t}{\BE} = c^2 \spacegrad \cross \BB - \inv{\epsilon} \BJ
\begin{equation}\label{eqn:maxwellsFiniteTime:3419}
\begin{aligned}
\PD{t}{\BE} &= - \frac{\BJ}{\epsilon} - c^2 I \lr{ \spacegrad \wedge \BB } \\
\PD{t}{I c \BB} &= - c I \BM - c \spacegrad \wedge \BE,
\end{aligned}
\end{equation}
or
\begin{equation}\label{eqn:maxwellsFiniteTime:3439}
\begin{aligned}
\PD{t}{\BE} &= - \frac{\BJ}{\epsilon} + c^2 \spacegrad \cross \BB \\
\PD{t}{\BB} &= - \BM - \spacegrad \cross \BE,
\end{aligned}
\end{equation}
just takes us full circle, since these are just the Ampere-Maxwell, and Maxwell-Faraday equations respectively.

%}
\EndArticle
