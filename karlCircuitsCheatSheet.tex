%{ % Ignore this.  It's a trick for brace matching in an editor called vim that Peeter uses.
\documentclass[letterpaper]{scrartcl}
\usepackage[margin=1in,top=0.5in]{geometry}
\usepackage{amsmath}
\usepackage{caption}
\usepackage{graphicx}
\usepackage{siunitx}
\usepackage{wrapfig}
\usepackage[english]{cleveref}

% for a centered figure with caption (not used here)
% \simpleFigure{path}{caption}{label}{width}
\newcommand{\simpleFigure}[4]{%
  \begin{figure}[htp]%
    \centering%
    \includegraphics[width=#4\textwidth]{#1}%
    \caption{#2}%
    \label{#3}%
  \end{figure}%
}

% for a figure off to the left or to the right (default is left), with a caption and label
% \sideFigure{path}{caption}{label}{width}[side]
\newcommand{\sideFigure}[5][l]{%
  \begin{wrapfigure}{#1}{#5\textwidth}%
    \centering%
    \includegraphics[width=#5\textwidth]{#2}%
    \caption{#3}%
    \label{#4}%
  \end{wrapfigure}%
}
% for manually adjusting spaces before or after \sideFigure
% \figspace = Space after a figure (1.5 baselineskip)
\newcommand{\figspace}{\vspace{1.5\baselineskip}}
% \sectionspace = Space between sections (3 baselineskip)
\newcommand{\sectionspace}{\vspace{3\baselineskip}}

% This command is a shortcut meant to do a balanced set of ( ) braces (using \left and \right makes them resize themselves bigger automatically)
\newcommand{\lr}[1]{\left(#1\right)}

% This command is a shortcut:
% \inv{x} = 1/x.
\newcommand{\inv}[1]{\frac{1}{#1}}

% DeclareMathOperator{\Real}{Re} below is equivalent to \operatorname{Re} which puts Re in a mathrm font and adjusts spacing around it like built-in operators like \sin, \cos, ...
\DeclareMathOperator{\Real}{Re}
\DeclareMathOperator{\Arg}{Arg}
\DeclareMathOperator{\Imag}{Im}
\DeclareMathOperator{\conj}{conj}
\DeclareMathOperator{\atan2}{atan2}

% \phasor{X} = Bold phasor notation for X (e.g., \phasor{V} for voltage phasor)
\newcommand{\phasor}[1]{\mathbf{#1}}
% \thev = Thevenin subscript (Th)
\newcommand{\thev}[0]{\mathrm{Th}}
% \load = Load subscript (Load)
\newcommand{\load}[0]{\mathrm{Load}}

% \Max = Maximum subscript (max)
\newcommand{\Max}[0]{\mathrm{max}}
% \Avg = Average subscript (avg)
\newcommand{\Avg}[0]{\mathrm{avg}}
% \rms = RMS subscript (rms)
\newcommand{\rms}[0]{\mathrm{rms}}
% \Abs{x} = Absolute value of x with automatic sizing
\newcommand{\Abs}[1]{{\left\lvert{#1}\right\rvert}}

\begin{document}
\begin{center}
\Large \textbf{Circuits Review Cheatsheet}
\end{center}
% DC Circuit Analysis
\subsection*{Current Division}
\begin{minipage}{\textwidth}
For a DC current source with parallel resistors, such as the one illustrated in \cref{fig:currentSource:currentSourceFig1}, we have
\end{minipage}
\sideFigure{../figures/blogit/currentSourceFig1}{Current Source}{fig:currentSource:currentSourceFig1}{0.3}
\figspace
\begin{equation}\label{eqn:karlCircuitsCheatSheet:20}
I_1 = I_s \frac{R_2}{R_1 + R_2}
\end{equation}
\begin{equation}\label{eqn:karlCircuitsCheatSheet:40}
I_2 = I_s \frac{R_1}{R_1 + R_2}.
\end{equation}
\WFclear
\sectionspace
\subsection*{Voltage Division}
\begin{minipage}{\textwidth}
For a DC voltage source with series resistors, such as the one illustrated in \cref{fig:voltageSource:voltageSourceFig2}, we have
\end{minipage}
\sideFigure{../figures/blogit/voltageSourceFig2}{Voltage Source}{fig:voltageSource:voltageSourceFig2}{0.3}
\figspace
\begin{equation}\label{eqn:karlCircuitsCheatSheet:60}
V = V_2 \frac{R_2}{R_1 + R_2}.
\end{equation}
\WFclear
\figspace
\sectionspace
\subsection*{Ohm's Law and Power}
\begin{itemize}
\item Ohm's Law: \( V = I R \).
\item Power: \( P = V I = I^2 R = \frac{V^2}{R} \).
\end{itemize}
% Linear Algebra for Circuit Analysis
\subsection*{Cramer's Rule}
The solution of
\begin{equation}\label{eqn:karlCircuitsCheatSheet:380}
\begin{bmatrix}
a & b \\
c & d
\end{bmatrix}
\begin{bmatrix}
x \\
y
\end{bmatrix}
=
\begin{bmatrix}
e \\
f
\end{bmatrix},
\end{equation}
is given by
\begin{equation}\label{eqn:karlCircuitsCheatSheet:100}
x = \frac{
   \begin{vmatrix}
   e & b \\
   f & d
   \end{vmatrix}
}{
   \begin{vmatrix}
   a & b \\
   c & d
   \end{vmatrix}
},
\qquad
y = \frac{
   \begin{vmatrix}
   a & e \\
   c & f
   \end{vmatrix}
}{
   \begin{vmatrix}
   a & b \\
   c & d
   \end{vmatrix}
}.
\end{equation}
\subsection*{Matrix inverse}
\Cref{eqn:karlCircuitsCheatSheet:380} may also be solved directly with less hassle than Cramer's rule (when you aren't specifically asked to use Cramer's)
\begin{equation}\label{eqn:karlCircuitsCheatSheet:420}
\begin{bmatrix}
x \\
y
\end{bmatrix}
=
\inv{a d - b c}
\begin{bmatrix}
d  & -b \\
-c & a
\end{bmatrix}
\begin{bmatrix}
e \\
f
\end{bmatrix}.
\end{equation}

Note that this works because
\begin{equation}\label{eqn:karlCircuitsCheatSheet:440}
\begin{bmatrix}
a & b \\
c & d
\end{bmatrix}
\begin{bmatrix}
d  & -b \\
-c & a
\end{bmatrix}
=
\begin{bmatrix}
a d - b c & 0         \\
0         & a d - b c
\end{bmatrix}.
\end{equation}

\subsection*{RLC Circuit equations}
\begin{minipage}{\textwidth}
For time domain analysis of circuits with resistors, capacitors, and inductors, for example, those of the simple circuit sketched in \cref{fig:rcCircuit:rcCircuitFig2}, we have
\end{minipage}
\sideFigure{../figures/blogit/rcCircuitFig2}{Circuit element}{fig:rcCircuit:rcCircuitFig2}{0.3}
\figspace
\begin{equation}\label{eqn:karlCircuitsCheatSheet:120}
v_R = i_R R
\end{equation}
\begin{equation}\label{eqn:karlCircuitsCheatSheet:130}
v_L = L \frac{di_L}{dt}
\end{equation}
\begin{equation}\label{eqn:karlCircuitsCheatSheet:140}
i_C = C \frac{dv_C}{dt}.
\end{equation}

See Phasors below for the frequency domain.
\WFclear
\figspace
\sectionspace
\subsection*{Complex numbers}
\begin{minipage}{\textwidth}
The rectangular form of a complex number (or phasor), such as the one illustrated in \cref{fig:cheatSheat:cheatSheatFig3}, is
\end{minipage}
\sideFigure{../figures/blogit/cheatSheatFig3}{Complex number}{fig:cheatSheat:cheatSheatFig3}{0.3}
\figspace
\begin{equation}\label{eqn:karlCircuitsCheatSheet:160}
z = a + j b,
\end{equation}
where \( a \) is the real part, and \( j b \) is the imaginary part.

The polar form of a complex number (i.e.: of a phasor) is
\begin{equation}\label{eqn:karlCircuitsCheatSheet:180}
z = r \angle \theta,
\end{equation}
where \( r = \sqrt{a^2 + b^2} \), and \( \theta = \Arg z = \atan2\lr{b, a} \).

The exponential form of a complex number is
\begin{equation}\label{eqn:karlCircuitsCheatSheet:200}
z = r e^{j\theta} = r \lr{ \cos\theta + j \sin\theta }.
\end{equation}
\WFclear
\figspace
%\sectionspace

The polar and rectangular forms of a complex number can be related by noting that
\begin{equation}\label{eqn:karlCircuitsCheatSheet:520}
\begin{aligned}
a &= r \cos\theta \\
b &= r \sin \theta.
\end{aligned}
\end{equation}

Note that the \( \atan2 \) function, as used above, is what you get when you use the Arg feature of your calculator, it is like \( \tan^{-1}\lr{ {b}/{a} } \), but
corrects for the quadrant that the point \(\lr{a,b}\) is in (since arctan needs correction by \( \pm \pi \) in some quadrants.)

Observe that
\begin{equation}\label{eqn:karlCircuitsCheatSheet:220}
\begin{aligned}
\sqrt{z} &= \sqrt{r} e^{j\theta/2}   = \sqrt{r} \angle \frac{\theta}{2} \\
\inv{z}  &= \inv{r} e^{-j\theta}     = \inv{r} \angle -\theta \\
\conj z &= a - j b = r e^{-j\theta} = r \angle -\theta.
\end{aligned}
\end{equation}

Given two complex numbers \( z_1 = r_1 e^{j\theta_1}, z_2 = r_2 e^{j\theta_2} \), then their products and ratios are
\begin{equation}\label{eqn:karlCircuitsCheatSheet:240}
\begin{aligned}
z_1 z_2         &= r_1 r_2 e^{j \lr{\theta_1 + \theta_2 } }         = r_1 r_2 \angle \lr{ \theta_1 + \theta_2 } \\
\frac{z_1}{z_2} &= \frac{r_1}{r_2} e^{j \lr{\theta_1 - \theta_2 } } = \frac{r_1}{r_2} \angle \lr{ \theta_1 - \theta_2 }.
\end{aligned}
\end{equation}

Let's applying the phasor notation to voltages and currents.  For example, given a time dependent cosinusoidal voltage with constant magnitude \( V \), we have the following equivalences

% AC Circuit Analysis with Phasors
\subsection*{Phasors}
Phasors simplify AC circuit analysis by converting differential equations to algebraic ones in the frequency domain. For a time-dependent cosinusoidal voltage with constant magnitude \( V \), we have
\begin{equation}\label{eqn:karlCircuitsCheatSheet:260}
\begin{aligned}
v(t)       &= V \cos\lr{ \omega t + \theta } = \Real \lr{ V e^{j \lr{\omega t + \theta} } } \\
\phasor{V} &= V e^{j \theta} = V \angle \theta = V e^{j \theta} \\
v(t)       &= \Real \lr{ \phasor{V} e^{j \omega t} },
\end{aligned}
\end{equation}
where the phasor has been written \( \phasor{V} \).  This representation applies to
sinusoidal signals at a fixed angular frequency \( \omega \), in radians per second, often related to frequency \(f\) by \( \omega = 2 \pi f \).
Bold has been used here for the phasor, which works nice in print, but there are other common conventions, including \(\underline{\phasor{V}}\) or \(\overline{\phasor{V}}\).

Similarly, given a sinusoidal time dependent voltage
\begin{equation}\label{eqn:karlCircuitsCheatSheet:320}
\begin{aligned}
v(t)       &= V \sin\lr{ \omega t + \theta }            = \Real \lr{ -j V e^{j\lr{ \omega t + \theta } } } \\
\phasor{V} &= V \angle \lr{ \theta - \SI{90}{\degree}}  = V e^{j \lr{ \theta - \SI{90}{\degree}} },
\end{aligned}
\end{equation}
since
\begin{equation}\label{eqn:karlCircuitsCheatSheet:400}
-j = e^{-j \pi/2} = \angle \SI{-90}{\degree}.
\end{equation}
Once again \( v(t) = \Real \lr{ \phasor{V} e^{j \omega t} } \).

For a sinusoidal current \( i(t) = I \cos(\omega t + \phi) \), the phasor is \( \phasor{I} = I e^{j \phi} = I \angle \phi \), and \( i(t) = \Real \lr{ \phasor{I} e^{j \omega t} } \).

\subsection*{Impedance of RLC Elements}
For resistors, inductors, and capacitors in the frequency domain:
\begin{equation}\label{eqn:karlCircuitsCheatSheet:460}
\begin{aligned}
Z_R &= R \\
Z_L &= j \omega L \\
Z_C &= \frac{1}{j \omega C} = -j \frac{1}{\omega C}.
\end{aligned}
\end{equation}

\subsection*{Kirchhoff's Laws}
KCL: The sum of currents entering a node equals the sum of currents leaving:
\begin{equation}\label{eqn:karlCircuitsCheatSheet:480}
\sum I_{\text{in}} = \sum I_{\text{out}}
\end{equation}
KVL: The sum of voltage drops around a closed loop is zero:
\begin{equation}\label{eqn:karlCircuitsCheatSheet:500}
\sum V = 0.
\end{equation}

\subsection*{Impedance and Maximum Power Transfer}
The Thevenin equivalent impedance of a circuit is
\begin{equation}\label{eqn:karlCircuitsCheatSheet:340}
Z_\thev = R_\thev + j X_\thev,
\end{equation}
where \( R_\thev \) is the resistance and \( X_\thev \) is the reactance (positive for inductive, negative for capacitive). For maximum power transfer to a load, the load impedance should be the conjugate:
\begin{equation}\label{eqn:karlCircuitsCheatSheet:360}
Z_\load = \conj{Z_\thev} = R_\thev - j X_\thev.
\end{equation}

\subsection*{Supermesh and Supernodes}
Use supermesh analysis for circuits with current sources between nodes, and supernode analysis for circuits with voltage sources between nodes. For a supermesh, exclude the branch with the current source and write a KVL equation around the resulting loop. For a supernode, treat the nodes connected by the voltage source as a single node and apply KCL.

\subsection*{Maximum Load Power}
\begin{equation}\label{eqn:karlCircuitsCheatSheet:540}
\begin{aligned}
R_\load &= R_\thev \\
Z_\thev + Z_\load &= 2 R_\thev
\end{aligned}
\end{equation}
\begin{equation}\label{eqn:karlCircuitsCheatSheet:560}
\begin{aligned}
P_\Max
&= I_\load^2 R_\load \\
&= \lr{ \frac{ \Abs{V_\thev} }{ Z_\thev + Z_\load } }^2 R_\load \\
&= \lr{ \frac{ \Abs{V_\thev} }{ 2 R_\thev} }^2 R_\thev \\
&= \frac{ \Abs{V_\thev}^2 }{ 4 R_\thev }.
\end{aligned}
\end{equation}

\subsection*{Average Power}
Given
\begin{equation}\label{eqn:karlCircuitsCheatSheet:580}
\begin{aligned}
V(t) &= V_\Max \cos\lr{ \omega t + \theta_V } \\
I(t) &= I_\Max \cos\lr{ \omega t + \theta_I },
\end{aligned}
\end{equation}
The average power is
\begin{equation}\label{eqn:karlCircuitsCheatSheet:600}
\begin{aligned}
P_\Avg
&= \inv{T} \int_0^T P(t) dt \\
&= \inv{2 T} \int_0^T I_\Max V_\Max \lr{ \cos\lr{\theta_V - \theta_I} + \cos\lr{ 2 \omega t + \theta_V + \theta_I } } dt \\
&= \inv{2 T} I_\Max V_\Max \cos\lr{\theta_V - \theta_I} T,
\end{aligned}
\end{equation}
or
\begin{equation}\label{eqn:karlCircuitsCheatSheet:620}
P_\Avg = \inv{2} I_\Max V_\Max \cos\lr{\theta_V - \theta_I}.
\end{equation}
\subsection*{Effective or RMS Value}
\begin{equation}\label{eqn:karlCircuitsCheatSheet:640}
\begin{aligned}
I_\rms &= \frac{I_\Max}{\sqrt{2}} \\
V_\rms &= \frac{V_\Max}{\sqrt{2}},
\end{aligned}
\end{equation}
so
\begin{equation}\label{eqn:karlCircuitsCheatSheet:660}
P_\Avg = I_\rms V_\rms \cos\lr{\theta_V - \theta_I}.
\end{equation}

\subsection*{Apparent Power and Power Factor}
\begin{equation}\label{eqn:karlCircuitsCheatSheet:680}
\begin{aligned}
S
&= I_\rms V_\rms \angle \lr{\theta_V - \theta_I} \\
&= I_\rms V_\rms e^{j \lr{\theta_V - \theta_I}} \\
&= I_\rms V_\rms \lr{ \cos \lr{\theta_V - \theta_I} +j \sin \lr{\theta_V - \theta_I} }.
\end{aligned}
\end{equation}
The apparent power is
\begin{equation}\label{eqn:karlCircuitsCheatSheet:700}
\Abs{S} = I_\rms V_\rms,
\end{equation}
so
\begin{equation}\label{eqn:karlCircuitsCheatSheet:720}
P_\Avg = \Abs{S} \cos \lr{\theta_V - \theta_I},
\end{equation}
where
\begin{equation}\label{eqn:karlCircuitsCheatSheet:740}
P_f = \frac{P_\Avg}{\Abs{S}} = \cos \lr{\theta_V - \theta_I},
\end{equation}
is called the power factor.

\subsection*{Complex Power}
\begin{minipage}{\textwidth}
Complex power can be written in rectangular form as \( S = P + j Q \), as illustrated in \cref{fig:cheatSheat:cheatSheatFig5}, where
\end{minipage}
\sideFigure{../figures/blogit/cheatSheatFig5}{Power Triangle}{fig:cheatSheat:cheatSheatFig5}{0.3}
\figspace
\begin{equation}\label{eqn:karlCircuitsCheatSheet:820}
\begin{aligned}
P       &= \Real(S) = \Abs{S} \cos\lr{ \theta_V - \theta_I } \\
Q       &= \Imag(S) = \Abs{S} \sin\lr{ \theta_V - \theta_I } \\
\Abs{S} &= \sqrt{ P^2 + Q^2 },
\end{aligned}
\end{equation}
and
\begin{equation}\label{eqn:karlCircuitsCheatSheet:800}
\frac{P}{\Abs{S}} = \cos\lr{ \theta_V - \theta_I }.
\end{equation}
\begin{equation}\label{eqn:karlCircuitsCheatSheet:840}
\boxed{\textrm{Real Power} = P_\Avg}.
\end{equation}

\WFclear
\figspace
\sectionspace

\subsection*{Maximum Average Power}
\begin{equation}\label{eqn:karlCircuitsCheatSheet:760}
\begin{aligned}
\Max P_\Avg
&= \frac{\Abs{V_\thev}^2}{8 R_\thev } \\
&= \inv{2} I_\load^2 R_\load.
\end{aligned}
\end{equation}

\subsection*{Impedence}
With \( \phasor{V} = V \angle \theta_V \),
\begin{equation}\label{eqn:karlCircuitsCheatSheet:780}
Z = \frac{V}{I} = \frac{V_\rms}{I_\rms} \angle \lr{ \theta_V - \theta_I }
\end{equation}

\end{document}
%} % Ignore this.  It's a trick for brace matching in an editor called vim that Peeter uses.
