%
% Copyright � 2021 Peeter Joot.  All Rights Reserved.
% Licenced as described in the file LICENSE under the root directory of this GIT repository.
%
%{
\input{../latex/blogpost.tex}
\renewcommand{\basename}{dotwedgedef}
%\renewcommand{\dirname}{notes/phy1520/}
\renewcommand{\dirname}{notes/ece1228-electromagnetic-theory/}
%\newcommand{\dateintitle}{}
%\newcommand{\keywords}{}

\input{../latex/peeter_prologue_print2.tex}

\usepackage{peeters_layout_exercise}
\usepackage{peeters_braket}
\usepackage{peeters_figures}
\usepackage{siunitx}
\usepackage{verbatim}
%\usepackage{mhchem} % \ce{}
%\usepackage{macros_bm} % \bcM
%\usepackage{macros_qed} % \qedmarker
%\usepackage{txfonts} % \ointclockwise

\beginArtNoToc

\generatetitle{XXX}

The best way to view all of these identities is consequences of grade selection.  We say that scalars have grade 0, vectors grade 1, bivectors are grade 2, and so forth.  Examples, assuming that \( \setlr{ \Be_1, \Be_2, \cdots } \) is an orthonormal Euclidean basis, then
\begin{equation*}
-3, \pi, 7, 9, \cdots
\end{equation*}
are all scalars with grade 0,
\begin{equation*}
   \Be_1, \Be_2, 3 \Be_3 + \pi \Be_7, \cdots
\end{equation*}
are all vectors with grade 1,
\begin{equation*}
   \Be_1 \Be_2, \Be_2 \Be_3, 3 \Be_3 \Be_7 + \pi \Be_7 \Be_1, \cdots
\end{equation*}
are all bivectors and have grade 2.
We write
\begin{equation*}
   \gpgrade{A}{k}
\end{equation*}
as the selection of all grade \( k \) elements from the multivector \( A \).  For example, if
\begin{equation*}
   A = -3 + \pi+ 7+ 9 + \Be_1+ \Be_2+ 3 \Be_3 + \pi \Be_7 + \Be_1 \Be_2+ \Be_2 \Be_3+ 3 \Be_3 \Be_7 + \pi \Be_7 \Be_1,
\end{equation*}
then
\begin{equation*}
   \gpgrade{A}{0} = \pi+ 13,
\end{equation*}
\begin{equation*}
   \gpgrade{A}{1} = \Be_1+ \Be_2+ 3 \Be_3 + \pi \Be_7 ,
\end{equation*}
\begin{equation*}
   \gpgrade{A}{2} =
   \Be_1 \Be_2+ \Be_2 \Be_3+ 3 \Be_3 \Be_7 + \pi \Be_7 \Be_1,
\end{equation*}
and \( \gpgrade{A}{k} \) for all \( k > 2 \) is zero (in this case.)

Now, we can consider the geometric product of two vectors \( a = \sum_i x_i \Be_i \) and \( b = \sum_i y_i \Be_i \)
\begin{equation*}
\begin{aligned}
   a b
   &= \sum_{i,j} x_i \Be_i y_j \Be_j \\
   &= \sum_{i = j} x_i y_i \Be_i^2 + \sum_{i \ne j} x_i y_j \Be_i \Be_j \\
   &= \sum_{i} x_i y_i + \sum_{i \ne j} x_i y_j \Be_i \Be_j \\
   &= \sum_{i} x_i y_i + \sum_{i < j} \lr{ x_i y_j - x_j y_i } \Be_i \Be_j \\
\end{aligned}
\end{equation*}
Observe that the first sum is a scalar, and the second sum has only grade two terms.  This means that for any product of two vectors we must have
\begin{equation*}
   a b = \gpgrade{a b}{0} + \gpgrade{a b}{2}.
\end{equation*}
Clearly we wish to identify the scalar component as the dot product, and we define the bivector component to be the wedge product:
\begin{equation*}
   a \cdot b = \gpgrade{ a b }{0},
\end{equation*}
\begin{equation*}
   a \wedge b = \gpgrade{ a b }{2}.
\end{equation*}

You should be able to convince yourself that a product of a vector and r-blade \( A_r \) (an entity with only grade r components) must have only grades \( r-1 \) and \( r + 1 \), so
\begin{equation*}
   a A_r = \gpgrade{a A_r}{r-1} + \gpgrade{a A_r}{r+1},
\end{equation*}
so we define
\begin{equation*}
   a \cdot A_r = \gpgrade{ a A_r }{r-1},
\end{equation*}
\begin{equation*}
   a \wedge A_r = \gpgrade{ a A_r }{r+1},
\end{equation*}
so that
\begin{equation*}
   a A_r = a \cdot A_r + a \wedge A_r.
\end{equation*}

Still more generally, answering your final question, one can define the dot and wedge products of two blades \( A_r \), \( B_s \) with grades \( r, s \) respectively, as
\begin{equation*}
   A_r \cdot B_s = \gpgrade{ A_r B_s }{\abs{r-s}},
\end{equation*}
\begin{equation*}
   A_r \wedge B_s = \gpgrade{ A_r B_s }{r+s}.
\end{equation*}
Note that there is some variability in conventions and notations for dot product like operators, but these are definitions that are consistent with those used in New Foundations.

%}
%\EndArticle
\EndNoBibArticle
