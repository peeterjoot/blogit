%
% Copyright � 2021 Peeter Joot.  All Rights Reserved.
% Licenced as described in the file LICENSE under the root directory of this GIT repository.
%
%{
\input{../latex/blogpost.tex}
\renewcommand{\basename}{mandelbrot}
%\renewcommand{\dirname}{notes/phy1520/}
\renewcommand{\dirname}{notes/ece1228-electromagnetic-theory/}
%\newcommand{\dateintitle}{}
%\newcommand{\keywords}{}

\input{../latex/peeter_prologue_print2.tex}

\usepackage{peeters_layout_exercise}
\usepackage{peeters_braket}
\usepackage{peeters_figures}
\usepackage{siunitx}
\usepackage{verbatim}
%\usepackage{mhchem} % \ce{}
%\usepackage{macros_bm} % \bcM
%\usepackage{macros_qed} % \qedmarker
%\usepackage{txfonts} % \ointclockwise

\beginArtNoToc

\generatetitle{XXX}
%\chapter{XXX}
%\label{chap:mandelbrot}
% \citep{sakurai2014modern} pr X.Y
% \citep{pozar2009microwave}
% \citep{qftLectureNotes}
% \citep{doran2003gap}
% \citep{jackson1975cew}
% \citep{griffiths1999introduction}

In ``Geometric Algebra for Computer Science'' is a fractal problem based on a vectorization of the Mandelbrot equation, which allows for generalization to \( N \) dimensions.

I finally got around to trying the 3D variation of this problem.  As setup, recall that the Mandlebrot set is a visualization of iteration of the following complex number equation:
\begin{equation}\label{eqn:mandelbrot:n}
z \rightarrow z^2 + c,
\end{equation}
where the idea is that \( z \) starts as the constant \( c \), and if this sequence converges to zero, then the point \( c \) is in the set.

The idea in the problem is that this equation can be cast as a vector equation, instead of a complex number equation.  All we have to do is set \( z = \Be_1 \Bx \), where \( \Be_1 \) is the x-axis unit vector, and \( \Bx \) is an \R{2} vector.  Expanding in coordinates, with \( \Bx = \Be_1 x + \Be_2 y \), we have
\begin{equation}\label{eqn:mandelbrot:n}
z 
= \Be_1 \lr{ \Be_1 x + \Be_2 y }
= x + \Be_1 \Be_2 y,
\end{equation}
but since the bivector \( \Be_1 \Be_2 \) squares to \( -1 \), we can represent complex numbers as even grade multivectors.  Making the same substitution in the Mandlebrot equation, we have
\begin{equation}\label{eqn:mandelbrot:n}
\Be_1 \Bx \rightarrow \Be_1 \Bx \Be_1 \Bx + \Be_1 \Bc,
\end{equation}
or
\begin{equation}\label{eqn:mandelbrot:n}
\Bx \rightarrow \Bx \Be_1 \Bx + \Bc.
\end{equation}
Viola!  This is a vector version of the Mandlebrot equation, and we can use it in 2 or 3 or N dimensions, as desired.

The problem is really one of visualization.  How can we visualize a 3D Mandelbrot set?  One idea is to use a ray tracing algorithm, so that only the points on the surface need be evaluated.  I don't think I've ever written a ray tracer, but I thought that there has got to be a quick and dirty way to do this.  This weekend, I wrote some code to do this with a brute force evaluation of all the points in the upper half plane in around the origin.  Here's the result

%}
\EndArticle
%\EndNoBibArticle
