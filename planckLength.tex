%
% Copyright � 2022 Peeter Joot.  All Rights Reserved.
% Licenced as described in the file LICENSE under the root directory of this GIT repository.
%
%{
\input{../latex/blogpost.tex}
\renewcommand{\basename}{planckLength}
%\renewcommand{\dirname}{notes/phy1520/}
\renewcommand{\dirname}{notes/ece1228-electromagnetic-theory/}
%\newcommand{\dateintitle}{}
%\newcommand{\keywords}{}

\input{../latex/peeter_prologue_print2.tex}

\usepackage{peeters_layout_exercise}
\usepackage{peeters_braket}
\usepackage{peeters_figures}
\usepackage{siunitx}
\usepackage{verbatim}
%\usepackage{mhchem} % \ce{}
%\usepackage{macros_bm} % \bcM
%\usepackage{macros_qed} % \qedmarker
%\usepackage{txfonts} % \ointclockwise

\beginArtNoToc

\generatetitle{Verifying dimensions of Planck length}
%\chapter{Verifying dimensions of Planck length}
%\label{chap:planckLength}

I'm reading \citep{rovelli2021general}, which has problems, despite being a sort of pop-sci book.  The first such problem is showing that the particular constant
\begin{equation*}
\sqrt{ \frac{h G}{c^3} }
\end{equation*}
has dimensions of length.

My first thought for this was that we have lots of ways of expressing energy in ways that bring in some, but not all of those constants.  Examples are
\begin{equation*}
   m c^2
   ,\quad
   h \nu
   ,\quad
   i \,\hbar \PD{t}{}
   ,\quad
   - \frac{\hbar^2}{2m} \PDSq{x}{}
   ,\quad
   - \frac{G m M}{r^2}.
\end{equation*}

Some of these are identical with respect to dimensions, for example:
\begin{equation*}
[h\nu] = [i \,\hbar \PD{t}{}] = [h]/T.
\end{equation*}
Let's use the fact that the dimensions of a particle's rest energy match that of the photon energy, to find a way to eliminate mass from the dimensions of the gravitation potential energy, that is
\begin{equation*}
   [ m c^2 ] = [m] \frac{L^2}{T^2} = [h]/T,
\end{equation*}
or
\begin{equation*}
M L^2/T^2 = [h]/T,
\end{equation*}
so
\begin{equation*}
   M
   = [h] \frac{T}{L^2}
   = [h/c] \inv{L}.
\end{equation*}

Now we can relate the photon energy dimensions with the dimensions of gravitational potential energy, to find

\begin{equation*}
\begin{aligned}
   \frac{[h]}{T}
   &=
   \frac{[G] M^2}{L} \\
   &=
   \frac{[G]}{L} 
   [h^2/c^2] \inv{L^2},
\end{aligned}
\end{equation*}
or
\begin{equation*}
[h G/c^3] = L^2.
\end{equation*}
so, we see that the root of this odd combination of units, does, as claimed, have dimensions of length.

%}
\EndArticle
