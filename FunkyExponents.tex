%
% Copyright � 2025 Peeter Joot.  All Rights Reserved.
% Licenced as described in the file LICENSE under the root directory of this GIT repository.
%
%{
\input{../latex/blogpost.tex}
\renewcommand{\basename}{FunkyExponents}
%\renewcommand{\dirname}{notes/phy1520/}
\renewcommand{\dirname}{notes/ece1228-electromagnetic-theory/}
%\newcommand{\dateintitle}{}
%\newcommand{\keywords}{}

\input{../latex/peeter_prologue_print2.tex}

\usepackage{peeters_layout_exercise}
\usepackage{peeters_braket}
\usepackage{peeters_figures}
\usepackage{siunitx}
\usepackage{verbatim}
%\usepackage{macros_cal} % \LL
%\usepackage{amsthm} % proof
%\usepackage{mhchem} % \ce{}
%\usepackage{macros_bm} % \bcM
%\usepackage{macros_qed} % \qedmarker
%\usepackage{txfonts} % \ointclockwise

\beginArtNoToc

\generatetitle{Some fun exponents}
%\chapter{Some fun exponents}
%\label{chap:FunkyExponents}
I saw a twitter post (but forgot to save the link) with a guy looking confused, captioned something like:
\begin{equation}\label{eqn:FunkyExponents:20}
\Abs{e^{i \pi}} = \Abs{\pi^{e i}} = \Abs{ i^{\pi e}} = 1.
\end{equation}

If this is true, then the arguments of each of the absolutes are complex numbers on the unit circle.  I suspect I'd seen that before, but forgot, so naturally, I had to verify for myself.

First, for \( \pi^{e i} \), we have
\begin{equation}\label{eqn:FunkyExponents:40}
\begin{aligned}
\pi^{e i}
&= \lr{ e^{\ln \pi} }^{e i} \\
&= \cos\lr{ e \ln \pi} + i \sin\lr{ e \ln \pi},
\end{aligned}
\end{equation}
and for \( i^{\pi e} \), we have
\begin{equation}\label{eqn:FunkyExponents:60}
\begin{aligned}
i^{\pi e}
&= \lr{ e^{i \pi/2}}^{\pi e} \\
&= \lr{ e^{i e \pi^2/2}}^{\pi e} \\
&= \cos \lr{ e \pi^2/2} + i \sin \lr{ e \pi^2/2}.
\end{aligned}
\end{equation}

Sure enough, this is true.  As it happens, two of these special values are nearly equal, because
\begin{equation}\label{eqn:FunkyExponents:80}
\begin{aligned}
e^{\pi} &= 23.1407 \\
\pi^e &= 22.4592,
\end{aligned}
\end{equation}
so \( e^{\pi i} \approx \pi^{e i} \).  We can see this visually if we plot the three points, as done in \cref{fig:FunkyExponents:FunkyExponentsFig1}.
\imageFigure{../figures/blogit/FunkyExponentsFig1}{The three points.}{fig:FunkyExponents:FunkyExponentsFig1}{0.3}

%}
%\EndArticle
\EndNoBibArticle
