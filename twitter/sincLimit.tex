%
% Copyright � 2025 Peeter Joot.  All Rights Reserved.
% Licenced as described in the file LICENSE under the root directory of this GIT repository.
%
%{
\input{../latex/blogpost.tex}
\renewcommand{\basename}{sincLimit}
%\renewcommand{\dirname}{notes/phy1520/}
\renewcommand{\dirname}{notes/ece1228-electromagnetic-theory/}
%\newcommand{\dateintitle}{}
%\newcommand{\keywords}{}

\input{../latex/peeter_prologue_print2.tex}

\usepackage{peeters_layout_exercise}
\usepackage{peeters_braket}
\usepackage{peeters_figures}
\usepackage{siunitx}
\usepackage{verbatim}
%\usepackage{macros_cal} % \LL
%\usepackage{amsthm} % proof
%\usepackage{mhchem} % \ce{}
%\usepackage{macros_bm} % \bcM
%\usepackage{macros_qed} % \qedmarker
%\usepackage{txfonts} % \ointclockwise

\beginArtNoToc

\generatetitle{XXX}
%\chapter{XXX}
%\label{chap:sincLimit}

Find
\begin{equation}\label{eqn:sincLimit:20}
\lim_{x \rightarrow 0} \frac{\sin x}{x}.
\end{equation}
We need only use the Taylor series for sine
\begin{equation}\label{eqn:sincLimit:40}
\sin x = x - \frac{x^3}{3!} + \cdots,
\end{equation}
to find
\begin{equation}\label{eqn:sincLimit:60}
\frac{\sin x}{x} = 1 - \frac{x^2}{3!} + \cdots,
\end{equation}
so
\begin{equation}\label{eqn:sincLimit:80}
\boxed{
\lim_{x \rightarrow 0} \frac{\sin x}{x} = 1.
}
\end{equation}

%}
%\EndArticle
\EndNoBibArticle
