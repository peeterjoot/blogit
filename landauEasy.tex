%
% Copyright � 2025 Peeter Joot.  All Rights Reserved.
% Licenced as described in the file LICENSE under the root directory of this GIT repository.
%
%{
\input{../latex/blogpost.tex}
\renewcommand{\basename}{landauEasy}
%\renewcommand{\dirname}{notes/phy1520/}
\renewcommand{\dirname}{notes/ece1228-electromagnetic-theory/}
%\newcommand{\dateintitle}{}
%\newcommand{\keywords}{}

\input{../latex/peeter_prologue_print2.tex}

\usepackage{peeters_layout_exercise}
\usepackage{peeters_braket}
\usepackage{peeters_figures}
\usepackage{siunitx}
\usepackage{verbatim}
\usepackage{macros_cal} % \LL
%\usepackage{amsthm} % proof
%\usepackage{mhchem} % \ce{}
%\usepackage{macros_bm} % \bcM
%\usepackage{macros_qed} % \qedmarker
%\usepackage{txfonts} % \ointclockwise

\beginArtNoToc

\generatetitle{XXX}
%\chapter{XXX}
%\label{chap:landauEasy}

\makeoproblem{Lorentz power and force relationship.}{problem:freespace:LorentzPower}{\S 17 \citep{landau1980classical}}{
Using the relativistic definitions of momentum and energy
\begin{equation*}
\begin{aligned}
	\Bp(\Bx, t) &= \frac{m \Bv}{\sqrt{1-\Bv^2/c^2}} \\
	\calE(\Bx, t) &= \frac{m c^2}{\sqrt{1-\Bv^2/c^2}},
\end{aligned}
\end{equation*}
show that \( d\calE/dt = \Bv \cdot d\Bp/dt \), and use this to derive
\cref{eqn:freespace:220} from \cref{eqn:freespace:200}.
} % problem
\makeanswer{problem:freespace:LorentzPower}{
We use the usual notation
\begin{equation}\label{eqn:landauEasy:20}
\gamma = \lr{1 - \Bv^2/c^2}^{-1/2},
\end{equation}
so the derivative of the (relativistic) energy of a charged particle is
\begin{equation}\label{eqn:landauEasy:40}
\ddt{\calE} = m c^2 \ddt{\gamma}
\end{equation}
The derivative of the (relativistic) momentum of that charged particle is
\begin{equation}\label{eqn:landauEasy:60}
\ddt{\Bp} = m \gamma \ddt{\Bv} + m \Bv \ddt{\gamma},
\end{equation}
so
\begin{equation}\label{eqn:landauEasy:80}
\Bv \cdot \ddt{\Bp} = m \gamma \Bv \cdot \ddt{\Bv} + m \Bv^2 \ddt{\gamma}.
\end{equation}
We may related the first term to the \(\gamma\) derivative, since
\begin{equation}\label{eqn:landauEasy:100}
\ddt{\gamma} = \lr{ -\inv{2} } \frac{\gamma}{1 - \Bv^2/c^2} \lr{-\frac{2}{c^2} } \Bv \cdot \ddt{\Bv},
\end{equation}
or
\begin{equation}\label{eqn:landauEasy:120}
\lr{ c^2 - \Bv^2 } \ddt{\gamma} = \gamma \Bv \cdot \ddt{\Bv},
\end{equation}
so
\begin{equation}\label{eqn:landauEasy:140}
\begin{aligned}
\Bv \cdot \ddt{\Bp}
&= m \lr{ \lr{ c^2 - \Bv^2 } \ddt{\gamma} } + m \Bv^2 \ddt{\gamma} \\
&= m c^2 \ddt{\gamma} \\
&= \ddt{\calE}.
\end{aligned}
\end{equation}
From the Lorentz force, we have
\begin{equation}\label{eqn:landauEasy:160}
v \cdot \ddt{\Bp} = q \Bv \cdot \BE + q \Bv \cdot \lr{ \Bv \cross \BB } = q \Bv \cdot \BE,
\end{equation}
as expected.
} % answer

%}
%\EndArticle
\EndNoBibArticle
