%
% Copyright � 2021 Peeter Joot.  All Rights Reserved.
% Licenced as described in the file LICENSE under the root directory of this GIT repository.
%
%{
\input{../latex/blogpost.tex}
\renewcommand{\basename}{mvi}
%\renewcommand{\dirname}{notes/phy1520/}
\renewcommand{\dirname}{notes/ece1228-electromagnetic-theory/}
%\newcommand{\dateintitle}{}
%\newcommand{\keywords}{}

\input{../latex/peeter_prologue_print2.tex}

\usepackage{peeters_layout_exercise}
\usepackage{peeters_braket}
\usepackage{peeters_figures}
\usepackage{siunitx}
\usepackage{verbatim}
%\usepackage{mhchem} % \ce{}
%\usepackage{macros_bm} % \bcM
%\usepackage{macros_qed} % \qedmarker
%\usepackage{txfonts} % \ointclockwise

\beginArtNoToc

\generatetitle{XXX}

Yes, $1$ is meant to be the unit scalar here.  For example, assume that you are considering the multivectors generated by a 2D Euclidean vector space with an orthonormal basis \( \setlr{ \Be_1, \Be_2 } \):
\begin{equation*}
   \Be_1 \Be_1 = 1,
\end{equation*}
\begin{equation*}
   \lr{ \Be_1 \Be_2 } \lr{ -\Be_1 \Be_2 } = 1,
\end{equation*}
\begin{equation*}
   \lr{ \Be_1 + \Be_2 } \frac{ \Be_1 + \Be_2 }{2} = 1,
\end{equation*}
\begin{equation*}
   \lr{ 1 + \Be_1 \Be_2 } \frac{ 1 - \Be_1 \Be_2 }{2} = 1.
\end{equation*}
All these inverses following directly from \( \Be_1^2 = \Be_2^2 = 1 \), and \( \Be_1 \Be_2 = -\Be_2 \Be_1 \).

%}
\EndArticle
%\EndNoBibArticle
