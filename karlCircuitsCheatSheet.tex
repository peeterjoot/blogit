%{ % for vim brace matching if unbalanced.
%\PassOptionsToPackage{square,numbers}{natbib}
\documentclass[letterpaper]{scrartcl}
\usepackage[margin=1in,top=0.5in]{geometry}

%\usepackage[svgnames]{xcolor}
\usepackage{amsmath}
\usepackage{caption}
%\usepackage{exercise}
\usepackage{graphicx}
%\usepackage{ifthen}
%\usepackage{parameters}
\usepackage{siunitx}
%\usepackage{stackengine}
%\usepackage{subfig}
%\usepackage{verbatim}
\usepackage{wrapfig}
\usepackage[english]{cleveref}

% for a centered figure with caption (not used here)
% \simpleFigure{path}{caption}{label}{width}
\newcommand{\simpleFigure}[4]{%
  \begin{figure}[htp]%
    \centering%
    \includegraphics[width=#4\textwidth]{#1}%
    \caption{#2}%
    \label{#3}%
  \end{figure}%
}

% \sideFigure{path}{caption}{label}{width}[side]
\newcommand{\sideFigure}[5][l]{%
  \begin{wrapfigure}{#1}{#5\textwidth}%
    \centering%
    \includegraphics[width=#5\textwidth]{#2}%
    \caption{#3}%
    \label{#4}%
  \end{wrapfigure}%
}

\newcommand{\lr}[1]{\left(#1\right)}
\newcommand{\inv}[1]{\frac{1}{#1}}

\DeclareMathOperator{\Real}{Re}
\DeclareMathOperator{\Arg}{Arg}
\DeclareMathOperator{\Imag}{Im}

\newcommand{\BV}[0]{\mathbf{V}}
\newcommand{\thev}[0]{{\mathrm{Th}}}
%\newcommand{\degree}[0]{\circ}

\begin{document}
\section*{Circuits Cheatsheet}
\subsection*{Current Division}
\begin{minipage}{\textwidth}
For a DC current source, such as the one illustrated in \cref{fig:currentSource:currentSourceFig1}, we have
\end{minipage}
\sideFigure{../figures/blogit/currentSourceFig1}{Current Source}{fig:currentSource:currentSourceFig1}{0.3}
\vspace{1.5\baselineskip}
\begin{equation}\label{eqn:karlCircuitsCheatSheet:20}
I_1 = I_s \frac{R_2}{R_1 + R_2}
\end{equation}
\begin{equation}\label{eqn:karlCircuitsCheatSheet:40}
I_2 = I_s \frac{R_1}{R_1 + R_2}
\end{equation}
\WFclear
\vspace{3\baselineskip}
\subsection*{Voltage Division}
\begin{minipage}{\textwidth}
For a DC voltage source illustrated in \cref{fig:voltageSource:voltageSourceFig2}, we have
\end{minipage}
\sideFigure{../figures/blogit/voltageSourceFig2}{Voltage Source}{fig:voltageSource:voltageSourceFig2}{0.3}
\vspace{2.5\baselineskip}
\begin{equation}\label{eqn:karlCircuitsCheatSheet:60}
V = V_2 \frac{R_2}{R_1 + R_2}
\end{equation}
\WFclear
\vspace{4\baselineskip}
\subsection*{Cramer's Rule}
The solution of
\(\begin{bmatrix}
a & b \\
c & d
\end{bmatrix}
\begin{bmatrix}
x \\
y
\end{bmatrix}
=
\begin{bmatrix}
e \\
f
\end{bmatrix}\)
is given by
\begin{equation}\label{eqn:karlCircuitsCheatSheet:100}
x = \frac{
   \begin{vmatrix}
   e & b \\
   f & d
   \end{vmatrix}
}{
   \begin{vmatrix}
   a & b \\
   c & d
   \end{vmatrix}
},
\qquad
y = \frac{
   \begin{vmatrix}
   a & e \\
   c & f
   \end{vmatrix}
}{
   \begin{vmatrix}
   a & b \\
   c & d
   \end{vmatrix}
}.
\end{equation}
\subsection*{RLC Circuit equations}
\begin{equation}\label{eqn:karlCircuitsCheatSheet:120}
v_L = L \frac{di_L}{dt}
\end{equation}
\begin{equation}\label{eqn:karlCircuitsCheatSheet:140}
i_C = C \frac{dv_C}{dt}.
\end{equation}
\subsection*{Complex numbers}
The rectangular form of a complex number (or phasor) is
\begin{equation}\label{eqn:karlCircuitsCheatSheet:160}
z = a + j b,
\end{equation}
where \( a \) is the real part, and \( j b \) is the imaginary part.

The polar form of a complex number (i.e.: of a phasor) is
\begin{equation}\label{eqn:karlCircuitsCheatSheet:180}
z = r \angle \phi,
\end{equation}
where \( r = \sqrt{a^2 + b^2} \), and \( \phi = \tan^{-1}\lr{\frac{b}{a}} \).

The exponential form of a complex number is
\begin{equation}\label{eqn:karlCircuitsCheatSheet:200}
z = r e^{j\phi} = r \lr{ \cos\phi + j \sin\phi }.
\end{equation}
This can be related to the rectangular form by noting that \( a = r \cos\phi \) and \( b = r \sin \phi \).

Observe that
\begin{equation}\label{eqn:karlCircuitsCheatSheet:220}
\begin{aligned}
\sqrt{z} &= \sqrt{r} e^{j\phi/2} = \sqrt{r} \angle \frac{\phi}{2} \\
\inv{z} &= \inv{r} e^{-j\phi} = \inv{r} \angle -\phi \\
a - j b &= r e^{-j\phi} = r \angle -\phi.
\end{aligned}
\end{equation}

Given two complex numbers \( z_1 = r_1 e^{j\phi_1}, z_2 = r_2 e^{j\phi_2} \), then their products and ratios are
\begin{equation}\label{eqn:karlCircuitsCheatSheet:240}
\begin{aligned}
z_1 z_2 &= r_1 r_2 e^{j \lr{\phi_1 + \phi_2 } } = r_1 r_2 \angle \lr{ \phi_1 + \phi_2 } \\
\frac{z_1}{z_2} &= \frac{r_1}{r_2} e^{j \lr{\phi_1 - \phi_2 } } = \frac{r_1}{r_2} \angle \lr{ \phi_1 - \phi_2 } \\
\end{aligned}
\end{equation}

Let's applying the phasor notation to voltages and currents.  For example, given a time dependent cosinusoidal voltage with constant magnitude \( V \), we have the following equivalences
\begin{equation}\label{eqn:karlCircuitsCheatSheet:260}
\begin{aligned}
v(t) &= V \cos\lr{ \omega t + \phi } = \Real \lr{ V e^{j \lr{\omega t + \phi} } } \\
\BV &= V e^{j \phi} = V \angle \phi = V e^{j \phi} \\
v(t) &= \Real \lr{ \BV e^{j \omega t} },
\end{aligned}
\end{equation}
where the phasor has been written \( \BV \).  Various notations are used for phasors, including overlines and underlines, or even over-arrows like vectors.

Similarly, given a sinusoidal time dependent voltage
\begin{equation}\label{eqn:karlCircuitsCheatSheet:320}
\begin{aligned}
v(t) &= V \sin\lr{ \omega t + \phi } = \Real \lr{ -j V e^{j\lr{ \omega t + \phi } } } \\
\BV &= V \angle \lr{ \phi - \SI{90}{\degree}} = V e^{j \lr{ \phi - \SI{90}{\degree}} },
\end{aligned}
\end{equation}
since \( -j = \angle \SI{-90}{\degree} \).  Once again \( v(t) = \Real \lr{ \BV e^{j \omega t} } \).

\subsection*{Impedance}
If
\begin{equation}\label{eqn:karlCircuitsCheatSheet:340}
Z_\thev = R_\thev + j X_\thev,
\end{equation}
then
\begin{equation}\label{eqn:karlCircuitsCheatSheet:360}
Z_{\mathrm{L}} = R_\thev - j X_\thev.
\end{equation}
KARL: what is this?

\subsection*{Karl's Reminder To Self}

REMEMBER SUPERMESH AND SUPERNODES.

\end{document}
%} % for vim brace matching if unbalanced.
