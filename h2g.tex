%
% Copyright � 2025 Peeter Joot.  All Rights Reserved.
% Licenced as described in the file LICENSE under the root directory of this GIT repository.
%
%{
\input{../latex/blogpost.tex}
\renewcommand{\basename}{h2g}
%\renewcommand{\dirname}{notes/phy1520/}
\renewcommand{\dirname}{notes/ece1228-electromagnetic-theory/}
%\newcommand{\dateintitle}{}
%\newcommand{\keywords}{}

\input{../latex/peeter_prologue_print2.tex}

\usepackage{peeters_layout_exercise}
\usepackage{peeters_braket}
\usepackage{peeters_figures}
\usepackage{siunitx}
\usepackage{verbatim}
%\usepackage{macros_cal} % \LL
%\usepackage{amsthm} % proof
%\usepackage{mhchem} % \ce{}
%\usepackage{macros_bm} % \bcM
%\usepackage{macros_qed} % \qedmarker
%\usepackage{txfonts} % \ointclockwise

\beginArtNoToc

\generatetitle{XXX}
%\chapter{XXX}
%\label{chap:h2g}

The goal is to integrate
\begin{equation}\label{eqn:h2g:20}
I_\epsilon(r)
= -\inv{2 \pi} \int_0^{2\pi} d\theta \int_0^\infty dp \frac{p e^{j p r \cos\theta}}{p^2 - \lr{k + j\epsilon}^2}
\end{equation}

ChatGPT suggests evaluating the \( p \) integral first using contour integration.  That is
\begin{equation}\label{eqn:h2g:40}
\begin{aligned}
I(\theta)
&= -\inv{2 \pi} \int_0^\infty dp \frac{p e^{j p r \cos\theta}}{p^2 - \lr{k + j\epsilon}^2} \\
&=
-\inv{4 \pi} \int_0^\infty dp
\lr{ \inv{p - \lr{k + j\epsilon}} + \inv{p + \lr{k + j\epsilon}} } e^{j p r \cos\theta}.
\end{aligned}
\end{equation}
Changing variables \( p \rightarrow -p \) in the second integral yields
\begin{equation}\label{eqn:h2g:60}
\begin{aligned}
-\inv{4 \pi} \int_0^\infty dp
\lr{ \inv{p + \lr{k + j\epsilon}} } e^{j p r \cos\theta}
&=
+\inv{4 \pi} \int_0^{-\infty} dp
\inv{-p + \lr{k + j\epsilon}}
e^{-j p r \cos\theta} \\
&=
\inv{4 \pi} \int_{-\infty}^0 dp
\inv{p - \lr{k + j\epsilon}}
e^{-j p r \cos\theta},
\end{aligned}
\end{equation}
so
\begin{equation}\label{eqn:h2g:80}
\begin{aligned}
I(\theta)
&=
-\inv{4 \pi} \int_0^\infty dp
\frac{e^{j p r \cos\theta}}{p - \lr{k + j\epsilon}}
+
\inv{4 \pi} \int_{-\infty}^0 dp
\frac{e^{-j p r \cos\theta}}{p - \lr{k + j\epsilon}} \\
&=
-\inv{4 \pi} \int_{-\infty}^\infty dp
\frac{\sgn(p) e^{j \Abs{p} r \cos\theta}}{p - \lr{k + j\epsilon}}.
\end{aligned}
\end{equation}

%}
%\EndArticle
\EndNoBibArticle
