%
% Copyright � 2024 Peeter Joot.  All Rights Reserved.
% Licenced as described in the file LICENSE under the root directory of this GIT repository.
%
%{
\input{../latex/blogpost.tex}
\renewcommand{\basename}{averagespeedproblem}
%\renewcommand{\dirname}{notes/phy1520/}
\renewcommand{\dirname}{notes/ece1228-electromagnetic-theory/}
%\newcommand{\dateintitle}{}
%\newcommand{\keywords}{}

\input{../latex/peeter_prologue_print2.tex}

\usepackage{peeters_layout_exercise}
\usepackage{peeters_braket}
\usepackage{peeters_figures}
\usepackage{siunitx}
\usepackage{verbatim}
%\usepackage{mhchem} % \ce{}
%\usepackage{macros_bm} % \bcM
%\usepackage{macros_qed} % \qedmarker
%\usepackage{txfonts} % \ointclockwise

\beginArtNoToc

\generatetitle{Sept 17, Karl's average speed problem.}
%\chapter{Sept 17, Karl's average speed problem.}
%\label{chap:averagespeedproblem}

Problem:

A car is travelling at 89.5 \si{km/h}, with a 22 min stop.
The average velocity of the vehicle at the end is 77.8 \si{km/h}. How long was the car travelling?

Solution:

Let \( \Delta t_1 \) (in hours) be the total time that the car travelled at \( v_1 = 89.5 \si{km/h} \), and \( \Delta t_2 = 22/60\, \textrm{hours} \) be the total time that the car travelled at \( v_2  = 0 \si{km/h} \).

In terms of \( \Delta t_1, \Delta t_2 \), and \( v_1, v_2 \), what is the average speed that the car travelled?

That is:

\begin{equation}\label{eqn:averagespeedproblem:20}
\begin{aligned}
77.8
&=
\overbar{v} \\
&= \frac{\Delta d}{\Delta t} \\
&= \frac{(\Delta t_1) v_1 + (\Delta t_2) v_2}{\Delta t_1 + \Delta t_2} \\
&= \frac{(\Delta t_1) v_1 + (\Delta t_2) v_2}{\Delta t_1 + \Delta t_2} \\
&= \frac{(\Delta t_1) 89.5 + (22/60) (0) }{\Delta t_1 + (22/60)} \\
\end{aligned}
\end{equation}

Now solve for \(\Delta t_1\).

%}
\EndNoBibArticle
