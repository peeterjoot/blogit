%
% Copyright � 2024 Peeter Joot.  All Rights Reserved.
% Licenced as described in the file LICENSE under the root directory of this GIT repository.
%
%{
\input{../latex/blogpost.tex}
\renewcommand{\basename}{more_feynmans_trick}
%\renewcommand{\dirname}{notes/phy1520/}
\renewcommand{\dirname}{notes/ece1228-electromagnetic-theory/}
%\newcommand{\dateintitle}{}
%\newcommand{\keywords}{}

\input{../latex/peeter_prologue_print2.tex}

\usepackage{peeters_layout_exercise}
\usepackage{peeters_braket}
\usepackage{peeters_figures}
\usepackage{siunitx}
\usepackage{verbatim}
\usepackage{amsthm} % proof
%\usepackage{mhchem} % \ce{}
%\usepackage{macros_bm} % \bcM
%\usepackage{macros_qed} % \qedmarker
%\usepackage{txfonts} % \ointclockwise

\beginArtNoToc

\generatetitle{More derivatives of integrals.}
%\chapter{More integral derivatives}
%\label{chap:more_feynmans_trick}
\section{Motivation.}
I was asked about geometric algebra equivalents for a couple identities found in \citep{kemmer1977vector}, one for line integrals
\begin{equation}\label{eqn:more_feynmans_trick:20}
\ddt{} \int_{C(t)} \Bf \cdot d\Bx =
\int_{C(t)} \lr{
   \PD{t}{\Bf} + \spacegrad \lr{ \Bv \cdot \Bf } - \Bv \cross \lr{ \spacegrad \cross \Bf }
}
\cdot d\Bx,
\end{equation}
and one for area integrals
\begin{equation}\label{eqn:more_feynmans_trick:40}
\ddt{} \int_{S(t)} \Bf \cdot d\BA =
\int_{S(t)} \lr{
   \PD{t}{\Bf} + \Bv \lr{ \spacegrad \cdot \Bf } - \spacegrad \cross \lr{ \Bv \cross \Bf }
}
\cdot d\BA.
\end{equation}

Both of these look questionable at first glance, because neither has boundary term.  However, they can be transformed with Stokes theorem to
\begin{equation}\label{eqn:more_feynmans_trick:60}
\ddt{} \int_{C(t)} \Bf \cdot d\Bx
=
\int_{C(t)} \lr{
   \PD{t}{\Bf} - \Bv \cross \lr{ \spacegrad \cross \Bf }
}
\cdot d\Bx
+
\evalbar{\Bv \cdot \Bf }{\Delta C},
\end{equation}
and
\begin{equation}\label{eqn:more_feynmans_trick:80}
\ddt{} \int_{S(t)} \Bf \cdot d\BA =
\int_{S(t)} \lr{
   \PD{t}{\Bf} + \Bv \lr{ \spacegrad \cdot \Bf }
}
\cdot d\BA
-
\oint_{\partial S(t)} \lr{ \Bv \cross \Bf } \cdot d\Bx.
\end{equation}
The area integral derivative is now seen to be a variation of one of the special cases of the Leibniz integral rule, see for example \citep{enwiki:1223666713}.  The author admits that the line integral relationship is not well used, and doesn't show up in the wikipedia page.

My end goal will be to evaluate the derivative of a general multivector line integral
\begin{equation}\label{eqn:more_feynmans_trick:100}
\ddt{} \int_{C(t)} F d\Bx G,
\end{equation}
and area integral
\begin{equation}\label{eqn:more_feynmans_trick:120}
\ddt{} \int_{S(t)} F d^2\Bx G.
\end{equation}
We've derived that line integral result in a different fashion previously, but it's interesting to see a different approach.  Perhaps this approach will lend itself nicely to non-scalar integrands?
\section{Time derivative of a scalar line integral.}
It is worthwhile to work through the analysis from the text in question for the (scalar) line integral case in detail.  I'll do so here, but will handle some of the details a bit differently.

\makedefinition{Convective derivative.}{dfn:more_feynmans_trick:1}{
The convective derivative,
of \( \phi(t, \Bx(t)) \) is defined as
\begin{equation*}
\frac{D \phi}{D t} = \lim_{\Delta t \rightarrow 0} \frac{ \phi(t + \delta t, \Bx + \delta t \Bv) - \phi(t, \Bx)}{\Delta t},
\end{equation*}
where \( \Bv = d\Bx/dt \).
} % definition
\maketheorem{Convective derivative.}{thm:more_feynmans_trick:1}{
The convective derivative operator may be written
\begin{equation*}
\frac{D}{D t} = \PD{t}{} + \Bv \cdot \spacegrad.
\end{equation*}
} % theorem
\begin{proof}
Let's write
\begin{equation}\label{eqn:more_feynmans_trick:140}
\begin{aligned}
v_0 &= 1 \\
u_0 &= t + v_0 h \\
u_k &= x_k + v_k h, k \in [1,3] \\
\end{aligned}
\end{equation}
%u_\alpha(0) &= \evalbar{u_\alpha}{h = 0}, \alpha \in [0,3]
The limit, if it exists, must equal the sum of the individual limits
\begin{equation}\label{eqn:more_feynmans_trick:160}
\frac{D \phi}{D t} = \sum_{\alpha = 0}^3 \lim_{\Delta t \rightarrow 0} \frac{ \phi(u_\alpha + v_\alpha h) - \phi(t, Bx)}{h},
\end{equation}
but that is just a sum of derivitives, which can be evaluated by chain rule
\begin{equation}\label{eqn:more_feynmans_trick:180}
\begin{aligned}
\frac{D \phi}{D t}
&= \sum_{\alpha = 0}^{3} \evalbar{ \PD{u_\alpha}{\phi(u_\alpha)} \PD{h}{u_\alpha} }{h = 0} \\
&= \PD{t}{\phi} + \sum_{k = 1}^3 v_k \PD{x_k}{\phi} \\
&= \lr{ \PD{t}{} + \Bv \cdot \spacegrad } \phi.
\end{aligned}
\end{equation}
\end{proof}
\makedefinition{Hestenes overdot notation.}{dfn:more_feynmans_trick:2}{
We may use a dot or a tick with a derivative operator, to designate the scope of that operator, allowing it to operate bidirectionally, or in a restricted fashion, holding specific multivector elements constant.  This is called the Hestenes overdot notation.

Illustrating by example, with multivectors \( F, G \), and allowing the gradient to act bidirectionally, we have
\begin{equation*}
\begin{aligned}
F \spacegrad G
&=
\dot{F} \dot{\spacegrad} G
+
F \dot{\spacegrad} \dot{G} \\
&=
\sum_i \lr{ \partial_i F } \Be_i G + \sum_i F \Be_i \lr{ \partial_i G }.
\end{aligned}
\end{equation*}
The last step is a precise statement of the meaning of the overdot notation, showing that we hold the position of the vector elements of the gradient constant, while the (scalar) partials are allowed to commute, acting on the designated elements.
} % definition
We will need one additional identity
\makelemma{Gradient of dot product (one constant vector.)}{lemma:more_feynmans_trick:1}{
Given vectors \( \Ba, \Bb \) the gradient of their dot product is given by
\begin{equation*}
\spacegrad \lr{ \Ba \cdot \Bb }
= \lr{ \Bb \cdot \spacegrad } \Ba + \Bb \cdot \lr{ \spacegrad \wedge \Ba }
+ \lr{ \Ba \cdot \spacegrad } \Bb + \Ba \cdot \lr{ \spacegrad \wedge \Bb }.
\end{equation*}
If \( \Bb \) is constant, this reduces to
\begin{equation*}
\spacegrad \lr{ \Ba \cdot \Bb }
=
\dot{\spacegrad} \lr{ \dot{\Ba} \cdot \Bb }
= \lr{ \Bb \cdot \spacegrad } \Ba + \Bb \cdot \lr{ \spacegrad \wedge \Ba }.
\end{equation*}
} % lemma
\begin{proof}
The \( \Bb \) constant case is trivial to prove.  We use \( \Ba \cdot \lr{ \Bb \wedge \Bc } = \lr{ \Ba \cdot \Bb} \Bc - \Bb \lr{ \Ba \cdot \Bc } \), and simply expand the vector, curl dot product
\begin{equation}\label{eqn:more_feynmans_trick:200}
\Bb \cdot \lr{ \spacegrad \wedge \Ba }
=
\Bb \cdot \lr{ \dot{\spacegrad} \wedge \dot{\Ba} }
=
\lr{ \Bb \cdot \dot{\spacegrad} } \dot{\Ba} - \dot{\spacegrad} \lr{ \dot{\Ba} \cdot \Bb }.
\end{equation}
Rearrangement proves that \( \Bb \) constant identity.  The more general statement is just a superposition of the \( \Ba \) constant case and the \( \Bb \) constant case, and follows trivially.
\end{proof}

%}
\EndArticle
