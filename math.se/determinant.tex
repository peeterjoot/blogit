%
% Copyright � 2020 Peeter Joot.  All Rights Reserved.
% Licenced as described in the file LICENSE under the root directory of this GIT repository.
%
%{
\input{../latex/blogpost.tex}
\renewcommand{\basename}{determinant}
%\renewcommand{\dirname}{notes/phy1520/}
\renewcommand{\dirname}{notes/ece1228-electromagnetic-theory/}
%\newcommand{\dateintitle}{}
%\newcommand{\keywords}{}

\input{../latex/peeter_prologue_print2.tex}

\usepackage{peeters_layout_exercise}
\usepackage{peeters_braket}
\usepackage{peeters_figures}
\usepackage{siunitx}
\usepackage{verbatim}
%\usepackage{mhchem} % \ce{}
%\usepackage{macros_bm} % \bcM
%\usepackage{macros_qed} % \qedmarker
%\usepackage{txfonts} % \ointclockwise

\beginArtNoToc

\generatetitle{XXX}
%\chapter{XXX}
%\label{chap:determinant}

Just because there is a matrix representation of a vector, does not mean that this representation is unique.  So, why would the determinant of any given matrix representation of a vector (or multivector) be meaningful?

A better question to ask might be what is the logical equivalent of a determinant geometrically?  Then we can look for a Clifford representation of that operation or entity.

One way that the determinant can be motivated (\citep{damiano1988cla}) is by computing the area of a parallelogram, or more generally a hyper-parallopiped.  For example, if the vectors \( \Ba = (a_1, a_2) \) and \( \Bb = (b_1, b_2) \) for the edges of a parallogram in \R{2}, then the oriented area of that parallelogram is given by
\begin{equation*}
\begin{vmatrix}
   a_1 & a_2 \\
   b_1 & b_2 \\
\end{vmatrix}.
\end{equation*}

Observe that this is the numerical coefficient of the wedge product of those two vectors.  Assuming an orthonormal basis \(\setlr{\Be_1, \Be_2} \), such a wedge is
\begin{equation*}
\begin{aligned}
   \Ba \wedge \Bb 
   &=
   \lr{ a_1 \Be_1 + a_2 \Be_2 }
   \wedge
   \lr{ b_1 \Be_1 + b_2 \Be_2 } 
   \\
   &=
   a_1 b_2 \Be_1 \wedge \Be_2 + a_2 b_1 \Be_2 \wedge \Be_1 \\
   &=
   (a_1 b_2 - a_2 b_1) \Be_1 \wedge \Be_2.
\end{aligned}
\end{equation*}
It's not too hard to show that the wedge of N vectors in an N dimensional space equals the determinant of the Euclidean coefffients of those vectors, scaled by the unit volume element for that space.  That determinant is an oriented magnitude for that hypervolume.

If you are looking for how the determinant is observed in a Clifford algebraic context, you'll find it in such wedge products.  That is representation independent, and has no connection with any possible matrix representation of the underlying vectors.

%}
\EndArticle
%\EndNoBibArticle
