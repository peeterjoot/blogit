%
% Copyright � 2025 Peeter Joot.  All Rights Reserved.
% Licenced as described in the file LICENSE under the root directory of this GIT repository.
%
%{
\input{../latex/blogpost.tex}
\renewcommand{\basename}{selfEnergyIntegral}
%\renewcommand{\dirname}{notes/phy1520/}
\renewcommand{\dirname}{notes/ece1228-electromagnetic-theory/}
%\newcommand{\dateintitle}{}
%\newcommand{\keywords}{}

\input{../latex/peeter_prologue_print2.tex}

\usepackage{peeters_layout_exercise}
\usepackage{peeters_braket}
\usepackage{peeters_figures}
\usepackage{siunitx}
\usepackage{verbatim}
%\usepackage{macros_cal} % \LL
%\usepackage{amsthm} % proof
%\usepackage{mhchem} % \ce{}
%\usepackage{macros_bm} % \bcM
%\usepackage{macros_qed} % \qedmarker
%\usepackage{txfonts} % \ointclockwise

\beginArtNoToc

\generatetitle{An ``easy'' integral from Jackson's electrodynamics}
%\chapter{An ``easy'' integral from Jackson's electrodynamics}
%\label{chap:selfEnergyIntegral}
Once again, I was reading my Jackson \citep{jackson1975cew}, which characteristically had the statement ``the [...] integral \textit{can easily be shown} to have the value \( 4 \pi \)'', in a discussion of electrostatic energy and self energy.

The integral is
\begin{equation}\label{eqn:selfEnergyIntegral:20}
I = \int \frac{\Brho}{\rho^3} \cdot \frac{\Brho + \Bn}{\Norm{\Brho + \Bn}^3} d^3 \rho.
\end{equation}

This is something that I once figured out once before (see \citep{pjootECE1228} appendix C).  However, trying to do it a second time around, I think that I found the ``easy'' way.

As Jackson hints, the starting point is
\begin{equation}\label{eqn:selfEnergyIntegral:40}
\frac{\Bx}{\Norm{\Bx}^3}
=
-\spacegrad \inv{\Norm{\Bx}},
\end{equation}
but we don't have to apply it to both the vector terms, as I did in my initial attempt (which results in a Laplacian to reduce.)  Inserting this and applying chain rule, we find
\begin{equation}\label{eqn:selfEnergyIntegral:60}
\begin{aligned}
I
&= -\int \frac{\Brho}{\rho^3} \cdot \spacegrad_\Brho \inv{\Norm{\Brho + \Bn}} d^3 \rho \\
&=
-\int
\spacegrad_\Brho \cdot \lr{
\frac{\Brho}{\rho^3} \cdot
\inv{\Norm{\Brho + \Bn}}
}
d^3 \rho
+
\int
\lr{
\spacegrad_\Brho \cdot
\frac{\Brho}{\rho^3}
}
\inv{\Norm{\Brho + \Bn}}
d^3 \rho
\\
\end{aligned}
\end{equation}

The first integral can be evaluated using an infinite spherical shell
\begin{equation}\label{eqn:selfEnergyIntegral:80}
\begin{aligned}
I_1
&= -\int
\spacegrad_\Brho \cdot \lr{
\frac{\Brho}{\rho^3} \cdot
\inv{\Norm{\Brho + \Bn}}
}
d^3 \rho \\
&=
\lim_{\rho \rightarrow \infty}
-\frac{\Brho}{\rho} \cdot \lr{
\frac{\Brho}{\rho^3} \cdot
\inv{\Norm{\Brho + \Bn}}
} 4 \pi \rho^2 \\
&=
\lim_{\rho \rightarrow \infty}
\frac{-4 \pi}{\Norm{\Brho + \Bn}} \\
&=
0.
\end{aligned}
\end{equation}

The divergence term in the second integral, provided \( \Bx \ne 0 \), has the form
\begin{equation}\label{eqn:selfEnergyIntegral:100}
\begin{aligned}
\spacegrad \cdot \frac{\Bx}{\Norm{\Bx}^3}
&=
\inv{\Norm{\Bx}^3} \spacegrad \cdot \Bx
+
\lr{ \Bx \cdot \spacegrad } \inv{\Norm{\Bx}^3} \\
&=
\frac{3}{\Norm{\Bx}^3}
+
2 \frac{x_k x_j}{\Norm{\Bx}^5} \lr{-\frac{3}{2}} \partial_k x_j \\
&=
\frac{3}{\Norm{\Bx}^3}
- \frac{3}{\Norm{\Bx}^3}
\end{aligned}
\end{equation}
However, in a neighbourhood of the origin, this actually has a delta function structure.  We can see that from Gauss's law, where we have
\begin{equation}\label{eqn:selfEnergyIntegral:120}
\spacegrad \cdot \BE = \frac{\rho}{\epsilon_0}.
\end{equation}
If we plug in the integral representation of \( \BE \) on the LHS, we have
\begin{equation}\label{eqn:selfEnergyIntegral:140}
\begin{aligned}
\spacegrad \cdot \BE
&=
\spacegrad \cdot \int \frac{\rho(\Bx')}{4 \pi \epsilon_0} \frac{\Bx - \Bx'}{\Norm{\Bx - \Bx'}^3} d^3 x \\
&=
\int \frac{\rho(\Bx')}{4 \pi \epsilon_0} \spacegrad \cdot \frac{\Bx - \Bx'}{\Norm{\Bx - \Bx'}^3} d^3 x \\
&=
-\int \frac{\rho(\Bx')}{4 \pi \epsilon_0} \spacegrad' \cdot \frac{\Bx - \Bx'}{\Norm{\Bx - \Bx'}^3} d^3 x.
\end{aligned}
\end{equation}
Comparing the LHS and RHS, we must have
\begin{equation}\label{eqn:selfEnergyIntegral:160}
\spacegrad' \cdot \frac{\Bx' - \Bx}{\Norm{\Bx' - \Bx}^3} = 4 \pi \delta^3\lr{\Bx' - \Bx}.
\end{equation}

We can now substitute that into the second integral to find
\begin{equation}\label{eqn:selfEnergyIntegral:n}
\begin{aligned}
I_2 &= 
\int
\lr{
\spacegrad_\Brho \cdot
\frac{\Brho}{\rho^3}
}
\inv{\Norm{\Brho + \Bn}}
d^3 \rho \\
&=
\frac{4 \pi}{\Norm{\Bn}} \\
&=
4 \pi.
\end{aligned}
\end{equation}

Sure enough, the integral has a \( 4 \pi \) value.  But was that easy?  \href{https://youtu.be/mm-4PltMB2A?t=93}{I think Hitler would disagree.}
%}
\EndArticle
