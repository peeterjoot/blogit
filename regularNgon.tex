%
% Copyright � 2022 Peeter Joot.  All Rights Reserved.
% Licenced as described in the file LICENSE under the root directory of this GIT repository.
%
%{
\input{../latex/blogpost.tex}
\renewcommand{\basename}{regularNgon}
%\renewcommand{\dirname}{notes/phy1520/}
\renewcommand{\dirname}{notes/ece1228-electromagnetic-theory/}
%\newcommand{\dateintitle}{}
%\newcommand{\keywords}{}

\input{../latex/peeter_prologue_print2.tex}

\usepackage{peeters_layout_exercise}
\usepackage{peeters_braket}
\usepackage{peeters_figures}
\usepackage{siunitx}
\usepackage{verbatim}
%\usepackage{mhchem} % \ce{}
%\usepackage{macros_bm} % \bcM
%\usepackage{macros_qed} % \qedmarker
%\usepackage{txfonts} % \ointclockwise

\beginArtNoToc

\generatetitle{Interior angles of a regular n-sided polygon.}
%\chapter{XXX}
%\label{chap:regularNgon}

For reasons that I can't explain, I woke up this morning dreaming about the interior angles of regular polygons.  i.e. the angles \( \pi - \theta \) \cref{fig:polygonProblem:polygonProblemFig1}.
\imageFigure{../figures/blogit/polygonProblemFig1}{Regular polygon, interior angles.}{fig:polygonProblem:polygonProblemFig1}{0.3}

The logical way to calculate that angle would be to slice the polygon up into triangles from the center, since each slice would have an interior angle would be \( 2 \pi / N \), and then the problem is just trigonometric.  However, in my dream, I was going around the outside, each time rotating by a constant angle, until reaching the original starting point.  This was a vector algebra problem, instead of a trigonometric problem, as illustrated in
\cref{fig:iteratingPolygon:iteratingPolygonFig2}.
\imageFigure{../figures/blogit/iteratingPolygonFig2}{Polygon vertex iteration.}{fig:iteratingPolygon:iteratingPolygonFig2}{0.3}

I didn't have the computational power in my dream to solve the problem, and had to write it down when I woke up, to do so.  The problem has the structure of a recurrence relation:
\begin{equation}\label{eqn:regularNgon:20}
\Bp_k  = \Bp_{k-1} + a \Be_1 \lr{ e^{i\theta} }^{k-1},
\end{equation}
where
\begin{equation}\label{eqn:regularNgon:40}
   \Bp_N = \Bp_0.
\end{equation}

We can write these out explicitly for the first few \( k \) to see the pattern
\begin{equation}\label{eqn:regularNgon:60}
\begin{aligned}
   \Bp_2
   &= \Bp_{1} + a \Be_1 \lr{ e^{i\theta} }^{2-1} \\
   &= \Bp_{0} + a \Be_1 \lr{ e^{i\theta} }^{1-1} + a \Be_1 \lr{ e^{i\theta} }^{2-1} \\
   &= \Bp_{0} + a \Be_1 \lr{ 1 + \lr{ e^{i\theta} }^{2-1} },
\end{aligned}
\end{equation}
or
\begin{equation}\label{eqn:regularNgon:80}
\Bp_k = \Bp_{0} + a \Be_1 \lr{ 1 + e^{i\theta} + \lr{ e^{i\theta} }^{2-1} + \cdots + \lr{ e^{i\theta} }^{k-1} },
\end{equation}
so the equation to solve (for \(\theta\)) is
\begin{equation}\label{eqn:regularNgon:100}
   \Bp_N = \Bp_0 + a \Be_1 \lr{ 1 + \cdots + \lr{ e^{i\theta} }^{N-1} } = \Bp_0,
\end{equation}
or
\begin{equation}\label{eqn:regularNgon:120}
   1 + \cdots + \lr{ e^{i\theta} }^{N-1} = 0.
\end{equation}
The LHS is a geometric series of the form
\begin{equation}\label{eqn:regularNgon:140}
   S_N = 1 + \alpha + \cdots \alpha^{N-1}.
\end{equation}
Recall that the trick to solve this is noting that
\begin{equation}\label{eqn:regularNgon:160}
   \alpha S_N = \alpha + \cdots \alpha^{N-1} + \alpha^N,
\end{equation}
so
\begin{equation}\label{eqn:regularNgon:180}
   \alpha S_N - S_N = \alpha^N - 1,
\end{equation}
or
\begin{equation}\label{eqn:regularNgon:200}
S_N = \frac{\alpha^N - 1}{\alpha - 1}.
\end{equation}
For our polygon, we seek a zero numerator, that is
\begin{equation}\label{eqn:regularNgon:220}
   e^{N i \theta} = 1,
\end{equation}
and the smallest \( \theta \) solution to this equation is
\begin{equation}\label{eqn:regularNgon:240}
   N \theta = 2 \pi,
\end{equation}
or
\begin{equation}\label{eqn:regularNgon:260}
   \theta = \frac{2 \pi}{N}.
\end{equation}
The interior angle is the complement of this, since we are going around the outside edge.  That is
\begin{equation}\label{eqn:regularNgon:280}
\begin{aligned}
   \pi - \theta &= \pi - \frac{2 \pi}{N} \\ &= \frac{ N - 2 }{N} \pi,
\end{aligned}
\end{equation}
and the sum of all the interior angles is
\begin{equation}\label{eqn:regularNgon:300}
   N \lr{ \pi - \theta } = \lr{N - 2 } \pi.
\end{equation}

Plugging in some specific values, for \( N = 3, 4, 5, 6 \), we find that the interior angles are \( \pi/3, \pi/2, 3 \pi/5, 4 \pi/6 \), and the respective sums of these interior angles for the entire polygons are \( \pi, 2 \pi, 3 \pi, 4 \pi \).

Like, I said, this isn't the simplest way to solve this problem.  Instead, we could solve for \( 2 \mu \) with respect to interior triangle illustrated in
\cref{fig:easyWayPolygon:easyWayPolygonFig3}, where
\imageFigure{../figures/blogit/easyWayPolygonFig3}{Polygon interior geometry.}{fig:easyWayPolygon:easyWayPolygonFig3}{0.3}
\begin{equation}\label{eqn:regularNgon:320}
   2 \mu + \frac{ 2\pi}{N} = \pi,
\end{equation}
or
\begin{equation}\label{eqn:regularNgon:340}
   2 \mu = \frac{N - 2}{N} \pi,
\end{equation}
as found the hard way.  The hard way was kind of fun though.

The toughest problem to solve would be ``why on earth was my brain pondering this in the early morning?'' I didn't even go to bed thinking about anything math or geometry related (we finished the night with the brain-dead activity of watching an episode of ``Stranger things''.)

%}
%\EndArticle
\EndNoBibArticle
