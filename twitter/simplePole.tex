%
% Copyright � 2025 Peeter Joot.  All Rights Reserved.
% Licenced as described in the file LICENSE under the root directory of this GIT repository.
%
%{
\input{../latex/blogpost.tex}
\renewcommand{\basename}{junk}
%\renewcommand{\dirname}{notes/phy1520/}
\renewcommand{\dirname}{notes/ece1228-electromagnetic-theory/}
%\newcommand{\dateintitle}{}
%\newcommand{\keywords}{}

\input{../latex/peeter_prologue_print2.tex}

\usepackage{peeters_layout_exercise}
\usepackage{peeters_braket}
\usepackage{peeters_figures}
\usepackage{siunitx}
\usepackage{verbatim}
%\usepackage{macros_cal} % \LL
%\usepackage{amsthm} % proof
%\usepackage{mhchem} % \ce{}
%\usepackage{macros_bm} % \bcM
%\usepackage{macros_qed} % \qedmarker
%\usepackage{txfonts} % \ointclockwise

\beginArtNoToc

\generatetitle{XXX}
%\chapter{XXX}
%\label{chap:junk}

Find
\begin{equation}\label{eqn:junk:100}
I = \int_{-\infty}^\infty \frac{dx}{1 + x^2}.
\end{equation}

We will evaluate
\begin{equation}\label{eqn:junk:120}
\oint \frac{dz}{1 + z^2} = I + \lim_{R\rightarrow \infty} \int_0^{2 \pi} \frac{R e^{i\theta} d\theta}{1 + R^2 e^{2 i \theta}},
\end{equation}
over the infinite semicircular contour \cref{fig:simpleSemicircularContour:simpleSemicircularContourFig1}.

\imageFigure{../figures/blogit/simpleSemicircularContourFig1}{Infinite semicircular contour.}{fig:simpleSemicircularContour:simpleSemicircularContourFig1}{0.3}

The contribution on the semicircle vanishes, and we have a single enclosed pole at \( z = i \), so the residue calculation is just
\begin{equation}\label{eqn:junk:140}
I = 2 \pi i \evalbar{\inv{z + i}}{z = i} = \pi.
\end{equation}

%}
%\EndArticle
\EndNoBibArticle
