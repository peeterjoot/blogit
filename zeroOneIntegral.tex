%
% Copyright � 2025 Peeter Joot.  All Rights Reserved.
% Licenced as described in the file LICENSE under the root directory of this GIT repository.
%
%{
\input{../latex/blogpost.tex}
\renewcommand{\basename}{zeroOneIntegral}
%\renewcommand{\dirname}{notes/phy1520/}
\renewcommand{\dirname}{notes/ece1228-electromagnetic-theory/}
%\newcommand{\dateintitle}{}
%\newcommand{\keywords}{}

\input{../latex/peeter_prologue_print2.tex}

\usepackage{peeters_layout_exercise}
\usepackage{peeters_braket}
\usepackage{peeters_figures}
\usepackage{siunitx}
\usepackage{verbatim}
%\usepackage{macros_cal} % \LL
%\usepackage{amsthm} % proof
%\usepackage{mhchem} % \ce{}
%\usepackage{macros_bm} % \bcM
%\usepackage{macros_qed} % \qedmarker
%\usepackage{txfonts} % \ointclockwise

\beginArtNoToc

\generatetitle{XXX}
%\chapter{XXX}
%\label{chap:zeroOneIntegral}
We wish to solve
\begin{equation}\label{eqn:zeroOneIntegral:20}
I = \int_0^1 \frac{dx}{x^{1/2} + x^{1/3}}.
\end{equation}
Let \( y = x^{1/6} \), so
\begin{equation}\label{eqn:zeroOneIntegral:40}
dy = \inv{6} x^{-5/6} dx,
\end{equation}
or
\begin{equation}\label{eqn:zeroOneIntegral:60}
dx = 6 y^{5} dy.
\end{equation}
The integral is reduced to
\begin{equation}\label{eqn:zeroOneIntegral:80}
I
= \int_0^1 \frac{6 y^5 dy}{y^3 + y^2}
= \int_0^1 \frac{6 y^3 dy}{y + 1}.
\end{equation}
Now let \( u = y + 1 \), so that we can clear most of the denominator.
\begin{equation}\label{eqn:zeroOneIntegral:100}
\begin{aligned}
I
&= 6 \int_1^2 \frac{\lr{u-1}^3}{u} du \\
&= 6 \int_1^2 \lr{ u^2 - 3 u + 3 - \inv{u} } du \\
&= 6 \evalrange{ \inv{3} u^3 - \frac{3}{2} u^2 + 3 u - \ln u }{1}{2} \\
&= 5 - 6 \ln 2.
\end{aligned}
\end{equation}

%}
%\EndArticle
\EndNoBibArticle
