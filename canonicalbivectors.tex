%
% Copyright � 2022 Peeter Joot.  All Rights Reserved.
% Licenced as described in the file LICENSE under the root directory of this GIT repository.
%
%{
\input{../latex/blogpost.tex}
\renewcommand{\basename}{canonicalbivectors}
%\renewcommand{\dirname}{notes/phy1520/}
\renewcommand{\dirname}{notes/ece1228-electromagnetic-theory/}
%\newcommand{\dateintitle}{}
%\newcommand{\keywords}{}

\input{../latex/peeter_prologue_print2.tex}

\usepackage{peeters_layout_exercise}
\usepackage{peeters_braket}
\usepackage{peeters_figures}
\usepackage{siunitx}
\usepackage{verbatim}
%\usepackage{mhchem} % \ce{}
%\usepackage{macros_bm} % \bcM
%\usepackage{macros_qed} % \qedmarker
%\usepackage{txfonts} % \ointclockwise

\beginArtNoToc

\generatetitle{Canonical bivectors in spacetime algebra.}
%\chapter{Canonical bivectors}
%\label{chap:canonicalbivectors}

I've been enjoying XylyXylyX's \href{https://www.youtube.com/watch?v=Bt0qLGami5M}{QED Prerequisites Geometric Algebra: Spacetime} YouTube series, which is doing a thorough walk through of \citep{dressel2015spacetime}, filling in missing details.  The last episode \href{https://youtu.be/2ZSjkVJW6m0}{QED Prerequisites Geometric Algebra 15: Complex Structure}, left things with a bit of a cliff hanger, mentioning a ``canonical'' form for STA bivectors that was intriguing.

The idea is that STA bivectors, like spacetime vectors can be spacelike, timelike, or lightlike (i.e.: positive, negative, or zero square), but can also have a complex signature (squaring to a 0,4-multivector.)

The only context that I knew of that one wanted to square an STA bivector is for the electrodynamic field Lagrangian, which has an \( F^2 \) term.  In no other context, was the signature of \( F \), the electrodynamic field, of interest that I knew of, so I'd never considered this ``Canonical form'' representation.

Here are some examples:
\begin{equation}\label{eqn:canonicalbivectors:20}
\begin{aligned}
F &= \gamma_{10}, \quad F^2 = 1 \\
F &= \gamma_{23}, \quad F^2 = -1 \\
F &= 4 \gamma_{10} + \gamma_{13}, \quad F^2 = 15 \\
F &= \gamma_{10} + \gamma_{13}, \quad F^2 = 0 \\
F &= \gamma_{10} + 4 \gamma_{13}, \quad F^2 = -15 \\
F &= \gamma_{10} + \gamma_{23}, \quad F^2 = 2 I \\
F &= \gamma_{10} - 2 \gamma_{23}, \quad F^2 = -3 + 4 I.
\end{aligned}
\end{equation}
You can see in this table that all the \( F \)'s that are purely electric, have a positive signature, and all the purely magnetic fields have a negative signature, but when there is a mix, anything goes.  The idea behind the canonical representation in the paper is to write
\begin{equation}\label{eqn:canonicalbivectors:40}
F = f e^{I \phi},
\end{equation}
where \( f^2 \) is real and positive, assuming that \( F \) is not lightlike.

The paper gives a formula for computing \( f \) and \( \phi\), but let's do this by example, putting all the \( F^2 \)'s above into their complex polar form representation, like so
\begin{equation}\label{eqn:canonicalbivectors:60}
\begin{aligned}
F &= \gamma_{10}, \quad F^2 = 1  \\
F &= \gamma_{23}, \quad F^2 = 1 e^{\pi I} \\
F &= 4 \gamma_{10} + \gamma_{13}, \quad F^2 = 15  \\
F &= \gamma_{10} + \gamma_{13}, \quad F^2 = 0 \\
F &= \gamma_{10} + 4 \gamma_{13}, \quad F^2 = 15 e^{\pi I} \\
F &= \gamma_{10} + \gamma_{23}, \quad F^2 = 2 e^{(\pi/2) I} \\
F &= \gamma_{10} - 2 \gamma_{23}, \quad F^2 = 5 e^{ (\pi - \arctan(4/3)) I}
\end{aligned}
\end{equation}

Since we can put \( F^2 \) in polar form, we can factor out half of that phase angle, so that we are left with a bivector that has a positive square.  If we write
\begin{equation}\label{eqn:canonicalbivectors:80}
F^2 = \Abs{F^2} e^{2 \phi I},
\end{equation}
we can then form
\begin{equation}\label{eqn:canonicalbivectors:100}
f = F e^{-\phi I}.
\end{equation}

If we want an equation for \( \phi \), we can just write
\begin{equation}\label{eqn:canonicalbivectors:120}
2 \phi = \Arg( F^2 ).
\end{equation}
This is a bit better (I think) than the form given in the paper, since it will uniformly rotate \( F^2 \) toward the positive region of the real axis, whereas the paper's formula sometimes rotates towards the negative reals, which is a strange seeming polar form to use.

Let's compute \( f \) for \( F = \gamma_{10} - 2 \gamma_{23} \), using
\begin{equation}\label{eqn:canonicalbivectors:140}
2 \phi = \pi - \arctan(4/3).
\end{equation}
The exponential expands to
\begin{equation}\label{eqn:canonicalbivectors:160}
e^{-\phi I} = \inv{\sqrt{5}} \lr{ 1 - 2 I }.
\end{equation}

Multiplying each of the bivector components by \(1 - 2 I\), we find
\begin{equation}\label{eqn:canonicalbivectors:180}
\begin{aligned}
   \gamma_{10} \lr{ 1 - 2 I}
   &=
   \gamma_{10} - 2 \gamma_{100123} \\
   &=
   \gamma_{10} - 2 \gamma_{1123} \\
   &=
   \gamma_{10} + 2 \gamma_{23},
\end{aligned}
\end{equation}
and
\begin{equation}\label{eqn:canonicalbivectors:200}
\begin{aligned}
   - 2 \gamma_{23} \lr{ 1 - 2 I}
   &=
   - 2 \gamma_{23}
   + 4 \gamma_{230123} \\
   &=
   - 2 \gamma_{23}
+ 4 \gamma_{23}^2 \gamma_{01} \\
   &=
   - 2 \gamma_{23}
+ 4 \gamma_{10},
\end{aligned}
\end{equation}
leaving
\begin{equation}\label{eqn:canonicalbivectors:220}
f = \sqrt{5} \gamma_{10},
\end{equation}
so the canonical form is
\begin{equation}\label{eqn:canonicalbivectors:240}
F = \gamma_{10} - 2 \gamma_{23} = \sqrt{5} \gamma_{10} \frac{1 + 2 I}{\sqrt{5}}.
\end{equation}

It's interesting here that \( f \), in this case, is a spatial bivector (i.e.: pure electric field), but that clearly isn't always going to be the case, since we can have a case like,
\begin{equation}\label{eqn:canonicalbivectors:260}
   F = 4 \gamma_{10} + \gamma_{13} = 4 \gamma_{10} + \gamma_{20} I,
\end{equation}
from the table above, that has both electric and magnetic field components, yet is already in the canonical form, with \( F^2 = 15 \).  The canonical \( f \), despite having a positive square, is not necessarily a spatial bivector (as it may have both grades 1,2 in the spatial representation, not just the electric field, spatial grade-1 component.)

%}
\EndArticle
