%
% Copyright � 2025 Peeter Joot.  All Rights Reserved.
% Licenced as described in the file LICENSE under the root directory of this GIT repository.
%
%{
\input{../latex/blogpost.tex}
\renewcommand{\basename}{partialFractions}
%\renewcommand{\dirname}{notes/phy1520/}
\renewcommand{\dirname}{notes/ece1228-electromagnetic-theory/}
%\newcommand{\dateintitle}{}
%\newcommand{\keywords}{}

\input{../latex/peeter_prologue_print2.tex}

\usepackage{peeters_layout_exercise}
\usepackage{peeters_braket}
\usepackage{peeters_figures}
\usepackage{siunitx}
\usepackage{verbatim}
%\usepackage{macros_cal} % \LL
%\usepackage{amsthm} % proof
%\usepackage{mhchem} % \ce{}
%\usepackage{macros_bm} % \bcM
%\usepackage{macros_qed} % \qedmarker
%\usepackage{txfonts} % \ointclockwise

\beginArtNoToc

\generatetitle{XXX}
%\chapter{XXX}
%\label{chap:partialFractions}
% \citep{sakurai2014modern} pr X.Y
% \citep{pozar2009microwave}
% \citep{qftLectureNotes}
% \citep{doran2003gap}
% \citep{jackson1975cew}
% \citep{griffiths1999introduction}

Solve
\begin{equation}\label{eqn:partialFractions:20}
I = \int \frac{ x^2 + 1 }{\lr{ x^2 + 2 }\lr{ x^2 + 3 }} dx.
\end{equation}
We clearly want a partial fractions representation of the integrand
\begin{equation}\label{eqn:partialFractions:40}
\begin{aligned}
I'
&= \frac{A}{x^2 + 2} + \frac{B}{x^2 + 3} \\
&= \frac{ A \lr{ x^2 + 3} + B \lr{ x^2 + 2 } }{ \lr{ x^2 + 2 }\lr{ x^2 + 3 } } \\
&= \frac{ \lr{ A + B }x^2 + \lr{ 3A + 2 B} }{ \lr{ x^2 + 2 }\lr{ x^2 + 3 } },
\end{aligned}
\end{equation}
or
\begin{equation}\label{eqn:partialFractions:60}
\begin{aligned}
A + B &= 1 \\
3A + 2B &= 1.
\end{aligned}
\end{equation}
This has solution \( A = -1, B = 2 \), so
\begin{equation}\label{eqn:partialFractions:80}
I' = -\inv{ x^2 + 2 } + 2 \inv{ x^2 + 3 }.
\end{equation}
If we remember that \( \arctan(x/a)' = a/(x^2 + a^2) \), we can immediately write
\begin{equation}\label{eqn:partialFractions:100}
I = -\inv{\sqrt{2}} \arctan(x/\sqrt{2}) + \frac{2}{\sqrt{3}} \arctan(x/\sqrt{3}) + C.
\end{equation}
Alternatively, since
\begin{equation}\label{eqn:partialFractions:120}
\begin{aligned}
\int \inv{x^2 + a^2} dx
&= \inv{2 a i} \int \lr{ \inv{x - a i} - \inv{x + a i} } dx \\
&= \inv{2 a i} \lr{ \ln\lr{x - a i} - \ln\lr{x + a i} } + C \\
&= \inv{2 a i} \ln \frac{x - a i}{x + a i} + C,
\end{aligned}
\end{equation}
we could also write
\begin{equation}\label{eqn:partialFractions:140}
I = -\inv{2 \sqrt{2}i} \ln \frac{ x - \sqrt{2} i }{x + \sqrt{2} i} + \frac{1}{\sqrt{3}i} \ln \frac{ x - \sqrt{3} i }{x + \sqrt{3} i} + C.
\end{equation}

%}
%\EndArticle
\EndNoBibArticle
