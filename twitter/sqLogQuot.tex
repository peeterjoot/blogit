%
% Copyright � 2025 Peeter Joot.  All Rights Reserved.
% Licenced as described in the file LICENSE under the root directory of this GIT repository.
%
%{
\input{../latex/blogpost.tex}
\renewcommand{\basename}{sqLogQuot}
%\renewcommand{\dirname}{notes/phy1520/}
\renewcommand{\dirname}{notes/ece1228-electromagnetic-theory/}
%\newcommand{\dateintitle}{}
%\newcommand{\keywords}{}

\input{../latex/peeter_prologue_print2.tex}

\usepackage{peeters_layout_exercise}
\usepackage{peeters_braket}
\usepackage{peeters_figures}
\usepackage{siunitx}
\usepackage{verbatim}
%\usepackage{macros_cal} % \LL
%\usepackage{amsthm} % proof
%\usepackage{mhchem} % \ce{}
%\usepackage{macros_bm} % \bcM
%\usepackage{macros_qed} % \qedmarker
%\usepackage{txfonts} % \ointclockwise

\beginArtNoToc

\generatetitle{XXX}
%\chapter{XXX}
%\label{chap:sqLogQuot}

Solve:
\begin{equation}\label{eqn:sqLogQuot:20}
I = \int \frac{ \lr{ \ln x - 1 }^2 }{ \lr{ 1 + \ln x }^2 } dx.
\end{equation}
To get rid of all the ugly logs to start with, let \( u = 1 + \ln x \).  This means that \( du = dx/x \), and \( x = e^{u-1} \).  Our integral is now reduced to
\begin{equation}\label{eqn:sqLogQuot:40}
\begin{aligned}
I
&= \int \frac{\lr{u-1}^2}{u^2} e^{u-1} du \\
&= \inv{e} \int \lr{ 1 - \frac{4}{u} + \frac{4}{u^2} } e^u du.
\end{aligned}
\end{equation}
At first this looks scary, since we don't want to deal with special functions like the exponential integral \( \mathrm{Ei}(x) = -\int_{-\infty}^x \frac{e^u}{u} du \).  Assuming instead that cancelations will work in our favor, let's guess:
\begin{equation}\label{eqn:sqLogQuot:60}
I = \inv{e} \lr{ A + \frac{B}{u} } e^u,
\end{equation}
which requires
\begin{equation}\label{eqn:sqLogQuot:80}
-\frac{B}{u^2} + A + \frac{B}{u} = 1 - \frac{4}{u} + \frac{4}{u^2},
\end{equation}
or \( A = 1, B = -4 \).  Back substutition yeilds
\begin{equation}\label{eqn:sqLogQuot:100}
\begin{aligned}
I
&= \inv{e} \lr{ 1 - \frac{4}{u} } e^u \\
&= \lr{ 1 - \frac{4}{1 + \ln x} } x.
\end{aligned}
\end{equation}
Let's check:
\begin{equation}\label{eqn:sqLogQuot:120}
\begin{aligned}
I'
&= 1 - \frac{4}{1 + \ln x} + \frac{4 x \inv{x}}{\lr{ 1 + \ln x}^2} \\
&=
\frac{\lr{ 1 + \ln x}^2 + 4 \lr{1 + \ln x} + 4}{\lr{ 1 + \ln x}^2} \\
&=
\frac{ 1 + \lr{\ln x}^2 + 2 \ln x - 4 \ln x }{\lr{ 1 + \ln x}^2} \\
&=
\frac{\lr{ 1 - \ln x}^2}{\lr{ 1 + \ln x}^2}.
\end{aligned}
\end{equation}

%}
%\EndArticle
\EndNoBibArticle
