%
% Copyright � 2025 Peeter Joot.  All Rights Reserved.
% Licenced as described in the file LICENSE under the root directory of this GIT repository.
%
%{
\input{../latex/blogpost.tex}
\renewcommand{\basename}{pascals}
%\renewcommand{\dirname}{notes/phy1520/}
\renewcommand{\dirname}{notes/ece1228-electromagnetic-theory/}
%\newcommand{\dateintitle}{}
%\newcommand{\keywords}{}

\input{../latex/peeter_prologue_print2.tex}

\usepackage{peeters_layout_exercise}
\usepackage{peeters_braket}
\usepackage{peeters_figures}
\usepackage{siunitx}
\usepackage{verbatim}
%\usepackage{macros_cal} % \LL
%\usepackage{amsthm} % proof
%\usepackage{mhchem} % \ce{}
%\usepackage{macros_bm} % \bcM
%\usepackage{macros_qed} % \qedmarker
%\usepackage{txfonts} % \ointclockwise

\beginArtNoToc

\generatetitle{Pascal's triangle: binomial coefficients, value and recurrence relation.}
%\chapter{Pascal's triangle: choose recurrence relation}
%\label{chap:pascals}

I saw an interesting solution of an introductory programming source sample problem, showing how to display Pascal's triangle up to a given size.  For example, given input \( n = 5 \), the output should be like:
\begin{equation*}
\begin{array}{c}
     1 \\
     1 \quad 1 \\
     1 \quad 2 \quad 1 \\
     1 \quad 3 \quad 3 \quad 1 \\
     1 \quad 4 \quad 6 \quad 4 \quad 1 \\
     1 \quad 5 \quad 10 \quad 10 \quad 5 \quad 1
\end{array}
\end{equation*}

One kind of neat property of Pascal's triangle is that any non-unity number on the triangle, is the sum of the two numbers above it.  That's one of the ways that I'd imagine a student would solve the programming problem of displaying this triangle.

Another way would be to use a formula for the binomial coefficients.  Recall that
\maketheorem{Binomial theorem for integer powers.}{thm:pascals:1}{
Given integer \( n \), and any values \( a, b \) that commute under multiplication
\begin{equation*}
\lr{ a + b }^n = \sum_{k = 0}^n \binom{n}{k} a^k b^{n-k},
\end{equation*}
where
\begin{equation*}
\binom{n}{k} = \frac{n!}{k!\lr{n-k}!}.
\end{equation*}
} % theorem

These \( \binom{n}{k} \) values are the binomial coefficients, and are, in fact the values in Pascal's triangle (we will show this in a bit.)

What surprised me about the programming solution that I saw was that it didn't use the binomial coefficients.  Instead if appeared to use a recurrence relationship, which was interesting, because I didn't remember any such relationship.  We'll also find that recurrence relationship below, which allows for evaluation of the binomial coefficients without evaluating any of the factorials.

\section{Combinatoric argument for the binomial coefficients}

Let's look at an explicit expansion of the LHS of the binomial theorem for the first few values of \( n \), starting at \(2\).
\begin{equation}\label{eqn:pascals:20}
\lr{ H + T }^2 = \lr{ H + T }\lr{ H + T} = H H + H T + T H + T T.
\end{equation}
We see that we have all possible combinations of \( H, T \) in the final expression, but because of assumed commutation, there is some redundancy, and we can write
\begin{equation}\label{eqn:pascals:40}
\lr{ H + T }^2 = H^2 + 2 H T + T T.
\end{equation}
If we enumerate all possible values for two flips of a coin, there can only be one \( H, H \) pair, there will be two \( H, T \) pairs, and one \( T, T \) pair.  These numbers are the second row of Pascal's triangle: \( 1, 2, 1 \).

Similarly, for \( n = 3 \), we have
\begin{equation}\label{eqn:pascals:60}
\begin{aligned}
\lr{ H + T }^3
&= \lr{ H H + H T + T H + T T } \lr{ H + T }  \\
&= H H H + H T H + T H H + T T H + H H T + H T T + T H T + T T T.
\end{aligned}
\end{equation}
This time, after grouping, we have
\begin{equation}\label{eqn:pascals:80}
\lr{ H + T }^3 = H^3 + 3 H^2 T + 3 H T^2 + T^3.
\end{equation}
If we enumerate all possible values for three flips of a coin, there can only be one \( H, H, H \) triplet, there will be three \( H, H, T \) triplets, three \( T, T, H \) triplets, and one \( T, T, T \) triplet.  These numbers are the third row of Pascal's triangle: \( 1, 3, 3, 1 \).

To understand the binomial coefficient, we have to consider how to pick indistinguishable items from a set.  Suppose that we want to pick three items from a set of 7, in a specific order.  There are 7 ways to pick the first item, 6 ways to pick the second, and 5 ways to pick the third.  We can express that as

\begin{equation}\label{eqn:pascals:120}
P(7,3) = 7 \times 6 \times 5 = \frac{7!}{4!} = \frac{7!}{\lr{7-3}!},
\end{equation}
or more generally, to pick \( k \) distinguishable items from a set of \( n \)
\begin{equation}\label{eqn:pascals:140}
P(n,k) = \frac{n!}{\lr{n-k}!}.
\end{equation}
In our \( \lr{H + T}^n \) expansion, we cannot distinguish any of the elements.  For each \( k \) items we pick, there are \( k! \) ways of arranging those, so the binomial coefficient is
\begin{equation}\label{eqn:pascals:160}
\binom{n}{k} = \frac{P(n,k)}{k!} = \frac{n!}{k!\lr{n-k}!},
\end{equation}
as stated earlier.
\section{Recurrence relation}
For the recurrence relation, let's assume that we have already found \( \binom{n}{k} \), let's see if we can use that to find the next on the line.  That is, for \( k < n \)
\begin{equation}\label{eqn:pascals:100}
\begin{aligned}
\binom{n}{k + 1}
&= \frac{n!}{\lr{k+1}!\lr{n-(k+1)}!} \\
&= \frac{n!}{\lr{k+1} k!\lr{n - k - 1)}!} \\
&= \frac{n! \lr{ n - k }}{\lr{k+1} k!\lr{n - k)}!} \\
&= \binom{n}{k} \frac{n - k}{k + 1}.
\end{aligned}
\end{equation}

This is a kind of cool result, and is the method that I saw used in the solution to the Pascal's triangle program.  We can start with the 1 value at the left of any row, and with a single multiply and divide, figure out the next entry in the triangle, and proceed from there.

\section{Pascal's triangle.}
For the Pascal's triangle property to hold, we must have, for any \( k \ne 0 \)
\begin{equation}\label{eqn:pascals:180}
\binom{n}{k} = \binom{n-1}{k-1} + \binom{n-1}{k},
\end{equation}
as illustrated in \cref{fig:pascals:pascalsFig1}.
\imageFigure{../figures/blogit/pascalsFig1}{Pascal's triangle, sum of elements above.}{fig:pascals:pascalsFig1}{0.3}

It's fairly straightforward to verify that this property holds.
\begin{equation}\label{eqn:pascals:200}
\begin{aligned}
\binom{n-1}{k-1} + \binom{n-1}{k}
&=
\frac{\lr{n-1}!}{\lr{k-1}!\lr{n-1-\lr{k-1}}!}
+
\frac{\lr{n-1}!}{\lr{k}!\lr{n-1-k}!} \\
&=
\frac{\lr{n-1}!}{\lr{k-1}!} \lr{ \inv{\lr{n-k}!} + \inv{k \lr{n-1-k}!} } \\
&=
\frac{\lr{n-1}!}{\lr{k-1}!\lr{n-1-k}!} \lr{ \inv{n-k} + \inv{k} } \\
&=
\frac{\lr{n-1}!}{\lr{k-1}!\lr{n-1-k}!} \frac{k + n - k }{k \lr{ n - k } } \\
&=
\frac{n!}{k!\lr{n-k}!} \\
&=
\binom{n}{k},
\end{aligned}
\end{equation}
as expected.

%}
%\EndArticle
\EndNoBibArticle
