%
% Copyright � 2025 Peeter Joot.  All Rights Reserved.
% Licenced as described in the file LICENSE under the root directory of this GIT repository.
%
%{
\input{../latex/blogpost.tex}
\renewcommand{\basename}{trigProp}
%\renewcommand{\dirname}{notes/phy1520/}
\renewcommand{\dirname}{notes/ece1228-electromagnetic-theory/}
%\newcommand{\dateintitle}{}
%\newcommand{\keywords}{}

\input{../latex/peeter_prologue_print2.tex}

\usepackage{peeters_layout_exercise}
\usepackage{peeters_braket}
\usepackage{peeters_figures}
\usepackage{siunitx}
\usepackage{verbatim}
%\usepackage{macros_cal} % \LL
%\usepackage{amsthm} % proof
%\usepackage{mhchem} % \ce{}
%\usepackage{macros_bm} % \bcM
%\usepackage{macros_qed} % \qedmarker
%\usepackage{txfonts} % \ointclockwise

\beginArtNoToc

\generatetitle{What will be the value of k to satisfy this integral equation}
%\chapter{What will be the value of k to satisfy this integral equation}
%\label{chap:trigProp}

Another problem from x/twitter (\citep{CalcInsightsTrigK}):

Find \( k \), where
\begin{equation}\label{eqn:trigProp:20}
\int_0^{2 \pi} \sin^4 x dx = k \int_0^{\pi/2} \cos^4 x dx.
\end{equation}
I initially misread the integration range in the second integral as \( 2 \pi \), not \( \pi/2 \), in which case the answer is just 1 by inspection.  However, solving the stated problem, is not much more difficult.

Since sine and cosine are equal up to a shift by \( \pi/2 \)
\begin{equation}\label{eqn:trigProp:40}
\sin(u + \pi/2) = \frac{e^{i(u + \pi/2)} - e^{-i(u + \pi/2)}}{2i} = \frac{e^{i u} + e^{-i u}}{2} = \cos u,
\end{equation}
we can make an \( x = u + \pi/2 \) substitution in the sine integral.  

Observe that \( \cos^4 x = \Abs{\cos x}^4 \), but the area under \( \Abs{\cos x} \) is the same for each \( \pi/2 \) interval.  This is shown in \cref{fig:trigProp:trigPropFig1}.
\imageFigure{../figures/blogit/trigPropFig1}{Plot of \(\Abs{\cos x}\)}{fig:trigProp:trigPropFig1}{0.3}

Of course, the area under \( \cos^4 x \), will also have the same periodicity, but those regions will be rounded out by the power operation, as shown in
\cref{fig:trigProp:trigPropFig2}.
\imageFigure{../figures/blogit/trigPropFig2}{Plot of \( \cos^4 x\).}{fig:trigProp:trigPropFig2}{0.3}

Since the area under \( \cos^4 x \) is the same for each \( \pi/2 \) wide interval, we have
\begin{equation}\label{eqn:trigProp:60}
\boxed{
k = 4.
}
\end{equation}

%}
\EndArticle
%\EndNoBibArticle
