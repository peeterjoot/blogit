%
% Copyright � 2022 Peeter Joot.  All Rights Reserved.
% Licenced as described in the file LICENSE under the root directory of this GIT repository.
%
%{
\input{../latex/blogpost.tex}
\renewcommand{\basename}{independent}
%\renewcommand{\dirname}{notes/phy1520/}
\renewcommand{\dirname}{notes/ece1228-electromagnetic-theory/}
%\newcommand{\dateintitle}{}
%\newcommand{\keywords}{}

\input{../latex/peeter_prologue_print2.tex}

\usepackage{peeters_layout_exercise}
\usepackage{peeters_braket}
\usepackage{peeters_figures}
\usepackage{siunitx}
\usepackage{verbatim}
%\usepackage{mhchem} % \ce{}
%\usepackage{macros_bm} % \bcM
%\usepackage{macros_qed} % \qedmarker
%\usepackage{txfonts} % \ointclockwise

\beginArtNoToc

\generatetitle{XXX}
%\chapter{XXX}
%\label{chap:independent}

Question: How can you show if $\Ba_1,\dots,\Ba_k$ are linearly independent, then their wedge product is non-zero?

You could reduce this problem to showing that the determinant of the coordinates is non-zero if the vectors are independent.

Consider the simplest two vector case \(\Ba, \Bb \in \mathbb{R}^N\) first.  Expansion of the wedge in coordinates gives

\begin{equation*}
\begin{aligned}
\Ba \wedge \Bb 
&= \sum_{ij} \lr{ a_i \Be_i } \wedge \lr{ b_i \Be_j } \\
&= \lr{ \sum_{i < j} + \sum_{i > j} } a_i b_j \Be_i \Be_j \\
&= \sum_{i < j} \lr{ a_i b_j - a_j b_i } \Be_i \Be_j \\
&= \sum_{i < j} \begin{vmatrix} a_i & a_j \\ b_i & b_j \end{vmatrix} \Be_i \Be_j.
\end{aligned}
\end{equation*}

For \(\Ba, \Bb \in \mathbb{R}^2\) this is just a single determinant, but for higher dimensional spaces, you can always find two orthonormal vectors in the plane of the wedge product,
\begin{equation*}
   \Bf_1 \cdot \Bf_2 = 0,
\end{equation*}
and with respect to that subspace basis, one has
\begin{equation*}
a \wedge b = \begin{vmatrix} a_1' & a_2' \\ b_1' & b_2' \end{vmatrix} \Bf_1 \Bf_2.
\end{equation*}
The problem is now reduced to showing that the determinant of the coordinates with respect to this basis is non-zero if the vectors are linearly independent, which is a well known theorem.

Similarly, at least for non-degenerate spaces, one can find an orthonormal basis \( \Bf_1, \cdots, \Bf_k \in \mathrm{Span} \{ \Ba_1, \cdots \Ba_k \}, \Bf_i \cdot \Bf_j = \pm \delta_{ij} \) for which
\begin{equation*}
   \Ba_1 \wedge \cdots \wedge \Ba_k = 
\begin{vmatrix}
   a_{11} & \cdots & a_{1k} \\
    & \vdots & \\
   a_{k1} & \cdots & a_{kk} \\
\end{vmatrix}
\Bf_1 \cdots \Bf_k
\end{equation*}
and this determinant of the coordinates is non-zero if the vectors are linearly independent.

%}
%\EndArticle
\EndNoBibArticle
