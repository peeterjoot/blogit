%
% Copyright � 2024 Peeter Joot.  All Rights Reserved.
% Licenced as described in the file LICENSE under the root directory of this GIT repository.
%
%{
\input{../latex/blogpost.tex}
\renewcommand{\basename}{gaproduct_sine}
%\renewcommand{\dirname}{notes/phy1520/}
\renewcommand{\dirname}{notes/ece1228-electromagnetic-theory/}
%\newcommand{\dateintitle}{}
%\newcommand{\keywords}{}

\input{../latex/peeter_prologue_print2.tex}

\usepackage{peeters_layout_exercise}
\usepackage{peeters_braket}
\usepackage{peeters_figures}
\usepackage{siunitx}
\usepackage{verbatim}
%\usepackage{mhchem} % \ce{}
%\usepackage{macros_bm} % \bcM
%\usepackage{macros_qed} % \qedmarker
%\usepackage{txfonts} % \ointclockwise

\beginArtNoToc

\generatetitle{Geometric product square, sines and exponentials}
%\chapter{Geometric product square, sines and exponentials}
%\label{chap:gaproduct_sine}
This is really two questions, first.
\section{Square.}
Given
\begin{equation*}
\Ba \cdot \Bb = \inv{2} \lr{ \Ba \Bb + \Bb \Ba },
\end{equation*}
\begin{equation*}
\Ba \wedge \Bb = \inv{2} \lr{ \Ba \Bb - \Bb \Ba },
\end{equation*}
we have
\begin{equation*}
\begin{aligned}
\lr{ \Ba \cdot \Bb }^2 - \lr{ \Ba \wedge \Bb }^2 
&=
\inv{4} \lr{ \Ba \Bb + \Bb \Ba }^2 - \inv{4} \lr{ \Ba \Bb - \Bb \Ba }^2 \\
&=
\inv{4} \lr{ 
\lr{\Ba \Bb}^2
\lr{\Bb \Ba}^2
+ \Ba \Bb \Bb \Ba
+ \Bb \Ba \Ba \Bb }
-
\inv{4} \lr{ 
\lr{\Ba \Bb}^2
\lr{\Bb \Ba}^2
- \Ba \Bb \Bb \Ba
- \Bb \Ba \Ba \Bb } \\
&=
\inv{2} \lr{ 
+ 2 \Ba^2 \Bb^2
}
-
\inv{2} \lr{ 
- 2 \Ba^2 \Bb^2
} 
&= \Ba^2 \Bb^2.
\end{aligned}
\end{equation*}
\section{Trig relations.}
For the second question, let's express one vector in terms of length and direction, say
\begin{equation*}
\Ba = \Norm{\Ba} \acap,
\end{equation*}
and then express the second vector as a scaled rotation of that unit vector \( \acap \).  That is
\begin{equation*}
\Bb = \Norm{\Bb} \acap e^{i\theta},
\end{equation*}
where \( \theta \) is the angle from \( \Ba \) to \( \Bb \), and
\begin{equation*}
i = \frac{\Ba \wedge \Bb}{\Norm{\Ba \wedge \Bb}},
\end{equation*}
is the unit pseudoscalar for the plane containing \( \Ba, \Bb \), and is oriented ``from'' \( \Ba \) ``to'' \( \Bb \) as sketched below.  Note that the bivector norm, assuming a Euclidean vector space, should be defined as
\begin{equation*}
\Norm{\Ba \wedge \Bb} = \sqrt{ -\lr{ \Ba \wedge \Bb}^2 }.
\end{equation*}
Also note that we did not need to use half angle sandwiched rotors, since we are rotating both in the plane formed by the span of the two vectors.

FIXME: figure.

You can now expand the product \( \Ba \Bb \) as 
\begin{equation*}
\begin{aligned}
\Ba \Bb 
&= \Norm{\Ba} \acap \Norm{\Bb} \acap e^{i\theta} \\
&= \Norm{\Ba} \Norm{ \Bb } \acap \acap e^{i\theta} \\
&= \Norm{\Ba} \Norm{ \Bb } e^{i\theta} \\
&= \Norm{\Ba} \Norm{ \Bb } \lr{ \cos\theta + i \sin\theta }.
\end{aligned}
\end{equation*}
Scalar and bivector grade selections show that
\begin{equation*}
\Ba \cdot \Bb = \Norm{\Ba} \Norm{ \Bb } \cos\theta,
\end{equation*}
and
\begin{equation*}
\Ba \wedge \Bb = \Norm{\Ba} \Norm{ \Bb } i \sin\theta.
\end{equation*}
In particular, we have
\begin{equation*}
\lr{ \Ba \wedge \Bb}^2 = \Norm{\Ba}^2 \Norm{ \Bb }^2 (-1) \sin^2\theta,
\end{equation*}
and
\begin{equation*}
\Norm{ \Ba \wedge \Bb}^2 = \Norm{\Ba}^2 \Norm{ \Bb }^2 \sin^2\theta,
\end{equation*}
or
\begin{equation*}
\Norm{ \Ba \wedge \Bb} = \Norm{\Ba} \Norm{ \Bb } \Abs{ \sin\theta },
\end{equation*}
which is what you meant to write.

%}
%\EndArticle
\EndNoBibArticle
