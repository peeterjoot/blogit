%
% Copyright � 2020 Peeter Joot.  All Rights Reserved.
% Licenced as described in the file LICENSE under the root directory of this GIT repository.
%
%{
\input{../latex/blogpost.tex}
\renewcommand{\basename}{crashem}
%\renewcommand{\dirname}{notes/phy1520/}
\renewcommand{\dirname}{notes/ece1228-electromagnetic-theory/}
%\newcommand{\dateintitle}{}
%\newcommand{\keywords}{}

\input{../latex/peeter_prologue_print2.tex}

\usepackage{peeters_layout_exercise}
\usepackage{peeters_braket}
\usepackage{peeters_figures}
\usepackage{siunitx}
\usepackage{verbatim}
%\usepackage{mhchem} % \ce{}
%\usepackage{macros_bm} % \bcM
%\usepackage{macros_qed} % \qedmarker
%\usepackage{txfonts} % \ointclockwise

\beginArtNoToc

\generatetitle{Crashing Mathematica with HatchShading}
%\chapter{Crashing Mathematica with HatchShading}

I attempted to modify a plot for an electric field solution that I had in my old Antenna-Theory notes:
\begin{dmath}\label{eqn:advancedantennaProblemSet3Problem1:n}
\BE
=
j \omega
\frac{\mu_0 I_{\textrm{eo}} l}{4 \pi r} e^{-j k r}
\lr{ 1 + \cos\theta }
\lr{
-\cos\phi \thetacap
+ \sin\phi \phicap
},
\end{dmath}
and discovered that you can crash Mathematica by combining PlotStyle with Opacity and HatchShading.

%}
\EndArticle
%\EndNoBibArticle
