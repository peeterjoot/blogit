%
% Copyright � 2022 Peeter Joot.  All Rights Reserved.
% Licenced as described in the file LICENSE under the root directory of this GIT repository.
%
%{
\input{../latex/blogpost.tex}
\renewcommand{\basename}{bivectorFactor}
%\renewcommand{\dirname}{notes/phy1520/}
\renewcommand{\dirname}{notes/ece1228-electromagnetic-theory/}
%\newcommand{\dateintitle}{}
%\newcommand{\keywords}{}

\input{../latex/peeter_prologue_print2.tex}

\usepackage{peeters_layout_exercise}
\usepackage{peeters_braket}
\usepackage{peeters_figures}
\usepackage{siunitx}
\usepackage{verbatim}
%\usepackage{mhchem} % \ce{}
%\usepackage{macros_bm} % \bcM
%\usepackage{macros_qed} % \qedmarker
%\usepackage{txfonts} % \ointclockwise

\beginArtNoToc

\generatetitle{XXX}
%\chapter{XXX}
%\label{chap:bivectorFactor}

Suppose that you have found one unit vector factor \( \hata \) of \( B \).  That is \( B \wedge \hata = 0 \), then
\begin{equation*}
B
= B \hata \hata
= (B \hata) \hata,
\end{equation*}
so the two factors are \( \Bb = B \hata = B \cdot \hata \), and \( \hata \).  The task becomes finding such a factor, normalized or otherwise.

Suppose that
\begin{equation*}
B = a_1 \Be_{23} + a_2 \Be_{31} + a_3 \Be_{12}
\end{equation*}

Pick the coefficient \( a_i \) that has the largest absolute magnitude (this is to avoid numerical instability in case the bivector is ill-conditioned and has a small non-zero component in one direction), and select one of the two vector factors of the unit blade that is associated with that component, calling this \( \Be \).  For example, if that coefficient is \( a_3 \) then pick either \( \Be = \Be_1 \) or \( \Be = \Be_2 \).

Now, compute
\begin{equation*}
\Ba = B \cdot \Be.
\end{equation*}
This vector lies in the plane that \( B \) represents.  Specifically, it is the projection of \( \Be \) onto \( B \), but rotated 90 degrees.  If we dot \( \Ba \) with \( B \) then we find another vector that lies in the plane represented by \( B \), but is rotated 90 degrees in the plane, away from \( \Ba \).  That is:
\begin{equation*}
\Bb = B \cdot \Ba
\end{equation*}
We've now found two perpendicular vectors that lie in the plane that \( B \) represents, so we have
\begin{equation*}
B \propto \Ba \Bb = \Ba \wedge \Bb.
\end{equation*}
For your rotor application, you probably want to normalize both \( \Ba \) and \( \Bb \).

Here's a numerical example:
\begin{equation*}
B = \Be_{23} + 2 \Be_{31} + 3 \Be_{12}.
\end{equation*}
We can pick \( \Be = \Be_2 \), so
\begin{equation*}
   \Ba = B \cdot \Be_2 =
   \lr{  \Be_{23} + 2 \Be_{31} + 3 \Be_{12} } \cdot \Be_2
   =  3 \Be_1 - \Be_3.
\end{equation*}
\begin{equation*}
\Bb = B \cdot \Ba
   =
   \lr{  \Be_{23} + 2 \Be_{31} + 3 \Be_{12} } \cdot \lr{ 3 \Be_1 - \Be_3 } = 2 \Be_1 - 10 \Be_2 + 6 \Be_3.
\end{equation*}
For which we find
\begin{equation*}
   \Bb \Ba = 10 \Be_{23} + 20 \Be_{31} + 30 \Be_{12} = 10 B.
\end{equation*}
Observe that the scale factor here is exactly \( \Ba^2 \).

%}
\EndArticle
%\EndNoBibArticle
