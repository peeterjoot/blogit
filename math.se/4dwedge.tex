%
% Copyright � 2021 Peeter Joot.  All Rights Reserved.
% Licenced as described in the file LICENSE under the root directory of this GIT repository.
%
%{
\input{../latex/blogpost.tex}
\renewcommand{\basename}{4dwedge}
%\renewcommand{\dirname}{notes/phy1520/}
\renewcommand{\dirname}{notes/ece1228-electromagnetic-theory/}
%\newcommand{\dateintitle}{}
%\newcommand{\keywords}{}

\input{../latex/peeter_prologue_print2.tex}

\usepackage{peeters_layout_exercise}
\usepackage{peeters_braket}
\usepackage{peeters_figures}
\usepackage{siunitx}
\usepackage{verbatim}
%\usepackage{mhchem} % \ce{}
%\usepackage{macros_bm} % \bcM
%\usepackage{macros_qed} % \qedmarker
%\usepackage{txfonts} % \ointclockwise

\beginArtNoToc

\generatetitle{4d wedge q}
%\chapter{4d wedge q}
%\label{chap:4dwedge}

%in another example, roughly based on what i read, in trying to compute the wedge product between (2,3,4,5) and (6,7,8,9) they assume a standard basis and then calculated [2(1,0,0,0) + 3(0,1,0,0) + 4(0,0,1,0) + 5(0,0,0,1)]^[6(1,0,0,0) + 7(0,1,0,0) + 8(0,0,1,0) + 9(0,0,0,1)] . in their example, they did this with 3d so the output would be like 18(1,0,0)^(0,1,0) would effectively give 18k if the coordinates were roughly represented as (i,j,k). the problem is that when i try this technique for 2 4d vectors, i get stuff like (1,0,0,0)^(0,0,0,1) where reducing this to i,j,k,l doesn't make sense to me.
%
%in the last example, the wedge product is supposed to be an orientated area of something like a bivector, so the result is supposed to be calculated as ||u||*||v||*sin(theta).

With $\Ba = \lr{2 \Bi + 3 \Bj + 4 \Bk + 5 \Bl}$ and $\Bb = 6 \Bi + 7 \Bj + 8 \Bk + 9 \Bl$, we have
\begin{equation*}
\begin{aligned}
B
&= \Ba \wedge \Bb \\
&= \lr{2 \Bi + 3 \Bj + 4 \Bk + 5 \Bl} \wedge \lr{ 6 \Bi + 7 \Bj + 8 \Bk + 9 \Bl } \\
&=
14 \Bi \wedge \Bj + 16 \Bi \wedge \Bk + 18 \Bi \wedge \Bl \\
&\quad+
18 \Bj \wedge \Bi + 24 \Bj \wedge \Bk + 27 \Bj \wedge \Bl \\
&\quad+
24 \Bk \wedge \Bi + 28 \Bk \wedge \Bj + 36 \Bk \wedge \Bl  \\
&\quad+
30 \Bl \wedge \Bi + 35 \Bl \wedge \Bj + 40 \Bl \wedge \Bk  \\
&=
(14 -18) \Bi \wedge \Bj
+ (16-24) \Bi \wedge \Bk
+ (18-30) \Bi \wedge \Bl
+ (24-28) \Bj \wedge \Bk
+ (27-35) \Bj \wedge \Bl
+ (36-40) \Bk \wedge \Bl \\
&=
4 \Bj \wedge \Bi + 8 \Bk \wedge \Bi + 12 \Bl \wedge \Bi + 4 \Bk \wedge \Bj + 8 \Bl \wedge \Bj + 4 \Bl \wedge \Bk.
\end{aligned}
\end{equation*}

In order to interpret this as an oriented area, a metric must be imposed on the underlying vector space.  For example, we can show that
\begin{equation*}
\begin{aligned}
B =
\lr{ 8 \sqrt{5} }
\frac{
\Bj \wedge \Bi + 2 \Bk \wedge \Bi + 3 \Bl \wedge \Bi + \Bk \wedge \Bj + 2 \Bl \wedge \Bj + \Bl \wedge \Bk
}
{
2 \sqrt{5}
},
\end{aligned}
\end{equation*}
so, assuming a Euclidean metric, the right bivector factor is a unit bivector (squares to $\pm 1$), and we can conclude that the area has magnitude \( \sqrt{320} = 8 \sqrt{5} \).
You can confirm that by computing the square of the numerator:
\begin{equation*}
\lr{ \Bj \wedge \Bi + 2 \Bk \wedge \Bi + 3 \Bl \wedge \Bi + \Bk \wedge \Bj + 2 \Bl \wedge \Bj + \Bl \wedge \Bk }^2
= -1 -4 -9 -1 -4 -1 = -20 = -\lr{ 2 \sqrt{5} }^2.
\end{equation*}

We can ask whether the \( \sqrt{320} \) value that we have computed makes sense.  The portion of the \( \Bb \) that is perpendicular to \( \Ba \) is
\begin{equation*}
\Bb_\perp = \Bb - \lr{ \Ba \cdot \Bb } \frac{\Ba}{\Ba \cdot \Ba}
=
52/27 \Bi + 8/9 \Bj -4/27 \Bk -32/27 \Bl.
\end{equation*}
Using this, we can compute the (squared) area, to find
\begin{equation*}
\Ba^2 \Bb_\perp^2 = 320,
\end{equation*}
as expected.

You also saw that the wedge product magnitude should have the form
\begin{equation*}
\Norm{\Ba}\Norm{\Bb} \sin\theta.
\end{equation*}
That is easy to show, first calculating
\begin{equation*}
\cos\theta = \frac{ \Ba . \Bb }{ \Norm{\Ba} \Norm{\Bb} },
\end{equation*}
after which we find
\begin{equation*}
\Norm{\Ba} \Norm{\Bb} \sin\theta = 8 \sqrt{5},
\end{equation*}
as seen above.

%}
\EndArticle
%\EndNoBibArticle
