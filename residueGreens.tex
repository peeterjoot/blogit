%
% Copyright � 2025 Peeter Joot.  All Rights Reserved.
% Licenced as described in the file LICENSE under the root directory of this GIT repository.
%
%{
\input{../latex/blogpost.tex}
\renewcommand{\basename}{residueGreens}
%\renewcommand{\dirname}{notes/phy1520/}
\renewcommand{\dirname}{notes/ece1228-electromagnetic-theory/}
%\newcommand{\dateintitle}{}
%\newcommand{\keywords}{}

\input{../latex/peeter_prologue_print2.tex}

\usepackage{peeters_layout_exercise}
\usepackage{peeters_braket}
\usepackage{peeters_figures}
\usepackage{siunitx}
\usepackage{verbatim}
%\usepackage{macros_cal} % \LL
%\usepackage{amsthm} % proof
%\usepackage{mhchem} % \ce{}
%\usepackage{macros_bm} % \bcM
%\usepackage{macros_qed} % \qedmarker
\usepackage{txfonts} % \ointclockwise

\beginArtNoToc

\generatetitle{A fun application of Green's functions: Residue calculus}
%\chapter{A fun application of Green's functions: Residue calculus}
%\label{chap:residueGreens}

\section{Motivation.}
A fun application of both Green's functions and geometric algebra is to show how the Cauchy integral equation can be expressed in terms of the Green's function for the 2D gradient.  This is covered, almost as an aside, in \citep{doran2003gap}.  I found that treatment a bit hard to understand, so I am going to work through it here at my own pace.
\subsection{Complex numbers in geometric algebra.}
Anybody who has studied geometric algebra is likely familiar with a variety of ways to construct complex numbers from geometric objects.  For example, complex numbers can be constructed for any plane.  If \( \Be_1, \Be_2 \) is a pair of orthonormal vectors for some plane in \R{N}, then any vector in that plane has the form
\begin{equation}\label{eqn:residueGreens:20}
\Bf = \Be_1 u + \Be_2 v,
\end{equation}
has an associated complex representation, by simply multiplying that vector one of those basis vectors.  For example, if we pre-multiply \( \Bf \) by \( \Be_1 \), forming
\begin{equation}\label{eqn:residueGreens:40}
\begin{aligned}
z
&= \Be_1 \Bf \\
&= \Be_1 \lr{ \Be_1 u + \Be_2 v } \\
&= u + \Be_1 \Be_2 v.
\end{aligned}
\end{equation}
We may now identify the unit bivector \( \Be_1 \Be_2 \) as an imaginary, designed by \( i \), since it has the expected behavior
\begin{equation}\label{eqn:residueGreens:60}
\begin{aligned}
i^2 &=
\lr{\Be_1 \Be_2}^2 \\
&=
\lr{\Be_1 \Be_2}
\lr{\Be_1 \Be_2} \\
&=
\Be_1 \lr{\Be_2
\Be_1} \Be_2 \\
&=
-\Be_1 \lr{\Be_1
\Be_2} \Be_2 \\
&=
-\lr{\Be_1 \Be_1}
\lr{\Be_2 \Be_2} \\
&=
-1.
\end{aligned}
\end{equation}
\subsection{Cauchy-equations in terms of the gradient.}
It is natural to wonder about the geometric algebra equivalents of various complex-number relationships and identities.  Of particular interest for this discussion is the geometric algebra equivalent of the Cauchy equations that specify required conditions for differentiability.

If a complex function \( f(z) = u(z) + i v(z) \) is differentiable, then we must be able to find the limit of
\begin{equation}\label{eqn:residueGreens:80}
\frac{\Delta f(z_0)}{\Delta z} = \frac{f(z_0 + h) - f(z_0)}{h},
\end{equation}
for any complex \( h \rightarrow 0 \), for any possible tradjectory of \( z_0 + h \) toward \( z_0 \).  In particular, for real \( h = \epsilon \),
\begin{equation}\label{eqn:residueGreens:100}
\lim_{\epsilon \rightarrow 0} \frac{u(x_0 + \epsilon, y_0) + i v(x_0 + \epsilon, y_0) - u(x_0, y_0) - i v(x_0, y_0)}{\epsilon}
=
\PD{x}{u(z_0)} + i \PD{x}{v(z_0)},
\end{equation}
and for imaginary \( h = i \epsilon \)
\begin{equation}\label{eqn:residueGreens:120}
\lim_{\epsilon \rightarrow 0} \frac{u(x_0, y_0 + \epsilon) + i v(x_0, y_0 + \epsilon) - u(x_0, y_0) - i v(x_0, y_0)}{i \epsilon}
=
-i\lr{ \PD{y}{u(z_0)} + i \PD{y}{v(z_0)} }.
\end{equation}
Equating real and imaginary parts, we see that existence of the derivative requires
\begin{equation}\label{eqn:residueGreens:140}
\begin{aligned}
\PD{x}{u} &= \PD{y}{v} \\
\PD{x}{v} &= -\PD{y}{u}.
\end{aligned}
\end{equation}
These are the Cauchy equations.  When the derivative exists in a given neighbourhood, we say that the function is analytic in that region.  If we use a bivector interpretation of the imaginary, with \( i = \Be_1 \Be_2 \), the Cauchy equations are also satisfied if the gradient of the complex function is zero, since
\begin{equation}\label{eqn:residueGreens:160}
\begin{aligned}
\spacegrad f
&=
\lr{ \Be_1 \partial_x + \Be_2 \partial_y } \lr{ u + \Be_1 \Be_2 v } \\
&=
\Be_1 \lr{ \partial_x u - \partial_y v } + \Be_2 \lr{ \partial_y u + \partial_x v }.
\end{aligned}
\end{equation}
For the gradient to equal zero, we see that \cref{eqn:residueGreens:140} must be satisfied.

%}
\EndArticle
%\EndNoBibArticle
