%
% Copyright � 2022 Peeter Joot.  All Rights Reserved.
% Licenced as described in the file LICENSE under the root directory of this GIT repository.
%
%{
\input{../latex/blogpost.tex}
\renewcommand{\basename}{dyadicVsGa}
%\renewcommand{\dirname}{notes/phy1520/}
\renewcommand{\dirname}{notes/ece1228-electromagnetic-theory/}
%\newcommand{\dateintitle}{}
%\newcommand{\keywords}{}

\input{../latex/peeter_prologue_print2.tex}

\usepackage{peeters_layout_exercise}
\usepackage{peeters_braket}
\usepackage{peeters_figures}
\usepackage{siunitx}
\usepackage{verbatim}
%\usepackage{mhchem} % \ce{}
%\usepackage{macros_bm} % \bcM
%\usepackage{macros_qed} % \qedmarker
%\usepackage{txfonts} % \ointclockwise

\beginArtNoToc

\generatetitle{Vector gradients in dyadic notation and geometric algebra.}
%\chapter{XXX}
%\label{chap:dyadicVsGa}

This is an exploration of the dyadic representation of the gradient acting on a vector in \R{3}, and an examination of how that representation corresponds with equivalent structures as expressed using geometric algebra (GA for short.)

\section{GA gradient of a vector.}
In GA we are free to express the product of the gradient and a vector field by adjacency.  In coordinates (summation over repeated indexes assumed), such a product has the form
\begin{equation}\label{eqn:dyadicVsGa:20}
\begin{aligned}
   \spacegrad \Bv
   &= \lr{ \Be_i \partial_i } \lr{ v_j \Be_j } \\
   &= \lr{ \partial_i v_j } \Be_i \Be_j.
\end{aligned}
\end{equation}
In this sum, any terms with \( i = j \) are scalars since \( \Be_i^2 = 1 \), and the remaining terms are bivectors.
This can be written compactly as
\begin{dmath}\label{eqn:dyadicVsGa:40}
   \spacegrad \Bv = \spacegrad \cdot \Bv + \spacegrad \wedge \Bv,
\end{dmath}
or for \R{3}
\begin{equation}\label{eqn:dyadicVsGa:60}
   \spacegrad \Bv = \spacegrad \cdot \Bv + I \lr{ \spacegrad \cross \Bv},
\end{equation}
either of which breaks the gradient into into divergence and curl components.  In \cref{eqn:dyadicVsGa:40} this vector gradient is expressed
using the bivector valued curl operator \( (\spacegrad \wedge \Bv) \), whereas
\cref{eqn:dyadicVsGa:60} is expressed using the vector valued dual form of the curl \( (\spacegrad \cross \Bv) \) from convential vector algebra.

It is worth noting that order matters in the GA coordinate expansion of \cref{eqn:dyadicVsGa:20}.  It is not correct to write
\begin{equation}\label{eqn:dyadicVsGa:80}
   \spacegrad \Bv
   = \lr{ \partial_i v_j } \Be_j \Be_i,
\end{equation}
which is only true when the curl, \( \spacegrad \wedge \Bv = 0 \), is zero.

\section{Dyadic representation.}
Given a vector field \( \Bv = \Bv(\Bx) \), the differential of that field can be computed by chain rule
\begin{equation}\label{eqn:dyadicVsGa:100}
   d\Bv = \PD{x_i}{\Bv} dx_i = \lr{ d\Bx \cdot \spacegrad} \Bv,
\end{equation}
where \( d\Bx = \Be_i dx_i \).  This is a representation invariant form of the differential, where we have a scalar operator \( d\Bx \cdot \spacegrad \) acting on the vector field \( \Bv \).  The matrix representation of this differential can be written as
\begin{equation}\label{eqn:dyadicVsGa:120}
   d\Bv = \lr{
      {\begin{bmatrix}
      d\Bx
\end{bmatrix}}^\dagger
\begin{bmatrix}
\spacegrad
\end{bmatrix}
}
\begin{bmatrix}
\Bv
\end{bmatrix}
,
\end{equation}
where
we are using the dagger to designate transposition, and each of the terms on the right are the coordinate matrixes of the vectors with respect to the standard basis
\begin{equation}\label{eqn:dyadicVsGa:140}
\begin{bmatrix}
   d\Bx
\end{bmatrix}
   =
\begin{bmatrix}
   dx_1 \\
   dx_2 \\
   dx_3
\end{bmatrix},\quad
\begin{bmatrix}
   \Bv
\end{bmatrix}
   =
\begin{bmatrix}
   v_1 \\
   v_2 \\
   v_3
\end{bmatrix},\quad % \mbox{and}
\begin{bmatrix}
   \spacegrad
\end{bmatrix}
   =
\begin{bmatrix}
   \partial_1 \\
   \partial_2 \\
   \partial_3
\end{bmatrix}.
\end{equation}

In \cref{eqn:dyadicVsGa:120} the parens are very important, as the expression is meaningless without them.  With the parens we have a \((1 \times 3)(3 \times 1)\) matrix (i.e. a scalar) multiplied with a \(3\times 1\) matrix.  That becomes ill-formed if we drop the parens since we are left with an incompatible product of a \((3\times1)(3\times1)\) matrix on the right.  The dyadic notation, which introducing a tensor product into the mix, is a mechanism to make sense of the possibility of such a product.  Can we make sense of an expression like \( \spacegrad \Bv \) without the geometric product in our toolbox?

Stepping towards that question, let's examine the coordinate expansion of our vector differential \cref{eqn:dyadicVsGa:100}, which is
\begin{equation}\label{eqn:dyadicVsGa:160}
d\Bv = dx_i \lr{ \partial_i v_j } \Be_j.
\end{equation}
If we allow a matrix of vectors, this has a block matrix form
\begin{equation}\label{eqn:dyadicVsGa:180}
   d\Bv =
   {\begin{bmatrix}
   d\Bx
\end{bmatrix}}^\dagger
\begin{bmatrix}
\spacegrad \directproduct \Bv
\end{bmatrix}
\begin{bmatrix}
   \Be_1 \\
   \Be_2 \\
   \Be_3
\end{bmatrix}
.
\end{equation}
Here we introduce the tensor product
\begin{dmath}\label{eqn:dyadicVsGa:200}
\spacegrad \directproduct \Bv
= \partial_i v_j \, \Be_i \directproduct \Be_j,
\end{dmath}
and designate the matrix of coordinates \( \partial_i v_j \), a second order tensor, by \(
\begin{bmatrix}
\spacegrad \directproduct \Bv
\end{bmatrix}
\).

We have succeeded in factoring out a vector gradient.  We can introduce dot product between vectors and a direct product of vectors, by observing that
\cref{eqn:dyadicVsGa:180} has the structure of a quadradic form, and define
\begin{equation}\label{eqn:dyadicVsGa:220}
   \Bx \cdot (\Ba \directproduct \Bb) \equiv
   {\begin{bmatrix}
   \Bx
\end{bmatrix}}^\dagger
\begin{bmatrix}
   \Ba \directproduct \Bb
\end{bmatrix}
\begin{bmatrix}
   \Be_1 \\
   \Be_2 \\
   \Be_3
\end{bmatrix},
\end{equation}
so that \cref{eqn:dyadicVsGa:180} takes the form
\begin{equation}\label{eqn:dyadicVsGa:240}
   d\Bv = d\Bx \cdot \lr{ \spacegrad \directproduct \Bv }.
\end{equation}
Such a dot product gives operational meaning to the gradient-vector tensor product.

\section{Symmetrization and antisymmetrization of the vector differential in GA.}

%}
%\EndArticle
\EndNoBibArticle
