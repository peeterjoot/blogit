%{ % Ignore this.  It's a trick for brace matching in an editor called vim that Peeter uses.
\documentclass[letterpaper]{scrartcl}
\usepackage[margin=1in,top=0.5in]{geometry}
\usepackage{amsmath}
\usepackage{caption}
\usepackage{graphicx}
\usepackage{siunitx}
\usepackage{wrapfig}
\usepackage[english]{cleveref}

% for a centered figure with caption (not used here)
% \simpleFigure{path}{caption}{label}{width}
\newcommand{\simpleFigure}[4]{%
  \begin{figure}[htp]%
    \centering%
    \includegraphics[width=#4\textwidth]{#1}%
    \caption{#2}%
    \label{#3}%
  \end{figure}%
}

% for a figure off to the left or to the right (default is left), with a caption and label
% \sideFigure{path}{caption}{label}{width}[side]
\newcommand{\sideFigure}[5][l]{%
  \begin{wrapfigure}{#1}{#5\textwidth}%
    \centering%
    \includegraphics[width=#5\textwidth]{#2}%
    \caption{#3}%
    \label{#4}%
  \end{wrapfigure}%
}
% for manually adjusting spaces before or after \sideFigure
\newcommand{\figspace}{\vspace{1.5\baselineskip}}
\newcommand{\sectionspace}{\vspace{4\baselineskip}}

% This command is a shortcut meant to do a balanced set of ( ) braces (using \left and \right makes them resize themselves bigger automatically)
\newcommand{\lr}[1]{\left(#1\right)}

% This command is a shortcut:
% \inv{x} = 1/x.
\newcommand{\inv}[1]{\frac{1}{#1}}

\DeclareMathOperator{\Real}{Re}
\DeclareMathOperator{\Arg}{Arg}
\DeclareMathOperator{\Imag}{Im}
\DeclareMathOperator{\conj}{conj}
\DeclareMathOperator{\atan2}{atan2}

\newcommand{\phasor}[1]{\mathbf{#1}}
\newcommand{\thev}[0]{\mathrm{Th}}
\newcommand{\load}[0]{\mathrm{Load}}

\begin{document}
\section*{Circuits Cheatsheet}
\subsection*{Current Division}
\begin{minipage}{\textwidth}
For a DC current source, such as the one illustrated in \cref{fig:currentSource:currentSourceFig1}, we have
\end{minipage}
\sideFigure{../figures/blogit/currentSourceFig1}{Current Source}{fig:currentSource:currentSourceFig1}{0.3}
\figspace
\begin{equation}\label{eqn:karlCircuitsCheatSheet:20}
I_1 = I_s \frac{R_2}{R_1 + R_2}
\end{equation}
\begin{equation}\label{eqn:karlCircuitsCheatSheet:40}
I_2 = I_s \frac{R_1}{R_1 + R_2}.
\end{equation}
\WFclear
\sectionspace
\subsection*{Voltage Division}
\begin{minipage}{\textwidth}
For a DC voltage source illustrated in \cref{fig:voltageSource:voltageSourceFig2}, we have
\end{minipage}
\sideFigure{../figures/blogit/voltageSourceFig2}{Voltage Source}{fig:voltageSource:voltageSourceFig2}{0.3}
\figspace
\begin{equation}\label{eqn:karlCircuitsCheatSheet:60}
V = V_2 \frac{R_2}{R_1 + R_2}.
\end{equation}
\WFclear
\figspace
\sectionspace
\subsection*{Cramer's Rule}
The solution of
\begin{equation}\label{eqn:karlCircuitsCheatSheet:380}
\begin{bmatrix}
a & b \\
c & d
\end{bmatrix}
\begin{bmatrix}
x \\
y
\end{bmatrix}
=
\begin{bmatrix}
e \\
f
\end{bmatrix},
\end{equation}
is given by
\begin{equation}\label{eqn:karlCircuitsCheatSheet:100}
x = \frac{
   \begin{vmatrix}
   e & b \\
   f & d
   \end{vmatrix}
}{
   \begin{vmatrix}
   a & b \\
   c & d
   \end{vmatrix}
},
\qquad
y = \frac{
   \begin{vmatrix}
   a & e \\
   c & f
   \end{vmatrix}
}{
   \begin{vmatrix}
   a & b \\
   c & d
   \end{vmatrix}
}.
\end{equation}
\subsection*{RLC Circuit equations}
\begin{equation}\label{eqn:karlCircuitsCheatSheet:120}
v_L = L \frac{di_L}{dt}
\end{equation}
\begin{equation}\label{eqn:karlCircuitsCheatSheet:140}
i_C = C \frac{dv_C}{dt}.
\end{equation}

FIXME: add figures.

\subsection*{Complex numbers}
The rectangular form of a complex number (or phasor) is
\begin{equation}\label{eqn:karlCircuitsCheatSheet:160}
z = a + j b,
\end{equation}
where \( a \) is the real part, and \( j b \) is the imaginary part.

The polar form of a complex number (i.e.: of a phasor) is
\begin{equation}\label{eqn:karlCircuitsCheatSheet:180}
z = r \angle \phi,
\end{equation}
where \( r = \sqrt{a^2 + b^2} \), and \( \phi = \Arg z = \atan2\lr{\frac{b}{a}} \), where
\begin{equation*}
\atan2\lr{b, a} = \tan^{-1}\lr{ \frac{b}{a} },
\end{equation*}
but also correctly accounts for the quadrant that the point \(\lr{a,b}\) is in (since arctan needs correction by \( \pm \pi \) in some quadrants.)

The exponential form of a complex number is
\begin{equation}\label{eqn:karlCircuitsCheatSheet:200}
z = r e^{j\phi} = r \lr{ \cos\phi + j \sin\phi }.
\end{equation}
This can be related to the rectangular form by noting that \( a = r \cos\phi \) and \( b = r \sin \phi \).

FIXME: add a figure.

Observe that
\begin{equation}\label{eqn:karlCircuitsCheatSheet:220}
\begin{aligned}
\sqrt{z} &= \sqrt{r} e^{j\phi/2} = \sqrt{r} \angle \frac{\phi}{2} \\
\inv{z} &= \inv{r} e^{-j\phi} = \inv{r} \angle -\phi \\
\conj{z} &= a - j b = r e^{-j\phi} = r \angle -\phi.
\end{aligned}
\end{equation}

Given two complex numbers \( z_1 = r_1 e^{j\phi_1}, z_2 = r_2 e^{j\phi_2} \), then their products and ratios are
\begin{equation}\label{eqn:karlCircuitsCheatSheet:240}
\begin{aligned}
z_1 z_2 &= r_1 r_2 e^{j \lr{\phi_1 + \phi_2 } } = r_1 r_2 \angle \lr{ \phi_1 + \phi_2 } \\
\frac{z_1}{z_2} &= \frac{r_1}{r_2} e^{j \lr{\phi_1 - \phi_2 } } = \frac{r_1}{r_2} \angle \lr{ \phi_1 - \phi_2 }.
\end{aligned}
\end{equation}

Let's applying the phasor notation to voltages and currents.  For example, given a time dependent cosinusoidal voltage with constant magnitude \( V \), we have the following equivalences

Phasors simplify AC circuit analysis by converting differential equations to algebraic ones in the frequency domain.  For a time-dependent cosinusoidal voltage with constant magnitude \( V \), we have
\begin{equation}\label{eqn:karlCircuitsCheatSheet:260}
\begin{aligned}
v(t) &= V \cos\lr{ \omega t + \phi } = \Real \lr{ V e^{j \lr{\omega t + \phi} } } \\
\phasor{V} &= V e^{j \phi} = V \angle \phi = V e^{j \phi} \\
v(t) &= \Real \lr{ \phasor{V} e^{j \omega t} },
\end{aligned}
\end{equation}
where the phasor, assumed to be a sinusoidal signal at a fixed angular frequency \( \omega \), has been written \( \phasor{V} \).  Bold has been used here for the phasor, which works nice in print, but there are other common conventions, including \(\underline{\phasor{V}}\) or \(\overline{\phasor{V}}\).

Similarly, given a sinusoidal time dependent voltage
\begin{equation}\label{eqn:karlCircuitsCheatSheet:320}
\begin{aligned}
v(t) &= V \sin\lr{ \omega t + \phi } = \Real \lr{ -j V e^{j\lr{ \omega t + \phi } } } \\
\phasor{V} &= V \angle \lr{ \phi - \SI{90}{\degree}} = V e^{j \lr{ \phi - \SI{90}{\degree}} },
\end{aligned}
\end{equation}
since
\begin{equation}\label{eqn:karlCircuitsCheatSheet:n}
-j = e^{-j \pi/2} = \angle \SI{-90}{\degree}.
\end{equation}
Once again \( v(t) = \Real \lr{ \phasor{V} e^{j \omega t} } \).

\subsection*{Impedance of RLC Elements}
For a resistor, inductor, and capacitor in the frequency domain:
\begin{equation}
\begin{aligned}
Z_R &= R \\
Z_L &= j \omega L \\
Z_C &= \frac{1}{j \omega C} = -j \frac{1}{\omega C}.
\end{aligned}
\end{equation}

\subsection*{Kirchhoff's Laws}
\begin{itemize}
\item KCL: The sum of currents entering a node equals the sum of currents leaving: \(\sum I_{\text{in}} = \sum I_{\text{out}}\).
\item KVL: The sum of voltage drops around a closed loop is zero: \(\sum V = 0\).
\end{itemize}

\subsection*{Impedance and Maximum Power Transfer}
The Thevenin equivalent impedance of a circuit is
\begin{equation}\label{eqn:karlCircuitsCheatSheet:340}
Z_\thev = R_\thev + j X_\thev,
\end{equation}
where \( R_\thev \) is the resistance and \( X_\thev \) is the reactance (positive for inductive, negative for capacitive). For maximum power transfer to a load, the load impedance should be the conjugate:
\begin{equation}\label{eqn:karlCircuitsCheatSheet:360}
Z_\load = \conj{Z_\thev} = R_\thev - j X_\thev.
\end{equation}

\subsection*{Supermesh and Supernodes}
Use supermesh analysis for circuits with current sources between nodes, and supernode analysis for circuits with voltage sources between nodes.

FIXME: add a brief description or example of how to apply supermesh/supernode techniques, as they’re key for solving complex circuits.

\end{document}
%} % Ignore this.  It's a trick for brace matching in an editor called vim that Peeter uses.
