%
% Copyright � 2023 Peeter Joot.  All Rights Reserved.
% Licenced as described in the file LICENSE under the root directory of this GIT repository.
%
%{
\input{../latex/blogpost.tex}
\renewcommand{\basename}{formAndCurl}
%\renewcommand{\dirname}{notes/phy1520/}
\renewcommand{\dirname}{notes/ece1228-electromagnetic-theory/}
%\newcommand{\dateintitle}{}
%\newcommand{\keywords}{}

\input{../latex/peeter_prologue_print2.tex}

\usepackage{amsthm}
\usepackage{peeters_layout_exercise}
\usepackage{peeters_braket}
\usepackage{peeters_figures}
\usepackage{siunitx}
\usepackage{peeters_tablebox}
%\usepackage{macros_qed}
%\usepackage{xcolor}

%\usepackage{mhchem} % \ce{}
%\usepackage{macros_bm} % \bcM
%\usepackage{macros_qed} % \qedmarker
%\usepackage{txfonts} % \ointclockwise

%\newcommand{\makedefinition}[3]{%
%\begin{definition}[#1]\label{#2}%
%#3%
%\end{definition}%
%}

\beginArtNoToc

\generatetitle{Geometric calculus relationships to differential forms, and vector calculus identities.}
%\chapter{Geometric calculus equivalents for some vector calculus identities.}
%\label{chap:formAndCurl}

\section{Motivation.}
I was asked about the geometric algebra equivalents of some of the vector calculus identities from \citep{vboucharMATH215}.  I'll call the specific page of those calculus notes ``the article''.  The article includes identities like
\begin{equation}\label{eqn:formAndCurl:20}
\begin{aligned}
\spacegrad (f g) &= f \spacegrad g + g \spacegrad f \\
\spacegrad \cross (f \BF) &= f \spacegrad \cross \BF + (\spacegrad f) \cross \BF \\
\spacegrad \cdot (f \BF) &= f \spacegrad \cdot \BF + (\spacegrad f) \cdot \BF \\
\spacegrad \cdot \lr{ \BF \cross \BG } &= \BG \cdot \lr{ \spacegrad \cross \BF } - \BF \cdot \lr{ \spacegrad \cross \BG },
\end{aligned}
\end{equation}
but the point of these particular lecture notes is the interface between traditional Gibbs vector calculus and differential forms. That's a much bigger topic, and perhaps not what I was actually being asked about.  It is, however, an interesting topic, so let's explore it.

\section{Comparisons.}
The article identifies the cross product representation of the curl \( \spacegrad \cross \BF \) as the equivalent to the exterior derivative of a one form (which has been mapped to a vector function.) In geometric algebra, this isn't the identification we would use.  Instead we should identify the ``bivector curl'' \( \spacegrad \wedge \BF \) as the logical equivalent of the exterior derivative of that one form, and in general identify \( \spacegrad \wedge A_k \) as the exterior derivative of a k-form (k-blade).  In my notes to follow, the wedge of the gradient with a function, will be called the curl of that function, even if we are operating in \R{3} where the cross product is defined.

The starting place of the article was to define a one form and it's exterior derivative was essentially as follows
\makedefinition{The exterior derivative of a one form.}{dfn:formAndCurl:40}{
Let \( f : \mathbb{R}^N \rightarrow \mathbb{R} \) be a zero form.  It's exterior derivative is
\begin{equation*}
df = \sum_i dx_i \PD{x_i}{f}.
\end{equation*}
} % definition

I've stated that the GA equivalent of the exterior derivative was a curl \( \spacegrad \wedge A \), and this doesn't look anything curl like, so right away, we have some trouble to deal with.  To resolve that trouble, let's step back to the gradient, which we haven't defined yet.  In the article, the gradient (of a scalar function) was defined as a coordinate triplet
\begin{equation}\label{eqn:formAndCurl:60}
\spacegrad \Bf = \lr{ \PD{x}{f}, \PD{y}{f}, \PD{z}{f} }.
\end{equation}
In GA we don't like representations where the basis vectors are implicit, so we'd prefer to define
\makedefinition{The gradient of a function.}{dfn:formAndCurl:80}{
We define the gradient of multivector \( f(x_1, x_2, \cdots, x_N) \), and denote it by \( \spacegrad f \)
\begin{equation*}
\spacegrad f = \sum_{i=1}^N \Be_i \PD{x_i}{f},
\end{equation*}
where \( \setlr{ \Be_1, \cdots \Be_N } \) is an orthonormal basis for \R{N}.
} % definition
Unlike the article, we do not restrict \( f \) to be a scalar function, since we do not have a problem with a vector valued operator directly multiplying a vector or any product of vectors.  Instead \( f \) can be a multivector function, with scalar, vector, bivector, trivector, ... components, and we define the gradient the same way.

In order to define the curl of a k-blade, we need a reminder of how we define the wedge of a vector with a k-blade.  Recall that this is how we generally define the wedge between two blades.
\makedefinition{}{dfn:formAndCurl:100}{
Let \( A_r \) be a r-blade, and \( B_s \) a s-blade.  The wedge of \( A_r \) with \( B_s \) is
\begin{equation}\label{eqn:formAndCurl:120}
A_r \wedge B_s = \gpgrade{A_r B_s}{r+s}.
\end{equation}
} % definition
In particular, if \( \Ba \) is a vector, then the wedge with an s-blade \( B_s \) is
\begin{equation}\label{eqn:formAndCurl:140}
\Ba \wedge B_s = \gpgrade{\Ba B_s}{s+1},
\end{equation}
which is just the \( s+1 \) grade selection of their product.  Furthermore, if \( f \) is a scalar, then
\begin{equation}\label{eqn:formAndCurl:160}
\Ba \wedge f = \gpgrade{\Ba f}{1} = \Ba f.
\end{equation}
We can now state the curl of a k-blade
\makedefinition{Curl of a k-blade.}{dfn:formAndCurl:180}{
Let \( A_k \) be a k-blade.  We define the curl of a k-blade as the wedge product of the gradient with that k-blade, designated
\begin{equation*}
\spacegrad \wedge A_k.
\end{equation*}
} % definition
Observe, given our generalized wedge product definition above, that the curl of a scalar function \( f \), is in fact just the gradient of that function
\begin{equation}\label{eqn:formAndCurl:200}
\spacegrad \wedge f = \spacegrad f = \sum_i \Be_i \PD{x_i}{f}.
\end{equation}
This has exactly the structure of the exterior derivative of a one form, as stated in \cref{dfn:formAndCurl:40}, but we have replaced \( dx_i \) with a basis vector \( \Be_i \).

\makedefinition{Exterior derivative of a one-form.}{dfn:formAndCurl:820}{
Let \( \omega = f_i dx_i \) be a one-form.  The exterior derivative of \( d \omega \) is
\begin{equation*}
d\omega = \sum_i d( f_i ) \wedge dx_i.
\end{equation*}
} % definition
\makelemma{Exterior derivative of a one-form.}{lemma:formAndCurl:840}{
Let \( \omega = f_i dx_i \) be a one-form.  The exterior derivative \( d \omega \) can be expanding into a Jacobian form
\begin{equation*}
d\omega
=
\sum_{i < j} \lr{
\PD{x_i}{f_j}
-
\PD{x_j}{f_i}
} dx_i \wedge dx_j.
\end{equation*}
} % lemma
\begin{proof}
\begin{equation}\label{eqn:formAndCurl:220}
\begin{aligned}
d\omega
&= \sum_j d( f_j dx_j ) \\
&= \sum_j d( f_j ) \wedge dx_j \\
&= \sum_j \lr{ \sum_i dx_i \PD{x_i}{f_j} } \wedge dx_j \\
&= \sum_{ji} \PD{x_i}{f_j} dx_i \wedge dx_j \\
&= \sum_{j \ne i} \PD{x_i}{f_j} dx_i \wedge dx_j \\
&=
\sum_{i < j} \PD{x_i}{f_j} dx_i \wedge dx_j
+
\sum_{j < i} \PD{x_i}{f_j} dx_i \wedge dx_j \\
&=
\sum_{i < j} \PD{x_i}{f_j} dx_i \wedge dx_j
+
\sum_{i < j} \PD{x_j}{f_i} dx_j \wedge dx_i \\
&=
\sum_{i < j} \lr{
\PD{x_i}{f_j}
-
\PD{x_j}{f_i}
} dx_i \wedge dx_j.
\end{aligned}
\end{equation}
\end{proof}
\makelemma{Curl of a vector.}{lemma:formAndCurl:850}{
Let \( \Bf = \sum_i \Be_i f_i \in \mathbb{R}^N \) be a vector.  The curl of \( \Bf \) has a Jacobian structure
\begin{equation*}
\spacegrad \wedge \Bf
=
\sum_{i < j}
\lr{ \PD{x_i}{f_j} - \PD{x_j}{f_i} }
\lr{ \Be_i \wedge \Be_j }
.
\end{equation*}
} % lemma
\begin{proof}
The antisymmetry of the wedges of differentials in the exterior derivative and the curl clearly has a one to one correspondence.  Let's show this explicitly by expansion
\begin{equation}\label{eqn:formAndCurl:240}
\begin{aligned}
\spacegrad \wedge \Bf
&=
\sum_{ij} \lr{ \Be_i \PD{x_i}{} } \wedge \lr{ \Be_j f_j } \\
&=
\sum_{ij} \lr{ \Be_i \wedge \Be_j } \PD{x_i}{f_j} \\
&=
\sum_{i \ne j} \lr{ \Be_i \wedge \Be_j } \PD{x_i}{f_j} \\
&=
\sum_{i < j} \lr{ \Be_i \wedge \Be_j } \PD{x_i}{f_j}
+
\sum_{j < i} \lr{ \Be_i \wedge \Be_j } \PD{x_i}{f_j} \\
&=
\sum_{i < j} \lr{ \Be_i \wedge \Be_j } \PD{x_i}{f_j}
+
\sum_{i < j} \lr{ \Be_j \wedge \Be_i } \PD{x_j}{f_i} \\
&=
\sum_{i < j} \lr{ \Be_i \wedge \Be_j } \lr{ \PD{x_i}{f_j} - \PD{x_j}{f_i} }.
\end{aligned}
\end{equation}
\end{proof}
If we are translating from differential forms, again, we see that we simply replace any differentials \( dx_i \) with the basis vectors \( \Be_i \) (at least for the zero-form and one-form cases, which is all that we have looked at here.)

Note that in differential forms, we often assume that there is an implicit wedge product between any different one form elements, writing
\begin{equation}\label{eqn:formAndCurl:260}
dx_1 \wedge dx_2 = dx_1 dx_2.
\end{equation}
This works out fine when we map differentials to basis vectors, since
\begin{equation}\label{eqn:formAndCurl:280}
\Be_1 \Be_2 =
\Be_1 \cdot \Be_2
+
\Be_1 \wedge \Be_2
=
\Be_1 \wedge \Be_2.
\end{equation}
However, we have to be more careful in GA when using indexed expressions, since
\begin{equation}\label{eqn:formAndCurl:300}
\Be_i \Be_j = \Be_i \cdot \Be_j + \Be_i \wedge \Be_j.
\end{equation}
The dot product portion of the RHS is only zero if \( i \ne j \).

Now let's look at the equivalence between the exterior derivative of a two-form with the curl.
\makedefinition{Exterior derivative of a two-form.}{dfn:formAndCurl:840}{
Let \( \eta = \sum_{ij} f_{ij} dx_i \wedge dx_j \) be a two-form.  The exterior derivative of \( \eta \) is
\begin{equation*}
d\eta =
\sum_{ij} d( f_{ij} ) \wedge dx_i \wedge dx_j.
\end{equation*}
} % definition
\makelemma{Exterior derivative of a two-form.}{lemma:formAndCurl:860}{
Let \( \eta = \sum_{ij} f_{ij} dx_i \wedge dx_j \) be a two-form.  The exterior derivative of \( \eta \) can be expanded as
\begin{equation*}
d \eta
=
\sum_{i,j,k} \PD{x_k}{f_{ij}} dx_i \wedge dx_j \wedge dx_k.
\end{equation*}
} % lemma
\begin{proof}
The exterior derivative of \( \eta \) is
\begin{equation}\label{eqn:formAndCurl:340}
\begin{aligned}
d \eta
&=
\sum_{i,j} d( f_{ij} dx_i \wedge dx_j ) \\
&=
\sum_{i,j,k} \lr{ \PD{x_k}{f_{ij}} dx_k } \wedge dx_i \wedge dx_j \\
&=
\sum_{i,j,k} \PD{x_k}{f_{ij}} dx_i \wedge dx_j \wedge dx_k.
\end{aligned}
\end{equation}
\end{proof}
Let's compare that to the curl of a bivector valued function.
\makelemma{Curl of a 2-blade.}{lemma:formAndCurl:880}{
Let \( B = \sum_{i \ne j} f_{ij} \Be_i \wedge \Be_j \) be a 2-blade.  The curl of \( B \) is
\begin{equation*}
\spacegrad \wedge B
=
\sum_{i,j,k} \PD{x_k}{f_{ij}} \Be_i \wedge \Be_j \wedge \Be_k.
\end{equation*}
} % lemma
\begin{proof}
\begin{equation}\label{eqn:formAndCurl:380}
\begin{aligned}
\spacegrad \wedge B
&=
\lr{ \sum_k \Be_k \PD{x_k}{} } \wedge \lr{ \sum_{i \ne j} f_{ij} \Be_i \wedge \Be_j } \\
&=
\sum_{k, i \ne j} \PD{x_k}{f_{ij}} \Be_k \wedge \Be_i \wedge \Be_j \\
&=
\sum_{i,j,k} \PD{x_k}{f_{ij}} \Be_i \wedge \Be_j \wedge \Be_k.
\end{aligned}
\end{equation}
\end{proof}
Again, we see an exact correspondence with the exterior derivative \( d \eta \) of a two-form, and the curl \( \spacegrad \wedge B \), of a 2-blade.

The article established a coorespondence between the exterior derivative of a two form over \R{3} to the divergence.  The way we would express this in GA (also for \R{3}) is to write
\begin{equation}\label{eqn:formAndCurl:400}
B = I \Bb,
\end{equation}
where \( I = \Be_1 \Be_2 \Be_3 \) is the \R{3} pseudoscalar (a ``unit'' trivector.)  Forming the curl of \( B \) we have
\begin{equation}\label{eqn:formAndCurl:420}
\begin{aligned}
\spacegrad \wedge B
&= \gpgradethree{ \spacegrad B } \\
&= \gpgradethree{ \spacegrad I \Bb } \\
&= \gpgradethree{ I (\spacegrad \Bb) } \\
&= \gpgradethree{ I (\spacegrad \cdot \Bb + \spacegrad \wedge \Bb) } \\
&= I (\spacegrad \cdot \Bb).
\end{aligned}
\end{equation}
The equivalence relationships that we have developed must then imply that the differential forms representation of this relationship is
\begin{equation}\label{eqn:formAndCurl:440}
d B = dx_1 \wedge dx_2 \wedge dx_3 (\spacegrad \cdot \Bb)
    = dx \wedge dy \wedge dz \lr{ \PD{x}{b_1} + \PD{y}{b_2} + \PD{z}{b_3} },
\end{equation}
as defined in the article.

Here is the GA equivalent of Lemma 4.4.10 from the article
\makelemma{Repeated curl identities.}{lemma:formAndCurl:460}{
Let \( A \) be a smooth k-blade, then
\begin{equation*}
\spacegrad \wedge \lr{ \spacegrad \wedge A } = 0.
\end{equation*}
For \R{3}, this result, for a scalar function \( f \), and a vector function \( \Bf \), in terms of the cross product, as
\begin{equation}\label{eqn:formAndCurl:560}
\begin{aligned}
\spacegrad \cross \lr{ \spacegrad f } &= 0 \\
\spacegrad \cdot \lr{ \spacegrad \cross \Bf } &= 0.
\end{aligned}
\end{equation}
} % lemma
It shouldn't be surprising that this is the equivalent of \( d^2 A = 0 \) from differential forms.  Let's prove this, first considering the 0-blade case
\begin{proof}
\begin{equation}\label{eqn:formAndCurl:480}
\begin{aligned}
\spacegrad \wedge \lr{ \spacegrad \wedge A }
&=
\spacegrad \wedge \lr{ \spacegrad A } \\
&=
\sum_{ij} \Be_i \wedge \Be_j \frac{\partial^2 A}{\partial x_i \partial x_j} \\
&= 0.
\end{aligned}
\end{equation}
The smooth criteria of for the function \( A \) is assumed to imply that we have equality of mixed partials, and since this is a sum of an antisymmetric term with respect to indexes \( i, j \) (the wedge) and a symmetric term in indexes \( i, j \) (the partials), we have zero overall.

Now consider a k-blade \( A, k > 0 \).  Expanding the gradients, we have
\begin{equation}\label{eqn:formAndCurl:500}
\spacegrad \wedge \lr{ \spacegrad \wedge A }
=
\sum_{ij} \Be_i \wedge \Be_j \wedge \frac{\partial^2 A}{\partial x_i \partial x_j}.
\end{equation}
It may be obvious that this is zero for the same reasons as above (sum of product of symmetric and antisymmetric entities).  We can, however, make it more obvious, at the cost of some hellish indexing, by expressing \( A \) in coordinate form.  Let
\begin{equation}\label{eqn:formAndCurl:520}
A = \sum_{i_1, i_2, \cdots, i_k}
A_{i_1, i_2, \cdots, i_k} \Be_{i_1} \wedge \Be_{i_2} \wedge \cdots \wedge \Be_{i_k},
\end{equation}
then
\begin{equation}\label{eqn:formAndCurl:540}
\begin{aligned}
\spacegrad \wedge \lr{ \spacegrad \wedge A }
&=
\sum_{i,j,i_1, i_2, \cdots, i_k} \Be_i \wedge \Be_j \wedge \Be_{i_1} \wedge \Be_{i_2} \wedge \cdots \wedge \Be_{i_k}
\frac{\partial^2 }{\partial x_i \partial x_j}  A_{i_1, i_2, \cdots, i_k}  \\
&= 0.
\end{aligned}
\end{equation}
Now we clearly have a sum of an antisymmetric term (the wedges), and a symmetric term (assuming smooth \( A \) means that we have equality of mixed partials), so the sum is zero.

Finally, for the \R{3} identities, we have
\begin{equation}\label{eqn:formAndCurl:580}
\begin{aligned}
\spacegrad \cross \lr{ \spacegrad f}
&=
-I \lr{ \spacegrad \wedge \lr{ \spacegrad f } }
&=
0,
\end{aligned}
\end{equation}
since \( \spacegrad \wedge \lr{ \spacegrad f } = 0 \).  For a vector \( \Bf \), we have
\begin{equation}\label{eqn:formAndCurl:600}
\begin{aligned}
\spacegrad \cdot \lr{ \spacegrad \cross \Bf}
&=
\gpgradezero{
\spacegrad \lr{ \spacegrad \cross \Bf}
} \\
&=
\gpgradezero{
\spacegrad (-I) \lr{ \spacegrad \wedge \Bf}
} \\
&=
-\gpgradezero{
I \spacegrad \lr{ \spacegrad \wedge \Bf}
} \\
&=
-
I \spacegrad \wedge \lr{ \spacegrad \wedge \Bf} \\
&= 0,
\end{aligned}
\end{equation}
again, because \( \spacegrad \wedge \lr{ \spacegrad \wedge \Bf} = 0 \).
\end{proof}

\section{Identities.}
We have a number of chain rule identities in the article.  Here is the GA equivalent of that, and its corollaries
\makelemma{Chain rule identities.}{lemma:formAndCurl:620}{
Let \( f \) be a scalar function and \( A \) be a k-blade, then
\begin{equation*}
\spacegrad \lr{ f A } = \lr{ \spacegrad f } A + f \lr{ \spacegrad A }.
\end{equation*}
For \( A \) with grade \( k > 0 \), the grade \( k-1 \) and \( k+1 \) components of this product are
\begin{equation*}
\begin{aligned}
\spacegrad \cdot \lr{ f A } &= \lr{ \spacegrad f } \cdot A + f \lr{ \spacegrad \cdot A } \\
\spacegrad \wedge \lr{ f A } &= \lr{ \spacegrad f } \wedge A + f \lr{ \spacegrad \wedge A }.
\end{aligned}
\end{equation*}
For \R{3}, the wedge product relation above can be written in dual form as
\begin{equation*}
\spacegrad \cross \lr{ f A } = \lr{ \spacegrad f } \cross A + f \lr{ \spacegrad \cross A }.
\end{equation*}
} % lemma
Proving this is left to the reader.

We have some chain rule identities left in the article to verify and to find GA equivalents of.  Before doing so, we need a couple miscellaneous identities relating triple cross products to wedge-dots.
\makelemma{Triple cross products.}{lemma:formAndCurl:700}{
Let \( \Ba, \Bb, \Bc \) be vectors in \R{3}.  Then
\begin{equation*}
\begin{aligned}
\Ba \cross \lr{ \Bb \cross \Bc } &= - \Ba \cdot \lr{ \Bb \wedge \Bc } \\
\lr{ \Ba \cross \Bb } \cross \Bc &= - \lr{ \Ba \wedge \Bb } \cdot \Bc.
\end{aligned}
\end{equation*}
} % lemma
\begin{proof}
\begin{equation}\label{eqn:formAndCurl:720}
\begin{aligned}
\Ba \cross \lr{ \Bb \cross \Bc }
&=
\gpgradeone{ -I \lr{ \Ba \wedge \lr{ \Bb \cross \Bc } } } \\
&=
\gpgradeone{ -I \lr{ \Ba \lr{ \Bb \cross \Bc } } } \\
&=
\gpgradeone{ (-I)^2 \lr{ \Ba \lr{ \Bb \wedge \Bc } } } \\
&=
-\Ba \cdot \lr{ \Bb \wedge \Bc },
\end{aligned}
\end{equation}
\begin{equation}\label{eqn:formAndCurl:740}
\begin{aligned}
\lr{ \Ba \cross \Bb } \cross \Bc
&=
\gpgradeone{ -I \lr{ \Ba \cross \Bb } \wedge \Bc } \\
&=
\gpgradeone{ -I \lr{ \Ba \cross \Bb } \Bc } \\
&=
\gpgradeone{ (-I)^2 \lr{ \Ba \wedge \Bb } \Bc } \\
&=
- \lr{ \Ba \wedge \Bb } \cdot \Bc.
\end{aligned}
\end{equation}
\end{proof}

Next up is another chain rule identity
\makelemma{Gradient of dot product.}{lemma:formAndCurl:640}{
If \( \Ba, \Bb \) are vectors, then
\begin{equation*}
\spacegrad \lr{ \Ba \cdot \Bb } =
\lr{ \Ba \cdot \spacegrad } \Bb
+
\lr{ \Bb \cdot \spacegrad } \Ba
+
\lr{ \spacegrad \wedge \Bb }
\cdot
\Ba
+
\lr{ \spacegrad \wedge \Ba }
\cdot
\Bb
\end{equation*}
For \R{3}, this can be written as
\begin{equation*}
\spacegrad \lr{ \Ba \cdot \Bb }
=
\lr{ \Ba \cdot \spacegrad } \Bb
+
\lr{ \Bb \cdot \spacegrad } \Ba
+
\Ba \cross \lr{ \spacegrad \cross \Bb }
+
\Bb \cross \lr{ \spacegrad \cross \Ba }
\end{equation*}
} % lemma
\begin{proof}
We will use \( \rspacegrad \) to indicate that the gradient operates on everything to the right, \( \lrspacegrad \) to indicate that the gradient operates bidirectionally, and \( \spacegrad' A B' \) to indicate that the gradient's scope is limited to the ticked entity (just on \( B \) in this case.)
\begin{equation}\label{eqn:formAndCurl:760}
\begin{aligned}
\rspacegrad \lr{ \Ba \cdot \Bb }
&=
\gpgradeone{
\rspacegrad \lr{ \Ba \Bb - \Ba \wedge \Bb }
} \\
&=
\gpgradeone{
\spacegrad' \Ba' \Bb
+
\spacegrad' \Ba \Bb'
}
- \rspacegrad \cdot \lr{ \Ba \wedge \Bb }
\\
&=
\lr{ \spacegrad \cdot \Ba} \Bb
+
\lr{ \spacegrad \wedge \Ba} \cdot \Bb
+
\gpgradeone{
- \Ba \spacegrad \Bb + 2 \lr{ \Ba \cdot \spacegrad } \Bb
}
- \spacegrad' \cdot \lr{ \Ba' \wedge \Bb }
- \spacegrad' \cdot \lr{ \Ba \wedge \Bb' }
\\
&=
\lr{ \spacegrad \cdot \Ba} \Bb
+
\lr{ \spacegrad \wedge \Ba} \cdot \Bb
-
\Ba \lr{ \spacegrad \cdot \Bb }
-
\Ba \cdot \lr{ \spacegrad \wedge \Bb }
+ 2 \lr{ \Ba \cdot \spacegrad } \Bb
- \spacegrad' \cdot \lr{ \Ba' \wedge \Bb }
- \spacegrad' \cdot \lr{ \Ba \wedge \Bb' }.
\end{aligned}
\end{equation}
We are running out of room, and have not had any cancellation yet, so let's expand those last two terms separately
\begin{equation}\label{eqn:formAndCurl:780}
\begin{aligned}
- \spacegrad' \cdot \lr{ \Ba' \wedge \Bb }
- \spacegrad' \cdot \lr{ \Ba \wedge \Bb' }
&=
- \lr{ \spacegrad' \cdot \Ba' } \Bb
+ \lr{ \spacegrad' \cdot \Bb } \Ba'
- \lr{ \spacegrad' \cdot \Ba } \Bb'
+ \lr{ \spacegrad' \cdot \Bb' } \Ba
\\
&=
- \lr{ \spacegrad \cdot \Ba } \Bb
+ \lr{ \Bb \cdot \spacegrad } \Ba
- \lr{ \Ba \cdot \spacegrad } \Bb
+ \lr{ \spacegrad \cdot \Bb } \Ba.
\end{aligned}
\end{equation}
Now we can cancel some terms, leaving
\begin{equation}\label{eqn:formAndCurl:800}
\begin{aligned}
\rspacegrad \lr{ \Ba \cdot \Bb }
&=
\lr{ \spacegrad \wedge \Ba} \cdot \Bb
-
\Ba \cdot \lr{ \spacegrad \wedge \Bb }
+ \lr{ \Ba \cdot \spacegrad } \Bb
+ \lr{ \Bb \cdot \spacegrad } \Ba.
\end{aligned}
\end{equation}
After adjustment of the order and sign of the second term, we see that this is the result we wanted.  To show the \R{3} formulation, we have only apply \cref{lemma:formAndCurl:700}.
%{Triple cross products}
\end{proof}
\makelemma{Divergence of a bivector.}{lemma:formAndCurl:870}{
Let \( \Ba, \Bb \in \mathbb{R}^N \) be vectors.  The divergence of their wedge can be written
\begin{equation*}
\spacegrad \cdot \lr{ \Ba \wedge \Bb }
=
  \Bb \lr{ \spacegrad \cdot \Ba }
- \Ba \lr{ \spacegrad \cdot \Bb }
- \lr{ \Bb \cdot \spacegrad } \Ba
+ \lr{ \Ba \cdot \spacegrad } \Bb.
\end{equation*}
For \R{3}, this can also be written in triple cross product form
\begin{equation*}
\spacegrad \cdot \lr{ \Ba \wedge \Bb }
=
-\spacegrad \cross \lr{ \Ba \cross \Bb }.
\end{equation*}
} % lemma
\begin{proof}
\begin{equation}\label{eqn:formAndCurl:860}
\begin{aligned}
\rspacegrad \cdot \lr{ \Ba \wedge \Bb }
&=
\spacegrad' \cdot \lr{ \Ba' \wedge \Bb }
+ \spacegrad' \cdot \lr{ \Ba \wedge \Bb' } \\
&=
\lr{ \spacegrad' \cdot \Ba' } \Bb
- \lr{ \spacegrad' \cdot \Bb } \Ba'
+ \lr{ \spacegrad' \cdot \Ba } \Bb'
- \lr{ \spacegrad' \cdot \Bb' } \Ba
\\
&=
\lr{ \spacegrad \cdot \Ba } \Bb
- \lr{ \Bb \cdot \spacegrad } \Ba
+ \lr{ \Ba \cdot \spacegrad } \Bb
- \lr{ \spacegrad \cdot \Bb } \Ba.
\end{aligned}
\end{equation}
For the \R{3} part of the story, we have
\begin{equation}\label{eqn:formAndCurl:870}
\begin{aligned}
\spacegrad \cross \lr{ \Ba \cross \Bb }
&=
\gpgradeone{
-I \lr{ \spacegrad \wedge \lr{ \Ba \cross \Bb } }
} \\
&=
\gpgradeone{
-I \spacegrad \lr{ \Ba \cross \Bb }
} \\
&=
\gpgradeone{
(-I)^2 \spacegrad \lr{ \Ba \wedge \Bb }
} \\
&=
-
\spacegrad \cdot \lr{ \Ba \wedge \Bb }
\end{aligned}
\end{equation}
\end{proof}

We have just one identity left in the article to find the GA equivalent of, but will split that into two logical pieces.
\makelemma{Dual of triple wedge.}{lemma:formAndCurl:660}{
If \( \Ba, \Bb, \Bc \in \mathbb{R}^3 \) are vectors, then
\begin{equation*}
\Ba \cdot \lr{ \Bb \cross \Bc } = -I \lr{ \Ba \wedge \Bb \wedge \Bc }.
\end{equation*}
} % lemma
\begin{proof}
\begin{equation}\label{eqn:formAndCurl:680}
\begin{aligned}
\Ba \cdot \lr{ \Bb \cross \Bc }
&=
\gpgradezero{
\Ba \lr{ \Bb \cross \Bc }
} \\
&=
\gpgradezero{
\Ba (-I) \lr{ \Bb \wedge \Bc }
} \\
&=
\gpgradezero{
-I \lr{
\Ba \cdot \lr{ \Bb \wedge \Bc }
+
\Ba \wedge \lr{ \Bb \wedge \Bc }
}
} \\
&=
\gpgradezero{
-I \lr{ \Ba \wedge \lr{ \Bb \wedge \Bc } }
} \\
&=
-I \lr{ \Ba \wedge \lr{ \Bb \wedge \Bc } }.
\end{aligned}
\end{equation}
\end{proof}
\makelemma{Curl of a wedge of gradients (divergence of a gradient cross products.)}{lemma:formAndCurl:890}{
Let \( f, g, h \) be smooth functions with smooth derivatives.  Then
\begin{equation*}
\spacegrad \wedge \lr{ f \lr{ \spacegrad g \wedge \spacegrad h } }
=
\spacegrad f
\wedge
\spacegrad g
\wedge
\spacegrad h.
\end{equation*}
For \R{3} this can be written as
\begin{equation*}
\spacegrad \cdot \lr{ f \lr{ \spacegrad g \cross \spacegrad h } }
=
\spacegrad f
\cdot
\lr{
\spacegrad g
\cross
\spacegrad h
}.
\end{equation*}
} % lemma
\begin{proof}
The GA identity follows by chain rule and application of \cref{lemma:formAndCurl:460}.
\begin{equation}\label{eqn:formAndCurl:910}
\begin{aligned}
\spacegrad \wedge \lr{ f \lr{ \spacegrad g \wedge \spacegrad h } }
&=
\spacegrad f \wedge \lr{ \spacegrad g \wedge \spacegrad h }
+
f
\spacegrad \wedge \lr{ \spacegrad g \wedge \spacegrad h } \\
&=
\spacegrad f \wedge \spacegrad g \wedge \spacegrad h.
\end{aligned}
\end{equation}
The \R{3} part of the lemma can be shown using \cref{lemma:formAndCurl:660}, or we can compute it directly
\begin{equation}\label{eqn:formAndCurl:n}
\begin{aligned}
\spacegrad \cdot \lr{ f \lr{ \spacegrad g \cross \spacegrad h } }
&=
\gpgradezero{
\spacegrad \lr{ f \lr{ \spacegrad g \cross \spacegrad h } }
} \\
&=
\spacegrad f \cdot \lr{ \lr{ \spacegrad g \cross \spacegrad h } }
+
f \gpgradezero{
-I \spacegrad \lr{ \spacegrad g \wedge \spacegrad h }
} \\
&=
\spacegrad f \cdot \lr{ \lr{ \spacegrad g \cross \spacegrad h } }
-
f I \lr{ \spacegrad \wedge \lr{ \spacegrad g \wedge \spacegrad h } }.
\end{aligned}
\end{equation}
The last term is clearly zero, since after our chain rule application, we end up with a \( \spacegrad \wedge \spacegrad \) term on either branch of the chain rule expansion.
%\begin{equation}\label{eqn:formAndCurl:970}
%\begin{aligned}
%\spacegrad \cdot \lr{ f (\spacegrad g \cross \spacegrad h) }
%&=
%-I \lr{ \spacegrad \wedge (f (\spacegrad g \wedge \spacegrad h) } \\
%&=
%-I \lr{ \spacegrad f \wedge \spacegrad g \wedge \spacegrad h } \\
%&=
%\spacegrad f \cdot \lr{ \spacegrad g \cross \spacegrad h}.
%\end{aligned}
%\end{equation}
\end{proof}
\makelemma{Curl of a bivector.}{lemma:formAndCurl:930}{
Let \( \Ba, \Bb \) be vectors.  The curl of their wedge is
\begin{equation*}
\spacegrad \wedge \lr{ \Ba \wedge \Bb } = \Bb \wedge \lr{ \spacegrad \wedge \Ba } - \Ba \wedge \lr{ \spacegrad \wedge \Bb }
\end{equation*}
For \R{3}, this can be expressed as the divergence of a cross product
\begin{equation*}
\spacegrad \cdot \lr{ \Ba \cross \Bb } = \Bb \cdot \lr{ \spacegrad \cross \Ba } - \Ba \cdot \lr{ \spacegrad \cross \Bb }
\end{equation*}
} % lemma
\begin{proof}
The GA case is a trivial chain rule application
\begin{equation}\label{eqn:formAndCurl:950}
\begin{aligned}
\rspacegrad \wedge \lr{ \Ba \wedge \Bb }
&=
\lr{ \spacegrad' \wedge \Ba'} \wedge \Bb
+
\lr{ \spacegrad' \wedge \Ba } \wedge \Bb' \\
&= \Bb \wedge \lr{ \spacegrad \wedge \Ba } - \Ba \wedge \lr{ \spacegrad \wedge \Bb }.
\end{aligned}
\end{equation}
The \R{3} case, is less obvious by inspection, but follows from \cref{lemma:formAndCurl:660}.
\end{proof}
\section{Summary.}
We found that we have an isomorphism between the exterior derivative of differential forms and the curl operation of geometric algebra, as follows
\begin{equation}\label{eqn:formAndCurl:990}
\begin{aligned}
df &\rightleftharpoons \spacegrad \wedge f \\
dx_i &\rightleftharpoons \Be_i.
\end{aligned}
\end{equation}
We didn't look at how the Hodge dual translates to GA duality (pseudoscalar multiplication.)  The divergence relationship between the exterior derivative of an \R{3} two-form really requires that formalism, and has only been examined in a cursory fashion.

We also translated a number of vector and gradient identities from conventional vector algebra (i.e.: using cross products) and wedge product equivalents of the same.  The GA identities are often simpler, and in some cases, provide nice mechanisms to derive the conventional identities that would be more cumbersome to determine without the GA toolbox.
%}
\EndArticle
