%}}
% Copyright � 2020 Peeter Joot.  All Rights Reserved.
% Licenced as described in the file LICENSE under the root directory of this GIT repository.
%
%{
\input{../latex/blogpost.tex}
\renewcommand{\basename}{multivector}
\renewcommand{\dirname}{notes/ece1228-electromagnetic-theory/}

\input{../latex/peeter_prologue_print2.tex}

\usepackage{peeters_braket}
\usepackage{peeters_figures}
\usepackage{siunitx}
\usepackage{verbatim}
\usepackage{peeters_tablebox}
\usepackage{peeters_layout_exercise}
\usepackage{macros_qed}
\usepackage{xcolor}

% "#00aa00"
\definecolor{GreenDarker}{rgb}{0, 0.67, 0}
% "#0000aa"
\definecolor{BlueDarker}{rgb}{0, 0, 0.67}
% "#550055"
\definecolor{PurpleDarker}{rgb}{0.333333, 0, 0.333333}
% "#aa0000"
\definecolor{RedDarker}{rgb}{0.67, 0, 0}

\newcommand{\DarkerGreen}[1]{{\color{GreenDarker}#1}}
\newcommand{\DarkerBlue}[1]{{\color{BlueDarker}#1}}
\newcommand{\DarkerPurple}[1]{{\color{PurpleDarker}#1}}
\newcommand{\DarkerRed}[1]{{\color{RedDarker}#1}}

\newcommand{\nbcite}[2]{%
#2%
%\itemCite{GAelectrodynamics}{#1}{#2}%
}

% \mathImageFigure{path}{caption}{label}{width}{nbpath}
% nbpath like: ps2b:countItersAndPlot.m
\newcommand{\mathImageFigure}[5]{%
\imageFigure{#1}{\nbcite{#5}{#2}}{#3}{#4}
}

\beginArtNoToc

\generatetitle{An axiomatic introduction of multivectors, vector products, and geometric algebra.}
%\chapter{Multivector}
%\label{chap:multivector}

%%%\section{What's in the pipe.}
%%%It's been a while since I did any math or physics writing.
This is the first post in a series where I plan to work my way systematically from an introduction of vectors, to the axioms of geometric algebra.
I plan to start with an introduction of vectors as directed ``arrows'', building on that to discuss coordinates, tuples, and column matrix representations, and representation independent ideas.
With those basics established, I'll remind the reader about how generalized vector and dot product spaces are defined and give some examples.
Finally, with the foundation of vectors and vector spaces in place, I'll introduce the concept of a multivector space, and the geometric product, and start unpacking the implications of the axioms that follow naturally from this train of thought.

The applications that I plan to include in this series will be restricted to Euclidean spaces (i.e. where length is given by the Pythagorean law), primarily those of 2 and 3 dimensions.
However, it will be good to also lay the foundations for the non-Euclidean spaces that we encounter in relativistic electromagnetism (there is actually no other kind), and in computer graphics applications of geometric algebra, especially since we can do so nearly for free.
I plan to try to introduce the requisite ideas (i.e. the metric, which allows for a generalized dot product) by discussing Euclidean non-orthonormal bases.
Such bases have applications in condensed matter physics where there are useful for modeling crystal and lattice structure, and provide a hands conceptual bridge to a set of ideas that might otherwise seem abstract and without "real world" application.

\subsection{Motivation.}
Many introductions to geometric algebra start by first introducing the dot product, then bivectors and the wedge product, and eventually define the product of two vectors as the synthetic sum of the dot and wedge
\begin{equation}\label{eqn:inthepipe:20}
\Bx \By = \Bx \cdot \By + \Bx \wedge \By.
\end{equation}
It takes a fair amount of work to do this well.
In the seminal work \citep{hestenes1999nfc} a few pages are taken for each of the dot and wedge products, showing the similarities and building up ideas, before introducing the geometric product in this fashion.
In \citep{dorst2007gac} the authors take a phenomenal five chapters to build up the context required to introduce the geometric product.
I am not disparaging the authors for taking that long to build up the ideas, as their introduction of the subject is exceedingly clear and thorough, and they do a lot more than the minimum required to define the geometric product.

The strategy to introduce the geometric product as a sum of dot and wedge can result in considerable confusion, especially since the wedge product is often defined in terms of the geometric product
\begin{equation}\label{eqn:inthepipe:40}
\Bx \wedge \By =
\inv{2} \lr{
\Bx \By - \By \Bx
}.
\end{equation}
The whole subject can appear like a chicken and egg problem.
I personally found the subject very confusing initially, and had considerable difficulty understanding which of the many identities of geometric algebra were the most fundamental.
For this reason, I found the axiomatic approach of \citep{doran2003gap} very refreshing.
The caveat with that work is that is is exceptionally terse, as they jammed a reformulation of most of physics using geometric algebra into that single book, and it would have been thousands of pages had they tried to make it readable by mere mortals.

When I wrote my own book on the subject, I had the intuition that the way to introduce the subject ought to be like the vector space in abstract linear algebra.
The construct of a vector space is a curious and indirect way to define a vector.
Vectors are not defined as entities, but simply as members of a vector space, a space that is required to have a set of properties.
I thought that the same approach would probably work with multivectors, which could be defined as members of a multivector space, a mathematical construction with a set of properties.

I did try this approach, but was not fully satisfied with what I wrote.
I think that dissatisfaction was because I tried to define the multivector first.
To define the multivector, I first introduced a whole set of
prerequisite ideas (bivector, trivector, blade, k-vector, vector product, ...), but that was also problematic, since the vector multiplication idea required for those concepts wasn't fully defined until the multivector space itself was defined.

My approach shows some mathematical cowardliness.
Had I taken the approach of the vector space fully to heart, the multivector could have been defined as a member of a multivector space, and all the other ideas follow from that.
In this multi-part series, I'm going to play with this approach anew, and see how it works out.
\subsection{Review and background.}
For this discussion, I'm going to assume that the reader is familiar with a wide variety of concepts, including but not limited to
\begin{itemize}
\item vectors,
\item coordinates,
\item matrices,
\item basis,
\item change of basis,
\item dot and cross products,
\item real and complex numbers,
\item rotations and translations,
\item vector spaces, and
\item linear transformations.
\end{itemize}
Despite those assumptions, as mentioned above, I'm going to attempt to build up the basics of vector representation and vector spaces in a systematic fashion, starting from a very elementary level.

My reasons for doing so are mainly to explore the logical sequencing of the ideas required.
I've always found well crafted pedagogical sequences rewarding, and will hopefully construct one here that is appreciated by anybody who chooses to follow along.

%%%\section{Vectors as arrows.}
%%%Cast yourself back in time, all the way to high school, where the first definition of vector that you would have encountered was
probably very similar to the one made famous by the not very villainous \emph{Vector} in ``Despicable Me'' \citep{youtubeVectorVillian}.
His definition was not complete, but it is a good starting point:
\makedefinition{Vector}{dfn:vectorasarrow:180}{
A vector is a quantity represented by an arrow with both direction and magnitude.
} % definition
All the operations that make vectors useful are missing from this definition, such as
\begin{itemize}
\item a comparison operator,
\item a rescaling operation (i.e. a scalar multiplication operation that changes the length),
\item addition and subtraction operators,
\item an operator that provides the length of a vector,
\item multiplication or multiplication like operations.
\end{itemize}

The concept of vector, once supplemented with the operations above, will be useful since it models many directed physical quantities that we experience daily.� These include velocity, acceleration, forces, and electric and magnetic fields.
%We will introduce one multiplication like operator (the dot product), but this will be just to brige the gap
\subsection{Vector comparison.}
In \cref{fig:threeVectorsTx:threeVectorsTxFig1} (a), we have three vectors, labeled \( \DarkerGreen{\Ba}, \DarkerBlue{\Bb}, \DarkerPurple{\Bc} \), all with different directions and magnitudes, and in \cref{fig:threeVectorsTx:threeVectorsTxFig1} (b), those vectors have each been translated (moved without rotation or change of length) slightly.
Two vectors are considered equal if they have the same direction and magnitude.
That is, two vectors are equal if one is the image of the other after translation.
In these figures \( \DarkerGreen{\Ba} \ne \DarkerBlue{\Bb}, \DarkerBlue{\Bb} \ne \DarkerPurple{\Bc}, \DarkerPurple{\Bc} \ne \DarkerGreen{\Ba} \), whereas any same colored vectors are equal.
%This notion of vector has no origin, and a vector is considered equal to its translation anywhere in the space.
%\footnote{
%We will formalize the concept of space (as a vector space), but for now, the space can be considered the plane or the volume in which the vectors lie.
%}.
%\cref{fig:threeVectorsTx:threeVectorsTxFig1}.
%\imageFigure{../figures/GAelectrodynamics/threeVectorsFig1}{Three vectors.}{fig:threeVectors:threeVectorsFig1}{0.15}
%\imageFigure{../figures/GAelectrodynamics/threeVectorsTxFig1}{Translation examples.}{fig:threeVectorsTx:threeVectorsTxFig1}{0.15}
\imageTwoFigures{../figures/GAelectrodynamics/threeVectorsFig1}
{../figures/GAelectrodynamics/threeVectorsTxFig1}{Three vectors and example translations.}{fig:threeVectorsTx:threeVectorsTxFig1}{scale=0.4}
\subsection{Vector (scalar) multiplication.}
We can multiply vectors by scalars by changing their lengths appropriately.
%\footnote{
In this context a scalar is a real number (this is purposefully vague, as it will be useful to allow scalars to be complex valued later.)
%}
Using the example vectors, some rescaled vectors include
\( 2 \DarkerGreen{\Ba}, (-1) \DarkerBlue{\Bb}, \pi \DarkerPurple{\Bc} \), as illustrated in
\cref{fig:threeVectorsScaled:threeVectorsScaledFig1}.
\imageFigure{../figures/GAelectrodynamics/threeVectorsScaledFig1}{Scaled vectors.}{fig:threeVectorsScaled:threeVectorsScaledFig1}{0.1}
%%%%\mathImageFigure{../figures/GAelectrodynamics/VectorsWithOppositeOrientationFig1}{Scalar multiples of vectors.}{fig:VectorsWithOppositeOrientation:VectorsWithOppositeOrientationFig1}{0.15}{vectorOrientationAndAdditionFigures.nb}
\subsection{Vector addition.}
Scalar multiplication implicitly provides an algorithm for addition of vectors that have the same direction, as \( s \Bx + t \Bx = (s+t) \Bx \) for any scalars \( s, t \).  This is illustrated in \cref{fig:twiceVector:twiceVectorFig1} where \( 2 \DarkerGreen{\Ba} = \DarkerGreen{\Ba} + \DarkerGreen{\Ba} \) is formed in two equivalent forms.
We see that the addition of two vectors that have the same direction requires lining up those vectors
head to tail.  The sum of two such vectors is the vector that can be formed from the first tail to the final head.
\imageFigure{../figures/GAelectrodynamics/twiceVectorFig1}{Twice a vector.}{fig:twiceVector:twiceVectorFig1}{0.1}
This procedure is consistent with our experience of directed quantities like forces.  Should your buddies pull on your arms with equal forces, your shoulders might object, but you'll still be in one place, as illustrated in \cref{fig:equalForces:equalForcesFig1}.
\imageFigure{../figures/GAelectrodynamics/equalForcesFig1}{Pulled by opposing and equal forces.}{fig:equalForces:equalForcesFig1}{0.2}
However, if one of your friends is stronger, then assuming you haven't planted your feet too firmly to the ground, you'll be moving in the direction of your stronger friend, as illustrated in
\cref{fig:unequalForces:unequalForcesFig2}.
\imageFigure{../figures/GAelectrodynamics/unequalForcesFig2}{Pulled by unequal opposing forces.}{fig:unequalForces:unequalForcesFig2}{0.2}

It turns out that this arrow daisy chaining procedure is an appropriate way of defining addition for any vectors.
\makedefinition{Vector addition.}{dfn:vectorasarrow:200}{
The sum of two vectors can be found by connecting those two vectors head to tail in either order.  The sum of the two vectors is the vector that can be formed by drawing an arrow from the initial tail to the final head.  This can be generalized by chaining any number of vectors and joining the initial tail to the final head.
} % definition
This addition procedure is illustrated in
\cref{fig:vectorAddition:vectorAdditionFig1}, where \( \DarkerRed{\Bs} = \DarkerGreen{\Ba} + \DarkerBlue{\Bb} + \DarkerPurple{\Bc} \) has been formed.
\mathImageFigure{../figures/GAelectrodynamics/vectorAdditionFig1}{Addition of vectors.}{fig:vectorAddition:vectorAdditionFig1}{0.3}{vectorOrientationAndAdditionFigures.nb}
This definition of vector addition was inferred from the observation of the rules that must apply to addition of vectors that lay in the same direction (colinear vectors).
Is it a cheat to just declare that this rule for addition of colinear vectors also applies to arbitrary vectors?
Yes, it probably is, but it's a cheat that works nicely, and one that models physical quantities that we experience daily (velocities, acceleration, force, ...).
%If you collect two friends you can demonstrate the workability of this inferred rule easily, by putting your arms out, and having your friends pull on them.
%If you put your arms opposing to the sides, and have your friends pull with equal forces, you'll see that the force that can be represented by the pulling of your friends add to zero.
%If one of your friends is stronger, you'll move more in that direction.
Illustrating again with a force thought experiment,
if you put your arms out at 45 degree angles, and have your friends pull on them with equal forces, you'll move straight ahead.  That direction of motion is along the direction of the sum of the forces, if you model those forces as vectors (arrows, with magnitude proportional to the strength of your friends.)
This is crudely illustrated in \cref{fig:sumOfNonColinearForces:sumOfNonColinearForcesFig3}.
\imageFigure{../figures/GAelectrodynamics/sumOfNonColinearForcesFig3}{Sum of pulls separated by 90 degrees.}{fig:sumOfNonColinearForces:sumOfNonColinearForcesFig3}{0.2}
Hopefully, you had a high school physics teacher that had you do quantitative experiments of this nature, using springs and force gauges to
illustrate the vector nature of force.
Such experiments provide a nice first hand tangible justification for the vector addition rule above.

\subsection{Vector subtraction.}
Since we can scale a vector by \( -1 \) and we can add vectors, it is clear how to define vector subtraction
\makedefinition{Vector subtraction.}{dfn:vectorasarrow:220}{
The difference of vectors \( \DarkerGreen{\Ba}, \DarkerBlue{\Bb} \) is
\begin{equation*}
\DarkerGreen{\Ba} - \DarkerBlue{\Bb} \equiv \DarkerGreen{\Ba} + \DarkerPurple{(-1)\Bb}.
\end{equation*}
} % definition
Graphically,
subtracting a vector from another requires flipping the direction
of the vector to be subtracted (scaling by \(-1\)),
, and then adding both head to tail.  This is illustrated in
\cref{fig:vectorSubtractionFig1}.
\mathImageFigure{../figures/GAelectrodynamics/vectorSubtractionFig1}{Vector subtraction.}{fig:vectorSubtractionFig1}{0.25}{vectorOrientationAndAdditionFigures.nb}
%%%%In geometric algebra, we can also multiply vectors, but that is skipping ahead a bit -- for now, just note that we are going to contract your old high school teacher who said "No, you cannot multiply vectors."
XX

\section{Coordinate vectors.}
\subsection{Motivation.}
Having covered the basic operations of vectors in their arrow representation, we are ready to go algebraic on the subject.
In particular, while
it is easy to compute the length of a vector that has an arrow representation, where
one simply lines a ruler of appropriate units along the vector and measures, this graphical procedure is cumbersome when we need to calculate.

Some mathematical baggage from linear algebra is required to do this for general N-dimensional spaces, including
\begin{itemize}
\item coordinates,
\item basis (plural bases),
\item linear dependence and independence,
\item span,
\item dot product, and
\item metric.
\end{itemize}
None of these are hard concepts, but they interfere with the flow of the story, so let's cheat.
We can temporarily avoid the ideas of linear dependence, independence and span, by restricting the story to 2D and 3D spaces that we can describe geometrically.
%When we eventually finish constructing our geometric algebra toolbox, we will have a number of coordinate free methods available to us.
%but we need to understand coordinates to get to that point.
%We also need to understand coordinates, both to read the literature, and in practice.
%Coordinates and non-orthonormal bases are also a good way to introduce non-Euclidean metrics.
\subsection{Basis.}
\makedefinition{Basis (cheat).}{dfn:multivector:180}{
Given a periodic partitioning of a space into repeated parallelopipeds, an ordered set of vectors that lie between the vertices between the edges of one of the cells is called a basis.
} % definition
\index{basis}
The plural of basis is bases.
\index{bases}
A basis for a space subdivided into a parallopiped grid is illustrated with points at all the lattice vertices in
\cref{fig:fbasis:fbasisFig1}.
\imageFigure{../figures/GAelectrodynamics/fbasisFig1}{A basis for a 2D space with a parallopiped lattice.}{fig:fbasis:fbasisFig1}{0.3}
The basis in this example is the ordered set \( \setlr{\Bf_1, \Bf_2} \).
There is no unique choice of basis for a given lattice, as different orderings and directions are possible (
\( \setlr{\pm \Bf_2, \pm \Bf_1} \) or \( \setlr{\pm \Bf_1, \pm \Bf_2} \)).
Alternate bases for this lattice include \( \setlr{\Bf_2, \Bf_1} \), and \( \setlr{-\Bf_1, -\Bf_2} \).
A basis has one and only one vector that lies between the vertices of the smallest cell of the lattice.
%\imageTwoFigures
%{../figures/GAelectrodynamics/fbasisFig1}
%{../figures/GAelectrodynamics/ebasisFig1}
%{Oblique and rectangular two dimensional bases.}{fig:ebasis:ebasisFig1}{scale=0.4}

Of course, the simplest possible lattice is that of a uniform square grid as illustrated in
\cref{fig:ebasis:ebasisFig1}.
\imageFigure{../figures/GAelectrodynamics/ebasisFig1}{A basis for a 2D space with a square lattice.}{fig:ebasis:ebasisFig1}{0.3}
In this example, the pair of vectors \( \setlr{\Be_1, \Be_2}\) is our basis, as they lie along the respective lattice directions.
Alternate bases for this square grid include \( \setlr{\Be_2, \Be_1} \), \( \setlr{-\Be_1, -\Be_2} \), and \( \setlr{\Be_2, -\Be_1} \).
\subsection{Coordinates.}
\makedefinition{Coordinates.}{dfn:multivector:200}{
Given a basis with \( N \) basis vectors \( \setlr{ \Bf_1, \cdots \Bf_N } \), and a vector \( \Ba = \sum_{i = 1}^N a^i \Bf_i \), the coordinates of the vector \( \Ba \) are \((a^1, a^2, \cdots a^N)\), or in matrix (column vector) notation
\begin{equation*}
\begin{bmatrix}
a^1 \\
a^2 \\
\vdots \\
a^N
\end{bmatrix}.
\end{equation*}
} % definition
Superscript is used for the coordinate indexes when the lattice is not composed of square (or cubic) units.
This superscript notation has the disadvantage of introducing an ambiguity, requiring context to determine whether an index or a exponentiation is intended, but will be seen to be worthwhile.

When the lattice is composed of regular cubic cells, where the basis vectors are all of unit length and mutually perpendicular, we can use the more conventional subscript index convention for our coordinates.  In this case, one can also make the engineering identification of a matrix of coordinates as \emph{the vector}.
We choose not to identify the coordinates of a vector as the vector, and consider coordinates to be a specific representation.

We may use the two previous lattice examples to illustrate the idea of coordinates.
For example, consider
\( \Bx = 5 \Bf_1 + 3 \Bf_2 \) as illustrated in
\cref{fig:fbasisSum:fbasisSumFig1}.
\imageFigure{../figures/GAelectrodynamics/fbasisSumFig1}{Decomposition of a particular vector in terms of an oblique set of basis vectors.}{fig:fbasisSum:fbasisSumFig1}{0.3}
The coordinates of this vector with respect the basis \(\setlr{\Bf_1, \Bf_2}\) are \((5,3)\).
Any set of coordinates must be implicitly associated with an underlying basis, and are meaningless without such an association.
For example, the
coordinates of \(\Bx\) with respect the basis \(\setlr{-\Bf_2, \Bf_1}\) would be \((-3,5)\).
As another example, consider the vector \( \By = 3 \Be_1 + 2 \Be_2 \) as illustrated in
\cref{fig:ebasisSum:ebasisSumFig1} for a square lattice.
\imageFigure{../figures/GAelectrodynamics/ebasisSumFig1}{Decomposition of a particular vector on a square lattice.}{fig:ebasisSum:ebasisSumFig1}{0.3}
The coordinates of \( \By \) with respect to the basis \( \setlr{\Be_1, \Be_2}\) are \((3,2)\).
The coordinates of \( \By \) with respect to the an alternate basis \( \setlr{\Be_2, -\Be_1}\) would be \((2,-3)\), that is
\(\By = 2 \Be_2 + (-3)(-\Be_1)\).

%%%The special case of a cubic lattice is so important that we give it a name
%%%\index{standard basis}
%%%\makedefinition{Standard basis.}{dfn:multivector:220}{
%%%Given a unit square(volume) lattice with basis vectors \( \Be_i \) of length 1 (unit vectors), a basis
%%%\( \setlr{\Be_1, \Be_2, \cdots, \Be_N}\) is called a \emph{standard basis}.
%%%} % definition
%%%There are often additional conventions imposed on a standard basis (such as the 3D right hand rule), but we don't care about those for now.
You may ask why we could possibly care to use anything but a cubic lattice with uniform spacing.
Here are a few reasons
\begin{itemize}
\item When we get to vector calculus, we require bases that vary from point to point, depending on the parametization of the space that we are using in our integrals.
Those parameterizations (spherical, cylindrical, toriodal, ...) need not be cubic.
\item There are important applications for non-cubic in solid state physics, as the crystal structures that occur due to molecular bonds are not friendly enough to restrict themselves to cubic configurations.
\item Developing the toolbox for more general bases and coordinates will leave us ready to tackle the non-Euclidean space (spacetime, or ``four-vectors'') that we encounter in special relativity and electromagnetism.
All electromagnetic theory is relativistic, so having the tools to express these ideas is not just academic.
\end{itemize}
\subsection{Scaling.}
Scaling a vector algebraically is so simple that it is almost trivial.  Let
\( \Ba = \sum_{i = 1}^N a^i \Bf_i \), and let \( \alpha \) be any scalar value, then
\begin{equation}\label{eqn:multivector:240}
\alpha \Ba
=
\alpha \sum_{i = 1}^N a^i \Bf_i
=
\sum_{i = 1}^N (\alpha a^i) \Bf_i.
\end{equation}
In the matrix representation of a vector, this is just a scalar matrix product
\begin{dmath}\label{eqn:multivector:260}
\alpha
\begin{bmatrix}
a^1 \\
a^2 \\
\vdots \\
a^N
\end{bmatrix}
=
\begin{bmatrix}
\alpha a^1 \\
\alpha a^2 \\
\vdots \\
\alpha a^N
\end{bmatrix}.
\end{dmath}
\subsection{Addition and subtraction.}
Like scaling, addition and subtraction are trivial algebraically.  Let
\( \Ba = \sum_{i = 1}^N a^i \Bf_i \) and
\( \Bb = \sum_{i = 1}^N b^i \Bf_i \).  Sums or differences of the two are
\begin{dmath}\label{eqn:multivector:280}
\Ba \pm \Bb =
\sum_{i = 1}^N a^i \Bf_i
\pm
\sum_{i = 1}^N b^i \Bf_i
=
\sum_{i = 1}^N (a^i \pm b^i) \Bf_i.
\end{dmath}
The matrix equivalent of this sum or difference is
\begin{dmath}\label{eqn:multivector:300}
\begin{bmatrix}
a^1 \\
a^2 \\
\vdots \\
a^N
\end{bmatrix}
\pm
\begin{bmatrix}
b^1 \\
b^2 \\
\vdots \\
b^N
\end{bmatrix}
=
\begin{bmatrix}
a^1 \pm b^1 \\
a^2 \pm b^2 \\
\vdots \\
a^N \pm b^N
\end{bmatrix}.
\end{dmath}
\subsection{Length.}

%\section{Vector space.}
%We have briefly reviewed the graphical ``arrow'' representation of a vector, the coordinate representation of a vector, and the dot product in its traditional and generalized forms.  The next step in the journey is to systematize and generalize these ideas.  We do so by introducing the concepts of vector and dot product spaces.
%We wish to extend vector spaces by introducing a vector multiplication operation, so an explicit reminder is in order.
\makedefinition{Vector space.}{def:prerequisites:vectorspace}{
A vector space is a set \( V = \setlr{\Bx, \By, \Bz, \cdots} \), the elements of which are called vectors, which has an addition operation designated \( + \) and a scalar multiplication operation designated by juxtaposition, where the following axioms are satisfied
for all vectors \( \Bx, \By, \Bz \in V \) and scalars \( a, b \in \bbR \).
\begin{tablebox}[tabularx={X|Y}]%{Vector space axioms.}
    V is closed under addition & \( \Bx + \By \in V \) \\ \hline
    V is closed under scalar multiplication & \( a \Bx \in V \) \\ \hline
    Addition is associative & \( (\Bx + \By) + \Bz = \Bx + (\By + \Bz) \) \\ \hline
    Addition is commutative & \( \By + \Bx = \Bx + \By \) \\ \hline
    There exists a zero element \( \Bzero \in V \)  & \( \Bx + \Bzero = \Bx \) \\ \hline
    For any \( \Bx \in V \) there exists a negative additive inverse \( -\Bx \in V \) & \( \Bx + (-\Bx) = \Bzero \) \\ \hline
    Scalar multiplication is distributive  & \( a( \Bx + \By ) = a \Bx + a \By \), \( (a + b)\Bx = a \Bx + b\Bx \) \\ \hline
    Scalar multiplication is associative & \( (a b) \Bx = a ( b \Bx ) \) \\ \hline
    There exists a multiplicative identity & \( 1 \Bx = \Bx \) \\ \hline
\end{tablebox}
} % makedefinition{Vector space.}
This vector space concept is an abstract beast, but it encapsulates the rules that underpin addition, subtraction, and rescaling of arrows, as well as their coordinate representation.

%For our geometric algebra applications, we care only about finite dimensional vector spaces.
It is fairly simple to show that coordinate vectors with the usual addition and scaling operations satisfy the axioms of a vector space, as the following problem and solution demonstrates.
\makeproblem{N dimensional finite vector space.}{problem:prerequisites:RN}{
Let \( V = \setlr{ {\begin{bmatrix} x_1 & x_2 & \cdots & x_N \end{bmatrix}}^\T }, x_i \in R \), be the set of \( N \times 1 \) real valued column vectors.  Let
\(
\Bx =
{\begin{bmatrix}
x_1 & x_2 & \cdots & x_N
\end{bmatrix}}^\T \in V \), \(
\By =
{\begin{bmatrix}
y_1 & y_2 & \cdots & y_N
\end{bmatrix}}^\T \in V \), where
an addition operation
\begin{equation*}
\Bx + \By =
{\begin{bmatrix}
x_1 + y_1 & x_2 + y_2 & \cdots & x_N + y_N
\end{bmatrix}}^\T,
\end{equation*}
and a scaling operation
\begin{equation*}
a \Bx =
{\begin{bmatrix}
a x_1 & a x_2 & \cdots & a x_N
\end{bmatrix}}^\T,
\end{equation*}
is defined for all \( \Bx, \By \in V \).
Show that \( V \) is a vector space.
} % problem
\makeanswer{problem:prerequisites:RN}{
\begin{itemize}
\item Closed with respect to addition:
Given \( \Bx, \By \) defined above, we have
\begin{equation*}
\Bx + \By = {\begin{bmatrix} x_1 + y_1& x_2 + y_2& \cdots &x_N + y_N \end{bmatrix}}^\T \in V.
\end{equation*}
\item Closed with respect to multiplication:
\begin{equation*}
a \Bx = {\begin{bmatrix} a x_1& a x_2& \cdots & a x_N \end{bmatrix}}^\T \in V.
\end{equation*}
\item Addition is associative, commutative: left to the reader to verify.
\item Zero element:
Given \( \Bzero = {\begin{bmatrix} 0& 0& \cdots & 0 \end{bmatrix}}^\T \),
clearly
\begin{equation*}
\Bx + \Bzero = {\begin{bmatrix} x_1 + 0& x_2 + 0& \cdots &x_N + 0 \end{bmatrix}}^\T = \Bx,
\end{equation*}
for any \( \Bx \in V. \)
\item Negative inverse.
Given \( -\Bx = {\begin{bmatrix} -x_1& -x_2& \cdots& - x_N \end{bmatrix}}^\T \), then
\begin{equation*}
\Bx + (-\Bx) = {\begin{bmatrix} x_1 - x_1& x_2 - x_2& \cdots& x_N - x_N \end{bmatrix}}^\T = \Bzero.
\end{equation*}
Clearly we can construct a negative additive inverse for any \( \Bx \).
\item Distributed and associative nature of scalar multiplication : left to the reader to verify.
\item Multiplicative identity:
\begin{equation*}
1 \Bx = 1
{\begin{bmatrix}
x_1& x_2& \cdots& x_N
\end{bmatrix}}^\T
 =
{\begin{bmatrix}
1 x_1, 1 x_2, \cdots, 1 x_N
\end{bmatrix}}^\T
 = \Bx \in V. \qedmarker
\end{equation*}
\end{itemize}
} % answer
Our intended
geometric algebra applications will actually be restricted to simple vector spaces with two or three spatial dimensions, usually with real valued vectors.  Up to four dimensions are required for spacetime applications (special relativity).  For computer graphics applications, where Euclidean space is extended with additional dimensions for the origin and point at infinity, up to five dimensions may be required.

The vector space is, in fact, a much more general construct, and can be used to represent a number of different mathematical constructs.  For completeness sake,
a couple examples of more general vector spaces (with function and matrix elements) are
are given as problems below, but it would be too big of a digression to explore those in detail.  See any good book on linear algebra to explore the some of the powerful applications of vector spaces.
\input{../GAelectrodynamics/functionvectorspace.tex}
\input{../GAelectrodynamics/paulivectorspace.tex}

\subsection{Dot product spaces.}
We've abstracted the rules for adding, subtracting, and scaling arrows, which are encapsulated in the axioms of the vector space.  The next job is to do the same thing for vector magnitude, which we abstract by defining the dot product in terms of it's properties.

\index{dot product}
\makedefinition{Dot product.}{dfn:prerequisites:dotproduct}{
Let \( \Bx, \By \) be real valued vectors from a vector space \( V \).
A dot product \( \Bx \cdot \By \) is a mapping \( V \cross V \rightarrow \bbR \)
with the following properties.
\begin{tablebox}[tabularx={X|Y}]%{Dot product properties.}
    Symmetric & \( \Bx \cdot \By = \By \cdot \Bx \) \\ \hline
    Bilinear & \( (a \Bx + b \By) \cdot \Bz = a \Bx \cdot \Bz + b \By \cdot \Bz,\,
\Bx \cdot (a \By + b \Bz) = a \Bx \cdot \By + b \Bx \cdot \Bz \)
\\ \hline
    (Optional) Positive length & \( \Bx \cdot \Bx > 0, \Bx \ne 0 \) \\ \hline
\end{tablebox}
} % definition

\makedefinition{Dot product space.}{dfn:prerequisites:dotproductspace}{
A vector space with an associated dot product is called a dot product space.
}

\section{JUNK: REWRITE FROM HERE.}

%In geometric algebra, we require what can loosely be called a dot product.
%A dot product usually has the following characteristics
%
%- Symmetric : $ \Bx \cdot \By = \By \cdot \Bx $
%- Bilinear : $ (a \Bx + b \By) \cdot \Bz = a \Bx \cdot \Bz + b \By \cdot \Bz,\quad \Bx \cdot (a \By + b \Bz) = a \Bx \cdot \By + b \Bx \cdot \Bz $
%- Positive length : $ \Bx \cdot \Bx > 0, \Bx \ne 0 $
%
%, but the positive definite nature of that dot product is not required.
%
If a vector space \( V \) contains elements \( \Bx, \By \), we designate that dot product as \( \Bx \cdot \By \), and require

Symmetric : \( \Bx \cdot \By = \By \cdot \Bx \)

Bilinear : \( (a \Bx + b \By) \cdot \Bz = a \Bx \cdot \Bz + b \By \cdot \Bz,\quad \Bx \cdot (a \By + b \Bz) = a \Bx \cdot \By + b \Bx \cdot \Bz \)

Positive length : \( \Bx \cdot \Bx > 0, \Bx \ne 0 \)


Recall that a vector space with an associated dot product is called a dot product space.

Given a finite dimensional (dot-product) vector space \( V = \setlr{ \Bx, \By, \Bz, \cdots } \), with a dot product where the dot product of elements
, with a dot product \( \Bx \cdot \By \)
a multivector space generated by \( V \) is a set \( M = \setlr{ x, y, z, \cdots } \) of multivectors (sums of scalars, vectors, or products of vectors), where the following axioms are satisfied

Contraction : \( \Bx^2 = \Bx \cdot \Bx, \,\forall \Bx \in V \)

\( M \) is closed under addition : \( x + y \in M \)

\( M \) is closed under multiplication : \( x y \in M \)

Addition is associative : \( (x + y) + z = x + (y + z) \)

Addition is commutative : \( y + x = x + y \)

There exists a zero element \( 0 \in M \)  : \( x + 0 = x \)

For all \( x \in M \) there exists a negative additive inverse \( -x \in M \) : \( x + (-x) = 0 \)

Multiplication is distributive  : \( x( y + z ) = x y + x z \), \( (x + y)z = x z + y z \)

Multiplication is associative : \( (x y) z = x ( y z ) \)

There exists a multiplicative identity \( 1 \in M \) : \( 1 x = x \)

Clearly $\mathbb{R}$, using scalar multiplication as the dot product, is a multivector space.

It's possible to show that $\mathbb{R}^2$, $\mathbb{R}^3$, and other vector spaces (with the normal Euclidean dot product) also generate multivector spaces.  Both of these first require that we show that \( \Bx \By = - \By \Bx \), if \( \Bx \cdot \By = 0 \), that is, the products of perpendicular vectors, assumed to be members of  anticommute.


%}
\EndArticle
