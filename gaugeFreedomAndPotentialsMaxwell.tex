%
% Copyright � 2022 Peeter Joot.  All Rights Reserved.
% Licenced as described in the file LICENSE under the root directory of this GIT repository.
%
%{
\input{../latex/blogpost.tex}
\renewcommand{\basename}{gaugeFreedomAndPotentialsMaxwell}
%\renewcommand{\dirname}{notes/phy1520/}
\renewcommand{\dirname}{notes/ece1228-electromagnetic-theory/}
%\newcommand{\dateintitle}{}
%\newcommand{\keywords}{}

\input{../latex/peeter_prologue_print2.tex}

\usepackage{peeters_layout_exercise}
\usepackage{peeters_braket}
\usepackage{peeters_figures}
\usepackage{siunitx}
\usepackage{verbatim}
%\usepackage{mhchem} % \ce{}
%\usepackage{macros_bm} % \bcM
\usepackage{macros_qed} % \qedmarker
%\usepackage{txfonts} % \ointclockwise

\beginArtNoToc

\generatetitle{Gauge freedom and four-potentials in the STA form of Maxwell's equation.}
%\chapter{Gauge freedom and four-potentials in the STA form of Maxwell's equation.}
%\label{chap:gaugeFreedomAndPotentialsMaxwell}

\section{Motivation.}
In a recent video on the tensor structure of Maxwell's equation, I made a little side trip down the road of potential solutions and gauge transformations.

The initial point of that side trip was just to point out that the Faraday tensor can be expressed in terms of four potential coordinates
\begin{equation}\label{eqn:gaugeFreedomAndPotentialsMaxwell:20}
   F_{\mu\nu} = \partial_\mu A_\nu - \partial_\nu A_\mu,
\end{equation}
but before I got there I tried to motivate this.  In this post, I'll outline the same ideas.
\section{STA representation of Maxwell's equation.}
We'd gone through the work to show that Maxwell's equation has the STA form
\begin{equation}\label{eqn:gaugeFreedomAndPotentialsMaxwell:40}
   \grad F = J.
\end{equation}
This is a deceptively compact representation, as it requires all of the following definitions
\begin{equation}\label{eqn:gaugeFreedomAndPotentialsMaxwell:60}
   \grad = \gamma^\mu \partial_\mu = \gamma_\mu \partial^\mu,
\end{equation}
\begin{equation}\label{eqn:gaugeFreedomAndPotentialsMaxwell:80}
   \partial_\mu = \PD{x^\mu}{},
\end{equation}
\begin{equation}\label{eqn:gaugeFreedomAndPotentialsMaxwell:100}
   \gamma^\mu \cdot \gamma_\nu = {\delta^\mu}_\nu,
\end{equation}
\begin{equation}\label{eqn:gaugeFreedomAndPotentialsMaxwell:160}
   \gamma_\mu \cdot \gamma_\nu = g_{\mu\nu},
\end{equation}
\begin{equation}\label{eqn:gaugeFreedomAndPotentialsMaxwell:120}
\begin{aligned}
   F
   &= \BE + I c \BB  \\
   %&= E^k \gamma_k \gamma_0 - c \lr{ B^3 \gamma_1 \gamma_3 + B^2 \gamma_3 \gamma_1 + B^1 \gamma_2 \gamma_3 } \\
   &= -E^k \gamma^k \gamma^0 - \inv{2} c B^r \gamma^s \gamma^t \epsilon^{r s t} \\
   &= \inv{2} \gamma^{\mu} \wedge \gamma^{\nu} F_{\mu\nu},
\end{aligned}
\end{equation}
and
\begin{equation}\label{eqn:gaugeFreedomAndPotentialsMaxwell:140}
\begin{aligned}
   J &= \gamma_\mu J^\mu  \\
   J^\mu &= \frac{\rho}{\epsilon} \gamma_0 + \eta (\BJ \cdot \Be_k).
\end{aligned}
\end{equation}

\section{Four-potentials in the STA representation.}
In order to find the tensor form of Maxwell's equation (starting from the STA representation), we first split the equation into two, since
\begin{equation}\label{eqn:gaugeFreedomAndPotentialsMaxwell:180}
\grad F = \grad \cdot F + \grad \wedge F = J.
\end{equation}
The dot product is a four-vector, the wedge term is a trivector, and the current is a four-vector, so we have one grade-1 equation and one grade-3 equation
\begin{equation}\label{eqn:gaugeFreedomAndPotentialsMaxwell:200}
\begin{aligned}
   \grad \cdot F &= J \\
   \grad \wedge F &= 0.
\end{aligned}
\end{equation}
The potential comes into the mix, since the curl equation above means that \( F \) necessarily can be written as the curl of some four-vector
\begin{equation}\label{eqn:gaugeFreedomAndPotentialsMaxwell:220}
   F = \grad \wedge A.
\end{equation}
One justification of this is that \( a \wedge (a \wedge b) = 0 \), for any vectors \( a, b \).  Expanding such a double-curl out in coordinates is also worthwhile
\begin{equation}\label{eqn:gaugeFreedomAndPotentialsMaxwell:240}
\begin{aligned}
   \grad \wedge \lr{ \grad \wedge A }
   &=
   \lr{ \gamma_\mu \partial^\mu }
   \wedge
   \lr{ \gamma_\nu \partial^\nu }
   \wedge
   A \\
   &=
   \gamma^\mu \wedge \gamma^\nu \wedge \lr{ \partial_\mu \partial_\nu A }.
\end{aligned}
\end{equation}
Provided we have equality of mixed partials, this is a product of an antisymmetric factor and a symmetric factor, so the full sum is zero.

Things get interesting if one imposes a \( \grad \cdot A = \partial_\mu A^\mu = 0 \) constraint on the potential.  If we do so, then
\begin{equation}\label{eqn:gaugeFreedomAndPotentialsMaxwell:260}
   \grad F = \grad^2 A = J.
\end{equation}
Observe that \( \grad^2 \) is the wave equation operator (often written as \( \delSquaredBox \).)  That is
\begin{equation}\label{eqn:gaugeFreedomAndPotentialsMaxwell:280}
\begin{aligned}
   \grad^2
   &= \partial^\mu \partial_\mu \\
   &= \partial_0 \partial_0
   - \partial_1 \partial_1
   - \partial_2 \partial_2
   - \partial_3 \partial_3 \\
   &= \inv{c^2} \PDSq{t}{}  - \spacegrad^2.
\end{aligned}
\end{equation}
This is also an operator for which the Green's function is well known (\citep{jackson1975cew}), which means that we can immediately write the solutions
\begin{equation}\label{eqn:gaugeFreedomAndPotentialsMaxwell:300}
   A(x) = \int G(x,x') J(x') d^4 x'.
\end{equation}
However, we have no a-priori guarantee that such a solution has zero divergence.  We can fix that by making a gauge transformation of the form
\begin{equation}\label{eqn:gaugeFreedomAndPotentialsMaxwell:320}
   A \rightarrow A - \grad \chi.
\end{equation}
Observe that such a transformation does not change the electromagnetic field
\begin{equation}\label{eqn:gaugeFreedomAndPotentialsMaxwell:340}
   F = \grad \wedge A \rightarrow \grad \wedge \lr{ A - \grad \chi },
\end{equation}
since
\begin{equation}\label{eqn:gaugeFreedomAndPotentialsMaxwell:360}
   \grad \wedge \grad \chi = 0,
\end{equation}
(also by equality of mixed partials.)  Suppose that \( \tilde{A} \) is a solution of \( \grad^2 \tilde{A} = J \), and \( \tilde{A} = A + \grad \chi \), where \( A \) is a zero divergence field to be determined, then
\begin{equation}\label{eqn:gaugeFreedomAndPotentialsMaxwell:380}
   \grad \cdot \tilde{A}
   =
   \grad \cdot A + \grad^2 \chi,
\end{equation}
or
\begin{equation}\label{eqn:gaugeFreedomAndPotentialsMaxwell:400}
\grad^2 \chi = \grad \cdot \tilde{A}.
\end{equation}
So if \( \tilde{A} \) does not have zero divergence, we can find a \( \chi \)
\begin{equation}\label{eqn:gaugeFreedomAndPotentialsMaxwell:420}
   \chi(x) = \int G(x,x') \grad' \cdot \tilde{A}(x') d^4 x',
\end{equation}
so that \( A = \tilde{A} - \grad \chi \) does have zero divergence.

%}
\EndArticle
%\EndNoBibArticle
