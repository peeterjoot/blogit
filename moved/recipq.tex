%
% Copyright � 2023 Peeter Joot.  All Rights Reserved.
% Licenced as described in the file LICENSE under the root directory of this GIT repository.
%
%{
\input{../latex/blogpost.tex}
\renewcommand{\basename}{recipq}
%\renewcommand{\dirname}{notes/phy1520/}
\renewcommand{\dirname}{notes/ece1228-electromagnetic-theory/}
%\newcommand{\dateintitle}{}
%\newcommand{\keywords}{}

\input{../latex/peeter_prologue_print2.tex}

\usepackage{peeters_layout_exercise}
\usepackage{peeters_braket}
\usepackage{peeters_figures}
\usepackage{siunitx}
\usepackage{verbatim}
%\usepackage{mhchem} % \ce{}
%\usepackage{macros_bm} % \bcM
%\usepackage{macros_qed} % \qedmarker
%\usepackage{txfonts} % \ointclockwise

\beginArtNoToc

\generatetitle{XXX}
%\chapter{XXX}
%\label{chap:recipq}

If the curvilinear basis vectors \( \setlr{ \Bx_i } \) are orthogonal, then the reciprocal vectors \( \Bx^i \) are easy to compute.  They must each satisfy \( \Bx^i \cdot \Bx_i = 1 \).  Let
\begin{equation}\label{eqn:recipq:20}
\Bx^i = \alpha \Bx_i,
\end{equation}
then
\begin{equation}\label{eqn:recipq:40}
\begin{aligned}
1
&= \Bx^i \cdot \Bx_i \\
&= \lr{ \alpha \Bx_i } \cdot \Bx_i \\
&= \alpha \lr{ \Bx_i \cdot \Bx_i },
\end{aligned}
\end{equation}
or
\begin{equation}\label{eqn:recipq:60}
\Bx^i = \inv{ \Bx_i \cdot \Bx_i } \Bx_i = \inv{\Bx_i}.
\end{equation}

%}
%\EndArticle
\EndNoBibArticle
