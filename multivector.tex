%}}
% Copyright � 2020 Peeter Joot.  All Rights Reserved.
% Licenced as described in the file LICENSE under the root directory of this GIT repository.
%
%{
\input{../latex/blogpost.tex}
\renewcommand{\basename}{multivector}
%\renewcommand{\dirname}{notes/phy1520/}
\renewcommand{\dirname}{notes/ece1228-electromagnetic-theory/}
%\newcommand{\dateintitle}{}
%\newcommand{\keywords}{}

\input{../latex/peeter_prologue_print2.tex}

\usepackage{peeters_braket}
\usepackage{peeters_figures}
\usepackage{siunitx}
\usepackage{verbatim}
\usepackage{peeters_tablebox}
\usepackage{peeters_layout_exercise}
\usepackage{macros_qed}
%\usepackage{mhchem} % \ce{}
%\usepackage{macros_bm} % \bcM
%\usepackage{macros_qed} % \qedmarker
%\usepackage{txfonts} % \ointclockwise

\newcommand{\nbcite}[2]{%
#2%
%\itemCite{GAelectrodynamics}{#1}{#2}%
}

% \mathImageFigure{path}{caption}{label}{width}{nbpath}
% nbpath like: ps2b:countItersAndPlot.m
\newcommand{\mathImageFigure}[5]{%
\imageFigure{#1}{\nbcite{#5}{#2}}{#3}{#4}
}

\beginArtNoToc

\generatetitle{An new axiomatic introduction of multivectors, vector products, and geometric algebra.}
%\chapter{Multivector}
%\label{chap:multivector}

\paragraph{Motivation.}
Many introductions to geometric algebra start by first introducing the dot product, then bivectors and the wedge product, and eventually define the product of two vectors as the synthetic sum of the dot and wedge
\begin{equation}\label{eqn:multivector:20}
\Bx \By = \Bx \cdot \By + \Bx \wedge \By.
\end{equation}
It takes a fair amount of work to do this well.  In the seminal work \citep{hestenes1999nfc} a few pages are taken for each of the dot and wedge products, showing the similarities and building up ideas, before introducing the geometric product in this fashion.  In \citep{dorst2007gac} the authors take a phenomenal five chapters to build up the context required to introduce the geometric product.
I am not disparaging the authors for taking that long to build up the ideas, as their introduction of the subject is exceedingly clear and thorough, and they do a lot more than the minumum required to define the geometric product.

The strategy to introduce the geometric product as a sum of dot and wedge can result in considerable confusion, especially since the wedge product is often defined in terms of the geometric product
\begin{equation}\label{eqn:multivector:40}
\Bx \wedge \By =
\inv{2} \lr{
\Bx \By - \By \Bx
}.
\end{equation}
The whole subject can appear like a chicken and egg problem.  I personally found the subject very confusing initially, and had considerable difficulty understanding which of the many
identities of geometric algebra were the most fundamental.  For this reason, I found the axiomatic approach of \citep{doran2003gap} very refreshing.  The cavaet with that work is that is is exceptionally terse, as they jammed a reformulation of most of physics using geometric algebra into that single book, and it would have been thousands of pages had they tried to make it readable by mere mortals.

When I wrote my own book on the subject, I had the intuition that the way to introduce the subject ought to be like the vector space in abstract linear algebra.  The construct of a vector space is a curious and indirect way to define a vector.  Vectors are not defined as entities, but simply as members of a vector space, a space that is required to have a set of properties.  I thought that the same approach would probably work with multivectors, which could be defined as members of a multivector space, a mathematical construction with a set of properties.

I did try this approach, but was not fully satisfied with what I wrote.  I think that dissatisfaction was because I tried to define the multivector first.  To define the multivector, I first introduced a whole set of
prerequisite ideas (bivector, trivector, blade, k-vector, vector product, ...), but that was also problematic, since the vector multiplication idea required for those concepts wasn't fully defined until the multivector space itself was defined.

My approach shows some mathematical cowardness.  Had I taken the approach of the vector space fully to heart, the multivector could have been defined as a member of a multivector space, and all the other ideas follow from that.  In this article, I'm going to play with this approach anew, and see how it works out.
\paragraph{Vectors.}
One of the simplest descriptions of a vector, as made famous by the supervillian \emph{Vector}, is ``a quantity represented by an arrow with both direction and magnitude'' \citep{youtubeVectorVillian}.
We can add vectors by connecting the arrows head to tail, and drawing the arrow that connects the initial tail to the final head, as illustrated in%
\cref{fig:vectorAddition:vectorAdditionFig1}.
\mathImageFigure{../figures/GAelectrodynamics/vectorAdditionFig1}{Addition of vectors.}{fig:vectorAddition:vectorAdditionFig1}{0.3}{vectorOrientationAndAdditionFigures.nb}
We can perform scalar multiplication of vectors by simply changing their their length appropriately, as illustrated in \cref{fig:VectorsWithOppositeOrientation:VectorsWithOppositeOrientationFig1}.
We can subtract vectors by flipping their directions (scaling by \(-1\)) and adding them, as illustrated in%
\cref{fig:vectorSubtractionFig1}.
%In geometric algebra, we can also multiply vectors, but that is skipping ahead a bit -- for now, just note that we are going to contract your old high school teacher who said "No, you cannot multiply vectors."
\mathImageFigure{../figures/GAelectrodynamics/VectorsWithOppositeOrientationFig1}{Scalar multiples of vectors.}{fig:VectorsWithOppositeOrientation:VectorsWithOppositeOrientationFig1}{0.15}{vectorOrientationAndAdditionFigures.nb}
\mathImageFigure{../figures/GAelectrodynamics/vectorSubtractionFig1}{Vector subtraction.}{fig:vectorSubtractionFig1}{0.25}{vectorOrientationAndAdditionFigures.nb}

In engineering vectors are usually represented as \( N \times 1 \) column matrixes\footnote{Matrices are like meth or crack to engineers, who will attempt to cast everything as a matrix problem.}.  A vector \( \Bx \) with coordinates \((x,y,z)\) would have the representation
\begin{dmath}\label{eqn:multivector:60}
\Bx =
\begin{bmatrix}
x \\
y \\
z
\end{bmatrix}.
\end{dmath}
With a coordinate representation, we can add and subtract vectors by simply adding and subtracting their coordinates.  For example, with
\(
\Bx =
{
\begin{bmatrix}
x_1 &
x_2 &
x_3
\end{bmatrix}
}^\T,
\By =
{\begin{bmatrix}
y_1 &
y_2 &
y_3
\end{bmatrix}}^\T \), addition and scaling operations are trivial
\begin{equation}\label{eqn:multivector:140}
a \Bx + b \By
=
a
\begin{bmatrix}
x_1 \\
x_2 \\
x_3
\end{bmatrix}
+
b
\begin{bmatrix}
y_1 \\
y_2 \\
y_3
\end{bmatrix}
=
\begin{bmatrix}
a x_1 + b y_1 \\
a x_2 + b y_2 \\
a x_3 + b y_3
\end{bmatrix}.
\end{equation}
However, we must also supplement the coordinate representation with an operation that provides the length of the vector.
That operation is the dot product, which has the following general form
\begin{dmath}\label{eqn:multivector:101}
\Bx \cdot \By = \Bx^\T G \By,
\end{dmath}
where \( G \) is a symmetric matrix.  When \( G = I \), the identity matrix, then the dot product provides a compact encoding of the Euclidean length
\begin{equation}\label{eqn:multivector:80}
\Norm{\Bx} = \sqrt{ \Bx \cdot \Bx } = x^2 + y^2 + z^2.
\end{equation}
The dot product has an absolute maximum value when the two vectors are colinear, and is zero when the two vectors are perpendicular.  In all the geometric algebra applications that I'm aware of, the metric is always a
diagonal matrix with diagonal values \( 0, \pm 1 \).  The non-identity metric of interest for electromagnetism is the metric for four-vectors of special relativity
\begin{dmath}\label{eqn:multivector:120}
G =
\pm
\begin{bmatrix}
-1 & 0 & 0 & 0 \\
0 & 1 & 0 & 0 \\
0 & 0 & 1 & 0 \\
0 & 0 & 0 & 1
\end{bmatrix}.
\end{dmath}
With this metric the squared-length of ``timelike'' or ``spacelike'' vectors have opposite signs.  Observe that this special relativistic dot product does not have the usual \( \Bx \cdot \Bx \ge 0 \) property that is often included in an axiomatic definition of a dot product.
There are four and five dimensional conformal and projective geometric algebras that also use a non-identity metric, but we won't discuss them here, and the interested reader is referred to the literature.

We have briefly reviewed the graphical ``arrow'' representation of a vector, the coordinate representation of a vector, and the dot product in its traditional and generalized forms.  The next step in the journey is to systematize and generalize these ideas.  We do so by introducing the concepts of vector and dot product spaces.
\paragraph{Vector space.}
We wish to extend vector spaces by introducing a vector multiplication operation, so an explicit reminder is in order.
\makedefinition{Vector space.}{def:prerequisites:vectorspace}{
A vector space is a set \( V = \setlr{\Bx, \By, \Bz, \cdots} \), the elements of which are called vectors, which has an addition operation designated \( + \) and a scalar multiplication operation designated by juxtaposition, where the following axioms are satisfied
for all vectors \( \Bx, \By, \Bz \in V \) and scalars \( a, b \in \bbR \).
\begin{tablebox}[tabularx={X|Y}]%{Vector space axioms.}
    V is closed under addition & \( \Bx + \By \in V \) \\ \hline
    V is closed under scalar multiplication & \( a \Bx \in V \) \\ \hline
    Addition is associative & \( (\Bx + \By) + \Bz = \Bx + (\By + \Bz) \) \\ \hline
    Addition is commutative & \( \By + \Bx = \Bx + \By \) \\ \hline
    There exists a zero element \( \Bzero \in V \)  & \( \Bx + \Bzero = \Bx \) \\ \hline
    For any \( \Bx \in V \) there exists a negative additive inverse \( -\Bx \in V \) & \( \Bx + (-\Bx) = \Bzero \) \\ \hline
    Scalar multiplication is distributive  & \( a( \Bx + \By ) = a \Bx + a \By \), \( (a + b)\Bx = a \Bx + b\Bx \) \\ \hline
    Scalar multiplication is associative & \( (a b) \Bx = a ( b \Bx ) \) \\ \hline
    There exists a multiplicative identity & \( 1 \Bx = \Bx \) \\ \hline
\end{tablebox}
} % makedefinition{Vector space.}
This vector space concept is an abstract beast, but it encapsulates the rules that underpin addition, subtraction, and rescaling of arrows, as well as their coordinate representation.

For our geometric algebra applications, we care only about finite dimensional vector spaces.
It is fairly simple to show that coordinate vectors with the usual addition and scaling operations satisfy the axioms of a vector space, as the following problem and solution demonstrates.
\makeproblem{N dimensional finite vector space.}{problem:prerequisites:RN}{
Let \( V = \setlr{ {\begin{bmatrix} x_1 & x_2 & \cdots & x_N \end{bmatrix}}^\T }, x_i \in R \), be the set of \( N \times 1 \) real valued column vectors.  Let
\(
\Bx =
{\begin{bmatrix}
x_1 & x_2 & \cdots & x_N
\end{bmatrix}}^\T \in V \), \(
\By =
{\begin{bmatrix}
y_1 & y_2 & \cdots & y_N
\end{bmatrix}}^\T \in V \), where
an addition operation
\begin{equation*}
\Bx + \By =
{\begin{bmatrix}
x_1 + y_1 & x_2 + y_2 & \cdots & x_N + y_N
\end{bmatrix}}^\T,
\end{equation*}
and a scaling operation
\begin{equation*}
a \Bx =
{\begin{bmatrix}
a x_1 & a x_2 & \cdots & a x_N
\end{bmatrix}}^\T,
\end{equation*}
is defined for all \( \Bx, \By \in V \).
Show that \( V \) is a vector space.
} % problem
\makeanswer{problem:prerequisites:RN}{
\begin{itemize}
\item Closed with respect to addition:
Given \( \Bx, \By \) defined above, we have
\begin{equation*}
\Bx + \By = {\begin{bmatrix} x_1 + y_1& x_2 + y_2& \cdots &x_N + y_N \end{bmatrix}}^\T \in V.
\end{equation*}
\item Closed with respect to multiplication:
\begin{equation*}
a \Bx = {\begin{bmatrix} a x_1& a x_2& \cdots & a x_N \end{bmatrix}}^\T \in V.
\end{equation*}
\item Addition is associative, commutative: left to the reader to verify.
\item Zero element:
Given \( \Bzero = {\begin{bmatrix} 0& 0& \cdots & 0 \end{bmatrix}}^\T \),
clearly
\begin{equation*}
\Bx + \Bzero = {\begin{bmatrix} x_1 + 0& x_2 + 0& \cdots &x_N + 0 \end{bmatrix}}^\T = \Bx,
\end{equation*}
for any \( \Bx \in V. \)
\item Negative inverse.
Given \( -\Bx = {\begin{bmatrix} -x_1& -x_2& \cdots& - x_N \end{bmatrix}}^\T \), then
\begin{equation*}
\Bx + (-\Bx) = {\begin{bmatrix} x_1 - x_1& x_2 - x_2& \cdots& x_N - x_N \end{bmatrix}}^\T = \Bzero.
\end{equation*}
Clearly we can construct a negative additive inverse for any \( \Bx \).
\item Distributed and associative nature of scalar multiplication : left to the reader to verify.
\item Multiplicative identity:
\begin{equation*}
1 \Bx = 1
{\begin{bmatrix}
x_1& x_2& \cdots& x_N
\end{bmatrix}}^\T
 =
{\begin{bmatrix}
1 x_1, 1 x_2, \cdots, 1 x_N
\end{bmatrix}}^\T
 = \Bx \in V. \qedmarker
\end{equation*}
\end{itemize}
} % answer
The vector space is, in fact, a much more general construct, and can be used to represent a number of different mathematical constructs.  However, our intended
geometric algebra applications will actually be restricted to simple vector spaces with two or three spatial dimensions, and up to four dimensions for spacetime (special relativity).  For computer graphics applications, up to five dimensions may be required (as Euclidean space is extended with additional dimensions for the origin and point at infinity.)

For completeness sake,
a couple examples of more general vector spaces (with function and matrix elements) are
are given as problems below, but it would be too big of a digression to explore those in detail.  See any good book on linear algebra to explore the some of the powerful applications of vector spaces.
\input{../GAelectrodynamics/functionvectorspace.tex}
\input{../GAelectrodynamics/paulivectorspace.tex}
\paragraph{REWRITE FROM HERE.}

%In geometric algebra, we require what can loosely be called a dot product.
%A dot product usually has the following characteristics
%
%- Symmetric : $ \Bx \cdot \By = \By \cdot \Bx $
%- Bilinear : $ (a \Bx + b \By) \cdot \Bz = a \Bx \cdot \Bz + b \By \cdot \Bz,\quad \Bx \cdot (a \By + b \Bz) = a \Bx \cdot \By + b \Bx \cdot \Bz $
%- Positive length : $ \Bx \cdot \Bx > 0, \Bx \ne 0 $
%
%, but the positive definite nature of that dot product is not required.
%
If a vector space \( V \) contains elements \( \Bx, \By \), we designate that dot product as \( \Bx \cdot \By \), and require

Symmetric : \( \Bx \cdot \By = \By \cdot \Bx \)

Bilinear : \( (a \Bx + b \By) \cdot \Bz = a \Bx \cdot \Bz + b \By \cdot \Bz,\quad \Bx \cdot (a \By + b \Bz) = a \Bx \cdot \By + b \Bx \cdot \Bz \)

Positive length : \( \Bx \cdot \Bx > 0, \Bx \ne 0 \)


Recall that a vector space with an associated dot product is called a dot product space.

Given a finite dimensional (dot-product) vector space \( V = \setlr{ \Bx, \By, \Bz, \cdots } \), with a dot product where the dot product of elements
, with a dot product \( \Bx \cdot \By \)
a multivector space generated by \( V \) is a set \( M = \setlr{ x, y, z, \cdots } \) of multivectors (sums of scalars, vectors, or products of vectors), where the following axioms are satisfied

Contraction : \( \Bx^2 = \Bx \cdot \Bx, \,\forall \Bx \in V \)

\( M \) is closed under addition : \( x + y \in M \)

\( M \) is closed under multiplication : \( x y \in M \)

Addition is associative : \( (x + y) + z = x + (y + z) \)

Addition is commutative : \( y + x = x + y \)

There exists a zero element \( 0 \in M \)  : \( x + 0 = x \)

For all \( x \in M \) there exists a negative additive inverse \( -x \in M \) : \( x + (-x) = 0 \)

Multiplication is distributive  : \( x( y + z ) = x y + x z \), \( (x + y)z = x z + y z \)

Multiplication is associative : \( (x y) z = x ( y z ) \)

There exists a multiplicative identity \( 1 \in M \) : \( 1 x = x \)

Clearly $\mathbb{R}$, using scalar multiplication as the dot product, is a multivector space.

It's possible to show that $\mathbb{R}^2$, $\mathbb{R}^3$, and other vector spaces (with the normal Euclidean dot product) also generate multivector spaces.  Both of these first require that we show that \( \Bx \By = - \By \Bx \), if \( \Bx \cdot \By = 0 \), that is, the products of perpendicular vectors, assumed to be members of  anticommute.

\paragraph{Unit vectors: include?}
In particular, the unit vectors in the x, y, z directions are
\begin{equation}\label{eqn:prerequisites:20}
\Be_1 =
\begin{bmatrix}
1 \\
0 \\
0 \\
\end{bmatrix},\quad
\Be_2 =
\begin{bmatrix}
0 \\
1 \\
0 \\
\end{bmatrix},\quad
\Be_3 =
\begin{bmatrix}
0 \\
0 \\
1 \\
\end{bmatrix}.
\end{equation}
I'll use the symbol \( \Be_i \) to designate the unit vector in the ith direction%
\footnote{
The notation for the unit vectors themselves varies by author.
For example, it's not uncommon in engineering texts to use \( \hat{a}_x, \hat{a}_y, \hat{a}_z \) instead of \( \Be_1, \Be_2, \Be_3 \).  I would guess that the \(\hat{a}\) notation evolved to avoid the overloading of the symbol \(e = 2.718\cdots\).}.
This symbolic designation allows any vector to be encoded in a
representation agnostic fashion.  For example a vector \( \Bx \) with coordinates \( x, y, z \) is
\begin{dmath}\label{eqn:prerequisites:40}
\Bx = x \Be_1 + y \Be_2 + z \Be_3,
\end{dmath}
independent of whether the underlying representation of the unit vectors themselves are tuple, row, column, or anything else.


%}
\EndArticle
