%
% Copyright � 2021 Peeter Joot.  All Rights Reserved.
% Licenced as described in the file LICENSE under the root directory of this GIT repository.
%
%{
\input{../latex/blogpost.tex}
\renewcommand{\basename}{vectordotbivector}
%\renewcommand{\dirname}{notes/phy1520/}
\renewcommand{\dirname}{notes/ece1228-electromagnetic-theory/}
%\newcommand{\dateintitle}{}
%\newcommand{\keywords}{}

\input{../latex/peeter_prologue_print2.tex}

\usepackage{peeters_layout_exercise}
\usepackage{peeters_braket}
\usepackage{peeters_figures}
\usepackage{siunitx}
\usepackage{verbatim}
%\usepackage{mhchem} % \ce{}
%\usepackage{macros_bm} % \bcM
%\usepackage{macros_qed} % \qedmarker
%\usepackage{txfonts} % \ointclockwise

\beginArtNoToc

\generatetitle{XXX}
%\chapter{XXX}
%\label{chap:vectordotbivector}

The identities

$$a\cdot b= (ab+ba)/2$$

$$a\wedge b=(ab-ba)/2$$

are correct, but they are only for vectors \( a, b \).  The equation

$$a\cdot(b\wedge c) = (a(b\wedge c) - (b\wedge c)a)/2$$

(with the negative sign), is also correct.  In general, if \( M \) is a k-vector, and \( a \) is a vector, then one has

\begin{equation*}
a \cdot M
=
\inv{2} \lr{ a M + (-1)^{k-1} M a }.
\end{equation*}

I don't know how the Matrix Gateway book derives eq 2.14.  Here's one way to show it, rewriting the dot product as a grade 1 selection (i.e. take just the vector parts of any multivector products)
\begin{equation*}
\begin{aligned}
a \cdot \lr{ b \wedge c }
&=
\gpgradeone{ a \lr{ b \wedge c } } \\
   &=
\gpgradeone{ a \lr{ b c - b \cdot c } } \\
&=
\gpgradeone{ a b c } - a \lr{ b \cdot c },
\end{aligned}
\end{equation*}
however, using \( b a = 2 a \cdot b - b a \), we have
\begin{equation*}
\begin{aligned}
\gpgradeone{ a b c }
&=
\gpgradeone{ \lr{ 2 a \cdot b - b a } c } \\
&=
2 \lr{ a \cdot b } c - \gpgradeone{ b a c } \\
&=
2 \lr{ a \cdot b } c - \gpgradeone{ b \lr{ 2 a \cdot c - c a } } \\
&=
2 \lr{ a \cdot b } c - 2 \lr{ a \cdot c } b + \gpgradeone{ b c a } \\
&=
2 \lr{ a \cdot b } c - 2 \lr{ a \cdot c } b + \lr{ b \wedge c } \cdot a + \lr{ b \cdot c } a.
\end{aligned}
\end{equation*}

Putting the pieces together, we have
\begin{equation*}
a \cdot \lr{ b \wedge c }
=
2 \lr{ a \cdot b } c - 2 \lr{ a \cdot c } b + \lr{ b \wedge c } \cdot a.
\end{equation*}
Finally, note that the reverse of vector is a vector (\(\tilde{a} = a\)), so if we apply the reversion operator to the last term:
\begin{equation*}
\begin{aligned}
\lr{ b \wedge c } \cdot a
%&=
%\widetilde{\lr{ b \wedge c } \cdot a} \\
&=
a \cdot \lr{ c \wedge b } \\
&=
-a \cdot \lr{ b \wedge c },
\end{aligned}
\end{equation*}
and then substitute this back in, we have derived eq 2.14 from the text:
\begin{equation*}
a \cdot \lr{ b \wedge c }
=
2 \lr{ a \cdot b } c - 2 \lr{ a \cdot c } b - a \lr{ b \wedge c },
\end{equation*}
or
\begin{equation*}
a \cdot \lr{ b \wedge c }
=
\lr{ a \cdot b } c - \lr{ a \cdot c } b.
\end{equation*}

%}
%\EndArticle
\EndNoBibArticle
