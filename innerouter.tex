%
% Copyright � 2021 Peeter Joot.  All Rights Reserved.
% Licenced as described in the file LICENSE under the root directory of this GIT repository.
%
%{
\input{../latex/blogpost.tex}
\renewcommand{\basename}{innerouter}
%\renewcommand{\dirname}{notes/phy1520/}
\renewcommand{\dirname}{notes/ece1228-electromagnetic-theory/}
%\newcommand{\dateintitle}{}
%\newcommand{\keywords}{}

\input{../latex/peeter_prologue_print2.tex}

\usepackage{peeters_layout_exercise}
\usepackage{peeters_braket}
\usepackage{peeters_figures}
\usepackage{siunitx}
\usepackage{verbatim}
%\usepackage{mhchem} % \ce{}
%\usepackage{macros_bm} % \bcM
%\usepackage{macros_qed} % \qedmarker
%\usepackage{txfonts} % \ointclockwise

\beginArtNoToc

\generatetitle{XXX}
As mentioned by Hans, these are not fundamental definitions.  Instead, you should view the following as fundamental:
\begin{equation*}
   a \cdot A_k = \gpgrade{ a A_k }{k-1},
\end{equation*}
\begin{equation*}
   a \wedge A_k = \gpgrade{ a A_k }{k+1},
\end{equation*}
and then go through the exercise of proving that
\begin{equation*}
\gpgrade{ a A_k }{k-1} = \frac12 \left( aA_k-(-1)^kA_k a\right),
\end{equation*}
and
\begin{equation*}
\gpgrade{a \wedge A_k}{k+1} = \frac12 \left( a A_k+ (-1)^kA_k a \right).
\end{equation*}

%}
%\EndArticle
\EndNoBibArticle
