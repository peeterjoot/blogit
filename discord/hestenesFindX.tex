%
% Copyright � 2022 Peeter Joot.  All Rights Reserved.
% Licenced as described in the file LICENSE under the root directory of this GIT repository.
%
%{
\input{../latex/blogpost.tex}
\renewcommand{\basename}{hestenesFindX}
%\renewcommand{\dirname}{notes/phy1520/}
\renewcommand{\dirname}{notes/ece1228-electromagnetic-theory/}
%\newcommand{\dateintitle}{}
%\newcommand{\keywords}{}

\input{../latex/peeter_prologue_print2.tex}

\usepackage{peeters_layout_exercise}
\usepackage{peeters_braket}
\usepackage{peeters_figures}
\usepackage{siunitx}
\usepackage{verbatim}
%\usepackage{mhchem} % \ce{}
%\usepackage{macros_bm} % \bcM
%\usepackage{macros_qed} % \qedmarker
%\usepackage{txfonts} % \ointclockwise

\beginArtNoToc

\generatetitle{XXX}
%\chapter{XXX}
%\label{chap:hestenesFindX}
%A simplified version of part of gabookI/basics/nfcmCh2.tex
%\section{New}
\subsection{\citep{hestenes1999nfc} Exercise 1.3}
To solve
\begin{equation}\label{eqn:hestenesFindX:20}
   \alpha \Bx + \Ba \lr{ \Bx \cdot \Bb } = \Bc,
\end{equation}
for \(\Bx\), we may dot or wedge the whole equation with the constant vectors.  In particular, respectively dotting with \( \Bb \) and wedging with \( \Ba \) gives
\begin{equation}\label{eqn:hestenesFindX:40}
\begin{aligned}
   \alpha \lr{ \Bx \cdot \Bb } + \Ba \cdot \Bb \lr{ \Bx \cdot \Bb } &= \Bc \cdot \Bb \\
   \alpha \lr{ \Bx \wedge \Ba } + \lr{ \Ba \wedge \Ba } \lr{ \Bx \cdot \Bb } = \Bc \wedge \Ba
\end{aligned}
\end{equation}
The first equation can now be solved for \( \Bx \cdot \Bb \), namely
\begin{equation}\label{eqn:hestenesFindX:60}
   \Bx \cdot \Bb = \frac{ \Bc \cdot \Bb }{ \alpha + \Ba \cdot \Bb },
\end{equation}
and the second equation can be solved for \( \Bx \wedge \Ba \), noting that \( \Ba \wedge \Ba = 0 \), to find
\begin{equation}\label{eqn:hestenesFindX:80}
\Bx \wedge \Ba = \inv{\alpha} \Bc \wedge \Ba.
\end{equation}
Now dot this bivector equation with \( \Bb \), noting that
\begin{equation}\label{eqn:hestenesFindX:100}
   \lr{ \Bx \wedge \Ba } \cdot \Bb
   =
   \Bx \lr{ \Ba \cdot \Bb } - \lr{ \Bx \cdot \Bb } \Ba,
\end{equation}
or
\begin{equation}\label{eqn:hestenesFindX:120}
\Bx \lr{ \Ba \cdot \Bb } = \lr{ \Bx \cdot \Bb } \Ba + \inv{\alpha} \lr{ \Bc \wedge \Ba } \cdot \Bb
\end{equation}
Substituting \cref{eqn:hestenesFindX:60} and rearranging we find
\begin{equation}\label{eqn:hestenesFindX:n}
%\begin{aligned}
\Bx 
%&
=
\inv{ \Ba \cdot \Bb } \lr{
\frac{ \Bc \cdot \Bb }{ \alpha + \Ba \cdot \Bb } \Ba + \inv{\alpha} \lr{ \Bc \wedge \Ba } \cdot \Bb
}
%\\
%&=
%\frac{\Ba}{ \Ba \cdot \Bb } 
%\frac{ \Bc \cdot \Bb }{ \alpha + \Ba \cdot \Bb } + \frac{\Bc}{\alpha} - \frac{\Bc \cdot \Bb}{\alpha \lr{ \Ba \cdot \Bb }}
%\\
%\end{aligned}
\end{equation}
%%\section{original}
%%\subsection{Exercise 1.3}
%%
%%Solve for \(x\)
%%
%%\begin{equation}\label{eqn:nfcmCh2:60}
%%\begin{aligned}
%%\alpha x + a x \cdot b = c
%%\end{aligned}
%%\end{equation}
%%
%%where \(\alpha\) is a scalar and all the rest are vectors.
%%
%%\subsubsection{Solution}
%%
%%Can dot or wedge the entire equation with the constant vectors.  In particular
%%
%%\begin{equation}\label{eqn:nfcmCh2:80}
%%\begin{aligned}
%%c \cdot b &= (\alpha x + a x \cdot b) \cdot b \\
%%&= (\alpha + a \cdot b) x \cdot b
%%\end{aligned}
%%\end{equation}
%%\begin{equation}\label{eqn:nfcmCh2:100}
%%\begin{aligned}
%%\implies
%%x \cdot b &= \frac{c \cdot b}{\alpha + a \cdot b} \\
%%\end{aligned}
%%\end{equation}
%%
%%and
%%\begin{equation}\label{eqn:nfcmCh2:120}
%%\begin{aligned}
%%c \wedge a &= (\alpha x + a x \cdot b) \wedge a \\
%%&= \alpha (x \wedge a) + \mathLabelBox
%%[
%%   labelstyle={xshift=2cm},
%%   linestyle={out=270,in=90, latex-}
%%]
%%{(a \wedge a)}{\(=0\)} (x \cdot b) \wedge a \\
%%\end{aligned}
%%\end{equation}
%%\begin{equation}\label{eqn:nfcmCh2:140}
%%\begin{aligned}
%%\implies
%%x \wedge a &= \inv{\alpha} (c \wedge a) \\
%%\end{aligned}
%%\end{equation}
%%
%%This last can be reduced by dotting with \(b\), and then substitute the
%%result for \(x \cdot b\) from above
%%
%%\begin{equation}\label{eqn:nfcmCh2:160}
%%\begin{aligned}
%%(x \wedge a) \cdot b
%%&= x (a \cdot b) - (x \cdot b) a \\
%%&= x (a \cdot b) - \frac{c \cdot b}{\alpha + a \cdot b} a \\
%%\end{aligned}
%%\end{equation}
%%
%%Thus the final solution is
%%
%%\begin{equation}\label{eqn:nfcmCh2:180}
%%\begin{aligned}
%%x = \inv{a \cdot b}\left(
%%\frac{c \cdot b}{\alpha + a \cdot b} a
%%+ \inv{\alpha} (c \wedge a) \cdot b
%%\right)
%%\end{aligned}
%%\end{equation}
%%
%%Question: was there a geometric or physical motivation for this question.  I can not recall one?
%}
\EndArticle
%\EndNoBibArticle
