%
% Copyright � 2021 Peeter Joot.  All Rights Reserved.
% Licenced as described in the file LICENSE under the root directory of this GIT repository.
%
%{
\input{../latex/blogpost.tex}
\renewcommand{\basename}{bladeDotWedgeSymmetryIdentitiesTheorem}
%\renewcommand{\dirname}{notes/phy1520/}
\renewcommand{\dirname}{notes/ece1228-electromagnetic-theory/}
%\newcommand{\dateintitle}{}
%\newcommand{\keywords}{}

\input{../latex/peeter_prologue_print2.tex}

\usepackage{peeters_layout_exercise}
\usepackage{peeters_braket}
\usepackage{peeters_figures}
\usepackage{siunitx}
\usepackage{verbatim}
%\usepackage{mhchem} % \ce{}
%\usepackage{macros_bm} % \bcM
%\usepackage{macros_qed} % \qedmarker
%\usepackage{txfonts} % \ointclockwise

\beginArtNoToc

\generatetitle{XXX}
%\chapter{XXX}

To prove these relations, split the blade \( A_k \) into components that intersect with and are disjoint from \( a \) as follows
\begin{equation*}
A_k
=
\inv{a} n_1 n_2 \cdots n_{k-1} + m_1 m_2 \cdots m_k,
\end{equation*}
where \( n_i \) orthogonal to \( a \) and each other, and where \( m_i \) are all orthogonal.  The products of \( A_k \) with \( a \) are
\begin{equation*}
\begin{aligned}
a A_k
&=
a \inv{a} n_1 n_2 \cdots n_{k-1} + a m_1 m_2 \cdots m_k \\
&=
n_1 n_2 \cdots n_{k-1} + a m_1 m_2 \cdots m_k,
\end{aligned}
\end{equation*}
and
\begin{equation*}
\begin{aligned}
A_k a
&=
\inv{a} n_1 n_2 \cdots n_{k-1} a + m_1 m_2 \cdots m_k a \\
&=
(-1)^{k-1} n_1 n_2 \cdots n_{k-1} + (-1)^k a m_1 m_2 \cdots m_k \\
&=
(-1)^k \lr{ - n_1 n_2 \cdots n_{k-1} + a m_1 m_2 \cdots m_k },
\end{aligned}
\end{equation*}
or
\begin{equation*}
(-1)^k A_k a
=
- n_1 n_2 \cdots n_{k-1} + a m_1 m_2 \cdots m_k.
\end{equation*}

Respective addition and subtraction gives
\begin{equation*}
\begin{aligned}
a A_k + (-1)^k A_k a
&= 2 a m_1 m_2 \cdots m_k \\
&= 2 \gpgrade{a A_k}{k+1},
\end{aligned}
\end{equation*}
and
\begin{equation*}
\begin{aligned}
a A_k - (-1)^k A_k a
&=
2
n_1 n_2 \cdots n_{k-1} \\
&= 2 \gpgrade{a A_k}{k-1},
\end{aligned}
\end{equation*}
completing the proof.

%}
\EndNoBibArticle
