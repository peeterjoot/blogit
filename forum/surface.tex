%
% Copyright � 2021 Peeter Joot.  All Rights Reserved.
% Licenced as described in the file LICENSE under the root directory of this GIT repository.
%
%{
\input{../latex/blogpost.tex}
\renewcommand{\basename}{surface}
%\renewcommand{\dirname}{notes/phy1520/}
\renewcommand{\dirname}{notes/ece1228-electromagnetic-theory/}
%\newcommand{\dateintitle}{}
%\newcommand{\keywords}{}

\input{../latex/peeter_prologue_print2.tex}

\usepackage{peeters_layout_exercise}
\usepackage{peeters_braket}
\usepackage{peeters_figures}
\usepackage{siunitx}
\usepackage{verbatim}
%\usepackage{mhchem} % \ce{}
%\usepackage{macros_bm} % \bcM
%\usepackage{macros_qed} % \qedmarker
%\usepackage{txfonts} % \ointclockwise

\beginArtNoToc

\generatetitle{XXX}
%\chapter{XXX}
%\label{chap:surface}
You want to form a wedge product of the differentials (of a parameterized surface) to construct an area element.  The wedge products encode an oriented area element, and do not require any notion of a normal that could be ambiguous in a higher dimensional space.  Such an area element can be described using differential forms, or in geometric algebra.  Here is a rough outline of such a surface encoding in the geometric algebra formalism:

Suppose your surface is parameterized by
\begin{equation*}
   \Bx = \Bx(u,v).
\end{equation*}
The partials are tangent to the surface
\begin{equation*}
\begin{aligned}
\Bx_u &=
\PD{u}{\Bx}  \\
\Bx_v &=
\PD{v}{\Bx}
\end{aligned}
\end{equation*}
and you can form a surface area element by wedging the differentials \( d\Bx_u = \Bx_u du, d\Bx_v dv \) along each of those tangent space directions
\begin{equation*}
   d^2 \Bx = (\Bx_u \wedge \Bx_v) \, du dv.
\end{equation*}

If you happen to be in 3D, this area element can be related to the cross product by multiplying by the (trivector) pseudoscalar \( I = \Be_1 \Be_2 \Be_3 \),
\begin{equation*}
   -I d^2 \Bx = (\Bx_u \cross \Bx_v) \, du dv.
\end{equation*}

In differential forms, the notation and nomenclature is a bit different (2-form vs. bivector, hodge dual vs. pseudoscalar product) but a lot of the ideas should be similar.

%}
%\EndArticle
\EndNoBibArticle
