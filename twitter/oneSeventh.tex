%
% Copyright � 2025 Peeter Joot.  All Rights Reserved.
% Licenced as described in the file LICENSE under the root directory of this GIT repository.
%
%{
\input{../latex/blogpost.tex}
\renewcommand{\basename}{oneSeventh}
%\renewcommand{\dirname}{notes/phy1520/}
\renewcommand{\dirname}{notes/ece1228-electromagnetic-theory/}
%\newcommand{\dateintitle}{}
%\newcommand{\keywords}{}

\input{../latex/peeter_prologue_print2.tex}

\usepackage{peeters_layout_exercise}
\usepackage{peeters_braket}
\usepackage{peeters_figures}
\usepackage{siunitx}
\usepackage{verbatim}
%\usepackage{macros_cal} % \LL
%\usepackage{amsthm} % proof
%\usepackage{mhchem} % \ce{}
%\usepackage{macros_bm} % \bcM
%\usepackage{macros_qed} % \qedmarker
%\usepackage{txfonts} % \ointclockwise

\beginArtNoToc

\generatetitle{XXX}
%\chapter{XXX}
%\label{chap:oneSeventh}
% \citep{sakurai2014modern} pr X.Y
% \citep{pozar2009microwave}
% \citep{qftLectureNotes}
% \citep{doran2003gap}
% \citep{jackson1975cew}
% \citep{griffiths1999introduction}

Solve
\begin{equation}\label{eqn:oneSeventh:20}
I = \int \frac{dx}{\lr{x+4}^{8/7} \lr{ x - 3}^{6/7} }.
\end{equation}
Let's try guessing
\begin{equation}\label{eqn:oneSeventh:40}
I = \lr{ x + 4 }^a \lr{ x - 3 }^b + C.
\end{equation}
where \( a, b \) are probably constant multiples of \( 1/7 \), since \( 8/7 = 1 + 1/7 \) and \( 6/7 = 1 - 1/7 \).

We find
\begin{equation}\label{eqn:oneSeventh:60}
\begin{aligned}
I'
&= a \lr{ x + 4 }^{a-1} \lr{ x - 3 }^b  + b \lr{ x + 4 }^a \lr{ x - 3 }^{b - 1} \\
&= \lr{ x + 4 }^{a} \lr{ x - 3 }^b  \lr{ \frac{a}{x + 4} + \frac{b}{x - 3}  } \\
&= \lr{ x + 4 }^{a-1} \lr{ x - 3 }^{b-1}  \lr{ a \lr{x - 4} + b \lr{x + 4}  } \\
&= \lr{ x + 4 }^{a-1} \lr{ x - 3 }^{b-1}  \lr{ x \lr{ a + b } + 4 b - 3 a } \\
\end{aligned}
\end{equation}

We need \( a, b \) to satisfy the following set of over-specified equations
\begin{equation}\label{eqn:oneSeventh:80}
\begin{aligned}
a - 1 &= -\frac{8}{7} \\
b - 1 &= -\frac{6}{7} \\
a + b &= 0 \\
4 b - 3 a &= 1.
\end{aligned}
\end{equation}
This is solved by \( a = -1/7, b = 1/7 \), so
\begin{equation}\label{eqn:oneSeventh:100}
I = \lr{ \frac{ x - 3}{x + 4 }}^{1/7} + C.
\end{equation}
It will be interesting to see other solutions, especially one with a more systematic approach, since this guess was too lucky.

%}
%\EndArticle
\EndNoBibArticle
