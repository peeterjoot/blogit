%
% Copyright � 2025 Peeter Joot.  All Rights Reserved.
% Licenced as described in the file LICENSE under the root directory of this GIT repository.
%
%{
\input{../latex/blogpost.tex}
\renewcommand{\basename}{ellipicalIntegral}
%\renewcommand{\dirname}{notes/phy1520/}
\renewcommand{\dirname}{notes/ece1228-electromagnetic-theory/}
%\newcommand{\dateintitle}{}
%\newcommand{\keywords}{}

\input{../latex/peeter_prologue_print2.tex}

\usepackage{peeters_layout_exercise}
\usepackage{peeters_braket}
\usepackage{peeters_figures}
\usepackage{siunitx}
\usepackage{verbatim}
%\usepackage{macros_cal} % \LL
%\usepackage{amsthm} % proof
%\usepackage{mhchem} % \ce{}
%\usepackage{macros_bm} % \bcM
%\usepackage{macros_qed} % \qedmarker
%\usepackage{txfonts} % \ointclockwise

\beginArtNoToc

\generatetitle{A fun ellipse related integral.}
%\chapter{A fun ellipse related integral.}
%\label{chap:ellipicalIntegral}

\section{Motivation.}
This was a problem I found on twitter (\citep{CalcInsightsElliptical})

%\makeproblem{}{problem:ellipicalIntegral:1}{
Find
\begin{equation}\label{eqn:ellipicalIntegral:20}
I = \int_0^\pi \frac{dx}{a^2 \cos^2 x + b^2 \sin^2 x}.
\end{equation}
%} % problem

I posted my solution there (as a screenshot), but had a sign wrong.  Here's a correction.
\section{Solution.}
Let's first assume we aren't interested in the \( a^2 = b^2 \), nor either of the \( a = 0, b = 0\) cases (if either of \( a, b \) is zero, then the integral is divergent.) 

We can make a couple simple transformations to start with
\begin{equation}\label{eqn:ellipicalIntegral:40}
\begin{aligned}
\cos^2 x &= \frac{\cos(2x) + 1}{2} \\
\sin^2 x &= \frac{1 - \cos(2x)}{2},
\end{aligned}
\end{equation}
and then \( u = 2 x \), for \( dx = du/2 \)
\begin{equation}\label{eqn:ellipicalIntegral:60}
\begin{aligned}
I
&= \int_0^{2\pi} 2 \frac{du/2}{a^2 \lr{ 1 + \cos u } + b^2 \lr{1 - \cos u}} \\
&= \int_0^{2\pi} \frac{du}{ \lr{ a^2 - b^2 } \cos u + a^2 + b^2 }.
\end{aligned}
\end{equation}
There is probably a simple way to evaluate this integral, but let's try it the fun way, using contour integration.  Following examples from \citep{byron1992mca}, let \( z = e^{i u} \), where \( dz = i z du \), and \( \alpha = \lr{ a^2 + b^2 }/\lr{ a^2 - b^2 } \), for
\begin{equation}\label{eqn:ellipicalIntegral:80}
\begin{aligned}
I
&= \oint_{\Abs{z} = 1} \frac{dz/(i z)}{ \lr{ a^2 - b^2 } \lr{ z + \inv{z}}/2 + a^2 + b^2 } \\
&= \frac{2}{i \lr{ a^2 - b^2 }} \oint_{\Abs{z} = 1} \frac{dz}{ z \lr{ z + \inv{z} + 2 \alpha} } \\
&= \frac{2}{i \lr{ a^2 - b^2 }} \oint_{\Abs{z} = 1} \frac{dz}{ z^2 + 2 \alpha z + 1 } \\
&= \frac{2}{i \lr{ a^2 - b^2 }} \oint_{\Abs{z} = 1} \frac{dz}{ \lr{ z + \alpha - \sqrt{\alpha^2 - 1}}\lr{ z + \alpha + \sqrt{\alpha^2 - 1}} }.
\end{aligned}
\end{equation}

There is a single enclosed pole on the real axis.  For \( a^2 > b^2 \) where \( \alpha > 0 \) that pole is at \( z = -\alpha + \sqrt{ \alpha^2 - 1} \), so the integral is
\begin{equation}\label{eqn:ellipicalIntegral:100}
\begin{aligned}
I
&= 2 \pi i \frac{2}{i \lr{ a^2 - b^2 }} \evalbar{ \frac{1}{ z + \alpha + \sqrt{\alpha^2 - 1 } }}{z = -\alpha + \sqrt{ \alpha^2 - 1}} \\
&= \frac{4 \pi}{ a^2 - b^2 } \frac{1}{ 2 \sqrt{\alpha^2 - 1 } } \\
&= \frac{4 \pi }{ 2 \sqrt{\lr{a^2 + b^2}^2 - \lr{a^2 - b^2}^2 } } \\
&= \frac{2 \pi }{ \sqrt{4 a^2 b^2 } } \\
&= \frac{\pi }{ \Abs{ a b } },
\end{aligned}
\end{equation}
and for \( a^2 < b^2 \) where \( \alpha < 0 \), the enclosed pole is at \( z = -\alpha - \sqrt{ \alpha^2 - 1} \), where
\begin{equation}\label{eqn:ellipicalIntegral:120}
\begin{aligned}
I
&= 2 \pi i \frac{2}{i \lr{ a^2 - b^2 }} \evalbar{ \frac{1}{ z + \alpha - \sqrt{\alpha^2 - 1 } }}{z = -\alpha - \sqrt{ \alpha^2 - 1}} \\
&= \frac{4 \pi}{ a^2 - b^2 } \frac{1}{ -2 \sqrt{\alpha^2 - 1 } } \\
&= \frac{4 \pi}{ b^2 - a^2 } \frac{1}{ 2 \sqrt{\alpha^2 - 1 } } \\
&= \frac{4 \pi }{ 2 \sqrt{\lr{a^2 + b^2}^2 - \lr{b^2 - a^2}^2 } } \\
&= \frac{2 \pi }{ \sqrt{4 a^2 b^2 } } \\
&= \frac{\pi }{ \Abs{ a b } }.
\end{aligned}
\end{equation}
Observe that this also holds for the \( a = b \) case.

%}
\EndArticle
%\EndNoBibArticle
