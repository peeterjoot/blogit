%
% Copyright � 2025 Peeter Joot.  All Rights Reserved.
% Licenced as described in the file LICENSE under the root directory of this GIT repository.
%
%{
\input{../latex/blogpost.tex}
\renewcommand{\basename}{junk}
%\renewcommand{\dirname}{notes/phy1520/}
\renewcommand{\dirname}{notes/ece1228-electromagnetic-theory/}
%\newcommand{\dateintitle}{}
%\newcommand{\keywords}{}

\input{../latex/peeter_prologue_print2.tex}

\usepackage{peeters_layout_exercise}
\usepackage{peeters_braket}
\usepackage{peeters_figures}
\usepackage{siunitx}
\usepackage{verbatim}
%\usepackage{macros_cal} % \LL
%\usepackage{amsthm} % proof
%\usepackage{mhchem} % \ce{}
%\usepackage{macros_bm} % \bcM
%\usepackage{macros_qed} % \qedmarker
%\usepackage{txfonts} % \ointclockwise

\beginArtNoToc

\generatetitle{XXX}
%\chapter{XXX}
%\label{chap:junk}

\section{}
\begin{equation*}
\int f(x)\, dx\, g(x)
\end{equation*}

\section{}
\begin{equation*}
\begin{aligned}
i \sqrt{ i \sqrt{ i^{22} }}
&=
i \sqrt{ i \sqrt{ \lr{i^4}^5 i^2 }} \\
&=
i \sqrt{ i \sqrt{ {1}^5 \lr{ -1 } }} \\
&=
i \sqrt{ i^2 } \\
&=
i \sqrt{ -1 } \\
&=
i^2 \\
&=
-1.
\end{aligned}
\end{equation*}

\section{}
\begin{equation*}
6! = \lr{3 \cdot 2 } \cdot 5! = 3! 5!
\end{equation*}

\section{}
That first one works if \( a \) and \( b \) are orthogonal vectors (using the vector product from geometric algebra.)

An example, using a Pauli matrix representation of two such orthogonal vectors
\begin{equation*}
\begin{aligned}
a &= \sigma_x = \PauliX \\
b &= \sigma_z = \PauliZ.
\end{aligned}
\end{equation*}

In general, however, if \( a \cdot b \ne 0 \),
\begin{equation*}
\lr{ a + b }^2 = a^2 + 2 a \cdot b + b^2.
\end{equation*}
Observe that if \( a, b \in \bbR^1 \) (isomorphic to \( \bbR \)), then they cannot ever be orthogonal, so this reduces to the familiar
\begin{equation*}
\lr{ a + b }^2 = a^2 + 2 a b + b^2.
\end{equation*}


\section{}
Problem: Find the inverse of
\begin{equation}\label{eqn:junk:20}
y = \frac{10^x - 10^{-x}}{10^x + 10^{-x}} + 1.
\end{equation}

First rewrite the exponents
\begin{equation}\label{eqn:junk:40}
10^x = e^{\ln 10^x} = e^{x \ln 10},
\end{equation}
so
\begin{equation}\label{eqn:junk:60}
\begin{aligned}
y - 1
&= \frac{e^{x \ln 10} - e^{-x \ln 10}}{e^{x \ln 10} + e^{-x \ln 10}} \\
&= \frac{\sinh\lr{ x \ln 10}}{\cosh\lr{ x \ln 10 }} \\
&= \tanh\lr{ x \ln 10}.
\end{aligned}
\end{equation}
We can now invert to find
\begin{equation}\label{eqn:junk:80}
\boxed{
x = \frac{\tanh^{-1}\lr{ y - 1 }}{\ln 10}.
}
\end{equation}
\section{}
Here's a fun and easy way to compute the radial and tangential components of velocity and acceleration in circular coordinates, using a complex (polar) representation of the position vector.  First take derivatives
\begin{equation*}
\begin{aligned}
z &= r e^{i\theta} \\
z' &= r' e^{i\theta} + i r e^{i\theta} \theta' \\
z'' &= r'' e^{i\theta} + 2 i r' e^{i\theta} \theta' - r e^{i\theta} \lr{\theta'}^2 + i r e^{i\theta} \theta'',
\end{aligned}
\end{equation*}
Next identify \( \rcap = e^{i\theta} \) and \( \thetacap = i e^{i\theta} \), \( \omega = \theta' \), and \( \alpha = \theta'' \) to find
\begin{equation*}
\begin{aligned}
\Bx &= r \rcap \\
\Bv &= r' \rcap + r \omega \thetacap \\
\Ba &= \lr{ r'' - r \omega^2 } \rcap + \lr{ 2 r' \omega + r \alpha } \thetacap.
\end{aligned}
\end{equation*}
\section{}
If \( \Br(t) = \Br_0 + \Bv t \), then
\begin{equation*}
\Br(t) \wedge \Bv = \Br_0 \wedge \Bv + \lr{ \Bv \wedge \Bv } t = \Br_0 \wedge \Bv, \forall t.
\end{equation*}

%}
%\EndArticle
\EndNoBibArticle
