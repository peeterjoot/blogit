%
% Copyright � 2024 Peeter Joot.  All Rights Reserved.
% Licenced as described in the file LICENSE under the root directory of this GIT repository.
%
%{
\input{../latex/blogpost.tex}
\renewcommand{\basename}{hyperbolicIntegral}
%\renewcommand{\dirname}{notes/phy1520/}
\renewcommand{\dirname}{notes/ece1228-electromagnetic-theory/}
%\newcommand{\dateintitle}{}
%\newcommand{\keywords}{}

\input{../latex/peeter_prologue_print2.tex}

\usepackage{peeters_layout_exercise}
\usepackage{peeters_braket}
\usepackage{peeters_figures}
\usepackage{siunitx}
\usepackage{verbatim}
%\usepackage{macros_cal} % \LL
%\usepackage{amsthm} % proof
%\usepackage{mhchem} % \ce{}
%\usepackage{macros_bm} % \bcM
%\usepackage{macros_qed} % \qedmarker
%\usepackage{txfonts} % \ointclockwise

\beginArtNoToc

\generatetitle{n radical integral}
%\chapter{An radical integral}
%\label{chap:hyperbolicIntegral}

Problem posted on twitter.  Solve:
\begin{equation}\label{eqn:hyperbolicIntegral:20}
I = \int_2^4 \frac{\sqrt{x^2 - 4}}{x^4} dx.
\end{equation}
Let \( x = 2 \cosh u \), so \( dx = 2 \sinh u du \), \( x^2 - 4 = 4 \sinh^2 u \), and
\begin{equation}\label{eqn:hyperbolicIntegral:40}
I = \int_0^{\cosh^{-1}(2)} \frac{4 \sinh^2 u}{16 \cosh^4 u} du.
\end{equation}
After some trial and error (would have been cheating to look it up), we find
\begin{equation}\label{eqn:hyperbolicIntegral:60}
\lr{\tanh u}' = \inv{\cosh^2 u},
\end{equation}
so
\begin{equation}\label{eqn:hyperbolicIntegral:80}
\lr{\tanh^3 u}' = 3 \tanh^2 u \inv{\cosh^2 u} = 3 \frac{\sinh^2 u}{\cosh^4 u}.
\end{equation}
Substituting back, we find
\begin{equation}\label{eqn:hyperbolicIntegral:100}
\begin{aligned}
I &= \inv{12} \lr{ \tanh^3(\cosh^{-1}(2)) - \tanh^3 0 } \\
&= \inv{12} \frac{\sinh^3(\cosh^{-1}(2))}{8}.
\end{aligned}
\end{equation}
Let \( a = \sinh(\cosh^{-1}(2)) \), so
\begin{equation}\label{eqn:hyperbolicIntegral:120}
\begin{aligned}
a^2
&= \sinh^2(\cosh^{-1}(2))  \\
&= \cosh^2(\cosh^{-1}(2)) - 1  \\
&= 3,
\end{aligned}
\end{equation}
or \( \sinh(\cosh^{-1}(2)) = \sqrt{3} \).  Final substitution yields
\begin{equation}\label{eqn:hyperbolicIntegral:140}
\begin{aligned}
I
&= \inv{12 \, 8} 3 \sqrt{3} \\
&= \frac{\sqrt{3}}{32}.
\end{aligned}
\end{equation}


%}
%\EndArticle
\EndNoBibArticle
