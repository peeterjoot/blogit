%
% Copyright � 2020 Peeter Joot.  All Rights Reserved.
% Licenced as described in the file LICENSE under the root directory of this GIT repository.
%
%{
\input{../latex/blogpost.tex}
\renewcommand{\basename}{d4x}
%\renewcommand{\dirname}{notes/phy1520/}
\renewcommand{\dirname}{notes/ece1228-electromagnetic-theory/}
%\newcommand{\dateintitle}{}
%\newcommand{\keywords}{}

\input{../latex/peeter_prologue_print2.tex}

\usepackage{peeters_layout_exercise}
\usepackage{peeters_braket}
\usepackage{peeters_figures}
\usepackage{siunitx}
\usepackage{verbatim}
%\usepackage{mhchem} % \ce{}
%\usepackage{macros_bm} % \bcM
%\usepackage{macros_qed} % \qedmarker
%\usepackage{txfonts} % \ointclockwise

\beginArtNoToc

\generatetitle{XXX}
%\chapter{XXX}
%\label{chap:d4x}
% \citep{sakurai2014modern} pr X.Y
% \citep{pozar2009microwave}
% \citep{qftLectureNotes}
% \citep{doran2003gap}
% \citep{jackson1975cew}
% \citep{griffiths1999introduction}

I think that your construction is over-specified.  You want 4 parameters to parameterize the space time volume.  For example, suppose that you have a vector function of time and space coordinates that spans the integration volume of interest
\begin{equation}\label{eqn:d4x:20}
\Bu = t \Be_0 + x \Be_1 + y \Be_2 + z \Be_3,
\end{equation}
then the natural way to express the differential volume element is just
\begin{equation}\label{eqn:d4x:40}
   \sqrt{g} d^4 x =
   \lr{ \PD{t}{\Bu} dt } \wedge
   \lr{ \PD{x}{\Bu} dx } \wedge
   \lr{ \PD{y}{\Bu} dy } \wedge
   \lr{ \PD{z}{\Bu} dz }
   =
   \Be_0 \Be_1 \Be_2 \Be_3 dt dx dy dz.
\end{equation}
I've included your \( \sqrt{g} \) term on the left hand side since the wedge product of the differentials implicitly builds in the Jacobian structure that I assume this represents.

%}
\EndArticle
%\EndNoBibArticle
