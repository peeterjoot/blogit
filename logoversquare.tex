%
% Copyright � 2024 Peeter Joot.  All Rights Reserved.
% Licenced as described in the file LICENSE under the root directory of this GIT repository.
%
%{
\input{../latex/blogpost.tex}
\renewcommand{\basename}{logoversquare}
%\renewcommand{\dirname}{notes/phy1520/}
\renewcommand{\dirname}{notes/ece1228-electromagnetic-theory/}
%\newcommand{\dateintitle}{}
%\newcommand{\keywords}{}

\input{../latex/peeter_prologue_print2.tex}

\usepackage{peeters_layout_exercise}
\usepackage{peeters_braket}
\usepackage{peeters_figures}
\usepackage{siunitx}
\usepackage{verbatim}
%\usepackage{macros_cal} % \LL
%\usepackage{amsthm} % proof
%\usepackage{mhchem} % \ce{}
%\usepackage{macros_bm} % \bcM
%\usepackage{macros_qed} % \qedmarker
%\usepackage{txfonts} % \ointclockwise

\beginArtNoToc

\generatetitle{XXX}
%\chapter{XXX}
%\label{chap:logoversquare}

Goal is to find
\begin{equation}\label{eqn:logoversquare:20}
I = \int_0^\infty \frac{\ln x}{x^2 + 6 x + 9} dx.
\end{equation}

We can start by observing that
\begin{equation}\label{eqn:logoversquare:40}
\lr{ \frac{\ln x}{x + 3} }' =
-\frac{\ln x}{\lr{x+3}^2} + \inv{x\lr{x + 3}},
\end{equation}
so the indefinite integral is
\begin{equation}\label{eqn:logoversquare:60}
\begin{aligned}
\int \frac{\ln x}{\lr{x + 3}^2} dx
&=
-\frac{\ln x}{x + 3} + \int \inv{x\lr{x + 3}} dx \\
&=
-\frac{\ln x}{x + 3} + \inv{3} \int \lr{ \inv{x} - \inv{x+3} } dx \\
&=
-\frac{\ln x}{x + 3} + \inv{3} \lr{ \ln x - \ln\lr{x+3} } \\
&=
\inv{3} \lr{ \frac{x \ln x}{x+3} - \ln\lr{x+3} },
\end{aligned}
\end{equation}
where we performed a partial fractions expansion.

The limit at infinity can be evaluated using l'Hopitals rule, since
\begin{equation}\label{eqn:logoversquare:80}
\inv{3} \lr{ \frac{x \ln x}{x+3} - \ln\lr{x+3} }
=
\inv{3} \frac{ x \ln x - \lr{ x + 3} \ln \lr{ x + 3} }{ x+3 },
\end{equation}
has the required \(\infty/\infty\) form.  This gives
\begin{equation}\label{eqn:logoversquare:100}
\begin{aligned}
\lim_{x\rightarrow \infty}
\inv{3} \frac{ x \ln x - \lr{ x + 3} \ln \lr{ x + 3} }{ x+3 }
&=
\lim_{x\rightarrow \infty}
\inv{3} \frac{1 + \ln x - \lr{ 1 + ln\lr{x +3 } }}{1} \\
&=
\lim_{x\rightarrow \infty}
-\inv{3} \ln \lr{ 1 + 3/x } \\
&= 0.
\end{aligned}
\end{equation}
For the lower limit we have
\begin{equation}\label{eqn:logoversquare:120}
\lim_{x\rightarrow 0}
\inv{3} \lr{ \frac{x \ln x}{x+3} - \ln\lr{x+3} }=
-\frac{\ln 3}{3},
\end{equation}
so we have \( I = (1/3)\ln 3 \).

%}
%\EndArticle
\EndNoBibArticle
