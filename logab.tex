%
% Copyright � 2024 Peeter Joot.  All Rights Reserved.
% Licenced as described in the file LICENSE under the root directory of this GIT repository.
%
%{
\input{../latex/blogpost.tex}
\renewcommand{\basename}{logab}
%\renewcommand{\dirname}{notes/phy1520/}
\renewcommand{\dirname}{notes/ece1228-electromagnetic-theory/}
%\newcommand{\dateintitle}{}
%\newcommand{\keywords}{}

\input{../latex/peeter_prologue_print2.tex}

\usepackage{peeters_layout_exercise}
\usepackage{peeters_braket}
\usepackage{peeters_figures}
\usepackage{siunitx}
\usepackage{verbatim}
%\usepackage{mhchem} % \ce{}
%\usepackage{macros_bm} % \bcM
%\usepackage{macros_qed} % \qedmarker
%\usepackage{txfonts} % \ointclockwise

\beginArtNoToc

\generatetitle{A funny looking log identity}
%\chapter{A funny looking log identity}
%\label{chap:logab}

On twitter, I saw a \href{https://x.com/AlgebraFact/status/1866217553292300784/photo/1}{funny looking identity}

\begin{equation}\label{eqn:logab:20}
\log_{ab} x = \frac{ \log_a x \log_b x}{\log_a x + \log_b x}.
\end{equation}

To verify this, let
\begin{equation}\label{eqn:logab:40}
\begin{aligned}
u &= \log_a x \\
v &= \log_b x.
\end{aligned}
\end{equation}

This means that
\begin{equation}\label{eqn:logab:60}
\log_{ab} x = \log_{ab} a^u = \log_{ab} b^v.
\end{equation}

We may rewrite either of these in terms of \( a b \), for example
\begin{equation}\label{eqn:logab:80}
\begin{aligned}
\log_{ab} x
&= \log_{ab} b^v \\
&= v \log_{ab} b \\
&= v \log_{ab} \frac{ab}{a} \\
&= v \lr{ 1 - \log_{ab} a },
\end{aligned}
\end{equation}
so
\begin{equation}\label{eqn:logab:100}
u \log_{ab} a = v \lr{ 1 - \log_{ab} a },
\end{equation}
or
\begin{equation}\label{eqn:logab:120}
\lr{ u + v } \log_{ab} a = v,
\end{equation}
or
\begin{equation}\label{eqn:logab:140}
u \log_{ab} a = \frac{u v}{u + v},
\end{equation}
and since \( x = a^u \), our proof is complete.

%}
%\EndArticle
\EndNoBibArticle
