%
% Copyright � 2025 Peeter Joot.  All Rights Reserved.
% Licenced as described in the file LICENSE under the root directory of this GIT repository.
%
%{
\input{../latex/blogpost.tex}
\renewcommand{\basename}{deltaFunctions}
%\renewcommand{\dirname}{notes/phy1520/}
\renewcommand{\dirname}{notes/ece1228-electromagnetic-theory/}
%\newcommand{\dateintitle}{}
%\newcommand{\keywords}{}

\input{../latex/peeter_prologue_print2.tex}

\usepackage{peeters_layout_exercise}
\usepackage{peeters_braket}
\usepackage{peeters_figures}
\usepackage{siunitx}
\usepackage{verbatim}
\usepackage{macros_cal} % \LL
%\usepackage{amsthm} % proof
%\usepackage{mhchem} % \ce{}
%\usepackage{macros_bm} % \bcM
%\usepackage{macros_qed} % \qedmarker
%\usepackage{txfonts} % \ointclockwise

\beginArtNoToc

\generatetitle{Some gradient related Green's functions}
%\chapter{Some gradient related Green's functions}
%\label{chap:deltaFunctions}
Here are the Green's function representations for a few linear differential operators of interest in electrodynamics.  These Green's functions all satisfy
\begin{equation}\label{eqn:deltaFunctions:120}
\delta(\Bx - \Bx') = \LL G(\Bx, \Bx').
\end{equation}

Let \( \Br = \Bx - \Bx' \), \( r = \Norm{\Br} \), \( \rcap = \Br/r \), and \( \tau = t - t' \), then
\begin{enumerate}
\item Divergence operator, \( \LL \Bf = \spacegrad \cdot \Bf \)
\begin{equation}\label{eqn:deltaFunctions:40}
G\lr{ \Bx, \Bx' } = \frac{\rcap}{4 \pi}.
\end{equation}
\item Gradient operator, \( \LL = \spacegrad \)
%FIXME.
\item Laplacian operator, \( \LL = \spacegrad^2 \), in 1D, 2D and 3D respectively
\begin{equation}\label{eqn:deltaFunctions:20}
\begin{aligned}
G\lr{ \Bx, \Bx' } &= \frac{r}{2} \\
G\lr{ \Bx, \Bx' } &= \frac{1}{2 \pi} \ln r \\
G\lr{ \Bx, \Bx' } &= -\frac{1}{4 \pi r}.
\end{aligned}
\end{equation}
\item Second order Helmholtz operator, \( \LL = \spacegrad^2 + k^2 \) for 1D, 2D and 3D respectively
\begin{equation}\label{eqn:deltaFunctions:60}
\begin{aligned}
G\lr{ \Bx, \Bx' } &= -\frac{j}{2k} e^{j k r} \\
G(\Bx, \Bx') &= -\frac{j}{4} H_0^{(1)}(k r) \\
G\lr{ \Bx, \Bx' } &= -\frac{1}{4 \pi} \frac{e^{\pm j k r }}{r}.
\end{aligned}
\end{equation}

\item First order Helmholtz operator, \( \LL = \spacegrad + j k \), in 2D and 3D respectively
\begin{equation}\label{eqn:deltaFunctions:80}
\begin{aligned}
%G\lr{ \Bx, \Bx' } &= \inv{2} \lr{ \Be_1 - 1 } \sgn\lr{ \Be_1 \cdot \lr{\Bx - \Bx'}} e^{j k r} \\
G\lr{ \Bx, \Bx' } &= \frac{k}{4} \lr{ j \rcap H_1^{(1)}(k r) - H_0^{(1)}(k r) } \\
G\lr{ \Bx, \Bx' } &= \frac{e^{-j k r}}{4 \pi r} \lr{ jk \lr{ 1 + \rcap } + \frac{\rcap}{r} }.
\end{aligned}
\end{equation}
This is also the Green's function for a left acting operator \( G(\Bx, \Bx') \lr{ - \lspacegrad + j k } = \delta(\Bx - \Bx') \).
\item Spacetime gradient, \( \LL = \spacegrad + (1/c) \partial_t \), satisfying \( \LL G(\Bx - \Bx', t - t') = \delta(\Bx - \Bx') \delta(t - t') \)
\begin{equation}\label{eqn:deltaFunctions:100}
G( \Br, \tau ) = \inv{4 \pi} \lr{ \frac{\rcap}{r^2} + \inv{c r} \lr{ 1 + \rcap } } \delta\lr{\tau - \frac{r}{c} } \frac{d}{dt'}.
\end{equation}
\end{enumerate}

%}
%\EndArticle
\EndNoBibArticle
