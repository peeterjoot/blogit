%
% Copyright � 2024 Peeter Joot.  All Rights Reserved.
% Licenced as described in the file LICENSE under the root directory of this GIT repository.
%
%{
\input{../latex/blogpost.tex}
\renewcommand{\basename}{more_feynmans_trick}
%\renewcommand{\dirname}{notes/phy1520/}
\renewcommand{\dirname}{notes/ece1228-electromagnetic-theory/}
%\newcommand{\dateintitle}{}
%\newcommand{\keywords}{}

\input{../latex/peeter_prologue_print2.tex}

\usepackage{peeters_layout_exercise}
\usepackage{peeters_braket}
\usepackage{peeters_figures}
\usepackage{siunitx}
\usepackage{verbatim}
%\usepackage{mhchem} % \ce{}
%\usepackage{macros_bm} % \bcM
%\usepackage{macros_qed} % \qedmarker
%\usepackage{txfonts} % \ointclockwise

\beginArtNoToc

\generatetitle{More derivatives of integrals.}
%\chapter{More integral derivatives}
%\label{chap:more_feynmans_trick}
\section{Motivation.}
I was asked about geometric algebra equivalents for a couple identities found in \citep{kemmer1977vector}, one for line integrals
\begin{equation}\label{eqn:more_feynmans_trick:20}
\ddt{} \int_{C(t)} \Bf \cdot d\Bx =
\int_{C(t)} \lr{
   \PD{t}{\Bf} + \spacegrad \lr{ \Bv \cdot \Bf } - \Bv \cross \lr{ \spacegrad \cross \Bf }
}
\cdot d\Bx,
\end{equation}
and one for area integrals
\begin{equation}\label{eqn:more_feynmans_trick:40}
\ddt{} \int_{S(t)} \Bf \cdot d\BA =
\int_{S(t)} \lr{
   \PD{t}{\Bf} + \Bv \lr{ \spacegrad \cdot \Bf } - \spacegrad \cross \lr{ \Bv \cross \Bf }
}
\cdot d\BA.
\end{equation}

Both of these look questionable at first glance, because neither has boundary term.  However, they can be transformed with Stokes theorem to
\begin{equation}\label{eqn:more_feynmans_trick:60}
\ddt{} \int_{C(t)} \Bf \cdot d\Bx
=
\int_{C(t)} \lr{
   \PD{t}{\Bf} - \Bv \cross \lr{ \spacegrad \cross \Bf }
}
\cdot d\Bx
+
\evalbar{\Bv \cdot \Bf }{\Delta C},
\end{equation}
and
\begin{equation}\label{eqn:more_feynmans_trick:80}
\ddt{} \int_{S(t)} \Bf \cdot d\BA =
\int_{S(t)} \lr{
   \PD{t}{\Bf} + \Bv \lr{ \spacegrad \cdot \Bf }
}
\cdot d\BA
-
\oint_{\partial S(t)} \lr{ \Bv \cross \Bf } \cdot d\Bx.
\end{equation}
The area integral derivative is now seen to be a variation of one of the special cases of the Leibniz integral rule, see for example \citep{enwiki:1223666713}.  The author admits that the line integral relationship is not well used, and doesn't show up in the wikipedia page.

My end goal will be to evaluate the derivative of a general multivector line integral
\begin{equation}\label{eqn:more_feynmans_trick:100}
\ddt{} \int_{C(t)} F d\Bx G,
\end{equation}
and area integral
\begin{equation}\label{eqn:more_feynmans_trick:120}
\ddt{} \int_{S(t)} F d^2\Bx G.
\end{equation}
We've derived that line integral result in a different fashion previously, but it's interesting to see a different approach.  Perhaps this approach will lend itself nicely to non-scalar integrands?
\section{Time derivative of a scalar line integral.}
It is worthwhile to work through the analysis from the text in question for the (scalar) line integral case in detail.  I'll do so here, but will handle some of the details a bit differently.

%}
\EndArticle
