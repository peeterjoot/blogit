%
% Copyright � 2020 Peeter Joot.  All Rights Reserved.
% Licenced as described in the file LICENSE under the root directory of this GIT repository.
%
%{
\input{../latex/blogpost.tex}
\renewcommand{\basename}{cramersga}
%\renewcommand{\dirname}{notes/phy1520/}
\renewcommand{\dirname}{notes/ece1228-electromagnetic-theory/}
%\newcommand{\dateintitle}{}
%\newcommand{\keywords}{}

\input{../latex/peeter_prologue_print2.tex}

\usepackage{peeters_layout_exercise}
\usepackage{peeters_braket}
\usepackage{peeters_figures}
\usepackage{siunitx}
\usepackage{verbatim}
%\usepackage{mhchem} % \ce{}
%\usepackage{macros_bm} % \bcM
%\usepackage{macros_qed} % \qedmarker
%\usepackage{txfonts} % \ointclockwise

\beginArtNoToc

\generatetitle{Cramer's rule in geometric algebra}
%\chapter{Cramer's rule in geometric algebra}
%\label{chap:cramersga}

I've been asked to review a paper that uses linear algebra methods to evaluate repeated wedge products.  Part of the context presented is the use of wedge products to solve linear systems.
We learned Cramer's rule in high school without derivation, but the derivation is quite simple, and it just follows from the use of repeated wedge products.

In case you've forgotten, the Cramer's rule for the linear system
\begin{equation}\label{eqn:cramersga:120}
\Ba_1 x_1 + \Ba_2 x_2 \cdots + \Ba_n x_n = \Bb,
\end{equation}
is to form the ratio of determinants
\begin{equation}\label{eqn:cramersga:140}
\frac{
\begin{vmatrix}
   \Ba_1 & \Ba_2 & \cdots & \Ba_k & \cdots & \Ba_n \\
\end{vmatrix}
}
{
\begin{vmatrix}
   \Ba_1 & \Ba_2 & \cdots & \Ba_k & \cdots & \Ba_n \\
\end{vmatrix}
},
\end{equation}
and if you want to solve for \( x_k \), then replace \( \Ba_k \) with \( \Bb \) in the numerator
\begin{equation}\label{eqn:cramersga:160}
   x_k =
\frac{
\begin{vmatrix}
   \Ba_1 & \Ba_2 & \cdots & \Bb & \cdots & \Ba_n \\
\end{vmatrix}
}
{
\begin{vmatrix}
   \Ba_1 & \Ba_2 & \cdots & \Ba_k & \cdots & \Ba_n \\
\end{vmatrix}
}.
\end{equation}
The way that I knew of connecting the wedge product to linear system solution was as follows.  Given a linear system such as
\begin{equation}\label{eqn:cramersga:20}
\begin{aligned}
   a_1 x + b_1 y &= c_1 \\
   a_2 x + b_2 y &= c_2,
\end{aligned}
\end{equation}
we can restate the problem as
\begin{equation}\label{eqn:cramersga:40}
\Ba x + \Bb y = \Bc.
\end{equation}
Should we wish to solve for \( x \), we just wedge with \( \Bb \), and to solve for \( y \) we just wedge with \( \Ba \)
\begin{equation}\label{eqn:cramersga:60}
\begin{aligned}
\Ba \wedge \Bb x + \cancel{\Bb\wedge \Bb}  y &= \Bc \wedge \Bb \\
\cancel{\Ba \wedge \Ba} x + \Ba\wedge \Bb  y &= \Ba \wedge \Bc.
\end{aligned}
\end{equation}
If a solution exists, then \( \Ba \wedge \Bb \) is non zero, so we can write
\begin{equation}\label{eqn:cramersga:80}
\begin{aligned}
   x &= \frac{ \Bc \wedge \Bb }{ \Ba \wedge \Bb } \\
   y &= \frac{ \Ba \wedge \Bc }{ \Ba \wedge \Bb },
\end{aligned}
\end{equation}
but this ratio of wedge products is just the determinant of the matrix of these vectors (times \( \Be_{12} \)), so we discover the 2D case of Cramer's rule
\begin{equation}\label{eqn:cramersga:100}
\begin{aligned}
x &= \frac{ \begin{vmatrix} \Bc & \Bb \end{vmatrix} }{ \begin{vmatrix}\Ba & \Bb \end{vmatrix} } \\
y &= \frac{ \begin{vmatrix} \Ba & \Bc \end{vmatrix} }{ \begin{vmatrix}\Ba & \Bb \end{vmatrix} }.
\end{aligned}
\end{equation}

% discord chat:
%\section{foo}
%
%$$\mathbf{a} x + \mathbf{b} y = \mathbf{c}.$$
%
%Should we wish to solve for $ x $, we just wedge with $ \mathbf{y} $, and to solve for $ y $ we just wedge with $ \mathbf{x} $
%
%If a solution exists, then we get:
%
%$$\begin{aligned}   x &= \frac{ \mathbf{c} \wedge \mathbf{b} }{ \mathbf{a} \wedge \mathbf{b} } \\    y &= \frac{ \mathbf{a} \wedge \mathbf{c} }{ \mathbf{a} \wedge \mathbf{b} },\end{aligned}$$
%
%but this ratio of wedge products can be expanded as the respective determinants, yielding the 2D case of Cramer's rule:
%
%$$\begin{aligned} x &= \frac{ \begin{vmatrix} c_1 & b_1 \\ c_2 & b_2 \end{vmatrix} \mathbf{e}_1 \mathbf{e}_2 }{ \begin{vmatrix} a_1 & b_1 \\ a_2 & b_2 \end{vmatrix} \mathbf{e}_1 \mathbf{e}_2 } \\ y &= \frac{ \begin{vmatrix} a_1 & c_1 \\ a_2 & c_2 \end{vmatrix} \mathbf{e}_1 \mathbf{e}_2 }{ \begin{vmatrix} a_1 & b_1 \\ a_2 & b_2 \end{vmatrix} \mathbf{e}_1 \mathbf{e}_2 }\end{aligned}$$
%
%The same idea follows for higher dimensional cases.
%
%}
%\EndArticle
\EndNoBibArticle
