%
% Copyright � 2024 Peeter Joot.  All Rights Reserved.
% Licenced as described in the file LICENSE under the root directory of this GIT repository.
%
%{
\input{../latex/blogpost.tex}
\renewcommand{\basename}{2x2eigen}
\renewcommand{\dirname}{notes/ece1228-electromagnetic-theory/}

\input{../latex/peeter_prologue_print2.tex}

\usepackage{peeters_layout_exercise}
\usepackage{peeters_braket}
\usepackage{peeters_figures}
\usepackage{siunitx}
\usepackage{verbatim}
\usepackage{amsthm} % proof

\beginArtNoToc

\generatetitle{Eigenvalues of 2x2 matrix: another identity seen on twitter.}
%\chapter{Eigenvalues of 2x2 matrix: another identity seen on twitter.}
%\label{chap:2x2eigen}

Here's another interesting looking twitter math post, \href{https://x.com/AlgebraFact/status/1866818581465329805}{this time about 2x2 matrix eigenvalues}:
%https://x.com/AlgebraFact/status/1866818581465329805
%this time about 2x2 matrix eigenvalues

\maketheorem{Eigenvalues of a 2x2 matrix.}{thm:2x2eigen:1}{
Let \( m \) be the mean of the diagonal elements, and \( p \) be the determinant.  The eigenvalues of the matrix are given by
\begin{equation*}
m \pm \sqrt{ m^2 - p }.
\end{equation*}
} % theorem
This is also not hard to verify.
\begin{proof}
Let
\begin{equation}\label{eqn:2x2eigen:20}
A =
\begin{bmatrix}
a & b \\
c & d
\end{bmatrix},
\end{equation}
where we are looking for \( \lambda \) that satisfies the usual zero determinant condition
\begin{equation}\label{eqn:2x2eigen:40}
\begin{aligned}
0
&= \Abs{ A - \lambda I } \\
&=
\begin{vmatrix}
a - \lambda & b \\
c & d - \lambda
\end{vmatrix} \\
&=
\lr{ a - \lambda } \lr{ d - \lambda } - b c \\
&=
a d - b c - \lambda \lr{ a + d } + \lambda^2 \\
&=
\Det{A} - \lambda \Tr{A} + \lambda^2 \\
&=
\lr{ \lambda - \frac{\Tr{A}}{2} }^2 + \Det{A} - \lr{ \frac{\Tr{A}}{2}}^2,
\end{aligned}
\end{equation}
so
\begin{equation}\label{eqn:2x2eigen:60}
\lambda = \frac{\Tr{A}}{2} \pm \sqrt{ \lr{ \frac{\Tr{A}}{2}}^2 - \Det{A} }.
\end{equation}
substitution of the variables in the problem statement finishes the proof.
\end{proof}

Clearly the higher dimensional characteristic equation will also have both a trace and determinant dependency as well, but the cross terms will be messier (and nobody wants to solve cubic or higher equations by hand anyways.)

%}
%\EndArticle
\EndNoBibArticle
