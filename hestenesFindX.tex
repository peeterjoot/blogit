%
% Copyright � 2022 Peeter Joot.  All Rights Reserved.
% Licenced as described in the file LICENSE under the root directory of this GIT repository.
%
%{
\input{../latex/blogpost.tex}
\renewcommand{\basename}{hestenesFindX}
%\renewcommand{\dirname}{notes/phy1520/}
\renewcommand{\dirname}{notes/ece1228-electromagnetic-theory/}
%\newcommand{\dateintitle}{}
%\newcommand{\keywords}{}

\input{../latex/peeter_prologue_print2.tex}

\usepackage{peeters_layout_exercise}
\usepackage{peeters_braket}
\usepackage{peeters_figures}
\usepackage{siunitx}
\usepackage{verbatim}
%\usepackage{mhchem} % \ce{}
%\usepackage{macros_bm} % \bcM
%\usepackage{macros_qed} % \qedmarker
%\usepackage{txfonts} % \ointclockwise

\beginArtNoToc

\generatetitle{XXX}
%\chapter{XXX}
%\label{chap:hestenesFindX}
%A simplified version of part of gabookI/basics/nfcmCh2.tex
To solve
\begin{equation}\label{eqn:hestenesFindX:20}
   s \Bx + \Ba \lr{ \Bx \cdot \Bb } = \Bc,
\end{equation}
for \(\Bx\), we may dot or wedge the whole equation with the constant vectors.  In particular, respectively dotting with \( \Bb \) and wedging with \( \Ba \) gives
\begin{equation}\label{eqn:hestenesFindX:40}
\begin{aligned}
   s \lr{ \Bx \cdot \Bb } + \Ba \cdot \Bb \lr{ \Bx \cdot \Bb } &= \Bc \cdot \Bb \\
   s \lr{ \Bx \wedge \Ba } + \lr{ \Ba \wedge \Ba } \lr{ \Bx \cdot \Bb } = \Bc \wedge \Ba
\end{aligned}
\end{equation}
The first equation can now be solved for \( \Bx \cdot \Bb \), namely
\begin{equation}\label{eqn:hestenesFindX:60}
   \Bx \cdot \Bb = \frac{ \Bc \cdot \Bb }{ s + \Ba \cdot \Bb },
\end{equation}
and the second equation can be solved for \( \Bx \wedge \Ba \), noting that \( \Ba \wedge \Ba = 0 \), to find
\begin{equation}\label{eqn:hestenesFindX:80}
\Bx \wedge \Ba = \inv{s} \Bc \wedge \Ba.
\end{equation}
Now dot this bivector equation with \( \Bb \), noting that
\begin{equation}\label{eqn:hestenesFindX:100}
   \lr{ \Bx \wedge \Ba } \cdot \Bb
   =
   \Bx \lr{ \Ba \cdot \Bb } - \lr{ \Bx \cdot \Bb } \Ba,
\end{equation}
or
\begin{equation}\label{eqn:hestenesFindX:120}
\Bx \lr{ \Ba \cdot \Bb } = \lr{ \Bx \cdot \Bb } \Ba + \inv{s} \lr{ \Bc \wedge \Ba } \cdot \Bb
\end{equation}
Substitution of \( \Bx \cdot \Bb \) and some rearrangement solves for \( \Bx \).

%}
%\EndArticle
\EndNoBibArticle
