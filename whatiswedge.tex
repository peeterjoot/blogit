%
% Copyright � 2022 Peeter Joot.  All Rights Reserved.
% Licenced as described in the file LICENSE under the root directory of this GIT repository.
%
%{
\input{../latex/blogpost.tex}
\renewcommand{\basename}{whatiswedge}
%\renewcommand{\dirname}{notes/phy1520/}
\renewcommand{\dirname}{notes/ece1228-electromagnetic-theory/}
%\newcommand{\dateintitle}{}
%\newcommand{\keywords}{}

\input{../latex/peeter_prologue_print2.tex}

\usepackage{peeters_layout_exercise}
\usepackage{peeters_braket}
\usepackage{peeters_figures}
\usepackage{siunitx}
\usepackage{verbatim}
%\usepackage{mhchem} % \ce{}
%\usepackage{macros_bm} % \bcM
%\usepackage{macros_qed} % \qedmarker
%\usepackage{txfonts} % \ointclockwise

\beginArtNoToc

\generatetitle{XXX}
%\chapter{XXX}
%\label{chap:whatiswedge}

% # What is the wedge product?

You state ``I can't understand whether the wedge product is supposed to generate a matrix, a vector, or even an area?''

The wedge product of two vectors is not a matrix, nor a vector, but a bivector.  This product can represent an area, but it is a not a scalar area.  Instead, one would say that it is a sign area, and also has a specific orientation in space.

One need not assume a standard (or even Euclidian) basis to compute the wedge product, however can be helpful to expand the wedge product with respect to some basis, say \( \setlr{ \Be_k } \).
\begin{equation*}
\begin{aligned}
\Bx &= \sum_{i = 1, N} x^i \Be_i \\
\By &= \sum_{i = 1, N} y^i \Be_i,
\end{aligned}
\end{equation*}
where, upper indexes (which are not powers) are used for coordinates. The wedge of the two is

\begin{equation*}
\begin{aligned}
\Bx \wedge \By
&= \lr{ \sum_{i = 1, N} x^i \Be_i } \wedge \lr{ \sum_{j = 1, N} y^j \Be_j } \\
&= \sum_{i,j = 1, N} x^i y^j \lr{ \Be_i \wedge \Be_j } \\
&= \sum_{i \ne j} x^i y^j \lr{ \Be_i \wedge \Be_j } \\
&= \lr{ \sum_{i > j} + \sum_{j > i} } x^i y^j \lr{ \Be_i \wedge \Be_j }
\end{aligned}
\end{equation*}

We can change dummy indexes in the second sum:
\begin{equation*}
\begin{aligned}
\sum_{j > i} x^i y^j \lr{ \Be_i \wedge \Be_j }
&=
\sum_{{j'} > {i'}} x^{i'} y^{j'} \lr{ \Be_{i'} \wedge \Be_{j'} }  \\
&=
\sum_{i > j} x^j y^i \lr{ \Be_j \wedge \Be_i }  \\
&=
-\sum_{i > j} x^j y^i \lr{ \Be_i \wedge \Be_j }.
\end{aligned}
\end{equation*}

Putting the pieces together we have
\begin{equation*}
\begin{aligned}
\Bx \wedge \By
&= \sum_{i > j} \lr{ x^i y^j - x^j y^i } \Be_i \wedge \Be_j \\
&= \sum_{i > j} \begin{vmatrix} x^i & x^j \\ y^i & y^j \end{vmatrix} \Be_i \wedge \Be_j.
\end{aligned}
\end{equation*}


%}
\EndArticle
%\EndNoBibArticle
