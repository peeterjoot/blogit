%
% Copyright � 2025 Peeter Joot.  All Rights Reserved.
% Licenced as described in the file LICENSE under the root directory of this GIT repository.
%
%{
\input{../latex/blogpost.tex}
\renewcommand{\basename}{waveEquationGreens}
%\renewcommand{\dirname}{notes/phy1520/}
\renewcommand{\dirname}{notes/ece1228-electromagnetic-theory/}
%\newcommand{\dateintitle}{}
%\newcommand{\keywords}{}

\input{../latex/peeter_prologue_print2.tex}

\usepackage{peeters_layout_exercise}
\usepackage{peeters_braket}
\usepackage{peeters_figures}
\usepackage{siunitx}
\usepackage{verbatim}
%\usepackage{macros_cal} % \LL
%\usepackage{amsthm} % proof
%\usepackage{mhchem} % \ce{}
%\usepackage{macros_bm} % \bcM
%\usepackage{macros_qed} % \qedmarker
%\usepackage{txfonts} % \ointclockwise

\beginArtNoToc

\generatetitle{Green's function for the wave equation}
%\chapter{Green's function for the wave equation}
%\label{chap:waveEquationGreens}
The Green's function(s) \( G(\Br, \tau) \) for the 3D wave equation
\begin{equation}\label{eqn:waveEquationGreens:40}
\lr{ \spacegrad^2 - \inv{c^2}\frac{\partial^2}{\partial t^2} } G(\Br, \tau) = \delta(\Br) \delta(\tau),
\end{equation}
where
\begin{equation}\label{eqn:waveEquationGreens:20}
\begin{aligned}
\Br &= \Bx - \Bx' \\
r &= \Abs{\Br} \\
\tau &= t - t',
\end{aligned}
\end{equation}
is
\begin{equation}\label{eqn:waveEquationGreens:60}
G(\Br, \tau) = -\inv{4 \pi r} \delta( \pm \tau - r/c ).
\end{equation}
Here the positive case is the retarded solution, and negative the advanced solution.  The derivation of these Green's functions can be found derived in many places, including \citep{byron1992mca}, \citep{jackson1975cew}, and \citep{schwinger1998classical}

I wasn't familiar with the 1D and 2D Green's functions for the wave equation.  Grok says they are, respectively
\begin{equation}\label{eqn:waveEquationGreens:80}
\begin{aligned}
G(\Br, \tau) &= -\frac{c}{2} \Theta( \pm \tau - r/c ) \\
G(\Br, \tau) &= -\inv{2 \pi \sqrt{ \tau^2 - r^2/c^2 } } \Theta( \pm \tau - r/c ).
\end{aligned}
\end{equation}
I'm not going to try to derive these myself, at least for the time being.  Instead, let's verify that they work, starting with the retarded 1D case.
\section{1D Green's function verification.}
We will use the Heaviside theta representation of the absolute value.
\begin{equation}\label{eqn:waveEquationGreens:100}
\Abs{x} = x \Theta(x) - x \Theta(-x).
\end{equation}
Recall that the derivative of the absolute value function is a sign function
\begin{equation}\label{eqn:waveEquationGreens:120}
\begin{aligned}
\Abs{x}'
&= \Theta(x) - \Theta(-x) + x \delta(x) + x \delta(-x) \\
&= \Theta(x) - \Theta(-x) + 2 x \delta(x) \\
&= \Theta(x) - \Theta(-x) \\
&= \sgn(x),
\end{aligned}
\end{equation}
where \( x \delta(x) \) is zero in a distributional sense (zero if applied to a test function.)
\begin{equation}\label{eqn:waveEquationGreens:140}
\begin{aligned}
\sgn(x)'
&= \Theta(x)' - \Theta(-x)' \\
&= \delta(x) + \delta(-x) \\
&= 2 \delta(x).
\end{aligned}
\end{equation}

Now let's evaluate the \( x \) partials.
\begin{equation}\label{eqn:waveEquationGreens:160}
\begin{aligned}
\PD{x}{} \Theta(\tau - r/c)
&=
-\inv{c} \delta\lr{ \tau - r/c } \PD{x}{} \Abs{x - x'} \\
&=
-\inv{c} \delta\lr{ \tau - r/c } \sgn(x - x').
\end{aligned}
\end{equation}
The second derivative is
\begin{equation}\label{eqn:waveEquationGreens:180}
\begin{aligned}
\frac{\partial^2}{\partial x^2} \Theta(\tau - r/c)
&=
-\inv{c}
\lr{
    -\inv{c} \delta'\lr{ \tau - r/c } (\sgn(x - x'))^2
    +
    \delta\lr{ \tau - r/c } 2 \delta(x - x')
} \\
&=
\inv{c^2} \delta'\lr{ \tau - r/c } - \frac{2}{c} \delta\lr{ \tau} \delta(x - x').
\end{aligned}
\end{equation}
The transformation above from \( \delta\lr{ \tau - r/c } \rightarrow \delta(\tau) \) is because the spatial delta function \( \delta(x - x') \) is zero unless \( x = x' \), and \( r = 0 \) at that point.

The time derivatives are easier to compute
\begin{equation}\label{eqn:waveEquationGreens:200}
\begin{aligned}
\frac{\partial^2}{\partial t^2} \Theta(\tau - r/c)
&=
\PD{t}{} \delta(\tau - r/c) \\
&=
\delta'(\tau - r/c).
\end{aligned}
\end{equation}

Putting the pieces together, we have
\begin{equation}\label{eqn:waveEquationGreens:220}
\begin{aligned}
\lr{ \spacegrad^2 - \inv{c^2}\frac{\partial^2}{\partial t^2} } \Theta(\tau - r/c)
&=
\inv{c^2} \delta'\lr{ \tau - r/c } - \frac{2}{c} \delta\lr{ \tau} \delta(x - x')
- \inv{c^2} \delta'(\tau - r/c)
\\
&=
- \frac{2}{c} \delta\lr{ \tau} \delta(x - x').
\end{aligned}
\end{equation}
Dividing through by \( -2/c \) gives us
\begin{equation}\label{eqn:waveEquationGreens:240}
\lr{ \spacegrad^2 - \inv{c^2}\frac{\partial^2}{\partial t^2} } G(\Bx - \Bx', t - t') = \delta\lr{t - t'} \delta\lr{\Bx - \Bx'},
\end{equation}
as desired.  The \( \delta \) derivative terms can be given meaning, but they conveniently cancel out, so we don't have to think about that this time.

It's easy to see that the advanced Green's function has the same behaviour, since the two time partials will bring down a factor of \( (\pm 1)^2 = 1 \) in general, which does not change anything above.
\section{Attempted verification of the claimed 2D Green's function.}
Now let's try to verify Grok's claim for the 2D Green's function, starting with a few helpful side calculations.

\begin{equation}\label{eqn:waveEquationGreens:260}
\begin{aligned}
\spacegrad \Abs{r}
&= \Be_m \partial_m \sqrt{ \sum_n \lr{x_n - x_n'}^2 } \\
&= \inv{2} 2 \frac{\Bx - \Bx'}{\Abs{\Bx - \Bx'}} \\
&= \rcap
\end{aligned}
\end{equation}

\begin{equation}\label{eqn:waveEquationGreens:280}
\begin{aligned}
\spacegrad \lr{ \tau^2 - r^2/c^2 }^{-1/2}
&=
-\inv{2} \lr{ \tau^2 - r^2/c^2 }^{-3/2} \lr{-\frac{2 r}{c^2}} \spacegrad r \\
&=
-\inv{2} \lr{ \tau^2 - r^2/c^2 }^{-3/2} \lr{-\frac{2 r}{c^2}} \rcap \\
&=
\frac{r}{c^2} \lr{ \tau^2 - r^2/c^2 }^{-3/2} \rcap.
\end{aligned}
\end{equation}

\begin{equation}\label{eqn:waveEquationGreens:300}
\begin{aligned}
\spacegrad \lr{ \tau^2 - r^2/c^2 }^{-3/2}
&=
-\frac{3}{2} \lr{ \tau^2 - r^2/c^2 }^{-5/2} \lr{-\frac{2 r}{c^2}} \spacegrad r \\
&=
-\frac{3}{2} \lr{ \tau^2 - r^2/c^2 }^{-5/2} \lr{-\frac{2 r}{c^2}} \rcap \\
&=
\frac{3 r}{c^2} \lr{ \tau^2 - r^2/c^2 }^{-5/2} \rcap.
\end{aligned}
\end{equation}

\begin{equation}\label{eqn:waveEquationGreens:320}
\begin{aligned}
\spacegrad \Theta\lr{ \pm \tau - r/c }
&=
-\inv{c} \delta\lr{ \pm \tau - r/c } \spacegrad r \\
&=
-\inv{c} \delta\lr{ \pm \tau - r/c } \rcap.
\end{aligned}
\end{equation}

\begin{equation}\label{eqn:waveEquationGreens:340}
\begin{aligned}
\spacegrad \delta\lr{ \pm \tau - r/c }
&=
-\inv{c} \delta'\lr{ \pm \tau - r/c } \spacegrad r \\
&=
-\inv{c} \delta'\lr{ \pm \tau - r/c } \rcap.
\end{aligned}
\end{equation}

\begin{equation}\label{eqn:waveEquationGreens:360}
\begin{aligned}
\spacegrad \cdot \rcap
&=
\spacegrad \cdot \frac{\Bx - \Bx'}{r} \\
&=
\inv{r} \spacegrad \cdot \lr{\Bx - \Bx'} + \lr{\Bx - \Bx'} \cdot \spacegrad \inv{r} \\
&=
\frac{2}{r} + \lr{\Bx - \Bx'} \cdot \lr{ -\inv{r^2} \rcap } \\
&=
\frac{2}{r} - \inv{r} \\
&=
\frac{1}{r},
\end{aligned}
\end{equation}
or, in summary, with \( X = \tau^2 - r^2/c^2 \)
\begin{equation}\label{eqn:waveEquationGreens:540}
\begin{aligned}
\spacegrad \Abs{r} &= \rcap \\
\spacegrad X^{-1/2} &= \inv{c^2} r \rcap X^{-3/2} \\
\spacegrad X^{-3/2} &= \inv{c^2} 3 r \rcap X^{-5/2} \\
\spacegrad \Theta &= - \inv{c} \delta \rcap \\
\spacegrad \delta &= - \inv{c} \rcap \delta' \\
\spacegrad \cdot \rcap &= \frac{1}{r}.
\end{aligned}
\end{equation}

We want a couple helper Laplacians too
\begin{equation}\label{eqn:waveEquationGreens:580}
\begin{aligned}
\spacegrad^2 X^{-1/2}
&=
\spacegrad \cdot \lr{ \inv{c^2} r \rcap X^{-3/2} } \\
&=
  \inv{c^2} \lr{ \spacegrad \cdot \rcap} \lr{ r X^{-3/2} }
+ \inv{c^2} \lr{ \rcap \cdot \spacegrad r } X^{-3/2}
+ \frac{r}{c^2} \lr{ \rcap \cdot \spacegrad  X^{-3/2} } \\
&=
  \inv{c^2} X^{-3/2}
+ \inv{c^2} X^{-3/2}
+ \frac{r}{c^2} \lr{ \inv{c^2} 3 r X^{-5/2} } \\
&=
  \frac{2}{c^2} X^{-3/2}
+ \frac{3 r^2}{c^4} X^{-5/2} \\
\end{aligned}
\end{equation}

The Laplacian of the step is
\begin{equation}\label{eqn:waveEquationGreens:600}
\begin{aligned}
\spacegrad^2 \Theta
&=
\spacegrad \cdot \lr{ - \inv{c} \delta \rcap } \\
&=
-\inv{c}
\lr{ \spacegrad \cdot \rcap } \delta
-\inv{c}
\rcap \cdot \spacegrad \delta \\
&=
-\inv{r c} \delta
-\inv{c}
\rcap \cdot \lr{
- \inv{c} \rcap \delta'
}
&=
-\inv{r c} \delta
+\inv{c^2} \delta'
\end{aligned}
\end{equation}

We are now ready to compute the Laplacian of \( \Theta X^{-1/2} \).  Let's precompute the chain rule for that, so that the rest of the job is just algebra
\begin{equation}\label{eqn:waveEquationGreens:620}
\begin{aligned}
\spacegrad^2 \lr{ f g }
&=
\spacegrad \cdot \lr{ f \spacegrad g }
+
\spacegrad \cdot \lr{ g \spacegrad f } \\
&=
f \spacegrad^2 g + \spacegrad f \cdot \spacegrad g
+
g \spacegrad^2 f + \spacegrad g \cdot \spacegrad f \\
&=
f \spacegrad^2 g + 2 \spacegrad f \cdot \spacegrad g + g \spacegrad^2 f.
\end{aligned}
\end{equation}
We want to sub in
\begin{equation}\label{eqn:waveEquationGreens:640}
\begin{aligned}
\spacegrad^2 \Theta &= -\inv{r c} \delta +\inv{c^2} \delta' \\
\spacegrad^2 X^{-1/2} &= \frac{2}{c^2} X^{-3/2} + \frac{3 r^2}{c^4} X^{-5/2} \\
\spacegrad X^{-1/2} &= \inv{c^2} r \rcap X^{-3/2} \\
\spacegrad \Theta &= - \inv{c} \delta \rcap.
\end{aligned}
\end{equation}
We get
\begin{equation}\label{eqn:waveEquationGreens:660}
\begin{aligned}
\spacegrad^2 \lr{ \Theta X^{-1/2} }
&=
\lr{ -\inv{r c} \delta +\inv{c^2} \delta' } X^{-1/2}
+ \lr{ \frac{2}{c^2} X^{-3/2} + \frac{3 r^2}{c^4} X^{-5/2} } \Theta
- 2  \inv{c^2} r X^{-3/2} \inv{c} \delta \\
&=
\inv{c^2} X^{-1/2} \delta'
+ \inv{c^2} \lr{ 2  \lr{\tau^2 - r^2/c^2}  + \frac{3 r^2}{c^2} } X^{-5/2} \Theta
- \inv{r c} \lr{  \tau^2 - r^2/c^2 + 2 r^2/c^2 } X^{-3/2} \delta \\
&=
\inv{c^2} X^{-1/2} \delta'
+ \inv{c^2} \lr{ 2 \tau^2 + \frac{r^2}{c^2} } X^{-5/2} \Theta
- \inv{r c} \lr{  \tau^2 + \frac{r^2}{c^2} } X^{-3/2} \delta
\end{aligned}
\end{equation}

We are ready to evaluate the time derivatives now.  Let's try it the same way with
\begin{equation}\label{eqn:waveEquationGreens:680}
\begin{aligned}
\partial_{tt} \lr{ f g }
&=
\partial_t \lr{ f \partial_t g + g \partial_t f } \\
&=
g \partial_{tt} f
+
f \partial_{tt} g
+ 2 \lr{ \partial_t f } \lr{ \partial_t g }.
\end{aligned}
\end{equation}
A couple of the time partials can be computed by inspection
\begin{equation}\label{eqn:waveEquationGreens:700}
\begin{aligned}
\partial_t \Theta &= \pm \delta \\
\partial_{tt} \Theta &= \lr{\pm 1}^2 \delta' \\
\end{aligned}
\end{equation}
and for the rest, we have
\begin{equation}\label{eqn:waveEquationGreens:720}
\begin{aligned}
\partial_t X^{-1/2}
&=
-\inv{2} X^{-3/2} \partial_t X \\
&=
-\inv{2} X^{-3/2} 2 \tau \\
&=
-\tau X^{-3/2},
\end{aligned}
\end{equation}
and
\begin{equation}\label{eqn:waveEquationGreens:740}
\begin{aligned}
\partial_{tt} X^{-1/2}
&=
- X^{-3/2}
- \tau \partial_t X^{-3/2}  \\
&=
- X^{-3/2}
+ 3 \tau^2 X^{-5/2}.
\end{aligned}
\end{equation}
Assembling the pieces, we have
\begin{equation}\label{eqn:waveEquationGreens:760}
\begin{aligned}
\partial_{tt} \lr{ \Theta X^{-1/2} }
&=
\lr{
    - X^{-3/2}
    + 3 \tau^2 X^{-5/2}
} \Theta
+
\delta' X^{-1/2}
+ 2 \lr{  \pm \delta } \lr{ -\tau X^{-3/2} } \\
&=
 \delta' X^{-1/2}
+ \lr{ -\lr{ \tau^2 - r^2/c^2 } + 3 \tau^2 } X^{-5/2} \Theta
\mp 2 \tau X^{-3/2} \delta \\
&=
 \delta' X^{-1/2}
+ \lr{ 2 \tau^2 + r^2/c^2 } X^{-5/2} \Theta
\mp 2 \tau X^{-3/2} \delta.
\end{aligned}
\end{equation}

The wave equation operation on \( \Theta X^{-1/2} \) is
\begin{equation}\label{eqn:waveEquationGreens:780}
\begin{aligned}
\lr{ \spacegrad^2 - (1/c^2) \partial_{tt} } \Theta X^{-1/2}
&=
  \inv{c^2} \lr{ 2 \tau^2 + \frac{r^2}{c^2} } X^{-5/2} \Theta
- \inv{r c} \lr{  \tau^2 + \frac{r^2}{c^2} } X^{-3/2} \delta \\
&- \inv{c^2} \lr{ 2 \tau^2 + r^2/c^2 } X^{-5/2} \Theta
\pm \frac{2}{c^2} \tau X^{-3/2} \delta \\
&=
- \inv{r c} \lr{  \tau^2 + \frac{r^2}{c^2} } X^{-3/2} \delta
\pm \frac{2}{c^2} \tau X^{-3/2} \delta \\
&=
\inv{c^2} \lr{
    - \frac{c \tau^2}{r}
    - \frac{r}{c}
    \pm 2 \tau
}
X^{-3/2} \delta.
\end{aligned}
\end{equation}

So, after all that we have
\begin{equation}\label{eqn:waveEquationGreens:800}
\lr{ \spacegrad^2 - (1/c^2) \partial_{tt} } G =
-\inv{2 \pi c^2} \lr{
    - \frac{c \tau^2}{r}
    - \frac{r}{c}
    \pm 2 \tau
}
\frac{\delta(\pm \tau - r/c)}{\lr{\tau^2 - r^2/c^2}^{3/2}}.
\end{equation}

This is a very problematic expression.  The delta function is zero everywhere but \( \pm \tau = r/c \), but the denominator blows up at \( \pm \tau = r/c \), and the leading factor is also zero at that point:
\begin{equation}\label{eqn:waveEquationGreens:820}
\begin{aligned}
\evalbar{ \lr{ -\frac{c}{r} \tau^2 - \frac{r}{c} \pm 2 \tau }}{\pm \tau = r/c}
&=
-\frac{c}{r} \lr{ \frac{r}{c} }^2 - \frac{r}{c} + 2 \frac{r}{c} \\
&=
0.
\end{aligned}
\end{equation}
So, we've computed something that has a \( 0 \times \infty / 0 \) structure at \( \pm \tau = r/c \).  The question is whether or not that's exactly the infinity that we'd have for \( \delta(x - x') \delta(y - y') \delta(t - t') \) at that point?  Attempting to describe this physically, I think that we'd say that we have discovered that a constant velocity wave of this form has to propagate on the ``light cone''.  Is there a better representation of the 2D Green's function than this one?  I think it's time to look up some more advanced handling of Green's function to get a better handle on this.  I'd guess that there's a Green's function for the 2D wave equation related to Bessel functions, like that of the 2D Helmholtz operator.

%}
\EndArticle
