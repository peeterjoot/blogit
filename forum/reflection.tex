%
% Copyright � 2021 Peeter Joot.  All Rights Reserved.
% Licenced as described in the file LICENSE under the root directory of this GIT repository.
%
%{
\input{../latex/blogpost.tex}
\renewcommand{\basename}{reflection}
%\renewcommand{\dirname}{notes/phy1520/}
\renewcommand{\dirname}{notes/ece1228-electromagnetic-theory/}
%\newcommand{\dateintitle}{}
%\newcommand{\keywords}{}

\input{../latex/peeter_prologue_print2.tex}

\usepackage{peeters_layout_exercise}
\usepackage{peeters_braket}
\usepackage{peeters_figures}
\usepackage{siunitx}
\usepackage{verbatim}
%\usepackage{mhchem} % \ce{}
%\usepackage{macros_bm} % \bcM
%\usepackage{macros_qed} % \qedmarker
%\usepackage{txfonts} % \ointclockwise

\beginArtNoToc

\generatetitle{Reflection.}
The geometric algebra reflection expression is based on the following illustration

Observe that the projection of \( \Bv \) along the direction \( \ucap \) (assumed to be a unit vector for simplicity) is
\begin{equation}\label{eqn:reflection:20}
(\ucap \cdot \Bv) \ucap,
\end{equation}
and the component of \( \Bv \) perpendicular to \( \ucap \) (the rejection) is:
\begin{equation}\label{eqn:reflection:40}
\Bv - (\ucap \cdot \Bv) \ucap.
\end{equation}

From the diagram, it is clear that the reflection is just
\begin{equation}\label{eqn:reflection:60}
\begin{aligned}
\Bv'
&= \Bv - 2 \lr{ \Bv - (\ucap \cdot \Bv) \ucap }  \\
&= 2 (\ucap \cdot \Bv) \ucap - \Bv.
\end{aligned}
\end{equation}

There is no use of geometric algebra in this.  A geometric algebra is a vector space where the elements of the space are products of vectors, subject to an additional ``contraction identity'' \(\Bx^2 = \Bx \cdot \Bx\).
One can form various special product combinations that have useful applications.  In particular, it can be shown that a symmetric sum is related to the dot product
\begin{equation}\label{eqn:reflection:80}
\Bx \cdot \By = \inv{2} \lr{ \Bx \By + \By \Bx }.
\end{equation}
We can also form an antisymmetric sum
\begin{equation}\label{eqn:reflection:100}
\Bx \wedge \By = \inv{2} \lr{ \Bx \By - \By \Bx },
\end{equation}
where we call \( \wedge \) the wedge operator, an operator that happens to be related to the cross product in 3D.
The concepts above can be used to decompose a vector into projective and rejective components as follows
\begin{equation}\label{eqn:reflection:120}
\begin{aligned}
\Bv
&= \Bv \ucap \ucap \\
&= \lr{ \Bv \ucap } \ucap \\
&= \lr{ \Bv \cdot \ucap + \Bv \wedge \ucap } \ucap \\
&= \lr{ \Bv \cdot \ucap } \ucap + \lr{ \Bv \wedge \ucap } \ucap.
\end{aligned}
\end{equation}
The first term we recognize as the projection, allowing us to identify the remainder as the rejection.  It's also straightforward to show that
\begin{equation}\label{eqn:reflection:140}
\lr{ \lr{ \Bv \cdot \ucap } \ucap } \cdot \lr{ \lr{ \Bv \wedge \ucap } \ucap } = 0,
\end{equation}
satisfying our expections for the projection and rejection.

We can utilize this to compute the reflection, which is
\begin{equation}\label{eqn:reflection:160}
\begin{aligned}
\Bv'
&= \Bv - 2 \lr{ \Bv \wedge \ucap } \ucap \\
&= \Bv \ucap \ucap - 2 \lr{ \Bv \wedge \ucap } \ucap \\
&= \lr{ \Bv \ucap - 2 \lr{ \Bv \wedge \ucap } } \ucap \\
&= \lr{ \Bv \ucap - \lr{ \Bv \ucap - \ucap \Bv } } \ucap \\
&= \ucap \Bv \ucap.
\end{aligned}
\end{equation}
This sandwich of \( \Bv \) between \( \ucap \) is the usual way that reflection is expressed in geometric algebra.

%}
\EndArticle
%\EndNoBibArticle
