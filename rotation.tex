%
% Copyright � 2021 Peeter Joot.  All Rights Reserved.
% Licenced as described in the file LICENSE under the root directory of this GIT repository.
%
%{
\input{../latex/blogpost.tex}
\renewcommand{\basename}{rotation}
%\renewcommand{\dirname}{notes/phy1520/}
\renewcommand{\dirname}{notes/ece1228-electromagnetic-theory/}
%\newcommand{\dateintitle}{}
%\newcommand{\keywords}{}

\input{../latex/peeter_prologue_print2.tex}

\usepackage{peeters_layout_exercise}
\usepackage{peeters_braket}
\usepackage{peeters_figures}
\usepackage{siunitx}
\usepackage{verbatim}
%\usepackage{mhchem} % \ce{}
%\usepackage{macros_bm} % \bcM
%\usepackage{macros_qed} % \qedmarker
%\usepackage{txfonts} % \ointclockwise

\beginArtNoToc

\generatetitle{XXX}
%\chapter{XXX}
%\label{chap:rotation}

I'd suggest popping in some specific vectors to get a feeling for things.  For example, suppose that \( \Bi = \Be_1 \Be_2 \),
so that \( \Bu_\parallel = a \Be_1 + b \Be_2 \), and \( \Bu_\perp = c \Be_3 + \cdots \).

Since
\begin{equation*}
\begin{aligned}
   \Be_1 e^{\Bi \alpha}
   &= \Be_1 \lr{ \cos\alpha + \Be_1 \Be_2 \sin\alpha } \\
   &= \cos\alpha \Be_1 + \sin\alpha \Be_1 \Be_1 \Be_2 \\
   &= \cos\alpha \Be_1 - \sin\alpha \Be_1 \Be_2 \Be_1 \\
   &= \lr{ \cos\alpha - \Bi \sin\alpha } \Be_1 \\
   &= e^{-\Bi \alpha} \Be_1,
\end{aligned}
\end{equation*}
and
\begin{equation*}
\begin{aligned}
   \Be_2 e^{\Bi \alpha}
   &= \Be_2 \lr{ \cos\alpha + \Be_1 \Be_2 \sin\alpha } \\
   &= \cos\alpha \Be_2 + \sin\alpha \Be_2 \Be_1 \Be_2 \\
   &= \lr{ \cos\alpha + \Be_2 \Be_1 \sin\alpha } \Be_2 \\
   &= \lr{ \cos\alpha - \Bi \sin\alpha } \Be_2 \\
   &= e^{-\Bi \alpha} \Be_2.
\end{aligned}
\end{equation*}
You can use a superposition argument to conclude that
\begin{equation*}
   \Bx e^{\Bi \alpha} = e^{-\Bi \alpha} \Bx,
\end{equation*}
for any \( \Bx \in \Span \setlr{\Be_1, \Be_2} \).  In particular \( \Bu_\parallel e^{\Bi\theta/2} = e^{-\Bi\theta/2} \Bu_\parallel \).

For any component of the vector that is perpendicular to the plane of rotation, such as \( \Be_3 \) for example, we have
\begin{equation*}
\begin{aligned}
   \Be_3 e^{\Bi \alpha}
   &= \Be_3 \lr{ \cos\alpha + \Be_1 \Be_2 \sin\alpha } \\
   &= \cos\alpha \Be_3 + \sin\alpha \Be_3 \Be_1 \Be_2 \\
   &= \lr{ \cos\alpha + \Bi \sin\alpha } \Be_3 \\
   &= e^{\Bi \alpha} \Be_3,
\end{aligned}
\end{equation*}
since the sign changes twice as you permute the factors of the trivector (i.e.: \( \Be_3 \Be_1 \Be_2 = - \Be_1 \Be_3 \Be_2 = + \Be_1 \Be_2 \Be_3 \) ).  Again, a superposition argument allows you to conclude that \( \Bu_\perp e^{\Bi\theta/2} = e^{\Bi\theta/2} \Bu_\perp \).

In short, a complex exponential for a rotation in the plane \( \Bi \) commutes with components of the vector that lie off the plane, but conjugate-commutes with components of the vector that lie in the plane.

These two facts allow you to make sense of what you've labelled line 4.

%}
\EndArticle
%\EndNoBibArticle
