%
% Copyright � 2024 Peeter Joot.  All Rights Reserved.
% Licenced as described in the file LICENSE under the root directory of this GIT repository.
%
%{
\input{../latex/blogpost.tex}
\renewcommand{\basename}{eulers}
%\renewcommand{\dirname}{notes/phy1520/}
\renewcommand{\dirname}{notes/ece1228-electromagnetic-theory/}
%\newcommand{\dateintitle}{}
%\newcommand{\keywords}{}

\input{../latex/peeter_prologue_print2.tex}

\usepackage{peeters_layout_exercise}
\usepackage{peeters_braket}
\usepackage{peeters_figures}
\usepackage{siunitx}
\usepackage{verbatim}
%\usepackage{mhchem} % \ce{}
%\usepackage{macros_bm} % \bcM
%\usepackage{macros_qed} % \qedmarker
%\usepackage{txfonts} % \ointclockwise

\beginArtNoToc

\generatetitle{XXX}
%\chapter{XXX}
%\label{chap:eulers}

I'd start by assuming the Taylor series representation of the exponential
\begin{equation}\label{eqn:eulers:20}
\exp(x) = \sum_{k = 0}^\infty \frac{x^k}{k!},
\end{equation}
so
\begin{equation}\label{eqn:eulers:40}
\begin{aligned}
\exp(-i a/2)
&= \sum_{k = 0}^\infty \frac{i^k (-a/2)^k}{k!} \\
&= \sum_{\mbox{\(k\) even}} \frac{i^k (-a/2)^k}{k!}
+  \sum_{\mbox{\(k\) odd}} \frac{i^k (-a/2)^k}{k!} \\
&= \sum_{r = 0}^\infty \frac{i^{2r} (-a/2)^{2r}}{(2r)!}
+  \sum_{r = 0}^\infty \frac{i^{2r + 1} (-a/2)^{2r + 1}}{(2r + 1)!} \\
&= \sum_{r = 0}^\infty \frac{(-1)^{r}(-a/2)^{2r}}{(2r)!}
+ i \sum_{r = 0}^\infty \frac{(-1)^{r} (-a/2)^{2r + 1}}{(2r + 1)!} \\
\end{aligned}
\end{equation}
The even sum has the structure of a cosine, and is in fact a scalar, since $a^{2k} = \Abs{a}^{2k}$.
The odd sum has the structure of a sine, but it is a vector, but can be expressed as a scalar sum, by noting that
$a^{2k+1} = \Abs{a}^{2k} a = \Abs{a}^{2k+1} a / \Abs{a}$.

This leaves us with
\begin{equation}\label{eqn:eulers:60}
\begin{aligned}
&= \sum_{r = 0}^\infty \frac{(-1)^{r}(-\Abs{a}/2)^{2r}}{(2r)!}
+ i \frac{a}{\Abs{a}}\sum_{r = 0}^\infty \frac{(-1)^{r} (-\Abs{a}/2)^{2r + 1}}{(2r + 1)!} \\
&= \cos\lr{-1\Abs{a}/2} + i \frac{a}{\Abs{a}} \sin\lr{ -\Abs{a}/2 } \\
&= \cos\lr{\Abs{a}/2} - i \frac{a}{\Abs{a}} \sin\lr{ \Abs{a}/2 } \\
\end{aligned}
\end{equation}

%}
\EndArticle
%\EndNoBibArticle
