%
% Copyright � 2021 Peeter Joot.  All Rights Reserved.
% Licenced as described in the file LICENSE under the root directory of this GIT repository.
%
%{
\input{../latex/blogpost.tex}
\renewcommand{\basename}{exponentialmultivector}
%\renewcommand{\dirname}{notes/phy1520/}
\renewcommand{\dirname}{notes/ece1228-electromagnetic-theory/}
%\newcommand{\dateintitle}{}
%\newcommand{\keywords}{}

\input{../latex/peeter_prologue_print2.tex}

\usepackage{peeters_layout_exercise}
\usepackage{peeters_braket}
\usepackage{peeters_figures}
\usepackage{siunitx}
\usepackage{verbatim}
%\usepackage{mhchem} % \ce{}
%\usepackage{macros_bm} % \bcM
%\usepackage{macros_qed} % \qedmarker
%\usepackage{txfonts} % \ointclockwise

\beginArtNoToc

\generatetitle{XXX}
%\chapter{XXX}
%\label{chap:exponentialmultivector}

I don't think that you can generally find such an exponential factorization.  However, by the Baker-Campbell-Hausdorff theorem, we have
\begin{equation*}
   e^{A + B + \inv{2} \antisymmetric{A}{B} + \cdots} = e^A e^B,
\end{equation*}
where \( \antisymmetric{A}{B} \), is the commutator, the antisymmetric sum of the objects \( A, B \).
Consider that commutator for your case
\begin{equation*}
\begin{aligned}
\antisymmetric{v}{iw}
&=
v i w - i w v \\
&=
i \lr{ v w - w v} \\
&=
2 i \lr{ v \wedge w }.
\end{aligned}
\end{equation*}
Note that this commutator expansion is specific to \R{3} since the pseudoscalar \( i \) happens to commute with all grades in that case.

This expansion shows that if \( v, w \) are colinear, one must have
\begin{equation*}
e^{v + iw} = e^v e^{iw}.
\end{equation*}
If \( v, w \) were not colinear, but the higher order commutators happended to be zero, one could say
\begin{equation*}
   e^{v + iw + i (v \wedge w)} = e^v e^{iw},
\end{equation*}
but that is not the type of decomposition that you are seeking.

%}
\EndArticle
%\EndNoBibArticle
