\PassOptionsToPackage{square,numbers}{natbib}
%\documentclass[letterpaper]{scrreprt}
\documentclass[letterpaper,oneside]{scrreprt}
\usepackage[margin=1in]{geometry}
\usepackage[svgnames]{xcolor}
\usepackage{amsmath}
\usepackage{caption}
\usepackage{exercise}
\usepackage{graphicx}
\usepackage{ifthen}
\usepackage{parameters}
\usepackage{siunitx}
\usepackage{stackengine}
\usepackage{subfig}
\usepackage{verbatim}
\usepackage{wrapfig}
%\usepackage{peeters_macros}
\usepackage[english]{cleveref}

% \simpleFigure{path}{caption}{label}{width}
\newcommand{\simpleFigure}[4]{%
  \begin{figure}[htp]%
    \centering%
    \includegraphics[width=#4\textwidth]{#1}%
    \caption{#2}%
    \label{#3}%
  \end{figure}%
}

% \sideFigure{path}{caption}{label}{width}[side]
\newcommand{\sideFigure}[5][l]{%
  \begin{wrapfigure}{#1}{#5\textwidth}%
    \centering%
    \includegraphics[width=#5\textwidth]{#2}%
    \caption{#3}%
    \label{#4}%
  \end{wrapfigure}%
}

\begin{document}
\chapter{Circuits Cheatsheet.}
\paragraph{Current Division}
\begin{minipage}{\textwidth}
For a DC current source, such as the one illustrated in \cref{fig:currentSource:currentSourceFig1}, we have
\end{minipage}
\sideFigure{../figures/blogit/currentSourceFig1}{Current Source}{fig:currentSource:currentSourceFig1}{0.3}
\vspace{2\baselineskip}
\begin{equation}\label{eqn:karlCircuitsCheatSheet:20}
I_1 = I_s \frac{R_2}{R_1 + R_2}
\end{equation}
\begin{equation}\label{eqn:karlCircuitsCheatSheet:40}
I_2 = I_s \frac{R_1}{R_1 + R_2}.
\end{equation}
\WFclear
\vspace{4\baselineskip}
\paragraph{Voltage Division}
\begin{minipage}{\textwidth}
For a DC voltage source illustrated in \cref{fig:voltageSource:voltageSourceFig2}, we have
\end{minipage}
\sideFigure{../figures/blogit/voltageSourceFig2}{Voltage Source}{fig:voltageSource:voltageSourceFig2}{0.3}
\vspace{2\baselineskip}
\begin{equation}\label{eqn:karlCircuitsCheatSheet:60}
V = V_2 \frac{ R_2 }{R_1 + R_2}.
\end{equation}
\WFclear

\end{document}
