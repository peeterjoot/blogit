%
% Copyright � 2020 Peeter Joot.  All Rights Reserved.
% Licenced as described in the file LICENSE under the root directory of this GIT repository.
%
%{
\input{../latex/blogpost.tex}
\renewcommand{\basename}{fibonacci}
%\renewcommand{\dirname}{notes/phy1520/}
\renewcommand{\dirname}{notes/ece1228-electromagnetic-theory/}
%\newcommand{\dateintitle}{}
%\newcommand{\keywords}{}

\input{../latex/peeter_prologue_print2.tex}

\usepackage{mathtools}
\DeclarePairedDelimiter\ceil{\lceil}{\rceil}
\DeclarePairedDelimiter\floor{\lfloor}{\rfloor}

\usepackage{amsthm}
\usepackage{peeters_layout_exercise}
\usepackage{peeters_braket}
\usepackage{peeters_figures}
\usepackage{siunitx}
\usepackage{verbatim}
%\usepackage{mhchem} % \ce{}
%\usepackage{macros_bm} % \bcM
%\usepackage{macros_qed} % \qedmarker
%\usepackage{txfonts} % \ointclockwise
%\usepackage{amsmath} % \binom

%\renewcommand{\binom}[2]{{{#1}\choose{#2}}}

\beginArtNoToc

\generatetitle{The nth term of a Fibonacci series.}
%\chapter{The nth term of a Fibonacci series.}
%\label{chap:fibonacci}

I've just started reading \citep{strogatz2009calculus}, but already got distracted from the plot by a fun math fact.  Namely, a cute formula for the nth term of a Fibonacci series.  Recall
\makedefinition{Fibonacci series.}{dfn:fibonacci:20}{
With \( F_0 = 0 \), and \( F_1 = 1 \), the nth term \( F_n \) in the Fibonacci series is the sum of the previous two terms
\begin{equation*}
F_n = F_{n-2} + F_{n-1}.
\end{equation*}
} % definition
We can quickly find that the series has values \( 0, 1, 1, 2, 3, 5, 8, 13, \cdots \).  What's really cool, is that there's a closed form expression for the nth term in the series that doesn't require calculation of all the previous terms.
\maketheorem{Nth term of the Fibonacci series.}{thm:fibonacci:40}{
\begin{equation*}
F_n = \frac{ \lr{ 1 + \sqrt{5} }^n - \lr{ 1 - \sqrt{5} }^n }{ 2^n \sqrt{5} }.
\end{equation*}
} % theorem
This is a rather miraculous and interesting looking equation.  Other than the \(\sqrt{5}\) scale factor, this is exactly the difference of the nth powers of the golden ratio \( \phi = (1+\sqrt{5})/2 \), and \( 1 - \phi = (1-\sqrt{5})/2 \).  That is:
\begin{dmath}\label{eqn:fibonacci:60}
F_n = \frac{\phi^n - (1 -\phi)^n}{\sqrt{5}}.
\end{dmath}

How on Earth would somebody figure this out?
% https://books.google.com/books?id=QGgLbf2oFUYC&pg=PA29
According to \href{https://books.google.com/books?id=QGgLbf2oFUYC&pg=PA29}{Tattersal} \citep{tattersall2005elementary}, this relationship was discovered by Kepler.

Understanding this from the ground up looks like it's a pretty deep rabbit hole to dive into.  Let's save that game for another day, but try the more pedestrian task of proving that this formula works.
\begin{proof}
\begin{dmath}\label{eqn:fibonacci:80}
\sqrt{5} F_n =
\sqrt{5} \lr{ F_{n-2} + F_{n-1} }
=
\phi^{n-2} - \lr{ 1 - \phi}^{n-2}
+ \phi^{n-1} - \lr{ 1 - \phi}^{n-1}
=
\phi^{n-2} \lr{ 1 + \phi }
-\lr{1 - \phi}^{n-2} \lr{ 1 + 1 - \phi }
=
\phi^{n-2}
\frac{ 3 + \sqrt{5} }{2}
-\lr{1 - \phi}^{n-2}
\frac{ 3 - \sqrt{5} }{2}.
\end{dmath}
However,
\begin{dmath}\label{eqn:fibonacci:100}
\phi^2
= \lr{ \frac{ 1 + \sqrt{5} }{2} }^2
= \frac{ 1 + 2 \sqrt{5} + 5 }{4}
= \frac{ 3 + \sqrt{5} }{2},
\end{dmath}
and
\begin{dmath}\label{eqn:fibonacci:120}
(1-\phi)^2
= \lr{ \frac{ 1 - \sqrt{5} }{2} }^2
= \frac{ 1 - 2 \sqrt{5} + 5 }{4}
= \frac{ 3 - \sqrt{5} }{2},
\end{dmath}
so
\begin{dmath}\label{eqn:fibonacci:140}
\sqrt{5} F_n = \phi^n - (1-\phi)^n.
\end{dmath}
\end{proof}
\section{How the square root fives cancel out.}
One of the interesting things in this Fibonacci formula, is the \( \sqrt{5} \)'s that are all over the place, while the formula represents only integer values.  Expanding the formula in binomial series shows us exactly why those terms all vanish.  Consider the first few values of \( n \) explicitly.
\begin{dmath}\label{eqn:fibonacci:160}
F_1
= \frac{ 1 + \sqrt{5} - \lr{ 1 - \sqrt{5} } }{ 2^1 \sqrt{5} }
= \frac{ 2 \sqrt{5} }{ 2^1 \sqrt{5} }
= 1,
\end{dmath}
\begin{dmath}\label{eqn:fibonacci:180}
F_2
= \frac{ 1 + 2 \sqrt{5} + 5 - \lr{ 1 - 2 \sqrt{5} + 5 } }{ 2^2 \sqrt{5} }
= \frac{ 4 \sqrt{5} }{ 2^2 \sqrt{5} }
= 1,
\end{dmath}
\begin{dmath}\label{eqn:fibonacci:200}
F_3
	= \frac{ 1 + 3 \sqrt{5} + 3 (5) + \sqrt{5} 5 - \lr{ 1 - 3 \sqrt{5} + 3(5) - \sqrt{5} 5 } }{ 2^3 \sqrt{5} }
= \frac{ 2 \lr{ 3 \sqrt{5} + \sqrt{5} 5 } }{ 2^3 \sqrt{5} }
= \frac{ 3 + 5 }{ 2^2 }
= 2.
\end{dmath}
In the general case, we have
\begin{dmath}\label{eqn:fibonacci:220}
2^n \sqrt{5} F_n
=
\sum_{k = 0}^n
\binom{n}{k}
{\sqrt{5}}^k
-
\sum_{k = 0}^n \binom{n}{k} (-\sqrt{5})^k
=
2 \sum_{1 \le k \le n, \mbox{$k$ is odd}} \binom{n}{k} (\sqrt{5})^k
=
2 \sqrt{5} \sum_{m = 0}^{\floor*{(n-1)/2}} \binom{n}{2 m + 1} 5^m,
\end{dmath}
% k = 2 m + 1, m = 0, 2 m + 1 <= n ; m <= (n-1)/2
so (for any \( n > 0 \)),
\begin{dmath}\label{eqn:fibonacci:240}
F_n =
\inv{2^{n-1}} \sum_{m = 0}^{\floor*{(n-1)/2}} \binom{n}{2 m + 1} 5^m.
\end{dmath}
Since only the odd powers of \( \sqrt{5} \) in the binomial expansions survive, the root in the basement is obliterated every time, leaving only integers upstairs, and a power of two factor downstairs.
It is still somewhat remarkable seeming that there is always a perfect cancellation of all the factors of two in the basement.
%, but we can see easily why the end result, for any \( n \) has no \( \sqrt{5} \) terms in it.
\section{How the square root fives show up.}
We can rearrange the formula for the nth Fibonacci number as a difference equation
\begin{dmath}\label{eqn:fibonacci:260}
F_n - F_{n-1} = F_{n-2}.
\end{dmath}
This is a second order difference equation, so my naive expectation is that there are two particular solutions involved.  We know the answer, so it's not too hard to guess that the particular form of the solution has the following form
\begin{dmath}\label{eqn:fibonacci:280}
F_n = \alpha a^n + \beta b^n.
\end{dmath}
Given this guess, can we take some of the magic out of the formula, by just solving for \( \alpha, \beta, a, b \)?  Let's try that
\begin{equation}\label{eqn:fibonacci:300}
F_0 = \alpha + \beta = 0,
\end{equation}
\begin{dmath}\label{eqn:fibonacci:320}
F_1 = \alpha a + \beta b
    = \alpha \lr{ a - b } = 1,
\end{dmath}
and
\begin{dmath}\label{eqn:fibonacci:340}
F_n 
= F_{n-1} + F_{n-2}
= 
\alpha \lr{ a^{n-1} + a^{n-2} } 
-\alpha \lr{ b^{n-1} + b^{n-2} } 
=
\alpha a^{n-2} \lr{ 1 + a }
-\alpha b^{n-2} \lr{ 1 + b },
\end{dmath}
so
\begin{dmath}\label{eqn:fibonacci:360}
\begin{aligned}
a^2 &= a + 1 \\
b^2 &= b + 1.
\end{aligned}
\end{dmath}
If we complete the square we find
\begin{dmath}\label{eqn:fibonacci:380}
\lr{ a - \inv{2} }^2 = 1 + \inv{4} = \frac{5}{4},
\end{dmath}
or
\begin{dmath}\label{eqn:fibonacci:400}
a, b = \inv{2} \pm \frac{\sqrt{5}}{2}.
\end{dmath}
Out pop the golden ratio and it's complement.  Clearly we need to pick alternate roots for \( a \) and \( b \) or else we'd have zero for every value of \( n > 0 \).  Suppose we pick the positive root for \( a \), then to find the scaling constant \( \alpha \), we just compute
\begin{dmath}\label{eqn:fibonacci:420}
1 = 
\alpha \lr{ \frac{ 1 + \sqrt{5}}{2} - \frac{ 1 - \sqrt{5} }{2} }
= \alpha \sqrt{5},
\end{dmath}
so our system \cref{eqn:fibonacci:280} has the solution:
\begin{equation}\label{eqn:fibonacci:520}
\begin{aligned}
a &= \frac{1 + \sqrt{5}}{2} \\
b &= \frac{1 - \sqrt{5}}{2} \\
\alpha &= \inv{\sqrt{5}} \\
\beta &= -\inv{\sqrt{5}}.
\end{aligned}
\end{equation}
It would be interesting to study difference equations enough to understand why the guess \cref{eqn:fibonacci:280} works.  However, we see a path that will systematically lead us from the Fibonacci difference equation to the final result, and have only to fill in a few missing steps to understand that path.

\paragraph{Another approach.}
I showed this to Sofia, and she came up with a neat very direct way to motivate the \( \sqrt{5} \).  It follows naturally (again knowing the answer), by assuming the Fibonacci formula has the following form:
\begin{dmath}\label{eqn:fibonacci:440}
F_n = \inv{x} \lr{ 
   \lr{ \frac{1 + x}{2}}^n
   -
   \lr{ \frac{1 - x}{2}}^n
}.
\end{dmath}
We have only to plug in \( n = 3 \) to find
\begin{dmath}\label{eqn:fibonacci:460}
2 x = \inv{4} \lr{ 1 + 3 x + 3 x^2 + x^3 - \lr{ 1 - 3 x + 3 x^2 - x^3 } }
= \inv{2} \lr{ 3 x + x^3 },
\end{dmath}
or
\begin{dmath}\label{eqn:fibonacci:480}
8 = 3 + x^2,
\end{dmath}
so
\begin{dmath}\label{eqn:fibonacci:500}
x = \pm \sqrt{5}.
\end{dmath}
Again the \( \sqrt{5} \)'s pop out naturally, taking away some of the mystery of the cool formula.
%}
\EndArticle
