%
% Copyright � 2021 Peeter Joot.  All Rights Reserved.
% Licenced as described in the file LICENSE under the root directory of this GIT repository.
%
%{
\input{../latex/blogpost.tex}
\renewcommand{\basename}{jordan1inverse}
%\renewcommand{\dirname}{notes/phy1520/}
\renewcommand{\dirname}{notes/ece1228-electromagnetic-theory/}
%\newcommand{\dateintitle}{}
%\newcommand{\keywords}{}

\input{../latex/peeter_prologue_print2.tex}

\usepackage{peeters_layout_exercise}
\usepackage{peeters_braket}
\usepackage{peeters_figures}
\usepackage{siunitx}
\usepackage{verbatim}
%\usepackage{mhchem} % \ce{}
%\usepackage{macros_bm} % \bcM
%\usepackage{macros_qed} % \qedmarker
%\usepackage{txfonts} % \ointclockwise

\beginArtNoToc

\generatetitle{XXX}
%\chapter{XXX}
%\label{chap:jordan1inverse}
% misinterpretation of: https://math.stackexchange.com/questions/4338017/square-roots-of-the-basic-jordan-block-of-order-n-associated-with-the-eigenval

This isn't a full answer, since you were looking for a specific factored form for the Jordan block inverse.

Illustrating by example, with \( n = 5 \), the Jordan block and it's inverse has the following form
\begin{equation*}
\begin{bmatrix}
   1 & 0 & 0 & 0 & 0 \\
   1 & 1 & 0 & 0 & 0 \\
   0 & 1 & 1 & 0 & 0 \\
   0 & 0 & 1 & 1 & 0 \\
   0 & 0 & 0 & 1 & 1 \\
\end{bmatrix}
\begin{bmatrix}
   1 &  0 &  0 &  0 & 0 \\
  -1 &  1 &  0 &  0 & 0 \\
   1 & -1 &  1 &  0 & 0 \\
  -1 &  1 & -1 &  1 & 0 \\
   1 & -1 &  1 & -1 & 1 \\
\end{bmatrix}
= I.
\end{equation*}
I belive this is uniformly true, regardless of the characteristic of the field.

To show that the above is true, you can use a block matrix decomposition
\begin{equation*}
J_n =
\begin{bmatrix}
   1 & 0 \\
   x & J_{n-1} \\
\end{bmatrix}
\end{equation*}
\begin{equation*}
   J_n^{-1} =
\begin{bmatrix}
   1 & 0 \\
   y & J_{n-1}^{-1} \\
\end{bmatrix},
\end{equation*}
and proceed inductively.

%}
\EndArticle
%\EndNoBibArticle
