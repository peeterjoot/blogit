%
% Copyright � 2022 Peeter Joot.  All Rights Reserved.
% Licenced as described in the file LICENSE under the root directory of this GIT repository.
%
%{
\input{../latex/blogpost.tex}
\renewcommand{\basename}{tensorComponents}
%\renewcommand{\dirname}{notes/phy1520/}
\renewcommand{\dirname}{notes/ece1228-electromagnetic-theory/}
%\newcommand{\dateintitle}{}
%\newcommand{\keywords}{}

\input{../latex/peeter_prologue_print2.tex}

\usepackage{peeters_layout_exercise}
\usepackage{peeters_braket}
\usepackage{peeters_figures}
\usepackage{siunitx}
\usepackage{verbatim}
%\usepackage{mhchem} % \ce{}
%\usepackage{macros_bm} % \bcM
%\usepackage{macros_qed} % \qedmarker
%\usepackage{txfonts} % \ointclockwise

\beginArtNoToc

\generatetitle{Verifying the GA form for the symmetric and antisymmetric components of the different rate of strain.}
%\chapter{XXX}
%\label{chap:tensorComponents}

We found geometric algebra representations for the symmetric and antisymmetric components for a gradient-vector direct product.  In particular, given
\begin{equation}\label{eqn:tensorComponents:20}
d\Bv = d\Bx \cdot \lr{ \spacegrad \directproduct \Bv }
\end{equation}
we found
\begin{equation}\label{eqn:tensorComponents:40}
\begin{aligned}
   d\Bx \cdot \Bd
   &=
\inv{2} d\Bx \cdot \lr{
\spacegrad \directproduct \Bv
+
\lr{\spacegrad \directproduct \Bv }^\dagger
} \\
&=
\inv{2} \lr{
   d\Bx \lr{ \spacegrad \cdot \Bv }
   +
   \gpgradeone{ \spacegrad d\Bx \Bv }
},
\end{aligned}
\end{equation}
and
\begin{equation}\label{eqn:tensorComponents:60}
\begin{aligned}
   d\Bx \cdot \BOmega
   &=
\inv{2} d\Bx \cdot \lr{
\spacegrad \directproduct \Bv
-
\lr{\spacegrad \directproduct \Bv }^\dagger
} \\
&=
\inv{2} \lr{
   d\Bx \lr{ \spacegrad \cdot \Bv }
   -
   \gpgradeone{ d\Bx \Bv \spacegrad }
}.
\end{aligned}
\end{equation}

Let's expand each of these in coordinates to verify that these are correct.  For the symmetric component, that is
\begin{equation}\label{eqn:tensorComponents:80}
\begin{aligned}
   d\Bx \cdot \Bd
   &=
   \inv{2}
   \lr{
      dx_i \partial_j v_j \Be_i
      +
      \partial_j dx_i v_k \gpgradeone{ \Be_j \Be_i \Be_k }
   } \\
   &=
   \inv{2} dx_i
   \lr{
      \partial_j v_j \Be_i
      +
      \partial_j v_k \lr{ \delta_{ji} \Be_k + \lr{ \Be_j \wedge \Be_i } \cdot \Be_k }
   } \\
   &=
   \inv{2} dx_i
   \lr{
      \partial_j v_j \Be_i
      +
      \partial_j v_k \lr{ \delta_{ji} \Be_k + \delta_{ik} \Be_j - \delta_{jk} \Be_i }
   } \\
   &=
   \inv{2} dx_i
   \lr{
      \partial_j v_j \Be_i
      +
      \partial_i v_k \Be_k
      +
      \partial_j v_i \Be_j
      -
      \partial_j v_j \Be_i
   } \\
   &=
   \inv{2} dx_i
   \lr{
      \partial_i v_k \Be_k
      +
      \partial_j v_i \Be_j
   } \\
   &=
   dx_i \inv{2} \lr{ \partial_i v_j + \partial_j v_i } \Be_j.
\end{aligned}
\end{equation}
Sure enough, we that the product contains the matrix element of the symmetric component of \( \spacegrad \directproduct \Bv \).

Now let's verify that our GA antisymmetric tensor product representation works out.
\begin{equation}\label{eqn:tensorComponents:100}
\begin{aligned}
   d\Bx \cdot \BOmega
   &=
   \inv{2}
   \lr{
      dx_i \partial_j v_j \Be_i
      -
      dx_i \partial_k v_j \gpgradeone{ \Be_i \Be_j \Be_k }
   } \\
   &=
   \inv{2} dx_i
   \lr{
      \partial_j v_j \Be_i
      -
      \partial_k v_j
      \lr{ \delta_{ij} \Be_k + \delta_{jk} \Be_i - \delta_{ik} \Be_j }
   } \\
   &=
   \inv{2} dx_i
   \lr{
      \partial_j v_j \Be_i
      -
      \partial_k v_i \Be_k
      -
      \partial_k v_k \Be_i
      +
      \partial_i v_j \Be_j
   } \\
   &=
   \inv{2} dx_i
   \lr{
      \partial_i v_j \Be_j
      -
      \partial_k v_i \Be_k
   } \\
   &=
   dx_i
   \inv{2}
   \lr{
      \partial_i v_j
      -
      \partial_j v_i
   }
   \Be_j.
\end{aligned}
\end{equation}
As expected,
we that this product contains the matrix element of the antisymmetric component of \( \spacegrad \directproduct \Bv \).

We also found previously that \( \BOmega \) is just a curl, namely
\begin{equation}\label{eqn:tensorComponents:120}
   \BOmega = \inv{2} \lr{ \spacegrad \wedge \Bv } = \inv{2} \lr{ \partial_i v_j } \Be_i \wedge \Be_j,
\end{equation}
which directly encodes the antisymmetric component of \( \spacegrad \directproduct \Bv \).  We can also see that by fully expanding \( d\Bx \cdot \BOmega \), which gives
\begin{equation}\label{eqn:tensorComponents:140}
\begin{aligned}
   d\Bx \cdot \BOmega
   &=
   dx_i \inv{2} \lr{ \partial_j v_k }
   \Be_i \cdot \lr{ \Be_j \wedge \Be_k } \\
   &=
   dx_i \inv{2} \lr{ \partial_j v_k }
   \lr{
      \delta_{ij} \Be_k
      -
      \delta_{ik} \Be_j
   } \\
   &=
   dx_i \inv{2}
   \lr{
   \lr{ \partial_i v_k } \Be_k
      -
   \lr{ \partial_j v_i }
      \Be_j
   } \\
   &=
   dx_i \inv{2}
   \lr{
      \partial_i v_j - \partial_j v_i
   }
   \Be_j,
\end{aligned}
\end{equation}
as expected.

%}
%\EndArticle
\EndNoBibArticle
