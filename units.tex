%
% Copyright � 2021 Peeter Joot.  All Rights Reserved.
% Licenced as described in the file LICENSE under the root directory of this GIT repository.
%
%{
\input{../latex/blogpost.tex}
\renewcommand{\basename}{units}
%\renewcommand{\dirname}{notes/phy1520/}
\renewcommand{\dirname}{notes/ece1228-electromagnetic-theory/}
%\newcommand{\dateintitle}{}
%\newcommand{\keywords}{}

\input{../latex/peeter_prologue_print2.tex}

\usepackage{peeters_layout_exercise}
\usepackage{peeters_braket}
\usepackage{peeters_figures}
\usepackage{siunitx}
\usepackage{verbatim}
%\usepackage{mhchem} % \ce{}
%\usepackage{macros_bm} % \bcM
%\usepackage{macros_qed} % \qedmarker
%\usepackage{txfonts} % \ointclockwise

\beginArtNoToc

\generatetitle{XXX}
%\chapter{XXX}
%\label{chap:units}
% \citep{sakurai2014modern} pr X.Y
% \citep{pozar2009microwave}
% \citep{qftLectureNotes}
% \citep{doran2003gap}
% \citep{jackson1975cew}
% \citep{griffiths1999introduction}

\begin{comment}
\end{comment}

I think that in geometric algebra, as well as vector algebra, you need the units to ride along with the coordinates and not the basis vectors.  Consider a one dimensional vector space example with unit vector \( \Be_1 \)
\begin{equation*}
   \Bx = a \Be_1.
\end{equation*}
The vector square is
\begin{equation*}
   \Bx \cdot \Bx  = a^2 \Be_1 \cdot \Be_1 = a^2,
\end{equation*}
and the vector length is
\begin{equation*}
   \Norm{\Bx} = \sqrt{\Bx \cdot \Bx} = \Abs{a}.
\end{equation*}
In particular
\begin{equation*}
   \Be_1 = \frac{\Bx}{\Norm{\Bx}}.
\end{equation*}

If, say, \( a = 10\, \textrm{m} \), then
\begin{equation*}
   \Bx \cdot \Bx  = 10\, {\textrm{m}}^2
\end{equation*}
and
\begin{equation*}
   \Norm{\Bx} = 10\, \textrm{m}.
\end{equation*}
then you find that \( \Be_1 = \ifrac{\Bx}{\Norm{\Bx}} \) is dimensionless.

A simpler way to put this might be that we define a unit vector \( \ucap \) such that it's self dot product is a dimensionless scalar one value:
\begin{equation*}
   \ucap \cdot \ucap = 1,
\end{equation*}
and not \( \ucap \cdot \ucap = 1 \times \textrm{some-units}^2 \).

%}
%\EndArticle
\EndNoBibArticle
