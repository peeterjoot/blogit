%
% Copyright � 2021 Peeter Joot.  All Rights Reserved.
% Licenced as described in the file LICENSE under the root directory of this GIT repository.
%
%{
\input{../latex/blogpost.tex}
\renewcommand{\basename}{spinorinversion}
%\renewcommand{\dirname}{notes/phy1520/}
\renewcommand{\dirname}{notes/ece1228-electromagnetic-theory/}
%\newcommand{\dateintitle}{}
%\newcommand{\keywords}{}

\input{../latex/peeter_prologue_print2.tex}

\usepackage{peeters_layout_exercise}
\usepackage{peeters_braket}
\usepackage{peeters_figures}
\usepackage{siunitx}
\usepackage{verbatim}
%\usepackage{mhchem} % \ce{}
%\usepackage{macros_bm} % \bcM
%\usepackage{macros_qed} % \qedmarker
%\usepackage{txfonts} % \ointclockwise

\beginArtNoToc

\generatetitle{XXX}
%\chapter{XXX}
%\label{chap:spinorinversion}

For the special case of spinor \( Z_{ij} \in \mathbb{Cl}(2,0) \), 
we can setup this system of equations in matrix form, provided we are careful with the ordering of products.  For example:
\begin{equation*}
\begin{bmatrix}
   Z_{11} & Z_{12} \\
   Z_{21} & Z_{22} \\
\end{bmatrix}
\begin{bmatrix}
   x_1 \\
   x_2
\end{bmatrix}
=
\begin{bmatrix}
   a_1 \\
   a_2
\end{bmatrix}.
\end{equation*}
Here we must define matrix multiplication so that any products are taken in the order of the matrices.  That is
\begin{equation*}
\begin{bmatrix}
   Z_{11} & Z_{12} \\
   Z_{21} & Z_{22} \\
\end{bmatrix}
\begin{bmatrix}
   x_1 \\
   x_2
\end{bmatrix}
=
\begin{bmatrix}
   Z_{11} x_1 + Z_{12} x_2  \\
   Z_{21} x_1 + Z_{22} x_2
\end{bmatrix},
\end{equation*}
and not
\begin{equation*}
\begin{bmatrix}
   Z_{11} & Z_{12} \\
   Z_{21} & Z_{22} \\
\end{bmatrix}
\begin{bmatrix}
   x_1 \\
   x_2
\end{bmatrix}
=
\begin{bmatrix}
   x_1 Z_{11} + Z_{12} x_2  \\
   Z_{21} x_1 + x_2 Z_{22}
\end{bmatrix},
\end{equation*}
or any other permutation.  Because the even grade elements of \( Z_{ij} \in \mathbb{Cl}(2,0) \) all commute (this space is isomorphic to complex numbers),
we can just invert this matrix.
\begin{equation*}
\begin{bmatrix}
   x_1 \\
   x_2
\end{bmatrix}
=
\inv{ Z_{11} Z_{22} - Z_{12} Z_{21} }
\begin{bmatrix}
   Z_{22} & -Z_{12} \\
  -Z_{21} & Z_{11} \\
\end{bmatrix}
\begin{bmatrix}
   a_1 \\
   a_2
\end{bmatrix}.
\end{equation*}
We still must be careful with ordering when the \( 2x2 \) matrix inverse is multiplied with the matrix of vectors, but multiplying that out should solve the system.

This approach should also be possible for similar systems in this space with more variables.

%}
%\EndArticle
\EndNoBibArticle
