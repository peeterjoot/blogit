%
% Copyright � 2025 Peeter Joot.  All Rights Reserved.
% Licenced as described in the file LICENSE under the root directory of this GIT repository.
%
%{
\input{../latex/blogpost.tex}
\renewcommand{\basename}{thirdOrderPole}
%\renewcommand{\dirname}{notes/phy1520/}
\renewcommand{\dirname}{notes/ece1228-electromagnetic-theory/}
%\newcommand{\dateintitle}{}
%\newcommand{\keywords}{}

\input{../latex/peeter_prologue_print2.tex}

\usepackage{peeters_layout_exercise}
\usepackage{peeters_braket}
\usepackage{peeters_figures}
\usepackage{siunitx}
\usepackage{verbatim}
%\usepackage{macros_cal} % \LL
%\usepackage{amsthm} % proof
%\usepackage{mhchem} % \ce{}
%\usepackage{macros_bm} % \bcM
%\usepackage{macros_qed} % \qedmarker
%\usepackage{txfonts} % \ointclockwise

\beginArtNoToc

\generatetitle{A contour integral with a third order pole.}
%\chapter{A contour integral with a third order pole.}
%\label{chap:thirdOrderPole}

Here's problem 31(e) from \citep{byron1992mca}.  Find
\begin{equation}\label{eqn:thirdOrderPole:20}
I = \int_0^\infty \frac{x^2 dx}{\lr{ a^2 + x^2 }^3 }.
\end{equation}
Again, we use the contour \( C \) illustrated in \cref{fig:thirdOrderPole:thirdOrderPoleFig1}
\imageFigure{../figures/blogit/fourPolesFig1}{Standard above the x-axis, semicircular contour.}{fig:thirdOrderPole:thirdOrderPoleFig1}{0.3}

Along the infinite semicircle, with \( z = R e^{i\theta} \),
\begin{equation}\label{eqn:thirdOrderPole:40}
\Abs{ \int \frac{z^2 dz}{\lr{ a^2 + z^2 }^3 } } = O(R^3/R^6),
\end{equation}
which tends to zero.  We are left to just evaluate some residues
\begin{equation}\label{eqn:thirdOrderPole:60}
\begin{aligned}
I
&= \inv{2} \oint \frac{z^2 dz}{ \lr{ a^2 + z^2 }^3 } \\
&= \inv{2} \oint \frac{z^2 dz}{ \lr{ z - i a }^3 \lr{ z + i a }^3 } \\
&= \inv{2} \lr{ 2 \pi i } \inv{2!} \evalbar{ \lr{ \frac{z^2}{ \lr{ z + i a }^3 } }'' }{z = i a}
\end{aligned}
\end{equation}
Evaluating the derivatives, we have
\begin{equation}\label{eqn:thirdOrderPole:80}
\begin{aligned}
\lr{ \frac{z^2}{ \lr{ z + i a }^3 } }'
&= \frac{ 2 z \lr{ z + i a } - 3 z^2 }{ \lr{ z + i a }^4 } \\
&=
\frac{ - z^2 + 2 i a z }
{ \lr{ z + i a }^4 },
\end{aligned}
\end{equation}
and
\begin{equation}\label{eqn:thirdOrderPole:100}
\begin{aligned}
\lr{ \frac{z^2}{ \lr{ z + i a }^3 } }''
&= \lr{ \frac{ - z^2 + 2 i a z }
{ \lr{ z + i a }^4 } }' \\
&= \frac{ \lr{ - 2 z + 2 i a  }\lr{ z + i a} - 4 \lr{ - z^2 + 2 i a z }}{ \lr{ z + i a }^5 },
\end{aligned}
\end{equation}
so
\begin{equation}\label{eqn:thirdOrderPole:120}
\begin{aligned}
\evalbar{ \lr{ \frac{z^2}{ \lr{ z + i a }^3 } }'' }{z = i a}
&=
\frac{ \lr{ - 2 i a + 2 i a  }\lr{ 2 i a} - 4 \lr{ a^2 - 2 a^2 }}{ \lr{ 2 i a }^5 } \\
&=
\frac{ 4 a^2 }{ \lr{ 2 i a }^5 } \\
&=
\inv{8 a^3 i}.
\end{aligned}
\end{equation}
Putting all the pieces together, we have
\begin{equation}\label{eqn:thirdOrderPole:140}
\boxed{
I = \frac{\pi}{16 a^3}.
}
\end{equation}

%}
\EndArticle
%\EndNoBibArticle
