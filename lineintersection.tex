%
% Copyright � 2022 Peeter Joot.  All Rights Reserved.
% Licenced as described in the file LICENSE under the root directory of this GIT repository.
%
%{
\input{../latex/blogpost.tex}
\renewcommand{\basename}{lineintersection}
%\renewcommand{\dirname}{notes/phy1520/}
\renewcommand{\dirname}{notes/ece1228-electromagnetic-theory/}
%\newcommand{\dateintitle}{}
%\newcommand{\keywords}{}

\input{../latex/peeter_prologue_print2.tex}

\usepackage{peeters_layout_exercise}
\usepackage{peeters_braket}
\usepackage{peeters_figures}
\usepackage{siunitx}
\usepackage{verbatim}
%\usepackage{mhchem} % \ce{}
%\usepackage{macros_bm} % \bcM
%\usepackage{macros_qed} % \qedmarker
%\usepackage{txfonts} % \ointclockwise

\beginArtNoToc

\generatetitle{XXX}
%\chapter{XXX}

Given two lines:
\begin{equation*}
\begin{aligned}
\Br_1 &= \Bp + \alpha \Bu \\
\Br_2 &= \Bq + \beta \Bv \\
\end{aligned}
\end{equation*}
Intersection is:
\begin{equation*}
\Bq + \beta \Bv = \Bp + \alpha \Bu,
\end{equation*}
and wedging with \(\Bv\) (say), we have
\begin{equation*}
\Bq \wedge \Bv = \Bp \wedge \Bv + \alpha \Bu \wedge \Bv,
\end{equation*}
so if the solution exists
\begin{equation*}
\alpha = \frac{(\Bq - \Bp) \wedge \Bv}{\Bu \wedge \Bv},
\end{equation*}
which determines the point of intersection (from \(\Br_1\)).

%}
\EndArticle
%\EndNoBibArticle
