%
% Copyright � 2024 Peeter Joot.  All Rights Reserved.
% Licenced as described in the file LICENSE under the root directory of this GIT repository.
%
%{
\input{../latex/blogpost.tex}
\renewcommand{\basename}{limitSqrt}
%\renewcommand{\dirname}{notes/phy1520/}
\renewcommand{\dirname}{notes/ece1228-electromagnetic-theory/}
%\newcommand{\dateintitle}{}
%\newcommand{\keywords}{}

\input{../latex/peeter_prologue_print2.tex}

\usepackage{peeters_layout_exercise}
\usepackage{peeters_braket}
\usepackage{peeters_figures}
\usepackage{siunitx}
\usepackage{verbatim}
%\usepackage{macros_cal} % \LL
%\usepackage{amsthm} % proof
%\usepackage{mhchem} % \ce{}
%\usepackage{macros_bm} % \bcM
%\usepackage{macros_qed} % \qedmarker
%\usepackage{txfonts} % \ointclockwise

\beginArtNoToc

\generatetitle{XXX}
%\chapter{XXX}
%\label{chap:limitSqrt}
% \citep{sakurai2014modern} pr X.Y
% \citep{pozar2009microwave}
% \citep{qftLectureNotes}
% \citep{doran2003gap}
% \citep{jackson1975cew}
% \citep{griffiths1999introduction}

Find
\begin{equation}\label{eqn:limitSqrt:20}
\lim_{x \rightarrow \infty} \sqrt{ x^2 + x - 1 } - x.
\end{equation}
Factor out \( x \) from the root, and then expand in series
\begin{equation}\label{eqn:limitSqrt:40}
\begin{aligned}
\lim_{x \rightarrow \infty} \sqrt{ x^2 + x - 1 } - x
&=
\lim_{x \rightarrow \infty} x \lr{ -1 + \sqrt{ 1 + \inv{x} - \inv{x^2} } } \\
&=
\lim_{x \rightarrow \infty} x \lr{ -1 + \sqrt{ 1 + \lr{ \inv{x} - \inv{x^2}} } } \\
&=
\lim_{x \rightarrow \infty} x \lr{ -1 + 1 + \inv{2} \inv{1!} \lr{ \inv{x} - \inv{x^2}} + \inv{2}\lr{ \frac{-1}{2} } \inv{2!} \lr{ \inv{x} - \inv{x^2}}^2 + \cdots } \\
&=
\lim_{x \rightarrow \infty} \inv{2} \lr{ 1 - \inv{x}} - \frac{x}{4} \lr{ \inv{x} - \inv{x^2}}^2 + \cdots.
\end{aligned}
\end{equation}
All but the \( 1/2 \) term is \( O(1/x^k) \) with \( k > 0 \), so
\begin{equation}\label{eqn:limitSqrt:60}
\lim_{x \rightarrow \infty} \sqrt{ x^2 + x - 1 } - x = \inv{2}.
\end{equation}

%}
%\EndArticle
\EndNoBibArticle
