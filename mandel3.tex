%
% Copyright � 2021 Peeter Joot.  All Rights Reserved.
% Licenced as described in the file LICENSE under the root directory of this GIT repository.
%
%{
\input{../latex/blogpost.tex}
\renewcommand{\basename}{mandel3}
%\renewcommand{\dirname}{notes/phy1520/}
\renewcommand{\dirname}{notes/ece1228-electromagnetic-theory/}
%\newcommand{\dateintitle}{}
%\newcommand{\keywords}{}

\input{../latex/peeter_prologue_print2.tex}

\usepackage{peeters_layout_exercise}
\usepackage{peeters_braket}
\usepackage{peeters_figures}
\usepackage{siunitx}
\usepackage{verbatim}
%\usepackage{mhchem} % \ce{}
%\usepackage{macros_bm} % \bcM
%\usepackage{macros_qed} % \qedmarker
%\usepackage{txfonts} % \ointclockwise

\beginArtNoToc

\generatetitle{A better 3D generalization of the Mandelbrot set.}
%\chapter{A better 3D generalization of the Mandelbrot set.}
%\label{chap:mandel3}

I've been exploring 3D generalizations of the Mandelbrot set.  The iterative equation for the Mandelbrot set can be written in vector form (\citep{dorst2007gac}) as:
\begin{equation}
\begin{aligned}
\Bz 
&\rightarrow 
\Bz \Be_1 \Bz + \Bc  \\
&=
\Bz \lr{ \Be_1 \cdot \Bz }
+
\Bz \cdot \lr{ \Be_1 \wedge \Bz }
+ \Bc  \\
&=
2 \Bz \lr{ \Be_1 \cdot \Bz }
-
\Bz^2\, \Be_1
+ \Bc 
\end{aligned}
\end{equation}
Plotting this in 3D was an interesting challenge, but showed that the Mandelbrot set expressed above has rotational symmetry about the x-axis, which is kind of boring.

If all we require for a 3D fractal is to iterate a vector equation that is (presumably) at least quadratic, then we have lots of options.  Here's the first one that comes to mind:
\begin{equation}
\begin{aligned}
\Bz 
&\rightarrow 
\gpgradeone{ \Ba \Bz \Bb \Bz \Bc } + \Bd \\
&=
\lr{ \Ba \cdot \Bz } \lr{ \Bz \cross \lr{ \Bc \cross \Bz } }
+
\lr{ \Ba \cross \Bz } \lr{ \Bz \cdot \lr{ \Bc \cross \Bz } } 
+ \Bd
.
\end{aligned}
\end{equation}
where we iterate starting, as usual with \( \Bz = 0 \) where \( \Bd \) is the point of interest to test for inclusion in the set.  I tried this with
\begin{equation}\label{eqn:mandel3:n}
\begin{aligned}
\Ba &= (1,1,1) \\
\Bb &= (1,0,0) \\
\Bc &= (1,-1,0).
\end{aligned}
\end{equation}
Here are some slice plots

and an animation of the slices with respect to the z-axis:

%}
\EndArticle
