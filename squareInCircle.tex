%
% Copyright � 2023 Peeter Joot.  All Rights Reserved.
% Licenced as described in the file LICENSE under the root directory of this GIT repository.
%
%{
\input{../latex/blogpost.tex}
\renewcommand{\basename}{squareInCircle}
%\renewcommand{\dirname}{notes/phy1520/}
\renewcommand{\dirname}{notes/ece1228-electromagnetic-theory/}
%\newcommand{\dateintitle}{}
%\newcommand{\keywords}{}

\input{../latex/peeter_prologue_print2.tex}

\usepackage{peeters_layout_exercise}
\usepackage{peeters_braket}
\usepackage{peeters_figures}
\usepackage{siunitx}
\usepackage{verbatim}
%\usepackage{mhchem} % \ce{}
%\usepackage{macros_bm} % \bcM
%\usepackage{macros_qed} % \qedmarker
%\usepackage{txfonts} % \ointclockwise

\beginArtNoToc

\generatetitle{A silly geometry problem: length of side of square in circular quadrant}
%\chapter{A silly geometry problem: length of side of square in circular quadrant}
%\label{chap:squareInCircle}

Problem from \href{https://youtu.be/7ZIImuW0PBA}{Solving a geometry question that no one can figure out}.

My solution (before numerical reduction), using basic trig and complex numbers, is illustrated in \cref{fig:myway:mywayFig1}.
\imageFigure{../figures/blogit/mywayFig1}{With complex numbers.}{fig:myway:mywayFig1}{0.3}

We have
\begin{equation}\label{eqn:squareInCircle:20}
\begin{aligned}
s &= x \cos\theta \\
y &= x \sin\theta \\
p &= y + x e^{i\theta} \\
q &= i s + x e^{i\theta} \\
\Abs{q} &= y + 5 \\
\Abs{p - q} &= 2.
\end{aligned}
\end{equation}
This can be reduced to
\begin{equation}\label{eqn:squareInCircle:40}
\begin{aligned}
\Abs{ x e^{i\theta} - 5 } &= 2 \\
x \Abs{ i \cos\theta + e^{i\theta} } &= x \sin\theta + 5.
\end{aligned}
\end{equation}

My wife figured out how to do it with just Pythagoras, as illustrated in \cref{fig:sofiasWay:sofiasWayFig2}.
\imageFigure{../figures/blogit/sofiasWayFig2}{With Pythagoras.}{fig:sofiasWay:sofiasWayFig2}{0.3}
\begin{equation}\label{eqn:squareInCircle:60}
\begin{aligned}
\lr{ 5 - s }^2 + y^2 &= 4 \\
\lr{ s + y }^2 + s^2 &= \lr{ y + 5 }^2 \\
x^2 &= s^2 + y^2.
\end{aligned}
\end{equation}

Either way, the numerical solution is 4.12.  The geometry looks like \cref{fig:squareInCircle:squareInCircleFig1}.

\imageFigure{../figures/blogit/squareInCircleFig1}{Lengths to scale.}{fig:squareInCircle:squareInCircleFig1}{0.3}

A mathematica notebook to compute the numerical part of the problem (either way) and plot the figure to scale \href{https://github.com/peeterjoot/mathematica/blob/master/blogit/squareInCircleFig1.nb}{can be found in my mathematica github repo.}

%}
%\EndArticle
\EndNoBibArticle
