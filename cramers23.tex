%
% Copyright � 2025 Peeter Joot.  All Rights Reserved.
% Licenced as described in the file LICENSE under the root directory of this GIT repository.
%
%{
\input{../latex/blogpost.tex}
\renewcommand{\basename}{cramers23}
%\renewcommand{\dirname}{notes/phy1520/}
\renewcommand{\dirname}{notes/ece1228-electromagnetic-theory/}
%\newcommand{\dateintitle}{}
%\newcommand{\keywords}{}

\input{../latex/peeter_prologue_print2.tex}

\usepackage{peeters_layout_exercise}
\usepackage{peeters_braket}
\usepackage{peeters_figures}
\usepackage{siunitx}
\usepackage{verbatim}
%\usepackage{macros_cal} % \LL
%\usepackage{amsthm} % proof
%\usepackage{mhchem} % \ce{}
%\usepackage{macros_bm} % \bcM
%\usepackage{macros_qed} % \qedmarker
%\usepackage{txfonts} % \ointclockwise

\beginArtNoToc

\generatetitle{XXX}
%\chapter{XXX}
%\label{chap:cramers23}

You can derive Cramer's rule for the 2-variable and 3-variable cases using the cross product. Write
\begin{equation}\label{eqn:cramers23:20}
\begin{aligned}
\Ba &= a_1 \Be_1 + a_2 \Be_2 \\
\Bb &= b_1 \Be_1 + b_2 \Be_2 \\
\Bd &= d_1 \Be_1 + d_2 \Be_2,
\end{aligned}
\end{equation}
so the linear system is
\begin{equation}\label{eqn:cramers23:40}
\Ba x + \Bb y = \Bd.
\end{equation}
Crossing with \( \Ba \) and \( \Bb \) from the left and right respectively, gives
\begin{equation}\label{eqn:cramers23:60}
\begin{aligned}
\cancel{\lr{\Ba \cross \Ba}} x + \lr{\Ba \cross \Bb} y &= \Ba \cross \Bd \\
\lr{\Ba \cross \Bb} x + \cancel{\lr{\Bb \cross \Bb}} y &= \Bd \cross \Bb.
\end{aligned}
\end{equation}
Dotting with \( \zcap = \Be_3 \), we have
\begin{equation}\label{eqn:cramers23:80}
\begin{aligned}
\Be_3 \cdot \lr{ \Ba \cross \Bb } y &= \Be_3 \cdot \lr{ \Ba \cross \Bd } \\
\Be_3 \cdot \lr{ \Ba \cross \Bb } x &= \Be_3 \cdot \lr{ \Bd \cross \Bb }.
\end{aligned}
\end{equation}
Knowing that we can write this triple scalar product as a determinant, we are left with
\begin{equation}\label{eqn:cramers23:100}
\begin{aligned}
\begin{vmatrix}
0   & 0   & 1 \\
a_1 & a_2 & 0 \\
b_1 & b_2 & 0 \\
\end{vmatrix}
y
&=
\begin{vmatrix}
0   & 0   & 1 \\
a_1 & a_2 & 0 \\
d_1 & d_2 & 0 \\
\end{vmatrix} \\
\begin{vmatrix}
0   & 0   & 1 \\
a_1 & a_2 & 0 \\
b_1 & b_2 & 0 \\
\end{vmatrix}
x
&=
\begin{vmatrix}
0   & 0   & 1 \\
d_1 & d_2 & 0 \\
b_1 & b_2 & 0 \\
\end{vmatrix}.
\end{aligned}
\end{equation}
Partially expanding those determinants along the first row, gives
\begin{equation}\label{eqn:cramers23:120}
\begin{aligned}
\begin{vmatrix}
a_1 & a_2 \\
b_1 & b_2 \\
\end{vmatrix}
y
&=
\begin{vmatrix}
a_1 & a_2 \\
d_1 & d_2 \\
\end{vmatrix} \\
\begin{vmatrix}
a_1 & a_2 \\
b_1 & b_2 \\
\end{vmatrix}
x
&=
\begin{vmatrix}
d_1 & d_2 \\
b_1 & b_2 \\
\end{vmatrix}.
\end{aligned}
\end{equation}
After transposition of the remaining determinants (which doesn't change them), we have Cramer's rule.
\section{}
Similarly, for the 3-variable case, write
\begin{equation}\label{eqn:cramers23:160}
\begin{aligned}
\Ba &= a_1 \Be_1 + a_2 \Be_2 + a_3 \Be_3 \\
\Bb &= b_1 \Be_1 + b_2 \Be_2 + b_3 \Be_3 \\
\Bc &= c_1 \Be_1 + c_2 \Be_2 + c_3 \Be_3 \\
\Bd &= d_1 \Be_1 + d_2 \Be_2 + d_3 \Be_3 ,
\end{aligned}
\end{equation}
so the linear system is
\begin{equation}\label{eqn:cramers23:180}
\Ba x + \Bb y + \Bc z = \Bd.
\end{equation}
Illustrating by example, let's cross this with \( \Ba \) to eliminate one of the variables
\begin{equation}\label{eqn:cramers23:200}
\cancel{\lr{\Ba \cross \Ba}} x + \lr{\Ba \cross \Bb} y + \lr{ \Ba \cross \Bc } z = \Ba \cross \Bd.
\end{equation}
We can now dot with \( \Bb, \Bc \) respectively, to eliminate the other variables, to find
\begin{equation}\label{eqn:cramers23:220}
\begin{aligned}
\cancel{ \Bb \cdot \lr{\Ba \cross \Bb} } y + \Bb \cdot \lr{ \Ba \cross \Bc } z &= \Bb \cdot \lr{ \Ba \cross \Bd } \\
\Bc \cdot \lr{\Ba \cross \Bb} y + \cancel{ \Bc \cdot \lr{ \Ba \cross \Bc } } z &= \Bc \cdot \lr{ \Ba \cross \Bd }.
\end{aligned}
\end{equation}
This gives
\begin{equation}\label{eqn:cramers23:140}
\begin{aligned}
z &= \frac{\Bb \cdot \lr{ \Ba \cross \Bd }}{\Bb \cdot \lr{ \Ba \cross \Bc } } \\
y &= \frac{\Bc \cdot \lr{\Ba \cross \Bb}}{\Bb \cdot \lr{ \Ba \cross \Bc } },
\end{aligned}
\end{equation}
The triple products can be written out as determinants, and shuffled slightly to put them in the expected form for Cramer's rule.  The solution for \( x \) follows in the same way.

%}
%\EndArticle
\EndNoBibArticle
