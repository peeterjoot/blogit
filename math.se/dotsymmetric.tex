%
% Copyright � 2022 Peeter Joot.  All Rights Reserved.
% Licenced as described in the file LICENSE under the root directory of this GIT repository.
%
%{
\input{../latex/blogpost.tex}
\renewcommand{\basename}{dotsymmetric}
%\renewcommand{\dirname}{notes/phy1520/}
\renewcommand{\dirname}{notes/ece1228-electromagnetic-theory/}
%\newcommand{\dateintitle}{}
%\newcommand{\keywords}{}

\input{../latex/peeter_prologue_print2.tex}

\usepackage{peeters_layout_exercise}
\usepackage{peeters_braket}
\usepackage{peeters_figures}
\usepackage{siunitx}
\usepackage{verbatim}
%\usepackage{mhchem} % \ce{}
%\usepackage{macros_bm} % \bcM
%\usepackage{macros_qed} % \qedmarker
%\usepackage{txfonts} % \ointclockwise

\beginArtNoToc

\generatetitle{XXX}
%\chapter{XXX}
%\label{chap:dotsymmetric}

There isn't actually that much choice in what to call such a symmetric product of vectors.  Consider first the product of two vectors
\( a = \sum_i x_i \Be_i \) and \( b = \sum_i y_i \Be_i \), assuming that \( \setlr{ \Be_1, \cdots \Be_N } \) is an orthonormal basis.
\begin{equation*}
\begin{aligned}
   a b
   &= \sum_{1 \le i,j \le N} x_i \Be_i y_j \Be_j \\
   &= \sum_{i = j} x_i y_i \Be_i^2 + \sum_{i \ne j} x_i y_j \Be_i \Be_j \\
   &= \sum_{i = 1}^N x_i y_i + \sum_{i \ne j} x_i y_j \Be_i \Be_j \\
   &= \sum_{i = 1}^N x_i y_i + \sum_{1 \le i < j \le N} \lr{ x_i y_j - x_j y_i } \Be_i \Be_j \\
   &= \sum_{i = 1}^N x_i y_i + \sum_{1 \le i < j \le N}
\begin{vmatrix}
   x_i & x_j \\
   y_i & y_j
\end{vmatrix}
   \Be_i \Be_j.
\end{aligned}
\end{equation*}
Reversing the products, we find
\begin{equation*}
\begin{aligned}
   b a
   &= \sum_{i = 1}^N y_i x_i + \sum_{1 \le i < j \le N}
\begin{vmatrix}
   y_i & y_j \\
   x_i & x_j
\end{vmatrix}
   \Be_i \Be_j \\
   &= \sum_{i = 1}^N x_i y_i - \sum_{1 \le i < j \le N}
\begin{vmatrix}
   x_i & x_j \\
   y_i & y_j
\end{vmatrix}
   \Be_i \Be_j.
\end{aligned}
\end{equation*}

Forming the symmetrization, there isn't any choice about the specific value of this scalar term
\begin{equation*}
   \inv{2} \lr{ a b + b a } = \sum_{i = 1}^N x_i y_i,
\end{equation*}
and we are forced to identify the symmetric sum as the dot product.

Incidentally, we also have no choice about the specific value of the grade-2 portion of the vector product, which we may similarily pick out using an antisymmetrical sum
\begin{equation*}
   \inv{2} \lr{ a b - b a } = \sum_{1 \le i < j \le N}
\begin{vmatrix}
   x_i & x_j \\
   y_i & y_j
\end{vmatrix}
   \Be_i \Be_j.
\end{equation*}
We can then define the grade-2 part of the vector product as the wedge product.  From the coordinate expansion above, it is clear that this has the desired properties \( a \wedge a = 0, a \wedge b = - b \wedge a \).  It's also possible to show that this has the \( \sin\theta \) (parallelogram area) property that we desire of the wedge product.

%}
%\EndArticle
\EndNoBibArticle
