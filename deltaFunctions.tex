%
% Copyright � 2025 Peeter Joot.  All Rights Reserved.
% Licenced as described in the file LICENSE under the root directory of this GIT repository.
%
%{
\input{../latex/blogpost.tex}
\renewcommand{\basename}{deltaFunctions}
%\renewcommand{\dirname}{notes/phy1520/}
\renewcommand{\dirname}{notes/ece1228-electromagnetic-theory/}
%\newcommand{\dateintitle}{}
%\newcommand{\keywords}{}

\input{../latex/peeter_prologue_print2.tex}

\usepackage{peeters_layout_exercise}
\usepackage{peeters_braket}
\usepackage{peeters_figures}
\usepackage{siunitx}
\usepackage{verbatim}
\usepackage{macros_cal} % \LL
%\usepackage{amsthm} % proof
%\usepackage{mhchem} % \ce{}
%\usepackage{macros_bm} % \bcM
%\usepackage{macros_qed} % \qedmarker
%\usepackage{txfonts} % \ointclockwise

\beginArtNoToc

\generatetitle{Some gradient related Green's functions}
%\chapter{Some gradient related Green's functions}
%\label{chap:deltaFunctions}
Here are the Green's function representations for a few linear differential operators of interest in electrodynamics.  These Green's functions all satisfy
\begin{equation}\label{eqn:deltaFunctions:120}
\delta(\Bx - \Bx') = \LL G(\Bx, \Bx').
\end{equation}

Let \( \Br = \Bx - \Bx' \), \( r = \Norm{\Br} \), \( \rcap = \Br/r \), and \( \tau = t - t' \), then
\begin{enumerate}
\item Divergence operator (3D), \( \LL \Bf = \spacegrad \cdot \Bf \)
\begin{equation}\label{eqn:deltaFunctions:40}
G\lr{ \Bx, \Bx' } = \frac{\rcap}{4 \pi}.
\end{equation}
\item Gradient operator, \( \LL = \spacegrad \), in 1D, 2D and 3D respectively
\begin{equation}\label{eqn:deltaFunctions:25}
\begin{aligned}
G\lr{ \Bx, \Bx' } &= \frac{\rcap}{2} \\
G\lr{ \Bx, \Bx' } &= \frac{1}{2 \pi} \frac{\rcap}{r} \\
G\lr{ \Bx, \Bx' } &= \inv{4 \pi} \frac{\rcap}{r^2}.
\end{aligned}
\end{equation}
\item Laplacian operator, \( \LL = \spacegrad^2 \), in 1D, 2D and 3D respectively
\begin{equation}\label{eqn:deltaFunctions:20}
\begin{aligned}
G\lr{ \Bx, \Bx' } &= \frac{r}{2} \\
G\lr{ \Bx, \Bx' } &= \frac{1}{2 \pi} \ln r \\
G\lr{ \Bx, \Bx' } &= -\frac{1}{4 \pi r}.
\end{aligned}
\end{equation}
\item Second order Helmholtz operator, \( \LL = \spacegrad^2 + k^2 \) for 1D, 2D and 3D respectively
\begin{equation}\label{eqn:deltaFunctions:60}
\begin{aligned}
G\lr{ \Bx, \Bx' } &= \pm \frac{1}{2 j k} e^{\pm j k r} \\
G(\Bx, \Bx') &= \frac{1}{4 j} H_0^{(1)}(\pm k r) \\
G\lr{ \Bx, \Bx' } &= -\frac{1}{4 \pi} \frac{e^{\pm j k r }}{r}.
\end{aligned}
\end{equation}

\item First order Helmholtz operator, \( \LL = \spacegrad + j k \), in 1D, 2D and 3D respectively
% \spacegrad r = \rcap
% H_0^{(1)}' = -H_1^{(1)}
% G = (\spacegrad -j k)G_H, G_H from :60 above
\begin{equation}\label{eqn:deltaFunctions:80}
\begin{aligned}
G\lr{ \Bx, \Bx' } &= \frac{j}{2} \lr{ \rcap \mp 1 } e^{\pm j k r} \\
G\lr{ \Bx, \Bx' } &= \frac{k}{4} \lr{ \pm j \rcap H_1^{(1)}(\pm k r) - H_0^{(1)}(\pm k r) } \\
G\lr{ \Bx, \Bx' } &= \frac{e^{\pm j k r}}{4 \pi r} \lr{ jk \lr{ 1 \mp \rcap } + \frac{\rcap}{r} }.
\end{aligned}
\end{equation}

This is also the Green's function for a left acting operator \( G(\Bx, \Bx') \lr{ - \lspacegrad + j k } = \delta(\Bx - \Bx') \).
\item Spacetime gradient (3+1 D), \( \LL = \spacegrad + (1/c) \partial_t \), satisfying \( \LL G(\Bx - \Bx', t - t') = \delta(\Bx - \Bx') \delta(t - t') \)
\begin{equation}\label{eqn:deltaFunctions:100}
G( \Br, \tau ) = \inv{4 \pi} \lr{ \frac{\rcap}{r^2} + \inv{c r} \lr{ 1 + \rcap } } \delta\lr{\tau - \frac{r}{c} } \frac{d}{dt_R},
\end{equation}
where \( t_R = t - r/c \).
FIXME: calculate this for both the retarded and advanced Green's functions, and also the 1D and 2D results.
\item Wave equation, \( \spacegrad^2 - (1/c^2) \partial_{tt} \)
\begin{equation}\label{eqn:deltaFunctions:140}
\begin{aligned}
G(\Br, \tau) &= -\frac{c}{2} \Theta( \pm \tau - r/c ) \\
G(\Br, \tau) &= -\inv{2 \pi \sqrt{ \tau^2 - r^2/c^2 } } \Theta( \pm \tau - r/c ) \\
G(\Br, \tau) &= -\inv{4 \pi r} \delta( \pm \tau - r/c ),
\end{aligned}
\end{equation}
The positive sign is for the retarded solution, and the negative for advancing.
\end{enumerate}

%}
%\EndArticle
\EndNoBibArticle
