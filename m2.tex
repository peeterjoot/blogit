%
% Copyright � 2021 Peeter Joot.  All Rights Reserved.
% Licenced as described in the file LICENSE under the root directory of this GIT repository.
%
%{
\input{../latex/blogpost.tex}
\renewcommand{\basename}{m2}
%\renewcommand{\dirname}{notes/phy1520/}
\renewcommand{\dirname}{notes/ece1228-electromagnetic-theory/}
%\newcommand{\dateintitle}{}
%\newcommand{\keywords}{}

\input{../latex/peeter_prologue_print2.tex}

\usepackage{peeters_layout_exercise}
\usepackage{peeters_braket}
\usepackage{peeters_figures}
\usepackage{siunitx}
\usepackage{verbatim}
%\usepackage{mhchem} % \ce{}
%\usepackage{macros_bm} % \bcM
%\usepackage{macros_qed} % \qedmarker
%\usepackage{txfonts} % \ointclockwise

\beginArtNoToc

\generatetitle{3D Mandelbrot}
%\chapter{3D Mandelbrot}
%

\section{Analysis.}

The graphing play above shows some apparent rotational symmetry our vector equivalent to the Mandelbrot equation
\begin{equation}
\Bx \rightarrow \Bx \Be_1 \Bx + \Bc.
\end{equation}
It was not clear to me if this symmetry existed, as there were artifacts in the plots that made it appear that there was irregularity.  However, some thought shows that this irregularity is strictly due to sampling error, and perhaps also due to limitations in the plotting software, as such an uneven surface is probably tricky to deal with.

To see this, here are the first few iterations of the Mandlebrot sequence for an arbitary starting vector \( \Bc \).
\begin{equation}
\begin{aligned}
\Bx_0 &= \Bc \\
\Bx_1 &= \Bc \Be_1 \Bc + \Bc \\
\Bx_2 &= \lr{ \Bc \Be_1 \Bc + \Bc } \Be_1 \lr{ \Bc \Be_1 \Bc + \Bc } + \Bc \Be_1 \Bc + \Bc.
\end{aligned}
\end{equation}

Now, what happens when we rotate the starting vector \( \Bc \) in the \( y-z \) plane.  The rotor for such a rotation is
\begin{equation}
R = \exp\lr{ e_{23} \theta/2 },
\end{equation}
where
\begin{equation}
\Bc \rightarrow R \Bc \tilde{R}.
\end{equation}
Observe that if \( \Bc \) is parallel to the x-axis, then this rotation leaves the starting point invariant, as \( \Be_1 \) commutes with \( R \).  That is
\begin{equation}
R \Be_1 \tilde{R} = 
\Be_1 R \tilde{R} = \Be_1.
\end{equation}
Let \( \Bc' = R \Bc \tilde{R} \), so that
\begin{equation}
\Bx_0' = R \Bc \tilde{R} = R \Bx_0 \tilde{R} .
\end{equation}
\begin{equation}
\begin{aligned}
\Bx_1' 
&= R \Bc \tilde{R} \Be_1 R \Bc \tilde{R} + R \Bc \tilde{R}  \\
&= R \Bc \Be_1 \Bc \tilde{R} + R \Bc \tilde{R}  \\
&= R \lr{ \Bc \Be_1 \Bc R + \Bc } \tilde{R}  \\
&= R \Bx_1 \tilde{R}.
\end{aligned}
\end{equation}
\begin{equation}\label{eqn:m2:n}
\begin{aligned}
\Bx_2' 
&= \Bx_1' \Be_1 \Bx_1' + \Bc' \\
&= R \Bx_1 \tilde{R} \Be_1 R \Bx_1 \tilde{R} + R \Bc \tilde{R} \\
&= R \Bx_1 \Be_1 \Bx_1 \tilde{R} + R \Bc \tilde{R} \\
&= R \lr{ \Bx_1 \Be_1 \Bx_1 + \Bc } \tilde{R} \\
&= R \Bx_2 \tilde{R}.
\end{aligned}
\end{equation}

The pattern is clear.  If we rotate the starting point in the y-z plane, iterating the Mandelbrot sequence results in precisely the same rotation of the x-y plane Mandelbrot sequence.  So the apparent rotational symmetry in the 3D iteration of the Mandelbrot vector equation is exactly that.  This is an unfortunately boring 3D fractal.  All of the interesting fractal nature occurs in the 2D plane, and the rest is just a consequence of rotating that image around the x-axis.  We get some interesting fractal artifacts if we slice the rotated Mandelbrot image.

%}
\EndArticle
%\EndNoBibArticle
