%
% Copyright � 2025 Peeter Joot.  All Rights Reserved.
% Licenced as described in the file LICENSE under the root directory of this GIT repository.
%
%{
\input{../latex/blogpost.tex}
\renewcommand{\basename}{junk}
%\renewcommand{\dirname}{notes/phy1520/}
\renewcommand{\dirname}{notes/ece1228-electromagnetic-theory/}
%\newcommand{\dateintitle}{}
%\newcommand{\keywords}{}

\input{../latex/peeter_prologue_print2.tex}

\usepackage{peeters_layout_exercise}
\usepackage{peeters_braket}
\usepackage{peeters_figures}
\usepackage{siunitx}
\usepackage{verbatim}
%\usepackage{macros_cal} % \LL
%\usepackage{amsthm} % proof
%\usepackage{mhchem} % \ce{}
%\usepackage{macros_bm} % \bcM
%\usepackage{macros_qed} % \qedmarker
%\usepackage{txfonts} % \ointclockwise

\beginArtNoToc

\generatetitle{XXX}
%\chapter{XXX}
%\label{chap:junk}

\section{}
\begin{equation*}
\int f(x)\, dx\, g(x)
\end{equation*}

\section{}
\begin{equation*}
\begin{aligned}
i \sqrt{ i \sqrt{ i^{22} }}
&=
i \sqrt{ i \sqrt{ \lr{i^4}^5 i^2 }} \\
&=
i \sqrt{ i \sqrt{ {1}^5 \lr{ -1 } }} \\
&=
i \sqrt{ i^2 } \\
&=
i \sqrt{ -1 } \\
&=
i^2 \\
&=
-1.
\end{aligned}
\end{equation*}

\section{}
\begin{equation*}
6! = \lr{3 \cdot 2 } \cdot 5! = 3! 5!
\end{equation*}

\section{}
That first one works if \( a \) and \( b \) are orthogonal vectors (using the vector product from geometric algebra.)

An example, using a Pauli matrix representation of two such orthogonal vectors
\begin{equation*}
\begin{aligned}
a &= \sigma_x = \PauliX \\
b &= \sigma_z = \PauliZ.
\end{aligned}
\end{equation*}

In general, however, if \( a \cdot b \ne 0 \),
\begin{equation*}
\lr{ a + b }^2 = a^2 + 2 a \cdot b + b^2.
\end{equation*}
Observe that if \( a, b \in \bbR^1 \) (isomorphic to \( \bbR \)), then they cannot ever be orthogonal, so this reduces to the familiar
\begin{equation*}
\lr{ a + b }^2 = a^2 + 2 a b + b^2.
\end{equation*}


\section{}
Problem: Find the inverse of
\begin{equation}\label{eqn:junk:20}
y = \frac{10^x - 10^{-x}}{10^x + 10^{-x}} + 1.
\end{equation}

First rewrite the exponents
\begin{equation}\label{eqn:junk:40}
10^x = e^{\ln 10^x} = e^{x \ln 10},
\end{equation}
so
\begin{equation}\label{eqn:junk:60}
\begin{aligned}
y - 1
&= \frac{e^{x \ln 10} - e^{-x \ln 10}}{e^{x \ln 10} + e^{-x \ln 10}} \\
&= \frac{\sinh\lr{ x \ln 10}}{\cosh\lr{ x \ln 10 }} \\
&= \tanh\lr{ x \ln 10}.
\end{aligned}
\end{equation}
We can now invert to find
\begin{equation}\label{eqn:junk:80}
\boxed{
x = \frac{\tanh^{-1}\lr{ y - 1 }}{\ln 10}.
}
\end{equation}
\section{}
Here's a fun and easy way to compute the radial and tangential components of velocity and acceleration in circular coordinates, using a complex (polar) representation of the position vector.  First take derivatives
\begin{equation*}
\begin{aligned}
z &= r e^{i\theta} \\
z' &= r' e^{i\theta} + i r e^{i\theta} \theta' \\
z'' &= r'' e^{i\theta} + 2 i r' e^{i\theta} \theta' - r e^{i\theta} \lr{\theta'}^2 + i r e^{i\theta} \theta'',
\end{aligned}
\end{equation*}
Next identify \( \rcap = e^{i\theta} \) and \( \thetacap = i e^{i\theta} \), \( \omega = \theta' \), and \( \alpha = \theta'' \) to find
\begin{equation*}
\begin{aligned}
\Bx &= r \rcap \\
\Bv &= r' \rcap + r \omega \thetacap \\
\Ba &= \lr{ r'' - r \omega^2 } \rcap + \lr{ 2 r' \omega + r \alpha } \thetacap.
\end{aligned}
\end{equation*}
\section{}
If \( \Br(t) = \Br_0 + \Bv t \), then
\begin{equation*}
\Br(t) \wedge \Bv = \Br_0 \wedge \Bv + \lr{ \Bv \wedge \Bv } t = \Br_0 \wedge \Bv, \forall t.
\end{equation*}
\section{}
If you want to use a matrix representation, you can use Pauli or Dirac matrices.  For example, using Pauli matrices
\begin{equation*}
\begin{aligned}
\Be_1 &= \sigma_1 = \PauliX \\
\Be_2 &= \sigma_3 = \PauliZ
\end{aligned}
\end{equation*}
\section{}

Let \( x = e^u, dx = e^u du \), for
\begin{equation}\label{eqn:junk:100}
\begin{aligned}
\int x^i dx
&= \int e^{u(1 + i)} du \\
&= \frac{e^{u(1+i)}}{1 + i} + C \\
&= \frac{x^{1+i}}{1 + i} + C.
\end{aligned}
\end{equation}
If one wishes to see the contributions to the magnitude and phase, rewrite this as
\begin{equation}\label{eqn:junk:120}
\frac{x^{1+i}}{1 + i}
= \inv{\sqrt{2}} x e^{i\lr{ \ln x - \pi/4 }}.
\end{equation}

\section{}

\begin{equation}\label{eqn:junk:140}
\begin{aligned}
\lim_{\theta \rightarrow 0} \frac{\sin\theta}{\theta}
&=
\lim_{\theta \rightarrow 0} \frac{\sin\cancel{\theta}}{\cancel{\theta}} \\
&=
\sin
\end{aligned}
\end{equation}

\section{}

\begin{equation}\label{eqn:junk:160}
\begin{aligned}
\lim_{\theta \rightarrow 0} \frac{\sin\theta}{\theta}
&=
\lim_{\theta \rightarrow 0} \inv{\theta} \lr{ \theta - \inv{3!} \theta^3 + \inv{5!} \theta^5 + \cdots } \\
&=
\lim_{\theta \rightarrow 0} 1 - \inv{3!} \theta^2 + \inv{5!} \theta^4 + \cdots \\
&=
1.
\end{aligned}
\end{equation}

\section{}

That last one is a obscurified way of writing the ``reciprocal theorem'', normally written as
\begin{equation*}%\label{eqn:junk:180}
\Gamma(z)\Gamma(1-z) = \frac{\pi}{\sin(\pi z)}, \quad z \ne \mathbb{Z}.
\end{equation*}
Since \( \Gamma(z)\Gamma(1-z) = \infty \) for integer \(z\), and \( \sin(\pi z) = 0 \) for integer \( z \), the reciprocal of this reciprocal relation holds for all \( z \), allowing a solution for the sine.

\section{}

\begin{equation}\label{eqn:junk:200}
\begin{aligned}
\epsilon_{rst} \epsilon_{abt} A^{rsab}
&=
\epsilon_{rs1} \epsilon_{ab1} A^{rsab}
+
\epsilon_{rs2} \epsilon_{ab2} A^{rsab}
+
\epsilon_{rs3} \epsilon_{ab3} A^{rsab} \\
&=
\epsilon_{ab1} A^{23ab} - \epsilon_{ab1} A^{32ab}
+
\epsilon_{ab2} A^{31ab} - \epsilon_{ab2} A^{13ab}
+
\epsilon_{ab3} A^{12ab} - \epsilon_{ab3} A^{21ab}
\\
&=
\epsilon_{231} A^{2323} - \epsilon_{231} A^{3223}
+
\epsilon_{312} A^{3131} - \epsilon_{312} A^{1331}
+
\epsilon_{123} A^{1212} - \epsilon_{123} A^{2112}
+
\epsilon_{321} A^{2332} - \epsilon_{321} A^{3232}
+
\epsilon_{132} A^{3113} - \epsilon_{132} A^{1313}
+
\epsilon_{213} A^{1221} - \epsilon_{213} A^{2121} \\
&=
+ A^{2323} - A^{2332}
+ A^{3232} - A^{3223} \\
&\,+ A^{3131} - A^{3113}
+ A^{1313} - A^{1331}  \\
&\,+ A^{1212} - A^{1221}
+ A^{2121} - A^{2112}
\\
%&=
%A^{23[23]}
%+ A^{32[32]}
%+ A^{31[31]}
%+ A^{13[13]}
%+ A^{12[12}}
%+ A^{21[21]} \\
&=
\lr{ \delta_{ra}\delta_{sb} -\delta_{rb}\delta_{sa} } A^{rsab}
\end{aligned}
\end{equation}

\section{Frank's question: discord}

Is this supposed to be a line integral in two dimensions?

i.e.: Let
\begin{equation*}
\begin{aligned}
I &= \Be_1 \Be_2 \\
\spacegrad &= \Be_1 \partial_x + \Be_2 \partial_y \\
d\Bx &= dx \Be_1 + dy \Be_2.
\end{aligned}
\end{equation*}

and then compute
\begin{equation*}
\begin{aligned}
\spacegrad (xy) I d\Bx
&=
\lr{\Be_1 \partial_x + \Be_2 \partial_y }(xy)
\Be_1 \Be_2
\lr{ dx \Be_1 + dy \Be_2 } \\
&=
\lr{\Be_1 y + \Be_2 x }
\Be_1 \Be_2
\lr{ dx \Be_1 + dy \Be_2 } \\
&=
\lr{\Be_2 y - \Be_1 x }
\lr{ dx \Be_1 + dy \Be_2 } \\
&=
\lr{-I y - x } dx
+
\lr{y - I x } dy.
\end{aligned}
\end{equation*}

%}
%\EndArticle
\EndNoBibArticle
