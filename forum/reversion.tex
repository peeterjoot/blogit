%
% Copyright � 2021 Peeter Joot.  All Rights Reserved.
% Licenced as described in the file LICENSE under the root directory of this GIT repository.
%
%{
\input{../latex/blogpost.tex}
\renewcommand{\basename}{reversion}
%\renewcommand{\dirname}{notes/phy1520/}
\renewcommand{\dirname}{notes/ece1228-electromagnetic-theory/}
%\newcommand{\dateintitle}{}
%\newcommand{\keywords}{}

\input{../latex/peeter_prologue_print2.tex}

\usepackage{peeters_layout_exercise}
\usepackage{peeters_braket}
\usepackage{peeters_figures}
\usepackage{siunitx}
\usepackage{verbatim}
%\usepackage{mhchem} % \ce{}
%\usepackage{macros_bm} % \bcM
%\usepackage{macros_qed} % \qedmarker
%\usepackage{txfonts} % \ointclockwise

\beginArtNoToc

\generatetitle{XXX}
%\chapter{XXX}

1. "Is it necessary to decompose a multivector into a multiplication of two other multivectors?"

No, and as you suggest, decomposition of a multivector in terms of basis vectors, is a straightforward way to compute the reverse.

Take for example
\begin{equation*}
   A = a + b \Be_1 + c \Be_2 \Be_3 \Be_4,
\end{equation*}
where \( a, b, c\) are scalars.  Reversion can be performed term by term, resulting in a sign flip for the trivector term.

\begin{equation*}
   \tilde{A} = a + b \Be_1 - c \Be_2 \Be_3 \Be_4.
\end{equation*}

2. For your bivector example \( A = \Bn \Bm \),
first note that such a factorization may not
generally be possible.  For example,
an \R{4} bivector such as \( A = \Be_{12} + \Be_{34} \) cannot be decomposed into a product of two vectors (this is an example of a bivector that
is not a 2-blade.)

If the bivector has such a factorization, that decomposition amounts to a choice of basis.
For example:
\begin{equation*}
A = \Be_{12} + \Be_{23} + \Be_{31},
\end{equation*}
may be factored as
\begin{equation*}
A = \lr{ \Be_1 + \Be_2 - 2 \Be_3 } \frac{ \Be_2 - \Be_1 }{2},
\end{equation*}
or
\begin{equation*}
A = \frac{ \Be_3 - \Be_2 }{2} \lr{ 2 \Be_1 - \Be_2 - \Be_3 }.
\end{equation*}
We may compute the reverse from the original representation
\begin{equation*}
   \tilde{A} = \Be_{21} + \Be_{32} + \Be_{13},
\end{equation*}
or using the first factorization
\begin{equation*}
\begin{aligned}
\tilde{A}
&=
\frac{ \Be_2 - \Be_1 }{2}
\lr{ \Be_1 + \Be_2 - 2 \Be_3 }  \\
&=
\inv{2} \lr{ \Be_{21} - \Be_{11} + \Be_{22} - \Be_{12} - 2 \Be_{23} + 2 \Be_{13} } \\
&=
\Be_{21} + \Be_{32} + \Be_{13},
\end{aligned}
\end{equation*}
or using the second factorization
\begin{equation*}
\begin{aligned}
   \tilde{A}
   &=
\lr{ 2 \Be_1 - \Be_2 - \Be_3 }
\frac{ \Be_3 - \Be_2 }{2}  \\
   &=
\inv{2} \lr{
   2 \Be_{13} - 2 \Be_{12}
   - \Be_{23} + \Be_{22}
   - \Be_{33} + \Be_{32}
} \\
&=
\Be_{21} + \Be_{32} + \Be_{13}.
\end{aligned}
\end{equation*}
One should get the same answer regardless.

Also observe that such a 2-blade factorization depends on the orthogonality of the factors.  If that were not the case, then there would be a scalar term in the product, and the result would not be a 2-blade.  To see that, consider the following product of vectors
\begin{equation*}
A = \lr{ \Be_1 e^{i \theta} } \Be_1 = \cos\theta - i \sin\theta
\end{equation*}
where \( i = \Be_{1} \Be_{2} \) is the pseudoscalar for the plane that \( A \) lies in (i.e. \( A \wedge i = 0 \)).  For this to be a bivector, we require \( \theta \in \pi/2 + n \pi \).  That is, the two factors must be orthogonal.

If one wanted to prove that the reverse of any bivector \( A = n m \) is independent of the factorization (if such a factorization is possible), the task is essentially to show that the reverse is independent of any change of basis.

%}
\EndArticle
%\EndNoBibArticle
