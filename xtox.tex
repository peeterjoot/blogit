%
% Copyright � 2020 Peeter Joot.  All Rights Reserved.
% Licenced as described in the file LICENSE under the root directory of this GIT repository.
%
%{
\input{../latex/blogpost.tex}
\renewcommand{\basename}{xtox}
%\renewcommand{\dirname}{notes/phy1520/}
\renewcommand{\dirname}{notes/ece1228-electromagnetic-theory/}
%\newcommand{\dateintitle}{}
%\newcommand{\keywords}{}

\input{../latex/peeter_prologue_print2.tex}

\usepackage{peeters_layout_exercise}
\usepackage{peeters_braket}
\usepackage{peeters_figures}
\usepackage{siunitx}
\usepackage{verbatim}
%\usepackage{mhchem} % \ce{}
%\usepackage{macros_bm} % \bcM
%\usepackage{macros_qed} % \qedmarker
%\usepackage{txfonts} % \ointclockwise

\beginArtNoToc

\generatetitle{Exploring \(x^x\).}
%\chapter{XXX}
%\label{chap:xtox}

%My Youtube home page knows that I'm geeky enough to watch math videos and suggested a video today about \(0^0\).

Let \( z = r e^{i \theta} \), then
\begin{dmath}\label{eqn:xtox:20}
\ln z^z
= z \ln \lr{ r e^{i\theta} }
= z \ln r + i \theta z
= e^{i\theta} \ln r^r + i \theta z,
= \lr{ \cos\theta + i \sin\theta } \ln r^r + i r \theta \lr{ \cos\theta + i \sin\theta }
= \cos\theta \ln r^r - r \theta \sin\theta
+ i r \lr{ \sin\theta \ln r + \theta \cos\theta },
\end{dmath}
so
\begin{dmath}\label{eqn:xtox:40}
z^z =
e^{ r \lr{ \cos\theta \ln r - \theta \sin\theta}} \times
e^{i r \lr{ \sin\theta \ln r + \theta \cos\theta }}.
\end{dmath}
In particular, picking \( \theta = \pi \), for any \( x > 0 \), we have
\begin{dmath}\label{eqn:xtox:60}
   (-x)^{-x} = e^{-x \ln x - i x \pi } = \frac{e^{ - i x \pi }}{x^x}.
\end{dmath}

%}
\EndArticle
%\EndNoBibArticle
