%
% Copyright � 2024 Peeter Joot.  All Rights Reserved.
% Licenced as described in the file LICENSE under the root directory of this GIT repository.
%
%{
\input{../latex/blogpost.tex}
\renewcommand{\basename}{curlcurl}
%\renewcommand{\dirname}{notes/phy1520/}
\renewcommand{\dirname}{notes/ece1228-electromagnetic-theory/}
%\newcommand{\dateintitle}{}
%\newcommand{\keywords}{}

\input{../latex/peeter_prologue_print2.tex}

\usepackage{peeters_layout_exercise}
\usepackage{peeters_braket}
\usepackage{peeters_figures}
\usepackage{siunitx}
\usepackage{verbatim}
%\usepackage{mhchem} % \ce{}
%\usepackage{macros_bm} % \bcM
%\usepackage{macros_qed} % \qedmarker
%\usepackage{txfonts} % \ointclockwise

\beginArtNoToc

\generatetitle{XXX}
%\chapter{XXX}
%\label{chap:curlcurl}

% https://math.stackexchange.com/questions/4914846/help-with-proof-of-curl-double-product-identity-using-geometric-algebra-most-th/4915224#4915224

You write:

\begin{equation*}
\spacegrad (\spacegrad\cdot \BF) = \spacegrad\cdot(\spacegrad\cdot \BF) + \spacegrad\wedge(\spacegrad\cdot \BF),
\end{equation*}

but it appears that you are trying to use the vector identity

\begin{equation*}
\Ba \Bb = \Ba \cdot \Bb + \Ba \wedge \Bb,
\end{equation*}

(true for vectors \( \Ba, \Bb \)), but not neccessarily for multivectors.  In your case \( \spacegrad \cdot \BF \) is a scalar, not a vector, so there is not really any meaningful reduction that you can make of \( \spacegrad (\spacegrad\cdot \BF) \).

The essense of the identity you are trying to find is given by the following manipulation
\begin{equation*}
\begin{aligned}
\Ba \cross \lr{ \Ba \cross \Bb }
&=
-\lr{ I \Ba } \wedge \lr{ \Ba \cross \Bb }  \\
&=
\gpgradeone{
-\lr{ I \Ba } \wedge \lr{ \Ba \cross \Bb }
} \\
&=
-\gpgradeone{
I \Ba \lr{ \Ba \cross \Bb }
} \\
&=
-\gpgradeone{
I \Ba \lr{ -I \Ba \wedge \Bb }
} \\
&=
-\gpgradeone{
\Ba \lr{ \Ba \wedge \Bb }
} \\
&=
-\gpgradeone{
\Ba \cdot \lr{ \Ba \wedge \Bb }
+
\Ba \wedge \lr{ \Ba \wedge \Bb }
} \\
&=
-
\Ba \cdot \lr{ \Ba \wedge \Bb } \\
&=
-\Ba^2 \Bb + \Ba \lr{ \Ba \cdot \Bb }.
\end{aligned}
\end{equation*}
Notes:
In this case, it's also true that \( \Ba B = \Ba \cdot B + \Ba \wedge B \) for bivector \( B \) and vector \(\Ba \).  Here the dot and wedge products have the following meanings:
\begin{equation*}
\begin{aligned}
\Ba \cdot B &= \gpgradeone{ \Ba B } \\
\Ba \wedge B &= \gpgradethree{ \Ba B }.
\end{aligned}
\end{equation*}
Also observe that we are free to insert a no-op grade-one selection around any vector without changing the meaning.  That allows us to discard the trivector term of the vector-bivector product.
We also used the distribution identity
\begin{equation*}
\Bx \cdot \lr{ \By \wedge \Bz } = \lr{ \Bx \cdot \By } \Bz - \lr{ \Bx \cdot \Bz } \By.
\end{equation*}


If we summarize what we've done, essentially, we've first observed that

\begin{equation*}
\Ba \cross \lr{ \Bb \cross \Bc }
=
-
\Ba \cdot \lr{ \Bb \wedge \Bc },
\end{equation*}

for any vectors \( \Ba, \Bb, \Bc \).  In particular

\begin{equation*}
\Ba \cross \lr{ \Ba \cross \BF }
=
-
\Ba \cdot \lr{ \Ba \wedge \BF } = -\lr{ \Ba \cdot \Ba } \BF + \Ba \lr{ \Ba \cdot \BF },
\end{equation*}

This also holds for the case where \( \Ba = \spacegrad \), provided we take care to keep the gradients on the left (or use notational sugar to indicate the scope of their action.)

%}
\EndArticle
%\EndNoBibArticle
