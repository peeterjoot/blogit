%
% Copyright � 2024 Peeter Joot.  All Rights Reserved.
% Licenced as described in the file LICENSE under the root directory of this GIT repository.
%
%{
\input{../latex/blogpost.tex}
\renewcommand{\basename}{spherical}
%\renewcommand{\dirname}{notes/phy1520/}
\renewcommand{\dirname}{notes/ece1228-electromagnetic-theory/}
%\newcommand{\dateintitle}{}
%\newcommand{\keywords}{}

\input{../latex/peeter_prologue_print2.tex}

\usepackage{peeters_layout_exercise}
\usepackage{peeters_braket}
\usepackage{peeters_figures}
\usepackage{siunitx}
\usepackage{verbatim}
%\usepackage{mhchem} % \ce{}
%\usepackage{macros_bm} % \bcM
%\usepackage{macros_qed} % \qedmarker
%\usepackage{txfonts} % \ointclockwise

\beginArtNoToc

\generatetitle{XXX}
%\chapter{XXX}
%\label{chap:spherical}

% answering: https://discord.com/channels/607264339480674324/1232470033645436960/1232509219094269973
Quick back of scrap paper calculation,...

Let \( \Bx = r \rcap \), where \( \rcap = \Be_3 e^{j \theta} \), \( j = \Be_{31} e^{i\phi} \), and \( i = \Be_{12} \).  Then
\begin{equation}\label{eqn:spherical:20}
\dot{\Bx} = \rdot \rcap + r \dot{\rcap}.
\end{equation}
That radial unit vector's derivative is
\begin{equation}\label{eqn:spherical:40}
\begin{aligned}
\dot{\rcap}
&= \Be_3 j e^{j\theta} \dot{\theta} + \Be_3 \dot{\phi} \PD{\phi}{} \lr{ \cos\theta + j \sin\theta } \\
&= \Be_3 j e^{j\theta} \dot{\theta} + \Be_3 \sin\theta \dot{\phi} \PD{\phi}{} \lr{ \Be_{31} e^{i\phi} } \\
&= \Be_3 j e^{j\theta} \dot{\theta} + \Be_{33112} \sin\theta \dot{\phi} e^{i\phi} \\
&= \Be_3 j e^{j\theta} \dot{\theta} + \Be_{2} e^{i\phi} \sin\theta \dot{\phi}.
\end{aligned}
\end{equation}
This can be plugged in to find the spherical representation of the velocity
\begin{equation}\label{eqn:spherical:60}
\Bv = \rdot \rcap + r \lr{ \Be_3 j e^{j\theta} \dot{\theta} + \Be_{2} e^{i\phi} \sin\theta \dot{\phi} },
\end{equation}
which reduces to
\begin{equation}\label{eqn:spherical:70}
\Bv = \rdot \rcap + r \dot{\theta} \thetacap + r \sin\theta \dot{\phi} \phicap.
\end{equation}
The magnitude of the velocity is
\begin{equation}\label{eqn:spherical:80}
\Bv^2 = \rdot^2 + r^2 \dot{\theta}^2 + r^2 \sin^2 \theta \dot{\phi}^2,
\end{equation}
which has both radial and non-radial contributions.

%}
\EndArticle
%\EndNoBibArticle
