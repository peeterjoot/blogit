%
% Copyright � 2023 Peeter Joot.  All Rights Reserved.
% Licenced as described in the file LICENSE under the root directory of this GIT repository.
%
%{
\input{../latex/blogpost.tex}
\renewcommand{\basename}{test}
%\renewcommand{\dirname}{notes/phy1520/}
\renewcommand{\dirname}{notes/ece1228-electromagnetic-theory/}
%\newcommand{\dateintitle}{}
%\newcommand{\keywords}{}

\input{../latex/peeter_prologue_print2.tex}

\usepackage{peeters_layout_exercise}
\usepackage{peeters_braket}
\usepackage{peeters_figures}
\usepackage{siunitx}
\usepackage{verbatim}
%\usepackage{mhchem} % \ce{}
%\usepackage{macros_bm} % \bcM
%\usepackage{macros_qed} % \qedmarker
%\usepackage{txfonts} % \ointclockwise

\beginArtNoToc

\generatetitle{XXX}
%\chapter{XXX}
%\label{chap:test}

\begin{equation}\label{eqn:test:20}
\begin{aligned}
\spacegrad \cdot ( f \Bf )  &= (\spacegrad f ) \cdot \Bf + f ( \spacegrad \cdot \Bf ) \\
\spacegrad \times ( f \Bf ) &= (\spacegrad f ) \times \Bf + f ( \spacegrad \times \Bf ) \\
\spacegrad \wedge ( f A ) &= (\spacegrad f ) \wedge A + f ( \spacegrad \wedge A )
\end{aligned}
\end{equation}

\section{STA}

\begin{equation}\label{eqn:test:40}
\begin{aligned}
F &= \BE + I \eta \BH = \BE + I c \BB \\
\eta &= \sqrt{\mu/\epsilon} \\
c &= 1/\sqrt{\epsilon\mu} \\
\lr{ \spacegrad + \inv{c} \PD{t}{} } F &= \eta \lr{ c \rho - \BJ }.
\end{aligned}
\end{equation}
Now write the \R{3} basis vectors as
\begin{equation}\label{eqn:test:60}
\Be_i = \gamma_i \gamma_0,
\end{equation}
and multiply through on the left by \( \gamma_0 \)
\begin{equation}\label{eqn:test:80}
\gamma_0 \lr{ \spacegrad + \inv{c} \PD{t}{} } F = \gamma_0 \eta \lr{ c \rho - \BJ }.
\end{equation}

That's the STA form of Maxwell's equations, once you make the following identifications
\begin{equation}\label{eqn:test:100}
\begin{aligned}
\gamma_0 \gamma_\mu \gamma_0 &= g^{\nu\mu} \gamma_\nu = \gamma^\mu \\
\partial_0 &= \inv{c} \PD{t}{} = \PD{x^0}{} \\
\partial_k &= \PD{x^k}{} = \spacegrad \cdot \Be_k \\
\grad &= \gamma^\mu \partial_\mu \\
J^0 &= \eta c \rho = \rho/\epsilon \\
J^k &= \eta \BJ \cdot \Be_k \\
J &= J^\mu \gamma_\mu
\end{aligned}
\end{equation}

Note that if you write \( F = \BE + I c \BB \), where \( \BE = \sum E^k \Be_k, \BB = \sum B^k \Be_k \), then once you identify \( \Be_i = \gamma_i \gamma_0 \), you have
\begin{equation}\label{eqn:test:120}
\begin{aligned}
I &= \Be_1 \Be_2 \Be_3  \\
&= \gamma_1 \gamma_0 \gamma_2 \gamma_0 \gamma_3 \gamma_0 \\
&= -\gamma_1 \gamma_0 \gamma_0 \gamma_2 \gamma_3 \gamma_0 \\
&= -\gamma_1 \gamma_2 \gamma_3 \gamma_0 \\
&= \gamma_0 \gamma_1 \gamma_2 \gamma_3,
\end{aligned}
\end{equation}
so
\begin{equation}\label{eqn:test:140}
\begin{aligned}
F
&= \BE + I c \BB \\
&= \sum_k \lr{ E^k + \gamma_0 \gamma_1 \gamma_2 \gamma_3 c B^k } \gamma_k \gamma_0,
\end{aligned}
\end{equation}
is an STA bivector automatically.

Yes, we can put the gradient on the left.

With $N = A + I K$, we can write $F = \langle \nabla \tilde{N} \rangle {}_2 = \nabla \wedge A + I (\nabla \wedge K)$.

The Lagrangian is unchanged, and we still have $\mathcal{L} = (1/2) F^2 - \langle N (J - I M) \rangle {}_{0,4}$.  Variation of the associated action $S = \int d^4 x \mathcal{L}$, leads directly to Maxwell's equation: $\nabla F = J - I M$.

In conventional electromagnetism, we we use the identities
\begin{equation}\label{eqn:test:160}
\begin{aligned}
\spacegrad \cdot \lr{ \spacegrad \cross \Bf } &= 0 \\
\spacegrad \cross \lr{ \spacegrad \chi } &= 0.
\end{aligned}
\end{equation}
to pick
\begin{equation}\label{eqn:test:180}
\begin{aligned}
\BB &= \spacegrad \cross \BA \\
\BE &= -\spacegrad \phi - c \partial_0 \BA,
\end{aligned}
\end{equation}
to pick \( \phi, \BA \) so that, by construction, the source free Maxwell equations are satisfied.
\begin{equation}\label{eqn:test:200}
\begin{aligned}
\spacegrad \cdot \BB &= 0 \\
c \partial_0 \BB + \spacegrad \cross \BE &= 0.
\end{aligned}
\end{equation}
See Jackson 2nd edition, 6.4 Vector and Scalar Potentials.

Such choices are not unique, as we may alter the potentials with a gauge transformation
\begin{equation}\label{eqn:test:220}
\begin{aligned}
\BA &\rightarrow \BA + \spacegrad \chi \\
\phi &\rightarrow \phi - \partial_0 \chi,
\end{aligned}
\end{equation}
which does not change the observable fields \( \BE, \BB \).

The GA equivalent of the identities above is
\begin{equation}\label{eqn:test:240}
\spacegrad \wedge \spacegrad \wedge K = 0,
\end{equation}
for any blade \( K \).

Similarily, in STA, it's true that
\begin{equation}\label{eqn:test:260}
\grad \wedge \grad \wedge K = 0,
\end{equation}
for any blade \( K \).

Maxwell's equation in STA is
\begin{equation}\label{eqn:test:280}
\grad F = J,
\end{equation}
which we can split into vector and trivector grades, as
\begin{equation}\label{eqn:test:300}
\begin{aligned}
\grad \cdot F &= J \\
\grad \wedge F &= 0.
\end{aligned}
\end{equation}

Just as we did in conventional electromagnetism, that if we let
\begin{equation}\label{eqn:test:320}
F = \grad \wedge A,
\end{equation}
we automatically satisify the source free (trivector) grade of Maxwell's equation.

Is this the general solution?  No.  We are also able to make any gauge transformation of this potential
\begin{equation}\label{eqn:test:340}
A \rightarrow A + \grad \chi,
\end{equation}
which leaves \( F \) unchanged.

\section{On the nature of \( F \).}
In the STA form of Maxwell's equation, our field \( F \) is a bivector.  We've seen that any \( F \) defined as the curl of a four-vector potential plus the gradient of a scalar, satisfies the source free Maxwell's equations
\begin{equation}\label{eqn:test:360}
F = \grad \wedge \lr{ A + \grad \chi} = \grad \wedge A.
\end{equation}

\section{Q}

I think you need \( \Bv^2 = \Bv \cdot \Bv \), the contraction axiom, as the fundamental principle.  Once you have that, you can then expand something like \( \Be_1 + \Be_2 \) with both ways:
\begin{equation*}
\begin{aligned}
\lr{ \Be_1 + \Be_2 }^2 &= \Be_1 \Be_1 + \Be_1 \Be_2 + \Be_2 \Be_1 + \Be_2 \Be_2 = 2 + \Be_1 \Be_2 + \Be_2 \Be_1 \\
\lr{ \Be_1 + \Be_2 } \cdot \lr{ \Be_1 + \Be_2 } &= \Be_1 \cdot \Be_1 + \Be_1 \cdot \Be_2 + \Be_2 \cdot \Be_1 + \Be_2 \cdot \Be_2 = 2,
\end{aligned}
\end{equation*}
so that you can conclude that \( \Be_1 \Be_2 + \Be_2 \Be_1 = 0 \), or \( \Be_2 \Be_1 = -\Be_1 \Be_2 \).

%}
\EndArticle
%\EndNoBibArticle
