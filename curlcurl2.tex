%
% Copyright � 2025 Peeter Joot.  All Rights Reserved.
% Licenced as described in the file LICENSE under the root directory of this GIT repository.
%
%{
\input{../latex/blogpost.tex}
\renewcommand{\basename}{curlcurl2}
%\renewcommand{\dirname}{notes/phy1520/}
\renewcommand{\dirname}{notes/ece1228-electromagnetic-theory/}
%\newcommand{\dateintitle}{}
%\newcommand{\keywords}{}

\input{../latex/peeter_prologue_print2.tex}

\usepackage{peeters_layout_exercise}
\usepackage{peeters_braket}
\usepackage{peeters_figures}
\usepackage{siunitx}
\usepackage{verbatim}
%\usepackage{macros_cal} % \LL
%\usepackage{amsthm} % proof
%\usepackage{mhchem} % \ce{}
%\usepackage{macros_bm} % \bcM
%\usepackage{macros_qed} % \qedmarker
%\usepackage{txfonts} % \ointclockwise

\beginArtNoToc

\generatetitle{Curl of Curl.  Tensor and GA expansion, and GA equivalent identity.}
%\chapter{Curl of Curl}
%\label{chap:curlcurl2}

In this blog post, we will expand \(\spacegrad \cross \lr{ \spacegrad \cross \Bf } = -\spacegrad^2 \Bf + \spacegrad \lr{ \spacegrad \cdot \Bf } \) two different ways, using tensor index gymnastics and using geometric algebra.

\section{The tensor way.}
To expand the curl using a tensor expansion, let's first expand the cross product in coordinates
\begin{equation}\label{eqn:curlcurl2:20}
\begin{aligned}
\Ba \cross \Bb
&=
\lr{ \Be_r \cross \Be_s } a_r b_s \\
&=
\Be_t \cdot \lr{ \Be_r \cross \Be_s } \Be_t a_r b_s \\
&=
\epsilon_{rst} a_r b_s \Be_t.
\end{aligned}
\end{equation}
Here \( \epsilon_{rst} \) is the completely antisymmetric (Levi-Civita) tensor, and allows us to compactly express the geometrical nature of the triple product.

We can then expand the curl of the curl by applying this twice
\begin{equation}\label{eqn:curlcurl2:40}
\begin{aligned}
\spacegrad \cross \lr{ \spacegrad \cross \Bf }
&=
\epsilon_{rst} \partial_r \lr{ \spacegrad \cross \Bf }_s \Be_t \\
&=
\epsilon_{rst} \partial_r \lr{ \epsilon_{uvw} \partial_u f_v \Be_w }_s \Be_t \\
&=
\epsilon_{rst} \partial_r \epsilon_{uvs} \partial_u f_v \Be_t.
\end{aligned}
\end{equation}

It turns out that there's a nice identity to reduce the single index contraction of a pair of Levi-Civita tensors.
\begin{equation}\label{eqn:curlcurl2:60}
\epsilon_{abt} \epsilon_{cdt} = \delta_{ac} \delta_{bd} - \delta_{ad} \delta_{bc}.
\end{equation}
To show this, consider the \( t = 1 \) term of this sum \( \epsilon_{ab1} \epsilon_{cd1} \).  This is non-zero only for \( a,b,c,d \in \setlr{2,3} \).  If \( a,b = c,d \), this is one, and if \( a,b = d,c \), this is minus one.  We may summarize that as
\begin{equation}\label{eqn:curlcurl2:80}
\epsilon_{ab1} \epsilon_{cd1} = \delta_{ac} \delta_{bd} - \delta_{ad} \delta_{bc},
\end{equation}
but this holds for \( t = 2,3 \) too, so \cref{eqn:curlcurl2:60} holds generally.

We may now contract the tensors to find
\begin{equation}\label{eqn:curlcurl2:100}
\begin{aligned}
\spacegrad \cross \lr{ \spacegrad \cross \Bf }
&=
\epsilon_{rst} \epsilon_{uvs} \Be_t \partial_r \partial_u f_v \\
&=
-\epsilon_{rts} \epsilon_{uvs} \Be_t \partial_r \partial_u f_v \\
&=
-\lr{ \delta_{ru} \delta_{tv} - \delta_{rv} \delta_{tu} } \Be_t \partial_r \partial_u f_v \\
&=
- \Be_v \partial_u \partial_u f_v
+ \Be_u \partial_v \partial_u f_v \\
&=
-\spacegrad^2 \Bf + \spacegrad \lr{ \spacegrad \cdot \Bf }.
\end{aligned}
\end{equation}

\section{Using geometric algebra.}
Now let's pull out the GA toolbox.  We start with introducing a no-op grade-1 selection, and using the identity \( \Ba \cross \Bb = -I \lr{ \Ba \wedge \Bb } \)
\begin{equation}\label{eqn:curlcurl2:120}
\begin{aligned}
\spacegrad \cross \lr{ \spacegrad \cross \Bf }
&=
\gpgradeone{
\spacegrad \cross \lr{ \spacegrad \cross \Bf }
} \\
&=
\gpgradeone{
-I \lr{ \spacegrad \wedge \lr{ \spacegrad \cross \Bf } }
} \\
\end{aligned}
\end{equation}
We can now expand \( \Ba \wedge \Bb = \Ba \Bb - \Ba \cdot \Bb \)
\begin{equation}\label{eqn:curlcurl2:140}
\spacegrad \cross \lr{ \spacegrad \cross \Bf }
=
\gpgradeone{
-I \spacegrad \lr{ \spacegrad \cross \Bf }
+I \lr{ \spacegrad \cdot \lr{ \spacegrad \cross \Bf } }
}
\end{equation}
but that dot product is a scalar, leaving just a pseudoscalar, which has a zero grade-1 selection.  This leaves
\begin{equation}\label{eqn:curlcurl2:160}
\begin{aligned}
\spacegrad \cross \lr{ \spacegrad \cross \Bf }
&=
\gpgradeone{
-I \spacegrad \lr{ -I \lr{ \spacegrad \wedge \Bf } }
} \\
&=
-\gpgradeone{
\spacegrad \lr{ \spacegrad \wedge \Bf }
}.
\end{aligned}
\end{equation}
We use \( \Ba \wedge \Bb = \Ba \Bb - \Ba \cdot \Bb \) once more
\begin{equation}\label{eqn:curlcurl2:180}
\begin{aligned}
\spacegrad \cross \lr{ \spacegrad \cross \Bf }
&=
-\gpgradeone{
\spacegrad \lr{ \spacegrad \Bf }
-\spacegrad \lr{ \spacegrad \cdot \Bf }
}
\\
&=
-\spacegrad^2 \Bf
+\spacegrad \lr{ \spacegrad \cdot \Bf }.
\end{aligned}
\end{equation}

\section{GA identity.}
It's also worth noting that there's a natural GA formulation of the curl of a curl.  From the Laplacian and divergence relationship that we ended up with, we need only factor out the gradient
\begin{equation}\label{eqn:curlcurl2:200}
\begin{aligned}
\spacegrad \cross \lr{ \spacegrad \cross \Bf }
&=
-\spacegrad^2 \Bf +\spacegrad \lr{ \spacegrad \cdot \Bf } \\
&=
-\spacegrad \lr{ \spacegrad \Bf - \spacegrad \cdot \Bf } \\
&=
-\spacegrad \lr{ \spacegrad \wedge \Bf }.
\end{aligned}
\end{equation}
Because \( \spacegrad \wedge \lr{ \spacegrad \wedge \Bf } = 0 \), we may also write this as
\begin{equation}\label{eqn:curlcurl2:220}
\boxed{
\spacegrad \cdot \lr{ \spacegrad \wedge \Bf } = -\spacegrad \cross \lr{ \spacegrad \cross \Bf }.
}
\end{equation}
From the GA LHS, we see by inspection that
\begin{equation}\label{eqn:curlcurl2:240}
\spacegrad \cdot \lr{ \spacegrad \wedge \Bf } = \spacegrad^2 \Bf - \spacegrad \lr{ \spacegrad \cdot \Bf }.
\end{equation}

%}
%\EndArticle
\EndNoBibArticle
