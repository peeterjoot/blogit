%
% Copyright � 2021 Peeter Joot.  All Rights Reserved.
% Licenced as described in the file LICENSE under the root directory of this GIT repository.
%
%{
\input{../latex/blogpost.tex}
\renewcommand{\basename}{parallelogram}
%\renewcommand{\dirname}{notes/phy1520/}
\renewcommand{\dirname}{notes/ece1228-electromagnetic-theory/}
%\newcommand{\dateintitle}{}
%\newcommand{\keywords}{}

\input{../latex/peeter_prologue_print2.tex}

\usepackage{peeters_layout_exercise}
\usepackage{peeters_braket}
\usepackage{peeters_figures}
\usepackage{siunitx}
\usepackage{verbatim}
%\usepackage{mhchem} % \ce{}
%\usepackage{macros_bm} % \bcM
%\usepackage{macros_qed} % \qedmarker
%\usepackage{txfonts} % \ointclockwise

\beginArtNoToc

\generatetitle{Area of a parallelogram}
%\chapter{Area of a parallelogram}
%\label{chap:parallelogram}

\begin{dmath}\label{eqn:parallelogram:20}
\textrm{Area}^2
= \Norm{\Bx}^2 \Norm{\By - \lr{\By \cdot \xcap} \xcap }^2
= \Norm{\Bx}^2
\lr{\By - \lr{\By \cdot \xcap} \xcap }\cdot \lr{\By - \lr{\By \cdot \xcap} \xcap }
= \Norm{\Bx}^2
\lr{
\By \cdot \By + \lr{ \By \cdot \xcap }^2 \xcap \cdot \xcap - 2 \lr{ \xcap \cdot \By }^2
}
= \Norm{\Bx}^2
\lr{
\By \cdot \By - \lr{ \By \cdot \xcap }^2
}
=
\Norm{\Bx}^2 \Norm{\By}^2 - \lr{ \By \cdot \xcap }^2 \Norm{\Bx}^2
=
\Norm{\Bx}^2 \Norm{\By}^2 - \lr{ \Bx \cdot \By }^2.
\end{dmath}

In components
\begin{dmath}\label{eqn:parallelogram:40}
\textrm{Area}^2
=
\Norm{\Bx}^2 \Norm{\By}^2 - \lr{ \Bx \cdot \By }^2
=
\sum_{i = 1}^N x_i x_i
\sum_{j = 1}^N x_j x_j
-
\sum_{i = 1}^N x_i y_i
\sum_{j = 1}^N x_j y_j
=
\sum_{i,j= 1}^N x_i x_i y_j y_j - x_i y_i x_j y_j
=
\sum_{i,j= 1}^N x_i y_j \lr{ x_i y_j - y_i x_j }
=
\sum_{i,j= 1}^N
x_i y_j
\begin{vmatrix}
   x_i & y_i \\
   x_j & y_j \\
\end{vmatrix}
=
\lr{
\sum_{i<j}
+
\cancel{\sum_{i=j}}
+
\sum_{i>j}
}
x_i y_j
\begin{vmatrix}
   x_i & y_i \\
   x_j & y_j \\
\end{vmatrix}.
\end{dmath}

But
\begin{dmath}\label{eqn:parallelogram:60}
\sum_{i > j}
x_i y_j
\begin{vmatrix}
   x_i & y_i \\
   x_j & y_j \\
\end{vmatrix}
=
\sum_{a > b}
x_a y_b
\begin{vmatrix}
   x_a & y_a \\
   x_b & y_b \\
\end{vmatrix}
=
% b -> i
% a -> j
\sum_{j > i}
x_j y_i
\begin{vmatrix}
   x_j & y_j \\
   x_i & y_i \\
\end{vmatrix}
=
-\sum_{i < j}
x_j y_i
\begin{vmatrix}
   x_i & y_i \\
   x_j & y_j \\
\end{vmatrix},
\end{dmath}
so
\begin{dmath}\label{eqn:parallelogram:80}
\textrm{Area}^2
=
\sum_{i < j}
\lr{ x_i y_j - x_j y_i  }
\begin{vmatrix}
   x_i & y_i \\
   x_j & y_j \\
\end{vmatrix}
=
\sum_{i < j}
{\begin{vmatrix}
   x_i & y_i \\
   x_j & y_j \\
\end{vmatrix}}^2.
\end{dmath}

In \R{3} this happens to be the squared length of \( \Bx \cross \By \), but in \R{N} this 
has the look of the length of a two index entity with \( \binom{N}{2} \) components,
where the coordinates are determinants of the from above.  
More generally, when we study geometric algebra, we will see that the
beastie with these components is actually the bivector \( B = \Bx \wedge \By \).

${\left\lVert{\vec{u}}\right\rVert}^2{\left\lVert{\vec{v} - \left(\vec{v} \cdot \hat{u} \right)\hat{u}}\right\rVert}^2$


%${\left\lVert{\vec{u}}\right\rVert}^2\left({\left\lVert{\vec{v}}\right\rVert}^2 + {\left( \vec{v} \cdot \hat{u} \right)}^2 - 2 {\left( \vec{v} \cdot \hat{u} \right)}^2\right) $
${\left\lVert{\vec{u}}\right\rVert}^2\left({\left\lVert{\vec{v}}\right\rVert}^2 + {\left( \vec{v} \cdot \hat{u} \right)}^2 - 2 {\left( \vec{v} \cdot \hat{u} \right)}^2\right) $

%}
%\EndArticle
\EndNoBibArticle


