%
% Copyright � 2022 Peeter Joot.  All Rights Reserved.
% Licenced as described in the file LICENSE under the root directory of this GIT repository.
%
%{
\input{../latex/blogpost.tex}
\renewcommand{\basename}{directionalDerivatives}
%\renewcommand{\dirname}{notes/phy1520/}
\renewcommand{\dirname}{notes/ece1228-electromagnetic-theory/}
%\newcommand{\dateintitle}{}
%\newcommand{\keywords}{}

\input{../latex/peeter_prologue_print2.tex}

\usepackage{peeters_layout_exercise}
\usepackage{peeters_braket}
\usepackage{peeters_figures}
\usepackage{siunitx}
\usepackage{verbatim}
%\usepackage{mhchem} % \ce{}
%\usepackage{macros_bm} % \bcM
%\usepackage{macros_qed} % \qedmarker
%\usepackage{txfonts} % \ointclockwise

\beginArtNoToc

\generatetitle{Differentiation in Geometric Calculus, Computing Vector Derivatives of Multivector-Valued Functions}
%\chapter{XXX}
%\label{chap:directionalDerivatives}

\section{Question}
I haven't found an explicit formula and way to compute vector derivatives in geometric calculus. For instance, let $V \simeq \mathbb{R}^3$ with the usual orthonormal basis $\{\textbf{e}_i\}_{i=1}^3$ and $C \ell(V)$ its universal Clifford algebra. Consider the multivector-valued function of a vector, that is $F: P_1(C\ell(V)) \to C \ell (V)$ (where $P_1$ is the projection operator), defined as
$$F(x) = x(\textbf{e}_1 - \textbf{e}_2) + \textbf{e}_1\textbf{e}_2 \textbf{e}_3$$
where $x \in P_1(C \ell (V))$. Consider that $x = \textbf{e}_1$, then
$$F(\textbf{e}_1) = \textbf{e}_1(\textbf{e}_1 - \textbf{e}_2) + \textbf{e}_1\textbf{e}_2 \textbf{e}_3  = {\textbf{e}_1}^2 - \textbf{e}_1 \textbf{e}_2 + \textbf{e}_1\textbf{e}_2\textbf{e}_3$$
$$F(\textbf{e}_1) = 1 - \textbf{e}_1 \textbf{e}_2 + \textbf{e}_1\textbf{e}_2\textbf{e}_3$$

What would it mean to take the vector derivative $\partial_x$ of the function $F$? My line of reasoning is
$$\partial_x F(x) = \partial_x (x \textbf{e}_1) - \partial_x (x\textbf{e}_2) + \partial_x (\textbf{e}_1\textbf{e}_2\textbf{e}_3)$$
and, using $x=\textbf{e}_1$ for instance, we would have
$$\partial_{\textbf{e}_1} F = \partial_{\textbf{e}_1}({\textbf{e}_1}^2) - \partial_{\textbf{e}_1}(\textbf{e}_1)\textbf{e}_2 + \partial_{\textbf{e}_1}(\textbf{e}_1)\textbf{e}_2\textbf{e}_3$$
where $\partial_{\textbf{e}_1}({\textbf{e}_1}^2) = 2\textbf{e}_1$, but ${\textbf{e}_1}^2 = 1$ and $\partial_{\textbf{e}_1}(1) = 0$, this reasoning leads to an ambiguity. In the end
$$\partial_{\textbf{e}_1} F = 2\textbf{e}_1 -\textbf{e}_2 + \textbf{e}_2\textbf{e}_3$$
or
$$\partial_{\textbf{e}_1} F = 0 -\textbf{e}_2 + \textbf{e}_2\textbf{e}_3$$

This most likely isn't correct, i'm having a hard time undestanding how to compute those derivatives in the Clifford algebra. If the question is answered, I would also like to understand how to compute an $n$-vector derivative and even a multivector derivative.

In Alan Mcdonald's book, Vector and Geometric Calculus, he treats $\mathbb{R}^m$ as a vector space and simply defines the vector derivative as
$$\partial_{h} F = h^i \frac{\partial F}{\partial x^i} $$
where $h = h^i\textbf{e}_i$ and $x^i$ are coordinates on $\mathbb{R}^m$. But this makes any function $F$ be implicitly defined on $\mathbb{R}^m$ and not general subspaces of $C \ell(\mathbb{R}^m)$.

In David Hestenes and Garret Sobczyk's book, Clifford Algebra to Geometric Calculus A Unified Language for Mathematics and Physics, they define the vector derivative using the directional directional derivative as
$$a \cdot \partial_X F(x) = \left.\frac{\partial}{\partial \tau} F(x+a\tau ) \right\vert_{\tau =0}
= \lim_{\tau \to 0} \frac{F(x+a\tau) - F(x)}{\tau}$$
and, duo the generality desired, they never go on to give $\partial_x$ an explicit formula, since this would require a choice of basis. They do derive extensively its properties and its "algebra", and derive that
$$\partial_x F = \partial_x \cdot F + \partial_x \wedge F$$

In the Wikipedia article on geometric calculus (https://en.wikipedia.org/wiki/Geometric_calculus), the derivative
$$\partial_{\textbf{e}_i} = \partial_i$$
is simply stated as the derivative in the direction of $\textbf{e}_i$, does this imply to calculate $\partial/\partial x^i$ just like Alan does in his book?

If this is indeed the case, that is, the association of points in $\mathbb{R}^n$ to vectors in $P_1(C\ell(V))$ is "essential" to compute those derivatives, how would this theory come when considering manifolds as the base space, since this impossibilitates the use of points as vectors.

So, a recap. I haven't been able to understand how to compute vector derivatives of multivector-valued functions on $P_1(C\ell(V))$. From all I could see, this operation depends on the base space $\mathbb{R}^n \simeq V$ to allow for those calculations, but this seems to restrict those functions to just $\mathbb{R}^n$ and not really to vectors, $p$-vectors and multivectors.

\section{Answer}
As you noted, there is some notational inconsistency between different authors on this subject.  You mentioned \citep{aMacdonaldVAGC} who writes the directional derivative as
\begin{equation}\label{eqn:directionalDerivatives:20}
   \partial_\Bh F(\Bx) = \lim_{t\rightarrow 0} \frac{F(\Bx + t \Bh) - F(\Bx)}{t},
\end{equation}
where he makes the identification \( \partial_\Bh F(\Bx) = \lr{ \Bh \cdot \spacegrad } F(\Bx) \).  Similarly \citep{doran2003gap} writes
\begin{equation}\label{eqn:directionalDerivatives:40}
   A * \partial_X F(X) = \evalbar{\frac{dF(X + t A)}{dt}}{t = 0},
\end{equation}
where \( A * B = \gpgradezero{ A B } \) is a scalar grade operator.  In the first case, the domain of the function \( F \) was vectors, whereas the second construction is an explicit multivector formulation.  Should the domain of \( F \) be restricted to vectors, we may make the identification \( \partial_X = \spacegrad = \sum e^i \partial_i \), however we are interested in the form of the derivative operator for multivectors.  To see how that works, let's expand out the direction derivative in coordinates.

The first step is a coordinate expansion of our multivector \( X \).  We may write
\begin{equation}\label{eqn:directionalDerivatives:60}
   X = \sum_{i < \cdots < j} \lr{ X * \lr{ e_i \wedge \cdots \wedge e_j } } \lr{ e_i \wedge \cdots \wedge e_j }^{-1},
\end{equation}
or
\begin{equation}\label{eqn:directionalDerivatives:80}
   X = \sum_{i < \cdots < j} \lr{ X * \lr{ e^i \wedge \cdots \wedge e^j } } \lr{ e^i \wedge \cdots \wedge e^j }^{-1}.
\end{equation}
In either case, the basis \( \setlr{ e_1, \cdots, e_m } \), need not be orthonormal, not Euclidean.  In the latter case, we've written the components of the multivector in terms of the reciprocal frame satisfying \( e^i \cdot e_j = {\delta^i}_j \), and \( e^i \in \Span \setlr{ e_1, \cdots, e_m } \). Both of these expansions are effectively coordinate expansions.  We may make that more explicit, by writing
\begin{equation}\label{eqn:directionalDerivatives:100}
\begin{aligned}
   X^{i \cdots j} &= X * \lr{ e^j \wedge \cdots \wedge e^i } \\
   X_{i \cdots j} &= X * \lr{ e_j \wedge \cdots \wedge e_i },
\end{aligned}
\end{equation}
so
\begin{equation}\label{eqn:directionalDerivatives:120}
   X
   = \sum_{i < \cdots < j} X^{i \cdots j} \lr{ e_i \wedge \cdots \wedge e_j }
   = \sum_{i < \cdots < j} X_{i \cdots j} \lr{ e^i \wedge \cdots \wedge e^j }.
\end{equation}

%}
\EndArticle
%\EndNoBibArticle
