%
% Copyright � 2025 Peeter Joot.  All Rights Reserved.
% Licenced as described in the file LICENSE under the root directory of this GIT repository.
%
%{
\input{../latex/blogpost.tex}
\renewcommand{\basename}{generalizedGaussian}
%\renewcommand{\dirname}{notes/phy1520/}
\renewcommand{\dirname}{notes/ece1228-electromagnetic-theory/}
%\newcommand{\dateintitle}{}
%\newcommand{\keywords}{}

\input{../latex/peeter_prologue_print2.tex}

\usepackage{peeters_layout_exercise}
\usepackage{peeters_braket}
\usepackage{peeters_figures}
\usepackage{siunitx}
\usepackage{verbatim}
%\usepackage{macros_cal} % \LL
%\usepackage{amsthm} % proof
%\usepackage{mhchem} % \ce{}
%\usepackage{macros_bm} % \bcM
%\usepackage{macros_qed} % \qedmarker
%\usepackage{txfonts} % \ointclockwise

\beginArtNoToc

\generatetitle{Generalized Gaussian}
%\chapter{Generalized Gaussian}
%\label{chap:generalizedGaussian}

Here's another problem from \citep{byron1992mca}.  The point is to show that
\begin{equation}\label{eqn:generalizedGaussian:20}
G(x,x',\tau) = \inv{2\pi} \int_{-\infty}^\infty e^{i k\lr{ x - x' } } e^{-k^2 \tau} dk,
\end{equation}
has the value
\begin{equation}\label{eqn:generalizedGaussian:40}
G(x,x',\tau) = \inv{\sqrt{4 \pi \tau} } e^{-\lr{ x - x'}^2/4 \tau },
\end{equation}
not just for real \(\tau\), but also for purely imaginary \(\tau\).

\section{Real case.}
The authors claim the real case is easy, but I don't think the real case is that trivial.  The trivial part is basically just completing the square.  Writing \( x - x ' = \Delta x \), that is
\begin{equation}\label{eqn:generalizedGaussian:60}
\begin{aligned}
-k^2 \tau + i k\lr{ x - x' }
&=
-\tau \lr{ k^2 - i k \Delta x/\tau } \\
&=
-\tau \lr{ \lr{ k - i \Delta x/2\tau }^2 - \lr{ - i \Delta x/2\tau }^2 } \\
&=
-\tau \lr{ \lr{ k - i \Delta x/2\tau }^2 + \lr{ \Delta x/2\tau }^2 }.
\end{aligned}
\end{equation}
So we have
\begin{equation}\label{eqn:generalizedGaussian:80}
G(x,x',\tau) = \inv{2\pi} e^{-(\Delta x/2)^2/\tau} \int_{-\infty}^\infty e^{-\tau \lr{ k - i \Delta x/2\tau }^2 } dk.
\end{equation}
Let's call the integral factor part of this \( I \)
\begin{equation}\label{eqn:generalizedGaussian:100}
I = \int_{-\infty}^\infty e^{-\tau \lr{ k - i \Delta x/2\tau }^2 } dk.
\end{equation}
However, making a change of variables makes this an integral over a complex path
\begin{equation}\label{eqn:generalizedGaussian:120}
I = \int_{-\infty - i \Delta x/2\tau }^{\infty - i \Delta x/2\tau} e^{-\tau k^2 } dk.
\end{equation}
If you are lazy you could say that \( \pm \infty \) adjusted by a constant, even if that constant is imaginary, leaves the integration limits unchanged.  That's clearly true if the constant is real, but I don't think it's that obvious if the constant is imaginary.

To answer that question more exactly, let's consider the integral
\begin{equation}\label{eqn:generalizedGaussian:140}
0 = I + J + K + L = \oint e^{-\tau z^2} dz,
\end{equation}
where the path is illustrated in \cref{fig:RectangularContour:RectangularContourFig1}.
\imageFigure{../figures/blogit/RectangularContourFig1}{Contour for the Gaussian.}{fig:RectangularContour:RectangularContourFig1}{0.2}
Since there are no enclosed poles, we have
\begin{equation}\label{eqn:generalizedGaussian:160}
I = \int_{-\infty}^\infty e^{-\tau k^2 } dk + K + L,
\end{equation}
where
\begin{equation}\label{eqn:generalizedGaussian:180}
\begin{aligned}
K &= \int_{- i \Delta x/2\tau}^0 e^{-\tau z^2} dz \\
L &= \int_0^{- i \Delta x/2\tau} e^{-\tau z^2} dz.
\end{aligned}
\end{equation}
We see now that we see perfect cancellation of \( K \) and \( L \), which justifies the change of variables, and the corresponding integration limits.

We can use the usual trick to evaluate \( I^2 \), to find
\begin{equation}\label{eqn:generalizedGaussian:200}
\begin{aligned}
I^2
&=
\int_{-\infty}^\infty e^{-\tau k^2 } dk
\int_{-\infty}^\infty e^{-\tau m^2 } dm \\
&=
2 \pi \int_0^\infty r e^{-\tau r^2} r dr \\
&=
2 \pi \evalrange{ - \frac{e^{-\tau r^2}}{-2 \tau} }{0}{\infty} \\
&=
\frac{\pi}{\tau}.
\end{aligned}
\end{equation}

So, for real values of \( \tau \) we have
\begin{equation}\label{eqn:generalizedGaussian:220}
G(x,x',\tau) = \inv{2\pi} \sqrt{\frac{\pi}{\tau}}e^{-(\Delta x/2)^2/\tau},
\end{equation}
as expected.

\section{Imaginary case.}
For the imaginary case, let \( \tau = i \alpha \).  Let's recomplete the square from scratch with this substitution
\begin{equation}\label{eqn:generalizedGaussian:240}
\begin{aligned}
-k^2 i \alpha + i k \Delta x
&=
- i \alpha \lr{ k^2 - \frac{i k \Delta x}{i \alpha} } \\
&=
- i \alpha \lr{ k^2 - \frac{k \Delta x}{\alpha} } \\
&=
- i \alpha \lr{ \lr{ k - \frac{\Delta x}{2 \alpha} }^2 - \lr{ \frac{\Delta x}{2 \alpha} }^2 }.
\end{aligned}
\end{equation}
So we have
\begin{equation}\label{eqn:generalizedGaussian:260}
\begin{aligned}
G(x,x',\tau)
&= \inv{2\pi} e^{i \alpha(\Delta x/2\alpha)^2} \int_{-\infty}^\infty e^{- i \alpha \lr{ k - \frac{\Delta x}{2 \alpha } }^2 } dk \\
&= \inv{\sqrt{\pi^2 \alpha}} e^{- (\Delta x)^2/4\tau} \int_0^\infty e^{- i m^2 } dm.
\end{aligned}
\end{equation}
This time we can make a \( m = \sqrt{\alpha} \lr{k - \frac{\Delta x}{2 \alpha }} \) change of variables, but don't have to worry about imaginary displacements of the integration limits.

The task is now reduced to the evaluation of an imaginary Gaussian like integral, and we are given the hint integrate \( e^{-z^2} \) over a pie shaped contour \cref{fig:PieContour:PieContourFig2}.
\imageFigure{../figures/blogit/PieContourFig2}{Pie shaped integration contour.}{fig:PieContour:PieContourFig2}{0.2}

Over \( I \) we set \( z = x, x \in [0, R] \), over \( J \), we set \( z = R e^{i\theta}, \theta \in [0, \pi/4] \), and on \( K \) we set \( z = u e^{i\pi/4}, u \in [R, 0] \).  Since there are no enclosed poles we have
\begin{equation}\label{eqn:generalizedGaussian:280}
\begin{aligned}
0 &= I + J + K \\
  &= \int_0^R e^{- x^2} dx + \int_0^{\pi/4} e^{-R^2 e^{2 i \theta} } i R e^{i\theta} d\theta - \int_0^R e^{-i u^2 } e^{i \pi/4} du.
\end{aligned}
\end{equation}
In the limit we have
\begin{equation}\label{eqn:generalizedGaussian:300}
\begin{aligned}
\int_0^\infty e^{-i u^2 } du
&= e^{-i\pi/4} \int_0^\infty e^{- x^2} dx + L \\
&= e^{-i\pi/4} \sqrt{\pi}/2 + L,
\end{aligned}
\end{equation}
where
\begin{equation}\label{eqn:generalizedGaussian:320}
L = \lim_{R\rightarrow \infty} e^{-i\pi/4} \int_0^{\pi/4} e^{-R^2 e^{2 i \theta} } i R e^{i\theta} d\theta.
\end{equation}
We hope that this is zero in the limit, but showing this requires a bit of care around the \( \pi/4 \) endpoint.  We start with
\begin{equation}\label{eqn:generalizedGaussian:340}
\begin{aligned}
\Abs{L}
&\le \lim_{R\rightarrow \infty} \int_0^{\pi/4} R e^{-R^2 \cos\lr{2 \theta} } d\theta \\
&= \lim_{R\rightarrow \infty} \inv{2} \int_{\pi/4 - \epsilon/2}^{\pi/4} R e^{-R^2 \cos\lr{2 \theta} } (2 d\theta),
\end{aligned}
\end{equation}
Now we make a change of variables
\begin{equation}\label{eqn:generalizedGaussian:360}
t = \frac{\pi}{2} - 2 \theta.
\end{equation}
At the limits we have
\begin{equation}\label{eqn:generalizedGaussian:380}
\begin{aligned}
t(\pi/4 - \epsilon/2) &= \epsilon \\
t(\pi/4) &= 0.
\end{aligned}
\end{equation}
Also,
\begin{equation}\label{eqn:generalizedGaussian:400}
\begin{aligned}
\cos\lr{ 2 \theta }
&= \Real \lr{ e^{2 i \theta} } \\
&= \Real \lr{ e^{i \lr{\pi/2 - t} } } \\
&= \Real \lr{ i e^{-i t } } \\
&= \sin t,
\end{aligned}
\end{equation}
so
\begin{equation}\label{eqn:generalizedGaussian:420}
\Abs{L} \le \lim_{R\rightarrow \infty} \inv{2} \int_0^\epsilon R e^{-R^2 \sin t} dt.
\end{equation}
Since we have forced \( t \) small, we can use the small angle approximation for the sine
\begin{equation}\label{eqn:generalizedGaussian:440}
\begin{aligned}
\int_0^\epsilon R e^{-R^2 \sin t} dt
&\approx
\int_0^\epsilon R e^{-R^2 t} dt \\
&= \evalrange{ \inv{-2 R} e^{-R^2 t} }{0}{\epsilon} \\
&= \frac{ 1 - e^{-R^2 \epsilon }}{2 R}.
\end{aligned}
\end{equation}
The numerator goes to zero for either \( \epsilon \rightarrow 0 \), or \( R \rightarrow \infty \), and the denominator to infinity, so we have the desired zero in the limit.  This means that
\begin{equation}\label{eqn:generalizedGaussian:460}
\begin{aligned}
G(x,x',\tau)
&= \frac{e^{-i\pi/4}}{\sqrt{4 \pi \alpha}} e^{- (\Delta x)^2/4\tau} \\
&= \frac{1}{\sqrt{4 \pi i \alpha}} e^{- (\Delta x)^2/4\tau} \\
&= \frac{1}{\sqrt{4 \pi \tau}} e^{- (\Delta x)^2/4\tau},
\end{aligned}
\end{equation}
as expected.

A fun application, also noted in the problem, is that we can decompose the imaginary integral
\begin{equation}\label{eqn:generalizedGaussian:480}
\int_{-\infty}^\infty e^{-i u^2} du = \sqrt{\pi} e^{-i\pi/4},
\end{equation}
into real and imaginary parts, to find
\begin{equation}\label{eqn:generalizedGaussian:500}
\int_{-\infty}^\infty \cos u^2 du = \int_{-\infty}^\infty \sin u^2 du = \sqrt{\frac{\pi}{2}}.
\end{equation}
Despite being real valued integrals, it it not at all obvious how one would go about finding those without these contour integration tricks.

%}
\EndArticle
