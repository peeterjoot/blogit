%
% Copyright � 2025 Peeter Joot.  All Rights Reserved.
% Licenced as described in the file LICENSE under the root directory of this GIT repository.
%
%{
\input{../latex/blogpost.tex}
\renewcommand{\basename}{unitCircleContourIntegrals}
%\renewcommand{\dirname}{notes/phy1520/}
\renewcommand{\dirname}{notes/ece1228-electromagnetic-theory/}
%\newcommand{\dateintitle}{}
%\newcommand{\keywords}{}

\input{../latex/peeter_prologue_print2.tex}

\usepackage{peeters_layout_exercise}
\usepackage{peeters_braket}
\usepackage{peeters_figures}
\usepackage{siunitx}
\usepackage{verbatim}
%\usepackage{macros_cal} % \LL
%\usepackage{amsthm} % proof
%\usepackage{mhchem} % \ce{}
%\usepackage{macros_bm} % \bcM
%\usepackage{macros_qed} % \qedmarker
%\usepackage{txfonts} % \ointclockwise

\beginArtNoToc

\generatetitle{Evaluating some real trig integrals with unit circle contours.}
%\chapter{Evaluating some real trig integrals with unit circle contours.}
%\label{chap:unitCircleContourIntegrals}

Here are a couple of problems from \citep{byron1992mca}.

\section{A sine integral.}
This is problem (31(a)).  For \( a > b > 0 \), find
\begin{equation}\label{eqn:unitCircleContourIntegrals:20}
I = \int_0^{2 \pi} \frac{d\theta}{a + b \sin\theta}.
\end{equation}
We can proceed by making a change of variables \( z = e^{i\theta} \), for which \( dz = i z d\theta \).  Also let \( \alpha = a/b \), so
\begin{equation}\label{eqn:unitCircleContourIntegrals:40}
\begin{aligned}
I
&= \inv{b} \oint_{\Abs{z} = 1} \frac{-i dz}{z} \inv{\alpha + (1/2i)\lr{z - 1/z}} \\
&= \frac{2}{b} \oint_{\Abs{z} = 1} \frac{dz}{2 i z \alpha + z^2 - 1} \\
&= \frac{2}{b} \oint_{\Abs{z} = 1} \frac{dz}{\lr{ z + i \alpha + i\sqrt{\alpha^2 - 1}}\lr{ z + i \alpha - i\sqrt{\alpha^2 - 1}}}.
\end{aligned}
\end{equation}
Clearly the mixed sign factor represents the pole that falls within the unit circle, so we have only one residue to include
\begin{equation}\label{eqn:unitCircleContourIntegrals:60}
\begin{aligned}
I
&= \frac{2}{b} 2 \pi i \evalbar{ \inv{ z + i \alpha + i\sqrt{\alpha^2 - 1}} }{ z = -i \alpha + i \sqrt{ \alpha^2 - 1 } } \\
&= \frac{4 \pi i}{b} \inv{ 2 i \sqrt{\alpha^2 - 1}} \\
&= \frac{2 \pi}{\sqrt{a^2 - b^2}}.
\end{aligned}
\end{equation}

\section{Sines and cosines upstairs and downstairs.}
This is problem (31(b)).  Given \( a > b > 0 \) (again), this time we want to find
\begin{equation}\label{eqn:unitCircleContourIntegrals:80}
I = \int_0^{2 \pi} \frac{\sin^2 \theta d\theta}{a + b \cos\theta}.
\end{equation}
We'd like to make the same \( z = e^{i \theta} \) substitution, but have to prepare a bit.  We rewrite the sine
\begin{equation}\label{eqn:unitCircleContourIntegrals:100}
\begin{aligned}
\sin^2 \theta
&= \inv{2} \lr{ 1 - \cos(2\theta) } \\
&= \inv{2} - \inv{4}\lr{ e^{2 i \theta} + e^{-2 i \theta} },
\end{aligned}
\end{equation}
so, again setting \( \alpha = a/b \), we have
\begin{equation}\label{eqn:unitCircleContourIntegrals:120}
\begin{aligned}
I
&= \inv{b} \oint_{\Abs{z} = 1} \lr{ \inv{2} - \inv{4}\lr{ z^2 + 1/z^2 } } \frac{-i dz}{z} \inv{\alpha + (1/2)\lr{ z + 1/z} } \\
&= \frac{-i}{2 b} \oint_{\Abs{z} = 1} \lr{ 2 - z^2 - \inv{z^2} } \frac{dz}{ 2 \alpha z + z^2 + 1 } \\
&= \frac{-i}{2 b} \oint_{\Abs{z} = 1} dz \frac{ 2 z^2 - z^4 - 1 }{ z^2 \lr{ 2 \alpha z + z^2 + 1} } \\
&= \frac{-i}{2 b} \oint_{\Abs{z} = 1} dz \frac{ 2 z^2 - z^4 - 1 }{ z^2 \lr{ z + \alpha + \sqrt{ \alpha^2 - 1} }\lr{ z + \alpha - \sqrt{ \alpha^2 - 1} } }.
\end{aligned}
\end{equation}
The enclosed poles are at \( z = 0 \) (a second order pole) and \( z = -\alpha + \sqrt{ \alpha^2 - 1} \), so the integral is
\begin{equation}\label{eqn:unitCircleContourIntegrals:140}
\begin{aligned}
I
&= \lr{ 2 \pi i } \lr{ \frac{-i}{2 b} } \lr{
\evalbar{ \lr{ \frac{ 2 z^2 - z^4 - 1 }{ 2 \alpha z + z^2 + 1 } }' }{z = 0}
+
\evalbar{ \frac{ 2 z^2 - z^4 - 1 }{ z^2 \lr{ z + \alpha + \sqrt{ \alpha^2 - 1} } } }{ z = -\alpha + \sqrt{ \alpha^2 - 1} }
}
\end{aligned}
\end{equation}
The derivative residue simplifies to
\begin{equation}\label{eqn:unitCircleContourIntegrals:160}
\begin{aligned}
\evalbar{ \lr{ \frac{ 2 z^2 - z^4 - 1 }{ 2 \alpha z + z^2 + 1 } }' }{z = 0}
&=
\evalbar{ \frac{ 4 z - 4 z^3 }{2 \alpha z + z^2 + 1} - \frac{ 2 z^2 - z^4 - 1}{\lr{ 2 \alpha z + z^2 + 1 }^2 }\lr{ 2 \alpha + 2 z } }{z = 0} \\
&= 2 \alpha,
\end{aligned}
\end{equation}
whereas the remaining residue is
\begin{equation}\label{eqn:unitCircleContourIntegrals:180}
\evalbar{ -\frac{ \lr{z^2 - 1}^2 }{ z^2 \lr{ 2 \sqrt{ \alpha^2 - 1} } } }{ z = -\alpha + \sqrt{ \alpha^2 - 1} }
=
\evalbar{ -\lr{z - \inv{z}}^2 \inv{ 2 \sqrt{ \alpha^2 - 1} } }{ z = -\alpha + \sqrt{ \alpha^2 - 1} },
\end{equation}
but
\begin{equation}\label{eqn:unitCircleContourIntegrals:220}
\begin{aligned}
\inv{z}
&= \inv{ -\alpha + \sqrt{ \alpha^2 - 1 } } \frac{ \lr{ \alpha + \sqrt{ \alpha^2 - 1 }} }{ \lr{ \alpha + \sqrt{ \alpha^2 - 1 } } } \\
&= \frac{ \alpha + \sqrt{ \alpha^2 - 1 } }{ -\alpha^2 + \lr{ \alpha^2 - 1} } \\
&= -\lr{ \alpha + \sqrt{ \alpha^2 - 1 } },
\end{aligned}
\end{equation}
and
\begin{equation}\label{eqn:unitCircleContourIntegrals:240}
\begin{aligned}
z - \inv{z}
&= -\alpha + \sqrt{ \alpha^2 - 1 } + \alpha + \sqrt{ \alpha^2 - 1 }
&= 2 \sqrt{ \alpha^2 - 1 },
\end{aligned}
\end{equation}
so
\begin{equation}\label{eqn:unitCircleContourIntegrals:260}
\begin{aligned}
\evalbar{ -\frac{ \lr{z^2 - 1}^2 }{ z^2 \lr{ 2 \sqrt{ \alpha^2 - 1} } } }{ z = -\alpha + \sqrt{ \alpha^2 - 1} }
&= - \frac{ \lr{ 2 \sqrt{ \alpha^2 - 1 } }^2 }{ 2 \sqrt{ \alpha^2 - 1} } \\
&= - 2 \sqrt{ \alpha^2 - 1 },
\end{aligned}
\end{equation}
for a final answer of
\begin{equation}\label{eqn:unitCircleContourIntegrals:200}
\begin{aligned}
I
&= \frac{2 \pi}{b} \lr{ \alpha - \sqrt{\alpha^2 - 1} } \\
&= \frac{2 \pi}{b^2} \lr{ a - \sqrt{a^2 - b^2} }.
\end{aligned}
\end{equation}

\section{Another cosine integral.}
Last problem of this sort (31 (c)), was to find, again with \( a > b > 0 \)
\begin{equation}\label{eqn:unitCircleContourIntegrals:280}
I = \int_0^{2 \pi} \frac{ d\theta} {\lr{ a + b \cos \theta }^2 }.
\end{equation}
Making our \( z = e^{i \theta} \) substitution, and setting \( \alpha = a/b \), we have
\begin{equation}\label{eqn:unitCircleContourIntegrals:300}
\begin{aligned}
I &= \inv{b^2} \oint_{\Abs{z} = 1} \frac{ -i dz/z} {\lr{ \alpha + (1/2)\lr{ z + 1/z } }^2 } \\
  &= \frac{-4 i}{b^2} \oint_{\Abs{z} = 1} \frac{ z dz}{\lr{ 2 \alpha z + z^2 + 1 }^2 } \\
  &= \frac{-4 i}{b^2} \oint_{\Abs{z} = 1} \frac{ z dz}{\lr{ z + \alpha + \sqrt{\alpha^2 - 1} }^2\lr{ z + \alpha - \sqrt{\alpha^2 - 1} }^2}.
\end{aligned}
\end{equation}
Again, only this mixed sign pole will be within the unit circle, so
\begin{equation}\label{eqn:unitCircleContourIntegrals:320}
\begin{aligned}
I
  &= \lr{\frac{-4 i}{b^2} }\lr{ 2 \pi i }
\lr{
\evalbar{ \lr{ \frac{z}{\lr{ z + \alpha + \sqrt{\alpha^2 - 1} }^2} }' }{z = -\alpha + \sqrt{\alpha^2 - 1} }
}
\end{aligned}
\end{equation}

That derivative is
\begin{equation}\label{eqn:unitCircleContourIntegrals:340}
\begin{aligned}
\lr{ \frac{z}{\lr{ z + \alpha + \sqrt{\alpha^2 - 1} }^2} }'
&=
\inv{\lr{ z + \alpha + \sqrt{\alpha^2 - 1} }^2} - \frac{2 z}{\lr{ z + \alpha + \sqrt{\alpha^2 - 1} }^3} \\
&= \frac{z + \alpha + \sqrt{\alpha^2 - 1} - 2 z}{\lr{ z + \alpha + \sqrt{\alpha^2 - 1} }^3} \\
&= \frac{-z + \alpha + \sqrt{\alpha^2 - 1}}{\lr{ z + \alpha + \sqrt{\alpha^2 - 1} }^3}.
\end{aligned}
\end{equation}
Evaluating it at our pole \( z = -\alpha + \sqrt{\alpha^2 - 1} \), we have
\begin{equation}\label{eqn:unitCircleContourIntegrals:360}
\begin{aligned}
\frac{-z + \alpha + \sqrt{\alpha^2 - 1}}{\lr{ z + \alpha + \sqrt{\alpha^2 - 1} }^3}
&=
\frac{ \alpha - \sqrt{\alpha^2 - 1} + \alpha + \sqrt{\alpha^2 - 1}}{\lr{ -\alpha + \sqrt{\alpha^2 - 1} + \alpha + \sqrt{\alpha^2 - 1} }^3} \\
&= \frac{ 2 \alpha }{\lr{ 2 \sqrt{\alpha^2 - 1} }^3 } \\
&= \inv{4} \frac{ \alpha }{\lr{ \alpha^2 - 1}^{3/2} },
\end{aligned}
\end{equation}
so
\begin{equation}\label{eqn:unitCircleContourIntegrals:380}
\begin{aligned}
I
&= \frac{8 \pi}{b^2} \inv{4} \frac{ \alpha }{\lr{ \alpha^2 - 1}^{3/2} } \\
&= \frac{2 \pi a }{b^3 \lr{ \alpha^2 - 1}^{3/2} },
\end{aligned}
\end{equation}
but \( b^3 = \lr{ b^2}^{3/2} \), for
\begin{equation}\label{eqn:unitCircleContourIntegrals:400}
I = \frac{ 2 \pi a }{ \lr{ a^2 - b^2 }^{3/2} }.
\end{equation}

%}
\EndArticle
%\EndNoBibArticle
