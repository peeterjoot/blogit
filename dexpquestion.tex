%
% Copyright � 2023 Peeter Joot.  All Rights Reserved.
% Licenced as described in the file LICENSE under the root directory of this GIT repository.
%
%{
\input{../latex/blogpost.tex}
\renewcommand{\basename}{dexpquestion}
%\renewcommand{\dirname}{notes/phy1520/}
\renewcommand{\dirname}{notes/ece1228-electromagnetic-theory/}
%\newcommand{\dateintitle}{}
%\newcommand{\keywords}{}

\input{../latex/peeter_prologue_print2.tex}

\usepackage{peeters_layout_exercise}
\usepackage{peeters_braket}
\usepackage{peeters_figures}
\usepackage{siunitx}
\usepackage{verbatim}
%\usepackage{mhchem} % \ce{}
%\usepackage{macros_bm} % \bcM
%\usepackage{macros_qed} % \qedmarker
%\usepackage{txfonts} % \ointclockwise

\beginArtNoToc

\generatetitle{XXX}
%\chapter{XXX}
%\label{chap:dexpquestion}

% https://discord.com/channels/607264339480674324/785572671085608990/1181712099714535464

Consider a slightly simpler problem.  Let
\begin{equation*}
\begin{aligned}
i &= \Be_{12} \\
j &= \Be_{31} e^{i\phi},
\end{aligned}
\end{equation*}
where we want to compute
\begin{equation*}
\PD{\phi}{e^{j\theta}}.
\end{equation*}
One way to do this, only knowing the characteristic power series definition of the exponential is to write
\begin{equation*}
\begin{aligned}
\PD{\phi}{e^{j\theta}}
&= \sum_{k = 0}^\infty \PD{\phi}{} \frac{ (j \theta)^k }{k!} \\
&= \sum_{k = 1}^\infty \PD{\phi}{j^k} \frac{ \theta^k }{k!}.
\end{aligned}
\end{equation*}
If you treat \( j \) as a complex number, this then reduces to
\begin{equation*}
\begin{aligned}
\PD{\phi}{e^{j\theta}}
&= \sum_{k = 1}^\infty k \PD{\phi}{j} j^{k-1} \frac{ \theta^k }{k!} \\
&=
\theta \PD{\phi}{j} \sum_{k = 1}^\infty \frac{ (j\theta)^{k-1} }{(k-1)!} \\
&=
\theta \PD{\phi}{j} e^{j\theta}.
\end{aligned}
\end{equation*}
But this is wrong.  The problem is that \( \PDi{\phi}{j} \) does not commute with \( j \), so
\begin{equation*}
\PD{\phi}{j^k} = \PD{\phi}{j} j^{k-1} + j \PD{\phi}{j} j^{k-2} + \cdots,
\end{equation*}
not \( k (\PDi{\phi}{j}) j^{k-1} \).

This is where you have the lack of commutitivity, sneakily hiding in the power series for the exponential, when you attempt to take the derivative.

%}
%\EndArticle
\EndNoBibArticle
