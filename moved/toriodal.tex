%
% Copyright � 2023 Peeter Joot.  All Rights Reserved.
% Licenced as described in the file LICENSE under the root directory of this GIT repository.
%
%{
\input{../latex/blogpost.tex}
\renewcommand{\basename}{toriodal}
%\renewcommand{\dirname}{notes/phy1520/}
\renewcommand{\dirname}{notes/ece1228-electromagnetic-theory/}
%\newcommand{\dateintitle}{}
%\newcommand{\keywords}{}

\input{../latex/peeter_prologue_print2.tex}

\usepackage{peeters_layout_exercise}
\usepackage{peeters_braket}
\usepackage{peeters_figures}
\usepackage{siunitx}
\usepackage{verbatim}
%\usepackage{mhchem} % \ce{}
%\usepackage{macros_bm} % \bcM
%\usepackage{macros_qed} % \qedmarker
%\usepackage{txfonts} % \ointclockwise

\beginArtNoToc

\generatetitle{XXX}
%\chapter{XXX}
%\label{chap:toriodal}

\begin{subequations}
\begin{align}\label{eqn:torusCenterOfMassParameterization:1}
\Bx(\rho, \theta, \phi) &= e^{-j\theta/2} \left( \rho \Be_1 e^{ i \phi } + R \Be_3 \right) e^{j \theta/2} \\
i &= \Be_1 \Be_3 \\
j &= \Be_3 \Be_2
\end{align}
\end{subequations}
Let's compute \( \Bx_\theta \).
\begin{equation}\label{eqn:toriodal:21}
\begin{aligned}
\Bx_\theta &= \PD{\theta}{\Bx} \\
&=
-\frac{j}{2}
e^{-j\theta/2} \left( \rho \Be_1 e^{ i \phi } + R \Be_3 \right) e^{j \theta/2}
+
e^{-j\theta/2} \left( \rho \Be_1 e^{ i \phi } + R \Be_3 \right) e^{j \theta/2}
\frac{j}{2}
\end{aligned}
\end{equation}
The bivector \( j \) commutes with \( \Be_1 \), so the \( \rho \) dependent part of \( \Bx_\theta \) is
\begin{equation}\label{eqn:toriodal:41}
\begin{aligned}
& \frac{\rho}{2} e^{-j \theta/2} \Be_1 \lr{ -j e^{i \phi}  + e^{i \phi} j } e^{ j\theta/2 } \\
&=\frac{\rho}{2} e^{-j \theta/2} \Be_1 \lr{ - \Be_{32} \lr{ \cos\phi + \Be_{13} \sin\phi} + \lr{ \cos\phi + \Be_{13} \sin\phi} \Be_{32} } e^{ j\theta/2 } \\
&=\frac{\rho}{2} e^{-j \theta/2} \Be_1 \lr{ - \Be_{3213}  \sin\phi + \Be_{1332} \sin\phi } e^{ j\theta/2 } \\
&=\frac{\rho}{2} e^{-j \theta/2} \Be_1 \lr{ - \Be_{21}  \sin\phi + \Be_{12} \sin\phi } e^{ j\theta/2 } \\
&=      \rho e^{-j \theta/2} \Be_{112} \sin\phi e^{ j\theta/2 } \\
&=      \rho e^{-j \theta/2} \Be_{2} \sin\phi e^{ j\theta/2 }.
%&=      \rho \sin\phi \Be_2 e^{j \theta/2} e^{ j\theta/2 } \\
%&=      \rho \sin\phi \Be_2 e^{j \theta}.
\end{aligned}
\end{equation}
Similarly, the \( R \) dependent contribution is
\begin{equation}\label{eqn:toriodal:61}
\begin{aligned}
& \frac{R}{2} e^{-j \theta/2} \lr{ -j \Be_3  + \Be_3 j } e^{ j\theta/2 } \\
& \frac{R}{2} e^{-j \theta/2} \lr{ -\Be_{323}  + \Be_{332} } e^{ j\theta/2 } \\
& \frac{R}{2} e^{-j \theta/2} \lr{ \Be_{2}  + \Be_{2} } e^{ j\theta/2 } \\
&       R     e^{-j \theta/2} \Be_{2}  e^{ j\theta/2 }.
\end{aligned}
\end{equation}
Putting the pieces together, we have
\begin{equation}\label{eqn:toriodal:81}
\Bx_\theta = e^{-j \theta/2} \lr{ R + \rho \sin\phi } \Be_{2}  e^{ j\theta/2 }.
\end{equation}

%}
\EndArticle
%\EndNoBibArticle
