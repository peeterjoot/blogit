\documentclass{article}

\usepackage[utf8]{inputenc}
\usepackage{mathtools}
\usepackage{graphicx}
\usepackage{amsthm}
\usepackage{amsmath}
\usepackage{amssymb}
\usepackage{color}
\usepackage{hyperref}
\usepackage{tikz}
\usepackage{dsfont}
\usepackage{gensymb}
\usepackage{comment}
\usepackage{physics}

\title{Miroslav's challenge}
\author{Pepijn Cobben}
\date{November 2020}

\begin{document}

\maketitle

\section{Introduction}
%\begin{figure}[h!]
%    \centering
%    \includegraphics[width=1\linewidth]{Screenshot 2020-11-13 173529.jpg}
%\end{figure}

\section{Solution}
First we note that - as Miroslav notes - the problem of drawing a circle can be reduced to choosing two vectors a and b on this circle and drawing the two arcs between them. \\
Thus, what we want to do is a kind of spherical interpolation\footnote{Something like this: https://en.wikipedia.org/wiki/Slerp}: We want to generate a lot of little points on this arc and individually display these - together it will give us a full arc. \\
Considering quaternions are known for spherical interpolation, and rotors supersede quaternions, it is likely GA will be able to help us here. This we will show next: (also, my two vectors a,b are unit vectors for simplicity for the proof, though the tools developed also work for non-unit vectors)\\ \medskip

Rotating a  vector a onto b can be done with a rotor: \begin{equation}
    R = \cos\frac{\theta}{2} - B\sin\frac{\theta}{2} = e^{-B\frac{\theta}{2}}
\end{equation}
where $\theta$ is the angle between a and b, while the normalized bivector B represents the plane $a\wedge b$. \\
The equation to rotate a is thus:
\begin{equation}
    RaR^{-1} = b
\end{equation}
Now, if we know the rotor R (which we do not currently) - we can create as many points between a and b as we want! As an example, let's create n new vectors $v_1, v_2, ..., v_n$ that are on the arc between a and b with equal distance between each $v_i$.
This we do as follows:
\begin{equation}
    R^{\frac{1}{n}}aR^{\frac{-1}{n}} = v_1
\end{equation}
Then
\begin{equation}
    R^{\frac{1}{n}}v_1R^{\frac{-1}{n}} = v_2
\end{equation}
etc. (we note that $v_n$ would be the last point on the arc - and thus is b). Since when do fractional powers of rotors exist? Recall
\begin{equation}
    R = e^{-B\frac{\theta}{2}}
\end{equation}
and we know we can do that for $e$. 
We aren't done yet though - we don't have an explicit equation for the rotor yet. That we will do next. \\
\bigskip

First we note that any rotation is a composition of two reflections across two vectors. Recall as well that in Geometric Algebra, the reflection of a vector $v$ across a unit vector $u$ is simply \begin{equation}
%    -uvu = v_{reflected}
    -uvu = v_{\text{reflected}}
\end{equation}
Thus, if we find two unit vectors $v_1$ and $v_2$, where a reflection of a across $v_1$ and then $v_2$ gives b - then:
\begin{align*}
    a_{ref} &= v_1av_1 \\
    b &= v_2v_1av_1v_2
\end{align*}
Which implies that $R = v_2v_1$. Now, let's find two such vectors! First, we find the unit vector between $a$ and $b$:
\begin{equation}
    c = \frac{a+b}{|a+b|}
\end{equation}
By the definition of c, thus a reflection of a across c must give  b:
\begin{equation}
    -cac = b
\end{equation}
A reflection of b across b is, well, just b:
\begin{equation}
    bcacb = b^3 = b
\end{equation}
Thus
\begin{equation}
    R = bc = b\frac{a + b}{|a + b|} = \frac{ba + b^2}{|a + b|} = \frac{ba + 1}{|a + b|}
\end{equation}
\section{Conclusion}
We see we get a very simple equation for the rotors in the end - and this can easily be used to create as many points on the arc as we would like using $R^{1/n}$
\end{document}
