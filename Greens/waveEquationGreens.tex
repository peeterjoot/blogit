%
% Copyright � 2025 Peeter Joot.  All Rights Reserved.
% Licenced as described in the file LICENSE under the root directory of this GIT repository.
%
%{
%%\input{../latex/blogpost.tex}
%%\renewcommand{\basename}{waveEquationGreens}
%%%\renewcommand{\dirname}{notes/phy1520/}
%%\renewcommand{\dirname}{notes/ece1228-electromagnetic-theory/}
%%%\newcommand{\dateintitle}{}
%%%\newcommand{\keywords}{}
%%
%%\input{../latex/peeter_prologue_print2.tex}
%%
%%\usepackage{peeters_layout_exercise}
%%\usepackage{peeters_braket}
%%\usepackage{peeters_figures}
%%\usepackage{siunitx}
%%\usepackage{verbatim}
%%%\usepackage{macros_cal} % \LL
%%%\usepackage{amsthm} % proof
%%%\usepackage{mhchem} % \ce{}
%%%\usepackage{macros_bm} % \bcM
%%%\usepackage{macros_qed} % \qedmarker
%%%\usepackage{txfonts} % \ointclockwise
%%
%%\beginArtNoToc
%%
%%\generatetitle{Green's function for the wave equation}
%\chapter{Green's function for the wave equation}
%\label{chap:waveEquationGreens}
The Green's function(s) \( G(\Br, \tau) \) for the 3D wave equation
\begin{equation}\label{eqn:waveEquationGreens:40}
\lr{ \spacegrad^2 - \inv{c^2}\frac{\partial^2}{\partial t^2} } G(\Br, \tau) = \delta(\Br) \delta(\tau),
\end{equation}
where
\begin{equation}\label{eqn:waveEquationGreens:20}
\begin{aligned}
\Br &= \Bx - \Bx' \\
r &= \Abs{\Br} \\
\tau &= t - t',
\end{aligned}
\end{equation}
is
\begin{equation}\label{eqn:waveEquationGreens:60}
G(\Br, \tau) = -\inv{4 \pi r} \delta( \pm \tau - r/c ).
\end{equation}
Here the positive case is the retarded solution, and negative the advanced solution.  The derivation of these Green's functions can be found derived in many places, including \citep{byron1992mca}, \citep{jackson1975cew}, and \citep{schwinger1998classical}

I wasn't familiar with the 1D and 2D Green's functions for the wave equation.  Grok says they are, respectively
\begin{equation}\label{eqn:waveEquationGreens:80}
\begin{aligned}
G(\Br, \tau) &= -\frac{c}{2} \Theta( \pm \tau - r/c ) \\
G(\Br, \tau) &= -\inv{2 \pi \sqrt{ \tau^2 - r^2/c^2 } } \Theta( \pm \tau - r/c ).
\end{aligned}
\end{equation}
At least for the time being, I thought that I'll attempt to verify these, instead of deriving them.  For the 1D case, this turns out to be fairly straightforward.  Perhaps unexpectedly, that isn't true for the 2D case, and I'll have to revisit that case in other ways.  In this post, I'll show the verification of the 1D Green's function, and my partial attempt to verify the 2D case.

\subsection{1D Green's function verification.}
We will use the Heaviside theta representation of the absolute value.
\begin{equation}\label{eqn:waveEquationGreens:100}
\Abs{x} = x \Theta(x) - x \Theta(-x).
\end{equation}
Recall that the derivative of the absolute value function is a sign function
\begin{equation}\label{eqn:waveEquationGreens:120}
\begin{aligned}
\Abs{x}'
&= \Theta(x) - \Theta(-x) + x \delta(x) + x \delta(-x) \\
&= \Theta(x) - \Theta(-x) + 2 x \delta(x) \\
&= \Theta(x) - \Theta(-x) \\
&= \sgn(x),
\end{aligned}
\end{equation}
where \( x \delta(x) \) is zero in a distributional sense (zero if applied to a test function.)
\begin{equation}\label{eqn:waveEquationGreens:140}
\begin{aligned}
\sgn(x)'
&= \Theta(x)' - \Theta(-x)' \\
&= \delta(x) + \delta(-x) \\
&= 2 \delta(x).
\end{aligned}
\end{equation}

Now let's evaluate the \( x \) partials.
\begin{equation}\label{eqn:waveEquationGreens:160}
\begin{aligned}
\PD{x}{} \Theta(\tau - r/c)
&=
-\inv{c} \delta\lr{ \tau - r/c } \PD{x}{} \Abs{x - x'} \\
&=
-\inv{c} \delta\lr{ \tau - r/c } \sgn(x - x').
\end{aligned}
\end{equation}
The second derivative is
\begin{equation}\label{eqn:waveEquationGreens:180}
\begin{aligned}
\frac{\partial^2}{\partial x^2} \Theta(\tau - r/c)
&=
-\inv{c}
\lr{
    -\inv{c} \delta'\lr{ \tau - r/c } (\sgn(x - x'))^2
    +
    \delta\lr{ \tau - r/c } 2 \delta(x - x')
} \\
&=
\inv{c^2} \delta'\lr{ \tau - r/c } - \frac{2}{c} \delta\lr{ \tau} \delta(x - x').
\end{aligned}
\end{equation}
The transformation above from \( \delta\lr{ \tau - r/c } \rightarrow \delta(\tau) \) is because the spatial delta function \( \delta(x - x') \) is zero unless \( x = x' \), and \( r = 0 \) at that point.

The time derivatives are easier to compute
\begin{equation}\label{eqn:waveEquationGreens:200}
\begin{aligned}
\frac{\partial^2}{\partial t^2} \Theta(\tau - r/c)
&=
\PD{t}{} \delta(\tau - r/c) \\
&=
\delta'(\tau - r/c).
\end{aligned}
\end{equation}

Putting the pieces together, we have
\begin{equation}\label{eqn:waveEquationGreens:220}
\begin{aligned}
\lr{ \spacegrad^2 - \inv{c^2}\frac{\partial^2}{\partial t^2} } \Theta(\tau - r/c)
&=
\inv{c^2} \delta'\lr{ \tau - r/c } - \frac{2}{c} \delta\lr{ \tau} \delta(x - x')
- \inv{c^2} \delta'(\tau - r/c)
\\
&=
- \frac{2}{c} \delta\lr{ \tau} \delta(x - x').
\end{aligned}
\end{equation}
Dividing through by \( -2/c \) gives us
\begin{equation}\label{eqn:waveEquationGreens:240}
\lr{ \spacegrad^2 - \inv{c^2}\frac{\partial^2}{\partial t^2} } G(\Bx - \Bx', t - t') = \delta\lr{t - t'} \delta\lr{\Bx - \Bx'},
\end{equation}
as desired.  The \( \delta \) derivative terms can be given meaning, but they conveniently cancel out, so we don't have to think about that this time.

It's easy to see that the advanced Green's function has the same behaviour, since the two time partials will bring down a factor of \( (\pm 1)^2 = 1 \) in general, which does not change anything above.
\subsection{Attempted verification of the claimed 2D Green's function.}
Now let's try to verify Grok's claim for the 2D Green's function, starting with a few helpful side calculations.

\begin{equation}\label{eqn:waveEquationGreens:260}
\begin{aligned}
\spacegrad \Abs{r}
&= \sum_m \Be_m \partial_m \sqrt{ \sum_n \lr{x_n - x_n'}^2 } \\
&= \inv{2} 2 \frac{\Bx - \Bx'}{\Abs{\Bx - \Bx'}} \\
&= \rcap
\end{aligned}
\end{equation}

\begin{equation}\label{eqn:waveEquationGreens:280}
\begin{aligned}
\spacegrad \lr{ \tau^2 - r^2/c^2 }^{-1/2}
&=
-\inv{2} \lr{ \tau^2 - r^2/c^2 }^{-3/2} \lr{-\frac{2 r}{c^2}} \spacegrad r \\
&=
-\inv{2} \lr{ \tau^2 - r^2/c^2 }^{-3/2} \lr{-\frac{2 r}{c^2}} \rcap \\
&=
\frac{r}{c^2} \lr{ \tau^2 - r^2/c^2 }^{-3/2} \rcap
\end{aligned}
\end{equation}

\begin{equation}\label{eqn:waveEquationGreens:300}
\begin{aligned}
\spacegrad \lr{ \tau^2 - r^2/c^2 }^{-3/2}
&=
-\frac{3}{2} \lr{ \tau^2 - r^2/c^2 }^{-5/2} \lr{-\frac{2 r}{c^2}} \spacegrad r \\
&=
-\frac{3}{2} \lr{ \tau^2 - r^2/c^2 }^{-5/2} \lr{-\frac{2 r}{c^2}} \rcap \\
&=
\frac{3 r}{c^2} \lr{ \tau^2 - r^2/c^2 }^{-5/2} \rcap
\end{aligned}
\end{equation}

\begin{equation}\label{eqn:waveEquationGreens:320}
\begin{aligned}
\spacegrad \Theta\lr{ \pm \tau - r/c }
&=
-\inv{c} \delta\lr{ \pm \tau - r/c } \spacegrad r \\
&=
-\inv{c} \delta\lr{ \pm \tau - r/c } \rcap
\end{aligned}
\end{equation}

\begin{equation}\label{eqn:waveEquationGreens:340}
\begin{aligned}
\spacegrad \delta\lr{ \pm \tau - r/c }
&=
-\inv{c} \delta'\lr{ \pm \tau - r/c } \spacegrad r \\
&=
-\inv{c} \delta'\lr{ \pm \tau - r/c } \rcap
\end{aligned}
\end{equation}

\begin{equation}\label{eqn:waveEquationGreens:360}
\begin{aligned}
\spacegrad \cdot \rcap
&=
\spacegrad \cdot \frac{\Bx - \Bx'}{r} \\
&=
\inv{r} \spacegrad \cdot \lr{\Bx - \Bx'} + \lr{\Bx - \Bx'} \cdot \spacegrad \inv{r} \\
&=
\frac{2}{r} + \lr{\Bx - \Bx'} \cdot \lr{ -\inv{r^2} \rcap } \\
&=
\frac{2}{r} - \inv{r} \\
&=
\frac{1}{r}.
\end{aligned}
\end{equation}
In summary, with \( X = \tau^2 - r^2/c^2 \)
\begin{equation}\label{eqn:waveEquationGreens:540}
\begin{aligned}
\spacegrad \Abs{r} &= \rcap \\
\spacegrad X^{-1/2} &= \inv{c^2} r \rcap X^{-3/2} \\
\spacegrad X^{-3/2} &= \inv{c^2} 3 r \rcap X^{-5/2} \\
\spacegrad \Theta &= - \inv{c} \delta \rcap \\
\spacegrad \delta &= - \inv{c} \rcap \delta' \\
\spacegrad \cdot \rcap &= \frac{1}{r}.
\end{aligned}
\end{equation}

We will want a couple helper Laplacian operations, including
\begin{equation}\label{eqn:waveEquationGreens:580}
\begin{aligned}
\spacegrad^2 X^{-1/2}
&=
\spacegrad \cdot \lr{ \inv{c^2} r \rcap X^{-3/2} } \\
&=
  \inv{c^2} \lr{ \spacegrad \cdot \rcap} \lr{ r X^{-3/2} }
+ \inv{c^2} \lr{ \rcap \cdot \spacegrad r } X^{-3/2}
+ \frac{r}{c^2} \lr{ \rcap \cdot \spacegrad  X^{-3/2} } \\
&=
  \inv{c^2} X^{-3/2}
+ \inv{c^2} X^{-3/2}
+ \frac{r}{c^2} \lr{ \inv{c^2} 3 r X^{-5/2} } \\
&=
  \frac{2}{c^2} X^{-3/2}
+ \frac{3 r^2}{c^4} X^{-5/2}.
\end{aligned}
\end{equation}

The Laplacian of the step is
\begin{equation}\label{eqn:waveEquationGreens:600}
\begin{aligned}
\spacegrad^2 \Theta
&=
\spacegrad \cdot \lr{ - \inv{c} \delta \rcap } \\
&=
-\inv{c}
\lr{ \spacegrad \cdot \rcap } \delta
-\inv{c}
\rcap \cdot \spacegrad \delta \\
&=
-\inv{r c} \delta
-\inv{c}
\rcap \cdot \lr{
- \inv{c} \rcap \delta'
}
&=
-\inv{r c} \delta
+\inv{c^2} \delta'.
\end{aligned}
\end{equation}

We are now ready to compute the Laplacian of \( \Theta X^{-1/2} \).  Let's expand the chain rule for that, so that the rest of the job is just algebra
\begin{equation}\label{eqn:waveEquationGreens:620}
\begin{aligned}
\spacegrad^2 \lr{ f g }
&=
\spacegrad \cdot \lr{ f \spacegrad g }
+
\spacegrad \cdot \lr{ g \spacegrad f } \\
&=
f \spacegrad^2 g + \spacegrad f \cdot \spacegrad g
+
g \spacegrad^2 f + \spacegrad g \cdot \spacegrad f \\
&=
f \spacegrad^2 g + 2 \spacegrad f \cdot \spacegrad g + g \spacegrad^2 f.
\end{aligned}
\end{equation}
We want to sub in
\begin{equation}\label{eqn:waveEquationGreens:640}
\begin{aligned}
\spacegrad^2 \Theta &= -\inv{r c} \delta +\inv{c^2} \delta' \\
\spacegrad^2 X^{-1/2} &= \frac{2}{c^2} X^{-3/2} + \frac{3 r^2}{c^4} X^{-5/2} \\
\spacegrad X^{-1/2} &= \inv{c^2} r \rcap X^{-3/2} \\
\spacegrad \Theta &= - \inv{c} \delta \rcap.
\end{aligned}
\end{equation}
We get
\begin{equation}\label{eqn:waveEquationGreens:660}
\begin{aligned}
\spacegrad^2 \lr{ \Theta X^{-1/2} }
&=
\lr{ -\inv{r c} \delta +\inv{c^2} \delta' } X^{-1/2}
+ \lr{ \frac{2}{c^2} X^{-3/2} + \frac{3 r^2}{c^4} X^{-5/2} } \Theta
- 2  \inv{c^2} r X^{-3/2} \inv{c} \delta \\
&=
\inv{c^2} X^{-1/2} \delta'
+ \inv{c^2} \lr{ 2  \lr{\tau^2 - r^2/c^2}  + \frac{3 r^2}{c^2} } X^{-5/2} \Theta
- \inv{r c} \lr{  \tau^2 - r^2/c^2 + 2 r^2/c^2 } X^{-3/2} \delta \\
&=
\inv{c^2} X^{-1/2} \delta'
+ \inv{c^2} \lr{ 2 \tau^2 + \frac{r^2}{c^2} } X^{-5/2} \Theta
- \inv{r c} \lr{  \tau^2 + \frac{r^2}{c^2} } X^{-3/2} \delta
\end{aligned}
\end{equation}

We are ready to evaluate the time derivatives now.  Let's try it the same way with
\begin{equation}\label{eqn:waveEquationGreens:680}
\begin{aligned}
\partial_{tt} \lr{ f g }
&=
\partial_t \lr{ f \partial_t g + g \partial_t f } \\
&=
g \partial_{tt} f
+
f \partial_{tt} g
+ 2 \lr{ \partial_t f } \lr{ \partial_t g }.
\end{aligned}
\end{equation}
A couple of the time partials can be computed by inspection
\begin{equation}\label{eqn:waveEquationGreens:700}
\begin{aligned}
\partial_t \Theta &= \pm \delta \\
\partial_{tt} \Theta &= \lr{\pm 1}^2 \delta',
\end{aligned}
\end{equation}
and for the rest, we have
\begin{equation}\label{eqn:waveEquationGreens:720}
\begin{aligned}
\partial_t X^{-1/2}
&=
-\inv{2} X^{-3/2} \partial_t X \\
&=
-\inv{2} X^{-3/2} 2 \tau \\
&=
-\tau X^{-3/2},
\end{aligned}
\end{equation}
and
\begin{equation}\label{eqn:waveEquationGreens:740}
\begin{aligned}
\partial_{tt} X^{-1/2}
&=
- X^{-3/2}
- \tau \partial_t X^{-3/2}  \\
&=
- X^{-3/2}
+ 3 \tau^2 X^{-5/2}.
\end{aligned}
\end{equation}
Assembling the pieces, we have
\begin{equation}\label{eqn:waveEquationGreens:760}
\begin{aligned}
\partial_{tt} \lr{ \Theta X^{-1/2} }
&=
\lr{
    - X^{-3/2}
    + 3 \tau^2 X^{-5/2}
} \Theta
+
\delta' X^{-1/2}
+ 2 \lr{  \pm \delta } \lr{ -\tau X^{-3/2} } \\
&=
 \delta' X^{-1/2}
+ \lr{ -\lr{ \tau^2 - r^2/c^2 } + 3 \tau^2 } X^{-5/2} \Theta
\mp 2 \tau X^{-3/2} \delta \\
&=
 \delta' X^{-1/2}
+ \lr{ 2 \tau^2 + r^2/c^2 } X^{-5/2} \Theta
\mp 2 \tau X^{-3/2} \delta.
\end{aligned}
\end{equation}

The wave equation operation on \( \Theta X^{-1/2} \) is
\begin{equation}\label{eqn:waveEquationGreens:780}
\begin{aligned}
\lr{ \spacegrad^2 - (1/c^2) \partial_{tt} } \Theta X^{-1/2}
&=
  \inv{c^2} \lr{ 2 \tau^2 + \frac{r^2}{c^2} } X^{-5/2} \Theta
- \inv{r c} \lr{  \tau^2 + \frac{r^2}{c^2} } X^{-3/2} \delta \\
&- \inv{c^2} \lr{ 2 \tau^2 + r^2/c^2 } X^{-5/2} \Theta
\pm \frac{2}{c^2} \tau X^{-3/2} \delta \\
&=
- \inv{r c} \lr{  \tau^2 + \frac{r^2}{c^2} } X^{-3/2} \delta
\pm \frac{2}{c^2} \tau X^{-3/2} \delta \\
&=
\inv{c^2} \lr{
    - \frac{c \tau^2}{r}
    - \frac{r}{c}
    \pm 2 \tau
}
X^{-3/2} \delta.
\end{aligned}
\end{equation}

So, after all that we have
\begin{equation}\label{eqn:waveEquationGreens:800}
\lr{ \spacegrad^2 - (1/c^2) \partial_{tt} } G =
-\inv{2 \pi c^2} \lr{
    - \frac{c \tau^2}{r}
    - \frac{r}{c}
    \pm 2 \tau
}
\frac{\delta(\pm \tau - r/c)}{\lr{\tau^2 - r^2/c^2}^{3/2}}.
\end{equation}

This is a very problematic expression.  The delta function is zero everywhere but \( \pm \tau = r/c \), but the denominator blows up at \( \pm \tau = r/c \), and the leading factor is also zero at that point:
\begin{equation}\label{eqn:waveEquationGreens:820}
\begin{aligned}
\evalbar{ \lr{ -\frac{c}{r} \tau^2 - \frac{r}{c} \pm 2 \tau }}{\pm \tau = r/c}
&=
-\frac{c}{r} \lr{ \frac{r}{c} }^2 - \frac{r}{c} + 2 \frac{r}{c} \\
&=
0.
\end{aligned}
\end{equation}
So, we've computed something that has a \( 0 \times \infty / 0 \) structure at \( \pm \tau = r/c \).  Presumably, this has the infinite value \( \delta(x - x') \delta(y - y') \delta(t - t') \) at that point.

I think that the root problem here is that the derivatives of \( \lr{ \tau^2 - r^2/c^2 }^{-1/2} \) are not defined where \( \tau = \pm r/c \), so we have a zero result for any region of spacetime where that is not the case, but can't say much about it at other points without additional work.

Attempting to describe this physically, I think that we'd say that we have discovered that a constant velocity wave of this form has to propagate on the ``light cone''.  We see something like that for the 3D Green's function too, which is explicitly zero off the light cone, not just after application of the wave equation operator.

\paragraph{Followup:}

\begin{enumerate}
\item Is there a better representation of the 2D Green's function than this one?  I think it's time to look up some more advanced handling of Green's function to get a better handle on this.  I'd guess that there's a Green's function for the 2D wave equation related to Bessel functions, like that of the 2D Helmholtz operator.
\item It should also be possible to perform a limiting convolution verification, in the neighbourhood of the light cone, and then look at the limit of that convolution.  I'd expect that to be better behaved, as it should avoid the singularity itself.
\end{enumerate}

\subsection{2nd attempted verification of the claimed 2D Green's function.}
Let's try verifying the Green's function using a convolution
\begin{equation}\label{eqn:waveEquationGreens:840}
I(\Bx) = \lr{ \spacegrad^2 - \inv{c^2} \partial_{tt} } -\inv{2 \pi} \int \frac{\Theta(\pm (t - t') - \Abs{\Bx - \Bx'}/c)}{\lr{(t - t')^2 - \Abs{\Bx - \Bx'}/c^2}^{1/2}} d^2 \Bx' dt'
\end{equation}
\subsection{Deriving the Green's functions.}
Having had trouble verifying the 2D Green's function, let's try deriving them ourselves.
\subsubsection{Setup.}
Let's try solving the forced wave equation
\begin{equation}\label{eqn:waveEquationGreens:860}
\lr{ \spacegrad^2 - \inv{c^2}\frac{\partial^2}{\partial t^2} } f(x,t) = g(x,t),
\end{equation}
using Fourier transform pairs
\begin{equation}\label{eqn:waveEquationGreens:880}
\begin{aligned}
F(\Bx, t) &= \inv{\lr{\sqrt{2 \pi}}^{N+1}} \int e^{j \Bk \cdot \Bx + j \omega t} \hatF(\Bk, \omega) d^N \Bk d\omega \\
\hatF(\Bk, \omega) &= \inv{\lr{\sqrt{2 \pi}}^{N+1}} \int e^{-j \Bk \cdot \Bx - j \omega t} F(\Bx, t) d^N \Bx dt.
\end{aligned}
\end{equation}
We can now transform \cref{eqn:waveEquationGreens:860}, expressing \(f, g\) in terms of their transforms
\begin{equation}\label{eqn:waveEquationGreens:900}
\lr{ \lr{ j \Bk}^2 - \lr{ j \omega }^2/c^2 } \hatf = \hatg,
\end{equation}
or
\begin{equation}\label{eqn:waveEquationGreens:920}
\hatf = \frac{\hatg}{(\omega/c)^2 - \Bk^2},
\end{equation}
or
\begin{equation}\label{eqn:waveEquationGreens:940}
\begin{aligned}
f(\Bx, t)
&= \inv{\lr{\sqrt{2 \pi}}^{N+1}} \int e^{j \Bk \cdot \Bx + j \omega t} \frac{\hatg(\Bk, \omega)}{(\omega/c)^2 - \Bk^2} d^N \Bk d\omega \\
&= \inv{\lr{2 \pi}^{N+1}} \int e^{j \Bk \cdot \Bx + j \omega t} \frac{g(\Bx', t')}{(\omega/c)^2 - \Bk^2} d^N \Bk d\omega e^{-j \Bk \cdot \Bx' - j \omega t'} d^N \Bx' dt' \\
&=
\int d^N \Bx' dt' g(\Bx', t') G(\Bx, \Bx', t, t'),
\end{aligned}
\end{equation}
where
\begin{equation}\label{eqn:waveEquationGreens:960}
G(\Bx, \Bx', t, t')
=
\inv{\lr{2 \pi}^{N+1}} \int d^N \Bk d\omega \frac{e^{j \Bk \cdot (\Bx-\Bx') + j \omega (t- t')}}{(\omega/c)^2 - \Bk^2}.
\end{equation}
\subsubsection{Evaluating the 1D Green's function}
For the 1D case we have
\begin{equation}\label{eqn:waveEquationGreens:980}
G(\Bx, \Bx', t, t')
=
\inv{\lr{2 \pi}^2} \int dk d\omega \frac{e^{j k (x-x') + j \omega (t- t')}}{(\omega/c)^2 - k^2}
\end{equation}
Let's write \( u = x - x' \), and \( \tau = t - t' \), and displace the poles by an imaginary offset \( j \epsilon \)
\begin{equation}\label{eqn:waveEquationGreens:1000}
G_\epsilon(u, \tau)
=
-\inv{\lr{2 \pi}^2} \int dk d\omega \frac{e^{j k u + j \omega \tau }}{\lr{ k - \lr{ \omega/c + j \epsilon}}\lr{ k + \lr{ \omega/c + j \epsilon }}}.
\end{equation}

Let's start by assuming that \( \epsilon > 0 \).  When \( u > 0 \), we can use a upper half plane contour in the k-plane, enclosing \( \omega/c + j \epsilon \), to find
\begin{equation}\label{eqn:waveEquationGreens:1020}
\begin{aligned}
G_\epsilon(u, \tau)
&=
-\frac{2 \pi j}{\lr{2 \pi}^2} \int d\omega \evalbar{\frac{e^{j k u + j \omega \tau }}{k + \lr{ \omega/c + j \epsilon }}}{k = \omega/c + j \epsilon} \\
&=
\frac{1}{4 \pi j} \int d\omega \frac{e^{j \omega (\tau + u/c)}}{\omega/c + j \epsilon }.
\end{aligned}
\end{equation}
However, for \( u < 0 \) we need the lower half plane contour that encloses \( -\omega/c - j \epsilon \).  Our residue calculation is
\begin{equation}\label{eqn:waveEquationGreens:1040}
\begin{aligned}
G_\epsilon(u, \tau)
&=
-\frac{-2 \pi j}{\lr{2 \pi}^2} \int d\omega \evalbar{\frac{e^{j k u + j \omega \tau }}{k - \lr{ \omega/c + j \epsilon }}}{k = -\omega/c - j \epsilon} \\
&=
\frac{1}{4 \pi j} \int d\omega \frac{e^{j \omega (\tau - u/c)}}{\omega/c + j \epsilon }.
\end{aligned}
\end{equation}
Merging the two cases, we have
\begin{equation}\label{eqn:waveEquationGreens:1060}
\begin{aligned}
G_\epsilon(u, \tau)
&=
\frac{1}{4 \pi j} \int d\omega \frac{e^{j \omega (\tau + \Abs{u}/c)}}{\omega/c + j \epsilon } \\
&=
\frac{c}{4 \pi j} \int d\omega \frac{e^{j \omega (\tau + \Abs{u}/c)}}{\omega + j \epsilon c } \\
\end{aligned}
\end{equation}
This can be integrated in the \(\omega\)-plane, with the pole at \( -j \epsilon c \).  For \( \tau + \Abs{u}/c > 0 \), we need an upper half plane infinite semicircular contour, but have no enclosed pole.  For \( \tau + \Abs{u}/c < 0 \), we have
\begin{equation}\label{eqn:waveEquationGreens:1080}
\begin{aligned}
G_\epsilon(u, \tau)
&=
\frac{c (-2 \pi j)}{4 \pi j} \evalbar{ e^{j \omega (\tau + \Abs{u}/c)}}{\omega = -j \epsilon c} \\
&=
-\frac{c}{2},
\end{aligned}
\end{equation}
(in the limit.)
Putting both pieces together, we have found the advanced Green's function for the 1D wave equation
\begin{equation}\label{eqn:waveEquationGreens:1100}
\boxed{
G(u, \tau) = -\frac{c}{2} \Theta(-\tau - \Abs{u}/c).
}
\end{equation}

Having found the advanced solution with a positive pole displacement, it is reasonable to assume that we will get the retarded solution, with a negative pole displacement \( \epsilon < 0 \).  This time, the upper half plane infinite semicircular contour encloses the \( -\omega/c -j \epsilon \) pole, and the lower half plane contour encloses the \( \omega/c + j \epsilon \) pole.  This gives, us, for \( u > 0 \)
\begin{equation}\label{eqn:waveEquationGreens:1120}
\begin{aligned}
G_\epsilon(u, \tau)
&=
-\frac{2 \pi j}{\lr{2 \pi}^2} \int d\omega \evalbar{\frac{e^{j k u + j \omega \tau }}{k - \lr{ \omega/c + j \epsilon }}}{k = -\omega/c - j \epsilon} \\
&=
-\frac{1}{4 \pi j} \int d\omega \frac{e^{j \omega (\tau - u/c)}}{\omega/c + j \epsilon } \\
&=
-\frac{c}{4 \pi j} \int d\omega \frac{e^{j \omega (\tau - u/c)}}{\omega - (-j \epsilon c) } \\
&=
-\frac{2 \pi j c}{4 \pi j} \evalbar{e^{j \omega (\tau - u/c)}}{\omega = -j \epsilon c } \Theta(\tau - u/c) \\
&=
-\frac{c}{2} \Theta(\tau - u/c),
\end{aligned}
\end{equation}
and for \( u < 0 \)
\begin{equation}\label{eqn:waveEquationGreens:1140}
\begin{aligned}
G_\epsilon(u, \tau)
&=
-\frac{-2 \pi j}{\lr{2 \pi}^2} \int d\omega \evalbar{\frac{e^{j k u + j \omega \tau }}{k + \lr{ \omega/c + j \epsilon }}}{k = \omega/c + j \epsilon} \\
&=
\frac{1}{4 \pi j} \int d\omega \frac{e^{j \omega (\tau + u/c)}}{\omega/c + j \epsilon } \\
&=
\frac{c}{4 \pi j} \int d\omega \frac{e^{j \omega (\tau + u/c)}}{\omega - (-j \epsilon c) } \\
&=
-\frac{-2 \pi j c}{4 \pi j} \evalbar{e^{j \omega (\tau + u/c)}}{\omega = -j \epsilon c } \Theta(\tau + u/c) \\
&=
-\frac{c}{2} \Theta(\tau + u/c).
\end{aligned}
\end{equation}
Combining the two cases, we've found the retarded solution
\begin{equation}\label{eqn:waveEquationGreens:1160}
\boxed{
G(u, \tau) = -\frac{c}{2} \Theta(\tau - \Abs{u}/c).
}
\end{equation}
This matches Grok's claim (which we also verified.)

\subsubsection{The convolution integrals.}
Let's write out the convolution integrals for fun.  They are
\begin{equation}\label{eqn:waveEquationGreens:1180}
f(x,t) = -\frac{c}{2}
\int_{-\infty}^\infty dt'
\int_{-\infty}^\infty dx'
\Theta(\pm(t - t') - \Abs{x - x'}/c) g(x', t').
\end{equation}

For the retarded case, we need only evaluate the step over the region
\begin{equation}\label{eqn:waveEquationGreens:1220}
t - t' - \Abs{x - x'}/c > 0,
\end{equation}
or
\begin{equation}\label{eqn:waveEquationGreens:1240}
t - \Abs{x - x'}/c > t'.
\end{equation}
For the advanced case, we want the restriction
\begin{equation}\label{eqn:waveEquationGreens:1260}
-t + t' - \Abs{x - x'}/c > 0,
\end{equation}
or
\begin{equation}\label{eqn:waveEquationGreens:1280}
t' > t + \Abs{x - x'}/c,
\end{equation}
so the retarded convolution is
\begin{equation}\label{eqn:waveEquationGreens:1300}
f(x,t) = -\frac{c}{2}
\int_{-\infty}^\infty dx'
\int_{-\infty}^{t - \Abs{x - x'}/c} dt'
g(x', t'),
\end{equation}
and the advanced convolution is
\begin{equation}\label{eqn:waveEquationGreens:1320}
f(x,t) = -\frac{c}{2}
\int_{-\infty}^\infty dx'
\int_{t + \Abs{x - x'}/c}^\infty dt'
g(x', t').
\end{equation}
\subsubsection{2D Green's function derivation.}
While it was difficult to attempt to verify the 2D Green's function, it actually turns out to be fairly easy to derive it, provided we pick an alternate pole displacement to make our lives easier.

With \( \Br = \Bx - \Bx' \), and \( \tau = t - t' \), and \( \epsilon > 0 \), we can form
\begin{equation}\label{eqn:waveEquationGreens:1340}
G_\epsilon(\Br, \tau) = \frac{c^2}{\lr{2 \pi}^3} \int d^2 \Bk d\omega \frac{ e^{j \Bk \cdot \Br + j \omega \tau}}{\lr{\omega -j \epsilon}^2 - \Bk^2 c^2 }
\end{equation}
This pole displacement has the nice property that both poles live in the upper half plane, so for \( \tau > 0 \), we have
\begin{equation}\label{eqn:waveEquationGreens:1360}
\begin{aligned}
G_\epsilon(\Br, \tau)
&= \frac{c^2}{\lr{2 \pi}^3} \int d^2 \Bk d\omega \frac{ e^{j \Bk \cdot \Br + j \omega \tau}}{
\lr{\omega -\lr{ \Abs{\Bk} c - j \epsilon}}
\lr{\omega -\lr{ -\Abs{\Bk} c - j \epsilon}}
} \\
&=
\Theta(\tau) \frac{c^2 j}{\lr{2 \pi}^2} \int d^2 \Bk e^{j \Bk \cdot \Br}
\lr{
    \evalbar{
    \frac{e^{ j \omega \tau}}{ \lr{\omega -\lr{ -\Abs{\Bk} c - j \epsilon}} }
    }
    {\omega = \Abs{\Bk} c - j \epsilon}
+
    \evalbar{
    \frac{e^{ j \omega \tau}}{ \lr{\omega -\lr{ \Abs{\Bk} c - j \epsilon}} }
    }
    {\omega = -\Abs{\Bk} c - j \epsilon}
}
\\
&=
\Theta(\tau) \frac{c^2 j}{\lr{2 \pi}^2} \int d^2 \Bk e^{j \Bk \cdot \Br} e^{ -j \epsilon \tau }
\lr{
    \frac{e^{ j \Abs{\Bk} c \tau}}{ 2 \Abs{\Bk} c }
+
    \frac{e^{ -j \Abs{\Bk} c \tau}}{ -2 \Abs{\Bk} c }
} \\
&=
\Theta(\tau) \frac{j^2 c}{\lr{2 \pi}^2} \int_{k=0}^\infty k dk \int_{\phi=0}^{2 \pi} d\phi e^{j k\Abs{\Br} \cos\phi } e^{ -j \epsilon \tau } \frac{\sin\lr{ k c \tau }}{k}.
\end{aligned}
\end{equation}
We've now successfully removed the singularity, and can evaluate the \(\epsilon \rightarrow 0 \) limit.  We may also evaluate the \( \phi \) integral, remembering that
\begin{equation}\label{eqn:waveEquationGreens:1380}
\int_0^{2 \pi} e^{j \Abs{a} \cos\phi} d\phi = 2 \pi J_0(\Abs{a}),
\end{equation}
to find
\begin{equation}\label{eqn:waveEquationGreens:1400}
G(\Br, \tau) = -\Theta(\tau) \frac{c}{2 \pi} \int_{k=0}^\infty dk J_0(k\Abs{\Br}) \sin\lr{ k c \tau }.
\end{equation}
This integral yields easily to Mathematica, and we find
\begin{equation}\label{eqn:waveEquationGreens:1420}
G(\Br, \tau) = -\Theta(\tau) \frac{c}{2 \pi} \frac{\Theta(c \tau - \Abs{\Br})}{\sqrt{(c\tau)^2 - \Br^2}}.
\end{equation}
However, since \( \Theta(c \tau - \Abs{\Br}) = 1 \) only for \( \tau > \Abs{\Br}/c \), the \( \Theta(\tau) \) factor is redundant, and we find
\begin{equation}\label{eqn:waveEquationGreens:1440}
\boxed{
G(\Br, \tau) = - \frac{1}{2 \pi} \frac{\Theta(c \tau - \Abs{\Br})}{\sqrt{\tau^2 - \Br^2/c^2}},
}
\end{equation}
which matches the retarded Green's function claimed by Grok.

Repeating this analysis for \( \tau < 0, \epsilon < 0 \), we find
\begin{equation}\label{eqn:waveEquationGreens:1460}
G(\Br, \tau) = -\Theta(-\tau) \frac{c}{2 \pi} \frac{\Theta(-c \tau - \Abs{\Br})}{\sqrt{(c\tau)^2 - \Br^2}},
\end{equation}
which we also see matches the Grok result for the advanced Green's function.  Both of these computations can be trivially performed in Mathematica following the same steps (taking all the fun from the story.)  The advanced integral evaluation is shown in \cref{fig:2dGreensAdvanced:2dGreensAdvancedFig1} as an example.
\imageFigure{../figures/blogit/2dGreensAdvancedFig1}{Advanced 2D Green's function for wave equation operator.}{fig:2dGreensAdvanced:2dGreensAdvancedFig1}{0.5}
\subsubsection{3D Green's function derivation.}
For the sake of completeness, now let's evaluate the Green's function for the 3D wave equation operator.  Again with \( \Br = \Bx - \Bx', \tau = t - t' \) we want the \( \epsilon \rightarrow 0 \) limit of
\begin{equation}\label{eqn:waveEquationGreens:1480}
G_\epsilon(\Br, \tau)
=
\inv{\lr{2 \pi}^4} \int d^3 \Bk d\omega \frac{e^{j \Bk \cdot \Br + j \omega \tau}}{(\omega/c - j \epsilon/c)^2 - \Bk^2}.
\end{equation}
For \(\epsilon > 0 \) this will presumably give us the retarded solution, with advanced for \( \epsilon < 0 \).  We are using the nice pole displacement that leaves both poles on the same side of the upper or lower half plane, depending on the sign of \( \epsilon \).

Let's only do the \( \epsilon > 0 \) case by hand.  Evaluating the \( \omega \) integral first with an upper half plane contour, we have
\begin{equation}\label{eqn:waveEquationGreens:1500}
\begin{aligned}
G_\epsilon(\Br, \tau)
&=
\frac{c^2}{\lr{2 \pi}^4} \int d^3 \Bk e^{j \Bk \cdot \Br}
\frac{e^{j \omega \tau}}{
  \lr{\omega - \lr{ j \epsilon - \Abs{\Bk} c}}
  \lr{\omega - \lr{ j \epsilon + \Abs{\Bk} c}}
} \\
&=
\frac{j c^2}{\lr{2 \pi}^3} \Theta(\tau) \int d^3 \Bk e^{j \Bk \cdot \Br}
\lr{
\evalbar{\frac{e^{j \omega \tau}}{\omega - \lr{ j \epsilon - \Abs{\Bk} c}}}{\omega = j \epsilon + \Abs{\Bk} c}
+
\evalbar{\frac{e^{j \omega \tau}}{\omega - \lr{ j \epsilon + \Abs{\Bk} c}}}{\omega = j \epsilon - \Abs{\Bk} c}
} \\
&=
\frac{j c^2}{\lr{2 \pi}^3} \Theta(\tau) e^{-\epsilon \tau} \int d^3 \Bk e^{j \Bk \cdot \Br}
\lr{
    \frac{e^{j \Abs{\Bk} c \tau}}{2 \Abs{\Bk} c}
    -
    \frac{e^{-j \Abs{\Bk} c \tau}}{2 \Abs{\Bk} c}
} \\
&=
-\frac{c}{\lr{2 \pi}^3} \Theta(\tau) e^{-\epsilon \tau} \int \frac{d^3 \Bk}{\Abs{\Bk}} e^{j \Bk \cdot \Br}
\sin\lr{ \Abs{\Bk} c \tau }.
\end{aligned}
\end{equation}
We can evaluate the \( \epsilon \rightarrow 0 \) limit, and switch to spherical coordinates in k-space.  Let \( \Br = r \Be_3 \)
\begin{equation}\label{eqn:waveEquationGreens:1520}
G(\Br, \tau)
=
-\frac{c}{\lr{2 \pi}^3} \Theta(\tau)
\int_{k = 0}^\infty \frac{k^2 dk}{k}
\int_{\phi = 0}^{2 \pi} d\phi
\int_{\theta = 0}^{\pi} \sin\theta d\theta
e^{j k r \cos\theta} \sin\lr{ k c \tau }.
\end{equation}
With \( u = \cos\theta \), this gives
\begin{equation}\label{eqn:waveEquationGreens:1540}
\begin{aligned}
G(\Br, \tau)
&=
\frac{c}{\lr{2 \pi}^2} \Theta(\tau)
\int_{k = 0}^\infty k dk \sin\lr{ k c \tau }
\int_{u = -1}^{1} du
e^{j k r u} \\
&=
\frac{c}{\lr{2 \pi}^2} \Theta(\tau)
\int_{k = 0}^\infty k dk \sin\lr{ k c \tau }
\lr{ \frac{e^{-j k r }}{j k r} - \frac{e^{j k r }}{j k r} } \\
&=
-\frac{c}{2 \pi^2 r} \Theta(\tau) \int_{k = 0}^\infty dk \sin\lr{ k c \tau } \sin\lr{ k r} \\
&=
-\frac{c}{4 \pi^2 r} \Theta(\tau) \int_{k = 0}^\infty dk
\lr{
    \cos\lr{ k( c \tau - r ) }
    -
    \cos\lr{ k( c \tau + r ) }
} \\
&=
-\frac{c}{8 \pi^2 r} \Theta(\tau) \int_{k = -\infty}^\infty dk
\lr{
    \cos\lr{ k( c \tau - r ) }
    -
    \cos\lr{ k( c \tau + r ) }
} \\
&=
-\frac{c}{8 \pi^2 r} \Theta(\tau) \int_{-\infty}^\infty dk
\lr{
        e^{ j k( c \tau - r ) } - e^{ j k( c \tau + r ) }
} \\
&=
-\frac{c}{4 \pi r} \Theta(\tau)
\lr{
    \delta( c \tau - r )
    -
    \delta( c \tau + r )
} \\
&=
-\frac{1}{4 \pi r} \Theta(\tau)
\lr{
    \delta( \tau - r/c )
    -
    \delta( \tau + r/c )
}.
\end{aligned}
\end{equation}
Observe that the second delta function only has a value when \( \tau = -r/c \), but \( \Theta(-r/c) = 0 \).  Similarly, the first delta function only has a value for \( \tau = r/c \ge 0 \), where the Heaviside step function is unity.  That means we can simplify this to just
\begin{equation}\label{eqn:waveEquationGreens:1560}
\boxed{
G(\Br, \tau) = -\frac{1}{4 \pi \Abs{\Br}} \delta( \tau - \Abs{\Br}/c ),
}
\end{equation}
as expected.

Again, sort of sadly, we can skip all the fun and evaluate most of this in Mathematica.  It needs only minor hand-holding to extract the delta function semantics.  The retarded derivation is shown in \cref{fig:waveEquation3DGreensDerivationMathematica:waveEquation3DGreensDerivationMathematicaFig1}, and the advanced derivation in \cref{fig:waveEquation3DGreensDerivationMathematica:waveEquation3DGreensDerivationMathematicaFig2}.
\imageFigure{../figures/blogit/waveEquation3DGreensDerivationMathematicaFig1}{Retarded 3D Green's function for the wave equation.}{fig:waveEquation3DGreensDerivationMathematica:waveEquation3DGreensDerivationMathematicaFig1}{0.5}
\imageFigure{../figures/blogit/waveEquation3DGreensDerivationMathematicaFig2}{Advanced 3D Green's function for the wave equation.}{fig:waveEquation3DGreensDerivationMathematica:waveEquation3DGreensDerivationMathematicaFig2}{0.5}
%}
%%\EndArticle
