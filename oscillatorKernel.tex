%
% Copyright � 2025 Peeter Joot.  All Rights Reserved.
% Licenced as described in the file LICENSE under the root directory of this GIT repository.
%
%{
\input{../latex/blogpost.tex}
\renewcommand{\basename}{oscillatorKernel}
%\renewcommand{\dirname}{notes/phy1520/}
\renewcommand{\dirname}{notes/ece1228-electromagnetic-theory/}
%\newcommand{\dateintitle}{}
%\newcommand{\keywords}{}

\input{../latex/peeter_prologue_print2.tex}

\usepackage{peeters_layout_exercise}
\usepackage{peeters_braket}
\usepackage{peeters_figures}
\usepackage{siunitx}
\usepackage{verbatim}
%\usepackage{macros_cal} % \LL
%\usepackage{amsthm} % proof
%\usepackage{mhchem} % \ce{}
%\usepackage{macros_bm} % \bcM
%\usepackage{macros_qed} % \qedmarker
%\usepackage{txfonts} % \ointclockwise

\beginArtNoToc

\generatetitle{XXX}
%\chapter{XXX}
%\label{chap:oscillatorKernel}

Here's the second last real-integral sub-problem from \citep{byron1992mca}, problem 31(j).  Find
\begin{equation}\label{eqn:oscillatorKernel:20}
I = P \int_{-\infty}^\infty \inv{ \lr{ \omega' - \omega_0 }^2 + a^2 } \inv{ \omega' - \omega } d\omega'.
\end{equation}

The denominator of this integrand will submit to a partial fraction decomposition of the form:
\begin{equation}\label{eqn:oscillatorKernel:40}
\inv{z - \alpha} \inv{z - \beta} \inv{ z - \gamma} = \frac{u}{z-\alpha} + \frac{v}{z - \beta} + \frac{w}{z - \gamma}.
\end{equation}
It turns out that for such a fraction, those values of \( u, v, w \) are
\begin{equation}\label{eqn:oscillatorKernel:60}
\begin{aligned}
u &= -\frac{1}{(\alpha-\beta)(\gamma-\alpha)} \\
v &= -\frac{1}{(\alpha-\beta)(\beta-\gamma)} \\
w &= -\frac{1}{(\gamma-\alpha)(\beta-\gamma)}.
\end{aligned}
\end{equation}

Our poles are sitting at
\begin{equation}\label{eqn:oscillatorKernel:80}
\begin{aligned}
\alpha, \beta &= \omega_0 \pm i a \\
\gamma &= \omega,
\end{aligned}
\end{equation}
so
\begin{equation}\label{eqn:oscillatorKernel:100}
\begin{aligned}
u &= -\frac{1}{(\omega_0 + 2 i a - \omega_0)(\omega - \omega_0 - i a)} \\
v &= -\frac{1}{(\omega_0 + 2 i a - \omega_0)(\omega_0 - i a - \omega)} \\
w &= -\frac{1}{(\omega - \omega_0 - i a)(\omega_0 - i a - \omega)}.
\end{aligned}
\end{equation}

one of which sits above the x-axis, one below, and one on the line.
This means that if we compute the usual infinite semicircular contour integral, we have a \( 2 \pi i \) weighted residue above the line and one \( \pi i \) weighted residue for the x-axis pole.  That is, assuming \( \alpha \) is the above the line pole, and \( \gamma \) is the on the line pole,
\begin{equation}\label{eqn:oscillatorKernel:50}
\begin{aligned}
\oint &dz\,
\inv{z - \alpha} \inv{z - \beta} \inv{ z - \gamma}  \\
&= 
\oint dz\, \lr{
\frac{u}{z-\alpha} + \frac{v}{z - \beta} + \frac{w}{z - \gamma} } \\
&=
\lr{ 2 \pi i } \evalbar{ \lr{ \frac{v}{z - \beta} + \frac{w}{ z - \gamma} } }{ z = \alpha }
+
\lr{ \pi i } \evalbar{ \lr{ \frac{u}{z - \alpha} + \frac{v}{z - \beta} } }{ z = \gamma } \\
&=
\lr{ 2 \pi i } \lr{ \frac{v}{\alpha - \beta} + \frac{w}{ \alpha - \gamma} }
+
\lr{ \pi i } \lr{ \frac{u}{\gamma - \alpha} + \frac{v}{\gamma - \beta} } \\
&=
\lr{ 2 \pi i } \lr{ \frac{v}{ 2 i a} + \frac{w}{ \omega_0 + i a - \omega} }
+
\lr{ \pi i } \lr{ \frac{u}{\omega  - \omega_0 - i a} + \frac{v}{\omega  - \omega_0 + i a} }
\end{aligned}
\end{equation}

%}
\EndArticle
%\EndNoBibArticle
