%
% Copyright � 2021 Peeter Joot.  All Rights Reserved.
% Licenced as described in the file LICENSE under the root directory of this GIT repository.
%
%{
\input{../latex/blogpost.tex}
\renewcommand{\basename}{identities}
%\renewcommand{\dirname}{notes/phy1520/}
\renewcommand{\dirname}{notes/ece1228-electromagnetic-theory/}
%\newcommand{\dateintitle}{}
%\newcommand{\keywords}{}

\input{../latex/peeter_prologue_print2.tex}

\usepackage{peeters_layout_exercise}
\usepackage{peeters_braket}
\usepackage{peeters_figures}
\usepackage{siunitx}
\usepackage{verbatim}
%\usepackage{mhchem} % \ce{}
%\usepackage{macros_bm} % \bcM
%\usepackage{macros_qed} % \qedmarker
%\usepackage{txfonts} % \ointclockwise

\beginArtNoToc

\generatetitle{XXX}
%\chapter{XXX}
%\label{chap:identities}

A more fundamental approach is to use the grade selection operators and the contraction axiom \( \Bx^2 = \Bx \cdot \Bx \) to define wedge products, to demonstrate the anticommutitivity property, and to prove the symmetric product identity for the dot product.

The anticommutive property follows from the contraction and distribution axioms.  If \( \Ba \cdot \Bb = 0 \), then we have
\begin{equation}\label{eqn:identities:20}
\begin{aligned}
   0 
   &= 
   \lr{ \Ba + \Bb }^2 - 
   \lr{ \Ba + \Bb } \cdot \lr{ \Ba + \Bb } \\
   &= 
   \Ba^2 + \Bb^2 + \Ba \Bb + \Bb \Ba
   - \Ba \cdot \Ba
   - \Bb \cdot \Bb
   - \Ba \cdot \Bb
   - \Bb \cdot \Ba \\
   &= 
   \Ba \Bb + \Bb \Ba,
\end{aligned}
\end{equation}
so for any orthogonal vectors \( \Ba, \Bb \), we have
\begin{equation}\label{eqn:identities:40}
\Ba \Bb = -\Bb \Ba
\end{equation}

Now supposed that we have an Euclidean orthonormal basis \( \setlr{\Be_1, \Be_2, \cdots} \).
Let \( \Bu = \sum_k \Be_k u_k \) and \( \Bv = \sum_k \Be_k v_k \), then
\begin{equation}\label{eqn:identities:60}
\begin{aligned}
   \Bu \Bv
   &= \sum_{j,k} \Be_j \Be_k u_j v_k \\
   &= 
   \sum_{j = k} \Be_j \Be_k u_j v_k
   +
   \sum_{j \ne k} \Be_j \Be_k u_j v_k \\
   &= 
   \sum_{j} (\Be_j \cdot \Be_j) u_j v_j
   +
   \sum_{j < k} \Be_j \Be_k \lr{ u_j v_k - u_k v_j }.
\end{aligned}
\end{equation}
The scalar part of this sum is completely symmetric, whereas the bivector portion of this sum is completely antisymmetric (this is a general statement, and can also be shown easily for non-Eucidean bases).  We must then have
\begin{equation}\label{eqn:identities:80}
   \gpgradezero{\Bu \Bv} = \inv{2} \lr{ \Bu \Bv + \Bv \Bu },
\end{equation}
and
\begin{equation}\label{eqn:identities:100}
   \gpgradetwo{\Bu \Bv} = \inv{2} \lr{ \Bu \Bv - \Bv \Bu },
\end{equation}
It is also clear that the scalar portion of this coordinate expansion is the dot product of the two vectors, which means
\begin{equation}\label{eqn:identities:120}
   \Bu \cdot \Bv = \inv{2} \lr{ \Bu \Bv + \Bv \Bu },
\end{equation}
Now, we define the wedge or two vectors as
\begin{equation}\label{eqn:identities:140}
   \Bu \wedge \Bv = \gpgradetwo{ \Bu \Bv },
\end{equation}
from which we see
\begin{equation}\label{eqn:identities:160}
   \Bu \wedge \Bv = 
   \inv{2} \lr{ \Bu \Bv - \Bv \Bu }.
\end{equation}

We define the wedge of three vectors as
\begin{equation}\label{eqn:identities:180}
   \Bu \wedge \Bv \wedge \Bw = \gpgradethree{ \Bu \Bv \Bw }.
\end{equation}
More generally, we define the dot and wedge of a vector \( \Bu \) and a k-blade \( V_k \) (the wedge of k vectors) as
\begin{equation}\label{eqn:identities:200}
\Bu \cdot V_k = \gpgrade{ \Bu V_k }{k-1},
\end{equation}
and
\begin{equation}\label{eqn:identities:220}
\Bu \wedge V_k = \gpgrade{ \Bu V_k }{k+1}.
\end{equation}


%}
%\EndArticle
\EndNoBibArticle
