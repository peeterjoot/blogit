%
% Copyright � 2025 Peeter Joot.  All Rights Reserved.
% Licenced as described in the file LICENSE under the root directory of this GIT repository.
%
%{
\input{../latex/blogpost.tex}
\renewcommand{\basename}{absCos3}
%\renewcommand{\dirname}{notes/phy1520/}
\renewcommand{\dirname}{notes/ece1228-electromagnetic-theory/}
%\newcommand{\dateintitle}{}
%\newcommand{\keywords}{}

\input{../latex/peeter_prologue_print2.tex}

\usepackage{peeters_layout_exercise}
\usepackage{peeters_braket}
\usepackage{peeters_figures}
\usepackage{siunitx}
\usepackage{verbatim}
%\usepackage{macros_cal} % \LL
%\usepackage{amsthm} % proof
%\usepackage{mhchem} % \ce{}
%\usepackage{macros_bm} % \bcM
%\usepackage{macros_qed} % \qedmarker
%\usepackage{txfonts} % \ointclockwise

\beginArtNoToc

\generatetitle{XXX}
%\chapter{XXX}
%\label{chap:absCos3}
% \citep{sakurai2014modern} pr X.Y
% \citep{pozar2009microwave}
% \citep{qftLectureNotes}
% \citep{doran2003gap}
% \citep{jackson1975cew}
% \citep{griffiths1999introduction}

Find
\begin{equation}\label{eqn:absCos3:20}
I = \int_0^\pi \Abs{\cos x}^3 dx.
\end{equation}
First eliminate the annoying absolute value, doubling half the integral
\begin{equation}\label{eqn:absCos3:40}
I = 2 \int_0^{\pi/2} \cos^3 x dx.
\end{equation}
For the rest, remember that
\begin{equation}\label{eqn:absCos3:60}
\begin{aligned}
\cos(3x) + i \sin(3x)
&= \lr{ \cos x + i \sin x }^3 \\
&= \cos^3 x + 3 \cos^2 x \lr{ i \sin x } + 3 \cos x \lr{ i \sin x }^2 + \lr{ i \sin x }^3.
\end{aligned}
\end{equation}
In particular, the real part gives us
\begin{equation}\label{eqn:absCos3:80}
\begin{aligned}
\cos 3x
&= \cos^3 x - 3 \cos x \sin^2 x \\
&= \cos^3 x - 3 \cos x \lr{ 1 - \cos^2 x } \\
&= 4 \cos^3 x - 3 \cos x,
\end{aligned}
\end{equation}
or
\begin{equation}\label{eqn:absCos3:100}
\cos^3 x = \inv{4} \lr{ \cos 3x + 3 \cos x }.
\end{equation}
We can now state the value of the integral directly
\begin{equation}\label{eqn:absCos3:120}
\begin{aligned}
I
&= \inv{2} \evalbar{ \lr{ \frac{ \sin 3 x }{3} + 3 \sin x } }{ x = \pi/2 } \\
&= \inv{2} \lr{ \frac{-1}{3} + 3 } \\
&= \frac{4}{3}.
\end{aligned}
\end{equation}

%}
%\EndArticle
\EndNoBibArticle
