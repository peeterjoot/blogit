%
% Copyright � 2022 Peeter Joot.  All Rights Reserved.
% Licenced as described in the file LICENSE under the root directory of this GIT repository.
%
%{
\input{../latex/blogpost.tex}
\renewcommand{\basename}{multivectorDivergence}
%\renewcommand{\dirname}{notes/phy1520/}
\renewcommand{\dirname}{notes/ece1228-electromagnetic-theory/}
%\newcommand{\dateintitle}{}
%\newcommand{\keywords}{}

\input{../latex/peeter_prologue_print2.tex}

\usepackage{peeters_layout_exercise}
\usepackage{peeters_braket}
\usepackage{peeters_figures}
\usepackage{siunitx}
\usepackage{verbatim}
%\usepackage{mhchem} % \ce{}
%\usepackage{macros_bm} % \bcM
%\usepackage{macros_qed} % \qedmarker
%\usepackage{txfonts} % \ointclockwise

\beginArtNoToc

\generatetitle{XXX}
%\chapter{XXX}
%\label{chap:multivectorDivergence}

% https://math.stackexchange.com/q/4479569/359

Given a 2D multivector
\begin{equation*}
X = x^0 + x^1 e_1 + x^2 e_2 + x^{12} e_{12},
\end{equation*}
where \( e_{12} = e_1 \wedge e_2 \), you can show that the multivector gradient has the form
\begin{equation*}
\partial_X = \frac{\partial {}}{\partial {x^0}} + e^1 \frac{\partial {}}{\partial {x^1}} + e^2 \frac{\partial {}}{\partial {x^2}} + e^{21} \frac{\partial {}}{\partial {x^{12}}}.
\end{equation*}

I think that the only sensible definition of the divergence of \( F = F^0 + F^1 e_1 + F^2 e_2 + F^{12} e_{12} \) would be the scalar product of the gradient with \( F \).  That is
\begin{equation*}
\begin{aligned}
\partial_X * F
&=
\lr{
\frac{\partial {}}{\partial {x^0}} + e^1 \frac{\partial {}}{\partial {x^1}} + e^2 \frac{\partial {}}{\partial {x^2}} + e^{21} \frac{\partial {}}{\partial {x^{12}}}} * \lr{ F^0 + F^1 e_1 + F^2 e_2 + F^{12} e_{12} } \\
&=
\frac{\partial F^0}{\partial {x^0}} + \frac{\partial F^1}{\partial {x^1}} + \frac{\partial F^2}{\partial {x^2}} + \frac{\partial F^{12}}{\partial {x^{12}}}.
\end{aligned}
\end{equation*}

Note that the above expression allows for non-Euclidean basis vectors, and utilizes the reciprocal basis vectors \( e^k \cdot e_j = {\delta^{k}}_j \).  For the Euclidean case that can be simplified with \( e_1 = e^1, e_2 = e^2, e_1^2 = e_2^2 = 1 \).

%}
\EndArticle
%\EndNoBibArticle
