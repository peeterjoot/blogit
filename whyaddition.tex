%
% Copyright � 2021 Peeter Joot.  All Rights Reserved.
% Licenced as described in the file LICENSE under the root directory of this GIT repository.
%
%{
\input{../latex/blogpost.tex}
\renewcommand{\basename}{whyaddition}
%\renewcommand{\dirname}{notes/phy1520/}
\renewcommand{\dirname}{notes/ece1228-electromagnetic-theory/}
%\newcommand{\dateintitle}{}
%\newcommand{\keywords}{}

\input{../latex/peeter_prologue_print2.tex}

\usepackage{peeters_layout_exercise}
\usepackage{peeters_braket}
\usepackage{peeters_figures}
\usepackage{siunitx}
\usepackage{verbatim}
%\usepackage{mhchem} % \ce{}
%\usepackage{macros_bm} % \bcM
%\usepackage{macros_qed} % \qedmarker
%\usepackage{txfonts} % \ointclockwise

\beginArtNoToc

\generatetitle{XXX}
%\chapter{XXX}
%\label{chap:whyaddition}

A geometric algebra is a vector space where the elements are products of vectors.  Suppose, for example, we are working with a 3D Euclidian orthonormal basis \( \setlr{\Be_1, \Be_2, \Be_3 } \).  For example, the following are all elements of the geometric algebra generated by this basis
\begin{equation*}
\begin{aligned}
   &\Be_1 \Be_1  \\
   &\Be_1 \\
   &\Be_1 \Be_2 \\
   &\Be_1 \Be_1 \Be_1 \Be_2 \Be_3.
\end{aligned}
\end{equation*}
These are, respectively, scalars, vectors, bivectors, and trivectors.  The multivector formed by adding all of these is
\begin{equation*}
   1 + \Be_1 + \Be_1 \Be_2 + \Be_1 \Be_2 \Be_3.
\end{equation*}

%}
\EndArticle
%\EndNoBibArticle
