%
% Copyright � 2021 Peeter Joot.  All Rights Reserved.
% Licenced as described in the file LICENSE under the root directory of this GIT repository.
%
%{
\input{../latex/blogpost.tex}
\renewcommand{\basename}{units}
%\renewcommand{\dirname}{notes/phy1520/}
\renewcommand{\dirname}{notes/ece1228-electromagnetic-theory/}
%\newcommand{\dateintitle}{}
%\newcommand{\keywords}{}

\input{../latex/peeter_prologue_print2.tex}

\usepackage{peeters_layout_exercise}
\usepackage{peeters_braket}
\usepackage{peeters_figures}
\usepackage{siunitx}
\usepackage{verbatim}
%\usepackage{mhchem} % \ce{}
%\usepackage{macros_bm} % \bcM
%\usepackage{macros_qed} % \qedmarker
%\usepackage{txfonts} % \ointclockwise

\beginArtNoToc

\generatetitle{XXX}
%\chapter{XXX}
%\label{chap:units}

Consider the following projective-rejective split of a vector \( \Bv \) along the direction of a different vector \( \Bu \)
\begin{equation*}
   \Bv = \lr{ \frac{\Bu}{\Norm{\Bu}} \cdot \Bv } \frac{\Bu}{\Norm{\Bu}} + \lr{ \frac{\Bu}{\Norm{\Bu}} \cross \Bv } \cross \frac{\Bu}{\Norm{\Bu}}.
\end{equation*}
It should be clear that we want the unit vector \( \ucap = \Bu/\Norm{\Bu} \) to be dimensionless, because if you didn't then
\begin{equation*}
\Bv = \lr{ \ucap \cdot \Bv } \ucap + \lr{ \ucap \cross \Bv } \cross \ucap,
\end{equation*}
would have dimensions of \( \textrm{some-units} \), but on the right you'd have \( {\textrm{some-units}}^3 \).

Instead, if let the units ride along coefficients instead of the unit vectors, and this sort of trouble is resolved.  You could have, for example
\begin{equation*}
   \Bx = \lr{ 10\, \textrm{m} } \Be_1,
\end{equation*}
\begin{equation*}
   \Norm{\Bx} = \sqrt{\Bx \cdot \Bx} = \sqrt{ \lr{10\, \textrm{m}}^2 \Be_1 \cdot \Be_1 } = 10\, \textrm{m},
\end{equation*}
and the unit vector will still be dimensionless
\begin{equation*}
   \Be_1 = \frac{\Bx}{\Norm{\Bx}} = \frac{\lr{10\, \textrm{m}} \Be_1}{10\, \textrm{m}} = \Be_1.
\end{equation*}

As well as resolving all sorts of troubles for plain old dot-product spaces, dimensionless unit vectors resolve the dimension products for pseudoscalar dual products in geometric algebras that we generate from such dot product spaces.

%}
%\EndArticle
\EndNoBibArticle
