%
% Copyright � 2022 Peeter Joot.  All Rights Reserved.
% Licenced as described in the file LICENSE under the root directory of this GIT repository.
%
%{
\input{../latex/blogpost.tex}
\renewcommand{\basename}{symmetricMV}
%\renewcommand{\dirname}{notes/phy1520/}
\renewcommand{\dirname}{notes/ece1228-electromagnetic-theory/}
%\newcommand{\dateintitle}{}
%\newcommand{\keywords}{}

\input{../latex/peeter_prologue_print2.tex}

\usepackage{peeters_layout_exercise}
\usepackage{peeters_braket}
\usepackage{peeters_figures}
\usepackage{siunitx}
\usepackage{verbatim}
%\usepackage{mhchem} % \ce{}
%\usepackage{macros_bm} % \bcM
%\usepackage{macros_qed} % \qedmarker
%\usepackage{txfonts} % \ointclockwise

\beginArtNoToc

\generatetitle{XXX}
%\chapter{XXX}
%\label{chap:symmetricMV}

% https://math.stackexchange.com/q/4477736/359 Symmetric multi-vector
You picked the matrix representation
\begin{equation}\label{eqn:symmetricMV:20}
a +x \Be_1 +y \Be_1 + b \Be_1 \Be_2 \sim
\begin{bmatrix}
a+x & -b+y \\ b+y & a-x
\end{bmatrix},
\end{equation}
but there are lots of possible matrix representations of multivectors.  For example, an alternate representation for the 2D multivector basis elements could use the Pauli matrices
\begin{equation}\label{eqn:symmetricMV:40}
\begin{aligned}
   1 &=
\begin{bmatrix}
   1 & 0 \\
   0 & 1
\end{bmatrix} \\
\Be_1 &= \PauliX \\
\Be_2 &= \PauliY \\
\Be_1 \Be_2 &= \PauliX \PauliY =
\begin{bmatrix}
   i & 0 \\
   0 & -i
\end{bmatrix},
\end{aligned}
\end{equation}
for which you would have
\begin{equation}\label{eqn:symmetricMV:60}
a + x \Be_1 + y \Be_1 + b \Be_1 \Be_2 \sim
\begin{bmatrix}
   a + i b &  x - i y \\
   x + i y & a - i b
\end{bmatrix}.
\end{equation}
Like your construction, you can also recover all the multivector coordinates from such a representation.

If you construct an matrix representaiton of this sort, the meaning of symmetrical will vary based on the representation chosen.  As a general statement, such symmetry is not neccessarily meaningful, although it could be for some specific representations.

If you are looking for matrix representations for the (3,1) space time algebra case, the obvious candidate are the Dirac matrices (of which there are also many representations.)

%}
%\EndArticle
\EndNoBibArticle
