%
% Copyright � 2023 Peeter Joot.  All Rights Reserved.
% Licenced as described in the file LICENSE under the root directory of this GIT repository.
%
%{
\input{../latex/blogpost.tex}
\renewcommand{\basename}{radialderivatives}
%\renewcommand{\dirname}{notes/phy1520/}
\renewcommand{\dirname}{notes/ece1228-electromagnetic-theory/}
%\newcommand{\dateintitle}{}
%\newcommand{\keywords}{}

\input{../latex/peeter_prologue_print2.tex}

\usepackage{peeters_layout_exercise}
\usepackage{peeters_braket}
\usepackage{peeters_macros}
\usepackage{peeters_figures}
\usepackage{siunitx}
\usepackage{verbatim}
\usepackage{amsthm}
%\usepackage{mhchem} % \ce{}
%\usepackage{macros_bm} % \bcM
%\usepackage{macros_qed} % \qedmarker
%\usepackage{txfonts} % \ointclockwise

\beginArtNoToc

\generatetitle{Radial derivatives.}
%\chapter{Radial derivatives.}
%\label{chap:radialderivatives}

In my last couple GA YouTube videos, circular and spherical coordinates were examined.  We found the form of the unit vector derivatives in both cases.  Let's now look at a more general case, where we write
\begin{equation}\label{eqn:radialderivatives:20}
\Br = r \rcap,
\end{equation}
leaving the angular dependence of \( \rcap \) unspecified.  We want to find both \( \Br' \) and \( \rcap'\).

\makelemma{Radial length derivative.}{lemma:radialderivatives:40}{
The derivative of a spherical length \( r \) can be expressed as
\begin{equation*}
\frac{dr}{dt} = \rcap \cdot \frac{d\Br}{dt}.
\end{equation*}
} % lemma
\begin{proof}
We write \( r^2 = \Br \cdot \Br \), and take derivatives of both sides, to find
\begin{equation}\label{eqn:radialderivatives:60}
2 r \frac{dr}{dt} = 2 \Br \cdot \frac{d\Br}{dt},
\end{equation}
or
\begin{equation}\label{eqn:radialderivatives:80}
\frac{dr}{dt} = \frac{\Br}{r} \cdot \frac{d\Br}{dt} = \rcap \cdot \frac{d\Br}{dt}.
\end{equation}
\end{proof}

Application of the chain rule to \cref{eqn:radialderivatives:20} is straightforward
\begin{equation}\label{eqn:radialderivatives:100}
\Br' = r' \rcap + r \rcap',
\end{equation}
but we don't know the form for \( \rcap' \).  We could proceed with a niave expansion of
\begin{equation}\label{eqn:radialderivatives:120}
\frac{d}{dt} \lr{ \frac{\Br}{r} },
\end{equation}
but we can be sneaky, and perform a projective and rejective split of \( \Br' \) with respect to \( \rcap \).  That is
\begin{equation}\label{eqn:radialderivatives:140}
\begin{aligned}
\Br'
&= \rcap \rcap \Br' \\
&= \rcap \lr{ \rcap \Br' } \\
&= \rcap \lr{ \rcap \cdot \Br' + \rcap \wedge \Br'} \\
&= \rcap \lr{ r' + \rcap \wedge \Br'}.
\end{aligned}
\end{equation}
We used \cref{lemma:radialderivatives:40} in the last step above, and after distribution, find
\begin{equation}\label{eqn:radialderivatives:160}
\Br' = r' \rcap + \rcap \lr{ \rcap \wedge \Br' }.
\end{equation}
Comparing to \cref{eqn:radialderivatives:100}, we see that
\begin{equation}\label{eqn:radialderivatives:180}
r \rcap' = \rcap \lr{ \rcap \wedge \Br' }.
\end{equation}
We see that the radial unit vector derivative is proportional to the rejection of \( \rcap \) from \( \Br' \)
\begin{equation}\label{eqn:radialderivatives:200}
\rcap' = \inv{r} \Rej{\rcap}{\Br'} = \inv{r^3} \Br \lr{ \Br \wedge \Br' }.
\end{equation}
The vector \( \rcap' \) is perpendicular to \( \rcap \) for any parameterization of it's orientation, or in symbols
\begin{equation}\label{eqn:radialderivatives:220}
\rcap \cdot \rcap' = 0.
\end{equation}
We saw this for the circular and spherical parameterizations, and see now that this also holds more generally.

%\makeproblem{Perform a niave computation of \( \rcap' \).}{problem:radialderivatives:240}{
\makeproblem{}{problem:radialderivatives:240}{
Find \cref{eqn:radialderivatives:200} without being sneaky.
} % problem
\makeanswer{problem:radialderivatives:240}{
\begin{equation}\label{eqn:radialderivatives:280}
\begin{aligned}
\rcap'
&= \frac{d}{dt} \lr{ \frac{\Br}{r} } \\
&= \inv{r} \Br' - \inv{r^2} \Br r' \\
&= \inv{r} \Br' - \inv{r} \rcap r' \\
&= \inv{r} \lr{ \Br' - \rcap r' } \\
&= \inv{r} \lr{ \rcap \rcap \Br' - \rcap r' } \\
&= \inv{r} \rcap \lr{ \rcap \Br' - r' } \\
&= \inv{r} \rcap \lr{ \rcap \Br' - \rcap \cdot \Br' } \\
&= \inv{r} \rcap \lr{ \rcap \wedge \Br' }.
\end{aligned}
\end{equation}
} % answer

\makeproblem{}{problem:radialderivatives:300}{
Show that \cref{eqn:radialderivatives:200} can be expressed as a triple vector cross product
\begin{equation}\label{eqn:radialderivatives:230}
\rcap' = \inv{r^3} \lr{ \Br \cross \Br' } \cross \Br,
\end{equation}
} % problem
\makeanswer{problem:radialderivatives:300}{
While this may be familiar from elementary calculus, such as in \citep{salas1990coa}, we can show follows easily from our GA result
\begin{equation}\label{eqn:radialderivatives:300}
\begin{aligned}
\rcap'
&= \inv{r} \rcap \lr{ \rcap \wedge \Br' } \\
&= \inv{r} \gpgradeone{ \rcap \lr{ \rcap \wedge \Br' } } \\
&= \inv{r} \gpgradeone{ \rcap I \lr{ \rcap \cross \Br' } } \\
&= \inv{r} \gpgradeone{ I \lr{ \rcap \cdot \lr{ \rcap \cross \Br' } + \rcap \wedge \lr{ \rcap \cross \Br' } } } \\
&= \inv{r} \gpgradeone{ I^2 \rcap \cross \lr{ \rcap \cross \Br' } } \\
&= \inv{r} \lr{ \rcap \cross \Br' } \cross \rcap.
\end{aligned}
\end{equation}
} % answer

%}
\EndArticle
%\EndNoBibArticle
