%
% Copyright � 2025 Peeter Joot.  All Rights Reserved.
% Licenced as described in the file LICENSE under the root directory of this GIT repository.
%
%{
\input{../latex/blogpost.tex}
\renewcommand{\basename}{transverseField}
%\renewcommand{\dirname}{notes/phy1520/}
\renewcommand{\dirname}{notes/ece1228-electromagnetic-theory/}
%\newcommand{\dateintitle}{}
%\newcommand{\keywords}{}

\input{../latex/peeter_prologue_print2.tex}

\usepackage{peeters_layout_exercise}
\usepackage{peeters_braket}
\usepackage{peeters_figures}
\usepackage{siunitx}
\usepackage{verbatim}
%\usepackage{macros_cal} % \LL
%\usepackage{amsthm} % proof
%\usepackage{mhchem} % \ce{}
%\usepackage{macros_bm} % \bcM
%\usepackage{macros_qed} % \qedmarker
%\usepackage{txfonts} % \ointclockwise

\beginArtNoToc

\generatetitle{Transverse electric and magnetic field relations.}
%\chapter{Transverse electric and magnetic field relations.}
%\label{chap:transverseField}
I found a sign error in my book, and am going to rederive all the results for myself here in a standalone fashion to do a second verification of the signs.

Suppose that a field is propagating in a medium along the z-axis.  We may represent that field as the real part of
\begin{equation}\label{eqn:transverseField:20}
F = F(x,y) e^{j(\omega t - k z)}.
\end{equation}
This is a doubly complex relationship, as we have a scalar complex imaginary \( j \), as well as the spatial imaginary \(I = \Be_1 \Be_2 \Be_3 \) that is part of the multivector field itself
\begin{equation}\label{eqn:transverseField:40}
F = \BE + I \eta \BH.
\end{equation}

Let's call
\begin{equation}\label{eqn:transverseField:60}
F_z = \lr{ \BE \cdot \Be_3} \Be_3 + I \eta \lr{ \BH \cdot \Be_3 } \Be_3,
\end{equation}
the propagation component of the field and \( F_t = F - F_z \) the transverse component of the field.  We can write these in a more symmetric fashion by expanding the dot products and regrouping
\begin{equation}\label{eqn:transverseField:80}
\begin{aligned}
F_z
&= \lr{ \BE \cdot \Be_3} \Be_3 + I \eta \lr{ \BH \cdot \Be_3 } \Be_3 \\
&= \inv{2} \lr{ \BE \Be_3 + \Be_3 \BE } \Be_3 + \frac{I \eta}{2} \lr{ \BH \Be_3 + \Be_3 \BH} \Be_3 \\
&= \inv{2} \lr{ \BE + \Be_3 \BE \Be_3 } + \frac{I \eta}{2} \lr{ \BH + \Be_3 \BH \Be_3} \Be_3 \\
&= \inv{2} \lr{ F + \Be_3 F \Be_3 }.
\end{aligned}
\end{equation}
By subtraction, we also have
\begin{equation}\label{eqn:transverseField:100}
F_t = \inv{2} \lr{ F - \Be_3 F \Be_3 }.
\end{equation}

The multivector form of Maxwell's equation, for source free conditions, is
\begin{equation}\label{eqn:transverseField:120}
0 = \lr{ \spacegrad + \inv{c} \partial_t } F.
\end{equation}
We split the gradient into a propagation direction component and a transverse component \( \spacegrad_t \)
\begin{equation}\label{eqn:transverseField:140}
\spacegrad = \spacegrad_t + \Be_3 \partial_z,
\end{equation}
so
\begin{equation}\label{eqn:transverseField:160}
\begin{aligned}
0
&= \lr{ \spacegrad_t + \Be_3 \partial_z + \inv{c} \partial_t } F \\
&= \lr{ \spacegrad_t + \Be_3 \partial_z + \inv{c} \partial_t } F(x,y) e^{j(\omega t - k z) } \\
&= \lr{ \spacegrad_t - j\Be_3 k + j\frac{\omega}{c} } F(x,y) e^{j(\omega t - k z) },
\end{aligned}
\end{equation}
or
\begin{equation}\label{eqn:transverseField:180}
-j \lr{ \frac{\omega}{c} - k \Be_3 } F = \spacegrad_t F.
\end{equation}

Observe that
\begin{equation}\label{eqn:transverseField:200}
-j \lr{ \frac{\omega}{c} - k \Be_3 } \Be_3 F \Be_3 = -\spacegrad_t \Be_3 F \Be_3,
\end{equation}
which means that
\begin{equation}\label{eqn:transverseField:220}
-j \lr{ \frac{\omega}{c} - k \Be_3 } \inv{2} \lr{ F \pm \Be_3 F \Be_3 } = \spacegrad_t \inv{2} \lr{ F \mp \Be_3 F \Be_3 },
\end{equation}
or
\begin{equation}\label{eqn:transverseField:240}
\begin{aligned}
-j \lr{ \frac{\omega}{c} - k \Be_3 } F_z &= \spacegrad_t F_t \\
-j \lr{ \frac{\omega}{c} - k \Be_3 } F_t &= \spacegrad_t F_z.
\end{aligned}
\end{equation}

Provided \( \omega^2 \ne k^2 c^2 \), this can be inverted, meaning that \( F_t \) fully specifies \( F_z \) if known, as well as the opposite.

That inversion is
\begin{equation}\label{eqn:transverseField:260}
\begin{aligned}
F_z &= j \frac{ \frac{\omega}{c} + k \Be_3 }{ \lr{\frac{\omega}{c}}^2 - k^2 } \spacegrad_t F_t \\
F_t &= j \frac{ \frac{\omega}{c} + k \Be_3 }{ \lr{\frac{\omega}{c}}^2 - k^2 } \spacegrad_t F_z.
\end{aligned}
\end{equation}

Let's expand the latter.  First note that
\begin{equation}\label{eqn:transverseField:280}
\begin{aligned}
\spacegrad_t F_z
&= \spacegrad_t \Be_3 \lr{ E_z + I \eta H_z } \\
&= -\Be_3 \spacegrad_t \lr{ E_z + I \eta H_z }.
\end{aligned}
\end{equation}
so
\begin{equation}\label{eqn:transverseField:300}
F_t = -j \frac{ \frac{\omega}{c} \Be_3 + k }{ \lr{\frac{\omega}{c}}^2 - k^2 } \spacegrad_t \lr{ E_z + I \eta H_z }.
\end{equation}
This may now be split into electric and magnetic field components, but first note that the multivector operator
\begin{equation}\label{eqn:transverseField:320}
\begin{aligned}
\Be_3 \spacegrad_t
&=
\Be_3 \cdot \spacegrad_t + \Be_3 \wedge \spacegrad_t  \\
&=
\Be_3 \wedge \spacegrad_t,
\end{aligned}
\end{equation}
has only a bivector component.

For the transverse electric field component, we have
\begin{equation}\label{eqn:transverseField:340}
\begin{aligned}
\gpgradeone{ \lr{ \frac{\omega}{c} \Be_3 + k } \spacegrad_t \lr{ E_z + I \eta H_z } }
&=
k \spacegrad_t E_z + \frac{\omega}{c} \Be_3 \wedge \spacegrad_t \lr{ I \eta H_z } \\
&=
k \spacegrad_t E_z - \frac{\eta \omega}{c} \Be_3 \cross \spacegrad_t H_z.
\end{aligned}
\end{equation}
and for the magnetic field component
\begin{equation}\label{eqn:transverseField:360}
\begin{aligned}
\gpgradetwo{ \lr{ \frac{\omega}{c} \Be_3 + k } \spacegrad_t \lr{ E_z + I \eta H_z } }
=
\frac{\omega}{c} \Be_3 \wedge \spacegrad_t E_z + I \eta k \spacegrad_t H_z
\end{aligned}
\end{equation}

This means that
\begin{equation}\label{eqn:transverseField:380}
\begin{aligned}
\BE_t &= \frac{j}{\lr{\frac{\omega}{c}}^2 - k^2 } \lr{ -k \spacegrad_t E_z + \frac{\eta \omega}{c} \Be_3 \cross \spacegrad_t H_z } \\
\eta I \BH_t &= -\frac{j}{\lr{\frac{\omega}{c}}^2 - k^2 } \lr{ \frac{\omega}{c} \Be_3 \wedge \spacegrad_t E_z + I \eta k \spacegrad_t H_z }
\end{aligned}
\end{equation}

Cancelling out the \( \eta I \) factors in the magnetic field component, and substutiting \( \eta/c = \mu, 1/(c\eta) = \epsilon \), leaves us with
\begin{equation}\label{eqn:transverseField:400}
\begin{aligned}
\BE_t &= \frac{j}{\lr{\frac{\omega}{c}}^2 - k^2 } \lr{ -k \spacegrad_t E_z + \mu \omega \Be_3 \cross \spacegrad_t H_z } \\
\BH_t &= -\frac{j}{\lr{\frac{\omega}{c}}^2 - k^2 } \lr{ \epsilon \omega \Be_3 \cross \spacegrad_t E_z + k \spacegrad_t H_z }.
\end{aligned}
\end{equation}

%}
%\EndArticle
\EndNoBibArticle
