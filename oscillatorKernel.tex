%
% Copyright � 2025 Peeter Joot.  All Rights Reserved.
% Licenced as described in the file LICENSE under the root directory of this GIT repository.
%
%{
\input{../latex/blogpost.tex}
\renewcommand{\basename}{oscillatorKernel}
%\renewcommand{\dirname}{notes/phy1520/}
\renewcommand{\dirname}{notes/ece1228-electromagnetic-theory/}
%\newcommand{\dateintitle}{}
%\newcommand{\keywords}{}

\input{../latex/peeter_prologue_print2.tex}

\usepackage{peeters_layout_exercise}
\usepackage{peeters_braket}
\usepackage{peeters_figures}
\usepackage{siunitx}
\usepackage{verbatim}
%\usepackage{macros_cal} % \LL
%\usepackage{amsthm} % proof
%\usepackage{mhchem} % \ce{}
%\usepackage{macros_bm} % \bcM
%\usepackage{macros_qed} % \qedmarker
%\usepackage{txfonts} % \ointclockwise

\beginArtNoToc

\generatetitle{A PV integral using contour integration.}
%\chapter{A PV integral using contour integration.}
%\label{chap:oscillatorKernel}

Here's the second last real-integral sub-problem from \citep{byron1992mca}, problem 31(j).  Find
\begin{equation}\label{eqn:oscillatorKernel:20}
I = P \int_{-\infty}^\infty \inv{ \lr{ \omega' - \omega_0 }^2 + a^2 } \inv{ \omega' - \omega } d\omega'.
\end{equation}

Our poles are sitting at \( \omega \), and
\begin{equation}\label{eqn:oscillatorKernel:80}
\alpha, \beta = \omega_0 \pm i a
\end{equation}
one of which sits above the x-axis, one below, and one on the line.

This means that if we compute the usual infinite semicircular contour integral, we have a \( 2 \pi i \) weighted residue above the line and one \( \pi i \) weighted residue for the x-axis pole.  That is
\begin{equation}\label{eqn:oscillatorKernel:50}
\begin{aligned}
\oint \inv{ \lr{ z - \omega_0 }^2 + a^2 } \inv{ z - \omega } dz
&=
\lr{ 2 \pi i } \evalbar{ \inv{\lr{z - \lr{ \omega_0 - i a } } \lr{ z - \omega } } }{z = \omega_0 + i a }
+
\lr{ \pi i } \evalbar{ \inv{ \lr{ z - \omega_0 }^2 + a^2 }}{z = \omega } \\
&=
\lr{ 2 \pi i } \inv{\lr{\omega_0 + i a - \lr{ \omega_0 - i a } } \lr{ \omega_0 + i a - \omega } }
+
\lr{ \pi i } \inv{ \lr{ \omega - \omega_0 }^2 + a^2 } \\
&=
\frac{ 2 \pi i }{2 i a} \inv{ \omega_0 + i a - \omega } \frac{ \omega_0 - i a - \omega }{\omega_0 - i a - \omega }
+
\lr{ \pi i } \inv{ \lr{ \omega - \omega_0 }^2 + a^2 } \\
&=
\frac{ \pi }{ \lr{ \omega - \omega_0 }^2 + a^2 } \lr{ \frac{\omega_0 - \omega}{a} - i + i },
\end{aligned}
\end{equation}
or
\begin{equation}\label{eqn:oscillatorKernel:100}
\boxed{
I =
\frac{ \pi \lr{ \omega_0 - \omega } }{ a \lr{ \lr{ \omega - \omega_0 }^2 + a^2} }.
}
\end{equation}

%While this is the answer stated in the text, Mathematica doesn't seem to be able to solve this one, at least stated niavely.  .  I'm assuming Perhaps the PV aspect of the integral
%Interestingly, Mathematica doesn't

%}
\EndArticle
%\EndNoBibArticle
