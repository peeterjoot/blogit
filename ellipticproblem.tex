%
% Copyright � 2023 Peeter Joot.  All Rights Reserved.
% Licenced as described in the file LICENSE under the root directory of this GIT repository.
%
%{
\input{../latex/blogpost.tex}
\renewcommand{\basename}{ellipticproblem}
%\renewcommand{\dirname}{notes/phy1520/}
\renewcommand{\dirname}{notes/ece1228-electromagnetic-theory/}
%\newcommand{\dateintitle}{}
%\newcommand{\keywords}{}

\input{../latex/peeter_prologue_print2.tex}

\usepackage{peeters_layout_exercise}
\usepackage{peeters_braket}
\usepackage{peeters_figures}
\usepackage{siunitx}
\usepackage{verbatim}
%\usepackage{mhchem} % \ce{}
%\usepackage{macros_bm} % \bcM
%\usepackage{macros_qed} % \qedmarker
%\usepackage{txfonts} % \ointclockwise

\beginArtNoToc

\generatetitle{XXX}
%\chapter{XXX}
%\label{chap:ellipticproblem}

\makeproblem{Elliptic curvilinear and reciprocal basis.}{problem:ellipticproblem:10}{
\makesubproblem{}{problem:ellipticproblem:10:a}
Show that an ellipse can be parameterized by
\begin{equation}\label{eqn:ellipticproblem:20}
   \Bx = s \Be_1 \cosh\lr{ \mu + i \theta },
\end{equation}
where \( i = \Be_{12} \), and find the values of the semi-major and semi-minor axes.
\makesubproblem{}{problem:ellipticproblem:10:b}
Determine how \( \mu \) and the eccentricity \( \epsilon = \sqrt{1 - b^2/a^2} \) are related.
\makesubproblem{}{problem:ellipticproblem:10:c}
Compute the curvilinear and reciprocal frame vectors for the parameterization \( \Bx(s, \theta) \) above.
%, and use this to verify
%\cref{eqn:curvilinearDefined:520} and \cref{eqn:curvilinearDefined:540} respectively.
\makesubproblem{}{problem:ellipticproblem:10:d}
Check that \( \Bx^i \cdot \Bx_j = {\delta^i}_j \).
%Hints: Given \( \mu = \Atanh(1/2) \),
%\begin{itemize}
%\item \( \cosh( \mu + i \theta ) \Be_2 = \Be_2 \cosh( \mu - i \theta ) \).
%\item \( \Real\lr{ \cosh( \mu - i \theta ) \sinh( \mu + i \theta ) } = 2/3 \).
%\end{itemize}
} % problem
\makeanswer{problem:ellipticproblem:10}{
\makesubanswer{}{problem:ellipticproblem:10:a}
Expanding the \( \cosh \) in terms of exponentials, we find
\begin{equation}\label{eqn:ellipticproblem:40}
\begin{aligned}
\Be_1 \cosh\lr{ \mu + i \theta }
&=
\frac{\Be_1}{2} \lr{ e^{\mu + i \theta} + e^{-\mu - i\theta} } \\
&=
\Be_1 \frac{e^\mu}{2} \lr{ \cos\theta + i \sin\theta }
+
\Be_1 \frac{e^{-\mu}}{2} \lr{ \cos\theta - i \sin\theta } \\
&=
\Be_1 \frac{ e^\mu + e^{-\mu} }{2} \cos\theta
+ \Be_2 \frac{ e^\mu - e^{-\mu} }{2} \sin\theta \\
&=
\Be_1 \cosh\mu \cos\theta + \Be_2 \sinh\mu \sin\theta,
\end{aligned}
\end{equation}
so
\begin{equation}\label{eqn:ellipticproblem:60}
\Bx = s \Be_1 \cosh\lr{ \mu + i \theta } = \Be_1 a \cos\theta + \Be_2 b \sin\theta,
\end{equation}
where
\begin{equation}\label{eqn:ellipticproblem:80}
\begin{aligned}
   a &= s \cosh\mu \\
   b &= s \sinh\mu,
\end{aligned}
\end{equation}
are the semi-major and semi-minor axis values.
\makesubanswer{}{problem:ellipticproblem:10:b}
The eccentricity (squared) is
\begin{equation}\label{eqn:ellipticproblem:100}
\begin{aligned}
   \epsilon^2
   &= 1 - \tanh^2\mu \\
   &= \frac{\cosh^2\mu - \sinh^2\mu}{\cosh^2\mu} \\
   &= \inv{\cosh^2\mu},
\end{aligned}
\end{equation}
so the eccentricity is
\begin{equation}\label{eqn:ellipticproblem:120}
   \epsilon = \inv{\cosh\mu}.
\end{equation}
\makesubanswer{}{problem:ellipticproblem:10:c}
Our curvilinear basis vectors are
\begin{equation}\label{eqn:ellipticproblem:140}
\begin{aligned}
\Bx_s &= \Be_1 \cosh\lr{ \mu + i \theta } \\
\Bx_\theta &= \Be_2 s \sinh\lr{ \mu + i \theta } \\
\end{aligned}
\end{equation}

To compute the reciprocals we need the area element
\begin{equation}\label{eqn:ellipticproblem:160}
\begin{aligned}
\Bx_s \wedge \Bx_\theta
&=
\gpgradetwo{
   \Be_1 \cosh\lr{ \mu + i \theta } \Be_2 s \sinh\lr{ \mu + i \theta }
} \\
&=
s \gpgradetwo{
   i \cosh\lr{ \mu - i \theta } \sinh\lr{ \mu + i \theta }
} \\
&=
s \gpgradetwo{
  i  \cosh\lr{ \mu - i \theta } \sinh\lr{ \mu + i \theta }
} \\
&=
\frac{s}{4} \gpgradetwo{
   i \lr{ e^{\mu - i\theta} - e^{-\mu + i \theta } } \lr{ e^{\mu + i\theta} - e^{-\mu - i \theta } }
} \\
&=
\frac{s}{4} \gpgradetwo{
   i \lr{ e^{2 \mu} - e^{-2\mu} + e^{2 i \theta} - e^{-2 i \theta} }
} \\
&=
\frac{s}{2} \gpgradetwo{
   i \sinh(2 \mu) - \sin(2 \theta)
} \\
&=
\frac{s i}{2} \sinh (2 \mu) \\
&=
s i \cosh \mu \sinh \mu.
\end{aligned}
\end{equation}

Our recipocal basis vectors are
\begin{equation}\label{eqn:ellipticproblem:240}
\begin{aligned}
   \Bx^s
   &= \Bx_\theta \inv{ \Bx_s \wedge \Bx_\theta } \\
   &= \Be_2 s \sinh\lr{ \mu + i \theta } \inv{s i \cosh \mu \sinh \mu} \\
   &= \Be_1 \frac{\sinh\lr{ \mu + i \theta }}{\cosh \mu \sinh \mu},
\end{aligned}
\end{equation}
and
\begin{equation}\label{eqn:ellipticproblem:180}
\begin{aligned}
\Bx^\theta
   &= -\Bx_s \inv{ \Bx_s \wedge \Bx_\theta } \\
   &= -\lr{ \Be_1 \cosh\lr{ \mu + i \theta } } \inv{ s i \cosh \mu \sinh \mu} \\
   &= \frac{ \Be_2 \cosh\lr{ \mu + i \theta } }{ s \cosh \mu \sinh \mu}.
\end{aligned}
\end{equation}
\makesubanswer{}{problem:ellipticproblem:10:d}
%\Bx_s &= \Be_1 \cosh\lr{ \mu + i \theta } \\
%\Bx_\theta &= \Be_2 s \sinh\lr{ \mu + i \theta } \\
%\Bx^s &= \Be_1 \frac{\sinh\lr{ \mu + i \theta }}{\cosh \mu \sinh \mu},
%\Bx^\theta &= \frac{ \Be_2 \cosh\lr{ \mu + i \theta } }{ s \cosh \mu \sinh \mu}.
} % answer


%}
%\EndArticle
\EndNoBibArticle
