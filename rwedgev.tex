%
% Copyright � 2025 Peeter Joot.  All Rights Reserved.
% Licenced as described in the file LICENSE under the root directory of this GIT repository.
%
%{
\input{../latex/blogpost.tex}
\renewcommand{\basename}{rwedgev}
%\renewcommand{\dirname}{notes/phy1520/}
\renewcommand{\dirname}{notes/ece1228-electromagnetic-theory/}
%\newcommand{\dateintitle}{}
%\newcommand{\keywords}{}

\input{../latex/peeter_prologue_print2.tex}

\usepackage{peeters_layout_exercise}
\usepackage{peeters_braket}
\usepackage{peeters_figures}
\usepackage{siunitx}
\usepackage{verbatim}
%\usepackage{macros_cal} % \LL
%\usepackage{amsthm} % proof
%\usepackage{mhchem} % \ce{}
%\usepackage{macros_bm} % \bcM
%\usepackage{macros_qed} % \qedmarker
%\usepackage{txfonts} % \ointclockwise

\beginArtNoToc

\generatetitle{XXX}
%\chapter{XXX}
%\label{chap:rwedgev}

Given constant velocity \( \Bv \), and a vector following a trajectory in that direction from some initial point \( \Br_0 \)
\begin{equation}\label{eqn:rwedgev:20}
\Br(t) = \Br_0 + \Bv t,
\end{equation}
we have
\begin{equation}\label{eqn:rwedgev:40}
\Br(t) \wedge \Bv = \Br_0 \wedge \Bv,
\end{equation}
which is a constant.  This means that the derivative of \( \Br \wedge \Bv \) should be zero, which we can check:
\begin{equation}\label{eqn:rwedgev:60}
\begin{aligned}
\ddt{} \lr{ \Br(t) \wedge \Bv }
&=
\ddt{\Br} \wedge \Bv + \Br \wedge \ddt{\Bv}.
\end{aligned}
\end{equation}
The first term is equal to \( \Bv \wedge \Bv = 0 \), and the second term is zero because \( \Bv \) is constant.

\chapter{Circular coordinates.}

With \( i = \Be_1 \Be_2 \)
\begin{equation}\label{eqn:rwedgev:80}
\Br = r \Be_1 e^{i \theta} = r \rcap,
\end{equation}
where \( \rcap = \Be_1 e^{i \theta} \).
The velocity is
\begin{equation}\label{eqn:rwedgev:120}
\begin{aligned}
\Bv
&= \dot{r} \Be_1 e^{i \theta} + r \Be_2 e^{i\theta} \omega \\
&= \dot{r} \rcap + r \omega \thetacap,
\end{aligned}
\end{equation}
where \( \thetacap = \Be_2 e^{i\theta} \), and \( \omega = d\theta/dt \).

This time we have
\begin{equation}\label{eqn:rwedgev:140}
\begin{aligned}
\Br \wedge \Bv &= r^2 \omega \rcap \wedge \thetacap \\
&= r^2 \omega i.
\end{aligned}
\end{equation}

BTW, if it's not obvious that \( \rcap \wedge \thetacap = i \), you can show it easily using a grade selection representation of the wedge product
\begin{equation}\label{eqn:rwedgev:160}
\begin{aligned}
\rcap \wedge \thetacap
&= \gpgradetwo{ \Be_1 e^{i \theta} \Be_2 e^{i\theta} } \\
&= \gpgradetwo{ \Be_1 e^{i \theta} e^{-i\theta} \Be_2} \\
&= \gpgradetwo{ \Be_1 \Be_2} \\
&= \Be_1 \Be_2 \\
&= i.
\end{aligned}
\end{equation}

%\section{}
%
%0 = \ddt{ } r^2 \omega
%= 2 r \dot{r} + r^2 \dot{\omega}
%
%\dot{r} = - r \dot{\omega}
%
%\Bv
%&= r \lr{ -\dot{\omega}\rcap + \omega \thetacap} ,
%&= r \lr{ -\dot{\omega} \Be_1 + \omega \Be_2} e^{i\theta}

%}
%\EndArticle
\EndNoBibArticle
