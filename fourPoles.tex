%
% Copyright � 2025 Peeter Joot.  All Rights Reserved.
% Licenced as described in the file LICENSE under the root directory of this GIT repository.
%
%{
\input{../latex/blogpost.tex}
\renewcommand{\basename}{fourPoles}
%\renewcommand{\dirname}{notes/phy1520/}
\renewcommand{\dirname}{notes/ece1228-electromagnetic-theory/}
%\newcommand{\dateintitle}{}
%\newcommand{\keywords}{}

\input{../latex/peeter_prologue_print2.tex}

\usepackage{peeters_layout_exercise}
\usepackage{peeters_braket}
\usepackage{peeters_figures}
\usepackage{siunitx}
\usepackage{verbatim}
%\usepackage{macros_cal} % \LL
%\usepackage{amsthm} % proof
%\usepackage{mhchem} % \ce{}
%\usepackage{macros_bm} % \bcM
%\usepackage{macros_qed} % \qedmarker
%\usepackage{txfonts} % \ointclockwise

\beginArtNoToc

\generatetitle{Another real integral using contour integration.}
%\chapter{Another real integral using contour integration.}
%\label{chap:fourPoles}

Here's (31(d)) from \citep{byron1992mca}.  Find
\begin{equation}\label{eqn:fourPoles:20}
I = \int_0^\infty \frac{dx}{1 + x^4} = \inv{2}\int_{-\infty}^\infty \frac{dx}{1 + x^4}.
\end{equation}
This one is easy conceptually, but a bit messy algebraically.  We integrate over the contour \( C \) illustrated in \cref{fig:fourPoles:fourPolesFig1}.
\imageFigure{../figures/blogit/fourPolesFig1}{Standard above the x-axis, semicircular contour.}{fig:fourPoles:fourPolesFig1}{0.3}

We want to evaluate
\begin{equation}\label{eqn:fourPoles:40}
2 I = \oint_C \frac{dz}{1 + z^4},
\end{equation}
because the semicircular part of the integral is \( O(R^{-3}) \), which tends to zero in the \( R \rightarrow \infty \) limit.

The poles are at the points
\begin{equation}\label{eqn:fourPoles:60}
\begin{aligned}
z^4
&= -1 \\
&= e^{i \pi + 2 \pi i k},
\end{aligned}
\end{equation}
or
\begin{equation}\label{eqn:fourPoles:80}
\begin{aligned}
z
&= e^{i \pi/4 + \pi i k/2},
\end{aligned}
\end{equation}
These are the points \( z = (\pm 1 \pm i)/\sqrt{2} \), two of which are enclosed by our contour.  Specifically
\begin{equation}\label{eqn:fourPoles:100}
\begin{aligned}
2 I
&= \oint_C \frac{dz}{
\lr{ z - \frac{1 + i}{\sqrt{2}} }
\lr{ z - \frac{-1 + i}{\sqrt{2}} }
\lr{ z - \frac{1 - i}{\sqrt{2}} }
\lr{ z - \frac{-1 - i}{\sqrt{2}} }
} \\
&= \oint_C \frac{dz}{
\lr{ z - \frac{1 + i}{\sqrt{2}} }
\lr{ z - \frac{-1 + i}{\sqrt{2}} }
\lr{ \lr{z + \frac{i}{\sqrt{2}}}^2 - \inv{2} }
} \\
&=
\evalbar{
\frac{ 2 \pi i }
{
\lr{ z - \frac{-1 + i}{\sqrt{2}} }
\lr{ \lr{z + \frac{i}{\sqrt{2}}}^2 - \inv{2} }
}
}{z = \frac{1 + i}{\sqrt{2}}}
+
\evalbar{
\frac{ 2 \pi i }
{
\lr{ z - \frac{1 + i}{\sqrt{2}} }
\lr{ \lr{z + \frac{i}{\sqrt{2}}}^2 - \inv{2} }
}
}
{z = \frac{-1 + i}{\sqrt{2}} } \\
&=
\evalbar{
\frac{(2 \pi i )(2 \sqrt{2})}
{
\lr{ z' + 1 - i }
\lr{ \lr{z' + i}^2 - 1 }
}
}{z' = 1 + i}
+
\evalbar{
\frac{(2 \pi i )(2 \sqrt{2})}
{
\lr{ z' - 1 - i }
\lr{ \lr{z' + i}^2 - 1 }
}
}
{z' = -1 + i}
\\
&=
\frac{2 \pi i \sqrt{2}}
{
\lr{2 i + 1}^2 - 1 }
-
\frac{2 \pi i \sqrt{2}}
{ \lr{2 i - 1}^2 - 1 }
\\
&=
\frac{\pi i \sqrt{2}}
{
2 (-1 + i)
}
+
\frac{\pi i \sqrt{2}}
{ 2(1 + i) }
\\
&=
\lr{ -1 - i }
\frac{\pi i}
{
2 \sqrt{2}
}
+
\lr{ 1 - i }
\frac{\pi i}
{ 2 \sqrt{2} }
\\
&=
\frac{\pi}
{ \sqrt{2} }
\end{aligned}
\end{equation}
or
\begin{equation}\label{eqn:fourPoles:120}
\boxed{
I = \frac{\pi}{2 \sqrt{2}}.
}
\end{equation}

%}
\EndArticle
%\EndNoBibArticle
