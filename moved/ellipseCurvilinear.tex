%
% Copyright � 2023 Peeter Joot.  All Rights Reserved.
% Licenced as described in the file LICENSE under the root directory of this GIT repository.
%
%{
\input{../latex/blogpost.tex}
\renewcommand{\basename}{ellipseCurvilinear}
%\renewcommand{\dirname}{notes/phy1520/}
\renewcommand{\dirname}{notes/ece1228-electromagnetic-theory/}
%\newcommand{\dateintitle}{}
%\newcommand{\keywords}{}

\input{../latex/peeter_prologue_print2.tex}

\usepackage{peeters_layout_exercise}
\usepackage{peeters_braket}
\usepackage{peeters_figures}
\usepackage{siunitx}
\usepackage{verbatim}
%\usepackage{mhchem} % \ce{}
%\usepackage{macros_bm} % \bcM
%\usepackage{macros_qed} % \qedmarker
%\usepackage{txfonts} % \ointclockwise

\beginArtNoToc

\generatetitle{XXX}
%\chapter{XXX}
%\label{chap:ellipseCurvilinear}

\makeproblem{Elliptic parameterization.}{problem:ellipseCurvilinear:200}{
An elliptical area can be parameterized as
\begin{equation}\label{eqn:ellipseCurvilinear:20}
   \Bx(u_1, u_2) = u_1 \lr{ \Be_1 \cos u_2 + \beta \Be_2 \sin u_2 },
\end{equation}
where \( \beta = \sqrt{1 - \epsilon^2} \), and \( \epsilon \) is the eccentricity of the ellipse.
\makesubproblem{}{problem:ellipseCurvilinear:200:a}
Compute the curvilinear vectors
\begin{equation}\label{eqn:ellipseCurvilinear:200}
\begin{aligned}
   \Bx_1 &= \PDi{u_1}{\Bx} \\
   \Bx_2 &= \PDi{u_2}{\Bx}.
\end{aligned}
\end{equation}
\makesubproblem{}{problem:ellipseCurvilinear:200:b}
Compute the reciprocal frame vectors
\begin{equation}\label{eqn:ellipseCurvilinear:220}
\begin{aligned}
\Bx^1 &= \Bx_2 \cdot \inv{ \Bx_1 \wedge \Bx_2 } \\
\Bx^2 &= -\Bx_1 \cdot \inv{ \Bx_1 \wedge \Bx_2 }.
\end{aligned}
\end{equation}
\makesubproblem{}{problem:ellipseCurvilinear:200:c}
Verify that \( \Bx_i \cdot \Bx^j = {\delta_i}^j \).
} % problem

\makeanswer{problem:ellipseCurvilinear:200}{
\makesubanswer{}{problem:ellipseCurvilinear:200:a}
The curvilinear basis associated with this parameterization can be computed by inspection
\begin{equation}\label{eqn:ellipseCurvilinear:40}
\begin{aligned}
   \Bx_1 &= \Be_1 \cos u_2 + \beta \Be_2 \sin u_2 \\
   \Bx_2 &= u_1 \lr{ -\Be_1 \sin u_2 + \beta \Be_2 \cos u_2 }.
\end{aligned}
\end{equation}
\makesubanswer{}{problem:ellipseCurvilinear:200:b}

We need to compute the area element first
\begin{equation}\label{eqn:ellipseCurvilinear:60}
\begin{aligned}
\Bx_1 \wedge \Bx_2
&= \lr{ \Be_1 \cos u_2 + \beta \Be_2 \sin u_2 } \wedge u_1 \lr{ -\Be_1 \sin u_2 + \beta \Be_2 \cos u_2 } \\
&= u_1 \gpgradetwo{ \lr{ \Be_1 \cos u_2 + \beta \Be_2 \sin u_2 } \lr{ -\Be_1 \sin u_2 + \beta \Be_2 \cos u_2 } } \\
&= u_1 \lr{
\beta \Be_{12} \cos^2 u_2 - \beta \Be_{21} \sin^2 u_2
} \\
&= u_1 \beta i,
\end{aligned}
\end{equation}
where \( i = \Be_{12} \).

The reciprocal frame vectors are given by
\begin{equation}\label{eqn:ellipseCurvilinear:80}
\begin{aligned}
\Bx^1
   &= \Bx_2 \cdot \inv{ \Bx_1 \wedge \Bx_2 } \\
   &= u_1 \lr{ -\Be_1 \sin u_2 + \beta \Be_2 \cos u_2 } \inv{ u_1 \beta i } \\
   %&= \lr{ - \inv{\beta} \Be_1 \sin u_2 + \Be_2 \cos u_2 } (-i)
   &= \inv{\beta} \Be_2 \sin u_2 + \Be_1 \cos u_2,
\end{aligned}
\end{equation}
\begin{equation}\label{eqn:ellipseCurvilinear:100}
\begin{aligned}
   \Bx^2
   &= -\Bx_1 \cdot \inv{ \Bx_1 \wedge \Bx_2 } \\
   &= - \lr{ \Be_1 \cos u_2 + \beta \Be_2 \sin u_2 } \inv{ u_1 \beta i } \\
   %&= \lr{ \Be_1 \cos u_2 + \beta \Be_2 \sin u_2 } \frac{i}{ u_1 \beta }
   &= \inv{u_1} \lr{ \inv{\beta} \Be_2 \cos u_2 - \Be_1 \sin u_2 }.
\end{aligned}
\end{equation}

\makesubanswer{}{problem:ellipseCurvilinear:200:c}
To verify that \( \Bx_i \cdot \Bx^j = {\delta_i}^j \) we can compute each of the dot products
\begin{equation}\label{eqn:ellipseCurvilinear:120}
\begin{aligned}
\Bx^1 \cdot \Bx_1
&= \gpgradezero{
\lr{ \Be_1 \cos u_2 + \beta \Be_2 \sin u_2 } \lr{ \inv{\beta} \Be_2 \sin u_2 + \Be_1 \cos u_2 }
} \\
&=
\cos^2 u_2 + \sin^2 u_2 \\
&= 1,
\end{aligned}
\end{equation}
\begin{equation}\label{eqn:ellipseCurvilinear:140}
\begin{aligned}
\Bx^2 \cdot \Bx_2
&= \gpgradezero{
u_1 \lr{ -\Be_1 \sin u_2 + \beta \Be_2 \cos u_2 } \inv{u_1} \lr{ \inv{\beta} \Be_2 \cos u_2 - \Be_1 \sin u_2 }
} \\
&=
% \lr{ -\Be_1 \sin u_2 + \beta \Be_2 \cos u_2 } \lr{ \inv{\beta} \Be_2 \cos u_2 - \Be_1 \sin u_2 }
\sin^2 u_2 + \cos^2 u_2 \\
&= 1,
\end{aligned}
\end{equation}
\begin{equation}\label{eqn:ellipseCurvilinear:160}
\begin{aligned}
\Bx^1 \cdot \Bx_2
&= \gpgradezero{
\lr{ \inv{\beta} \Be_2 \sin u_2 + \Be_1 \cos u_2 } u_1 \lr{ -\Be_1 \sin u_2 + \beta \Be_2 \cos u_2 }
} \\
&=
u_1 \sin u_2 \cos u_2 - u_1 \cos u_2 \sin u_2 \\
&= 0.
\end{aligned}
\end{equation}
\begin{equation}\label{eqn:ellipseCurvilinear:180}
\begin{aligned}
\Bx^2 \cdot \Bx_1
&= \gpgradezero{
\inv{u_1} \lr{ \inv{\beta} \Be_2 \cos u_2 - \Be_1 \sin u_2 } \lr{ \Be_1 \cos u_2 + \beta \Be_2 \sin u_2 }
} \\
&=
\inv{u_1} \lr{
\cos u_2 \sin u_2 - \sin u_2 \cos u_2
}
\\
&= 0.
\end{aligned}
\end{equation}
} % answer

%}
%\EndArticle
\EndNoBibArticle
