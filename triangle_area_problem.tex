%
% Copyright � 2024 Peeter Joot.  All Rights Reserved.
% Licenced as described in the file LICENSE under the root directory of this GIT repository.
%
%{
\input{../latex/blogpost.tex}
\renewcommand{\basename}{triangle_area_problem}
%\renewcommand{\dirname}{notes/phy1520/}
\renewcommand{\dirname}{notes/ece1228-electromagnetic-theory/}
%\newcommand{\dateintitle}{}
%\newcommand{\keywords}{}

\input{../latex/peeter_prologue_print2.tex}

\usepackage{peeters_layout_exercise}
\usepackage{peeters_braket}
\usepackage{peeters_figures}
\usepackage{siunitx}
\usepackage{verbatim}
%\usepackage{mhchem} % \ce{}
%\usepackage{macros_bm} % \bcM
%\usepackage{macros_qed} % \qedmarker
%\usepackage{txfonts} % \ointclockwise

\beginArtNoToc

\generatetitle{Triangle area problem}
%\chapter{Triangle area problem}
%\label{chap:triangle_area_problem}

On \href{https://www.linkedin.com/feed/update/urn:li:activity:7179268039274622977}{LinkedIn, James asked} for ideas about how to solve \href{https://www.youtube.com/watch?v=R26qu6xJZnA}{What is the total area of ABC? You should be able to solve this!} using geometric algebra.

I found one way, but suspect it's not the easiest way to solve the problem.

To start with I've re-sketched the triangle with the areas slightly more to scale in \cref{fig:triangle_area_problem:triangle_area_problemFig1}, where areas \( A_1 = 40, A_2 = 30, A_3 = 35, A_4 = 84 \) are given.  The aim is to find the total area \( \sum A_i \).
\imageFigure{../figures/blogit/triangle_area_problemFig1}{Triangle with given areas.}{fig:triangle_area_problem:triangle_area_problemFig1}{0.3}

If we had the vertex and center locations as vectors, we could easily compute the total area, but we don't.  We also don't know the locations of the edge intersections, but can calculate those, as they satisfy
\begin{equation}\label{eqn:triangle_area_problem:20}
\begin{aligned}
\BD &= s_1 \BA = \BB + t_1 \lr{ \BC - \BB } \\
\BE &= s_2 \BC = \BA + t_2 \lr{ \BB - \BA } \\
\BF &= s_3 \BB = \BA + t_3 \lr{ \BC - \BA }.
\end{aligned}
\end{equation}
Eliminating the \( t_i \) constants by wedging appropriately (or equivalently, using Cramer's rule), we find
\begin{equation}\label{eqn:triangle_area_problem:40}
\begin{aligned}
s_1 \BA \wedge \lr{ \BC - \BB } &= \BB \wedge \lr{ \BC - \BB }  \\
s_2 \BC \wedge \lr{ \BB - \BA } &= \BA \wedge \lr{ \BB - \BA }  \\
s_3 \BB \wedge \lr{ \BC - \BA } &= \BA \wedge \lr{ \BC - \BA },
\end{aligned}
\end{equation}
or
\begin{equation}\label{eqn:triangle_area_problem:60}
\begin{aligned}
s_1 &= \frac{\BB \wedge \BC }{\BA \wedge \lr{ \BC - \BB }} \\
s_2 &= \frac{\BA \wedge \BB }{\BC \wedge \lr{ \BB - \BA }} \\
s_3 &= \frac{\BA \wedge \BC }{\BB \wedge \lr{ \BC - \BA }}.
\end{aligned}
\end{equation}
Introducing bivector (signed-area) unknowns with scalar coefficients
\begin{equation}\label{eqn:triangle_area_problem:80}
\begin{aligned}
X &= \begin{vmatrix} \BA & \BB \end{vmatrix} \\
Y &= \begin{vmatrix} \BC & \BB \end{vmatrix} \\
Z &= \begin{vmatrix} \BA & \BC \end{vmatrix},
\end{aligned}
\end{equation}
\begin{equation}\label{eqn:triangle_area_problem:81}
\begin{aligned}
\BA \wedge \BB &= X \Be_{12} \\
\BC \wedge \BB &= Y \Be_{12} \\
\BA \wedge \BC &= Z \Be_{12}.
\end{aligned}
\end{equation}
The intersection parameters, after cancelling the pseudoscalar factors, are
\begin{equation}\label{eqn:triangle_area_problem:100}
\begin{aligned}
s_1 &= \frac{\BB \wedge \BC }{\BA \wedge \BC - \BA \wedge \BB } = \frac{-Y}{Z - X} \\
s_2 &= \frac{\BA \wedge \BB }{\BC \wedge \BB - \BC \wedge \BA } = \frac{X}{Y + Z} \\
s_3 &= \frac{\BA \wedge \BC }{\BB \wedge \BC - \BB \wedge \BA } = \frac{Z}{-Y + X},
\end{aligned}
\end{equation}
so the intersection points are
\begin{equation}\label{eqn:triangle_area_problem:120}
\begin{aligned}
\BD &= \BA \frac{Y}{X - Z} \\
\BE &= \BC \frac{X}{Y + Z} \\
\BF &= \BB \frac{Z}{X - Y}.
\end{aligned}
\end{equation}

We may now express the known areas in terms of these unknown vectors
\begin{equation}\label{eqn:triangle_area_problem:140}
\begin{aligned}
A_1 &= \inv{2} \Abs{ \BD \wedge \BC }                         \\
A_2 &= \inv{2} \Abs{ \lr{\BC - \BA} \wedge \lr{ \BD - \BA } } \\
A_3 &= \inv{2} \Abs{ \lr{\BA - \BC} \wedge \lr{ \BE - \BC } } \\
A_5 &= \inv{2} \Abs{ \lr{\BB - \BC} \wedge \lr{ \BF - \BC } },
\end{aligned}
\end{equation}
but
\begin{equation}\label{eqn:triangle_area_problem:160}
\begin{aligned}
\BD - \BA &= \BA \lr{ -1 + \frac{Y}{X - Z} } \\
\BE - \BC &= \BC \lr{ -1 + \frac{X}{Y + Z} } \\
\BF - \BC &= \BB \frac{Z}{X - Y} - \BC,
\end{aligned}
\end{equation}
so
\begin{equation}\label{eqn:triangle_area_problem:180}
\begin{aligned}
A_1 &= \inv{2} \Abs{ \BA \wedge \BC \frac{Y}{X - Z} } = \inv{2} \Abs{ Z} \Abs{ \frac{Y}{X - Z} } \\
A_2 &= \inv{2} \Abs{ \lr{\BC - \BA} \wedge \BA } \Abs{ -1 + \frac{Y}{X - Z} } = \inv{2} \Abs{Z} \Abs{ 1 + \frac{Y}{Z - X} } \\
A_3 &= \inv{2} \Abs{ \lr{\BA - \BC} \wedge \BC } \Abs{ -1 + \frac{X}{Y + Z} } = \inv{2} \Abs{Z} \Abs{ 1 - \frac{X}{Y + Z} },
\end{aligned}
\end{equation}
and
\begin{equation}\label{eqn:triangle_area_problem:200}
\begin{aligned}
A_5
&= \inv{2} \Abs{ \lr{\BB - \BC} \wedge \lr{ \BB \frac{Z}{X - Y} - \BC } } \\
&= \inv{2} \Abs{ -\BB \wedge \BC + \BC \wedge \BB \frac{Z}{X - Y} } \\
&= \inv{2} \Abs{ Y } \Abs{ 1 + \frac{Z}{Y - X} }.
\end{aligned}
\end{equation}

This gives us four equations in two (scalar) unknowns
\begin{equation}\label{eqn:triangle_area_problem:300}
\begin{aligned}
4 A_1^2 \lr{ X - Z }^2 &= Z^2 Y^2 \\
4 A_2^2 \lr{ X - Z }^2 &= Z^2 \lr{ X - Y - Z }^2 \\
4 A_3^2 \lr{ Z + Y }^2  &= Z^2 \lr{ X - Y - Z }^2 \\
4 A_4^2 \lr{ X - Y }^2  &= Y^2 \lr{ X - Y - Z }^2.
\end{aligned}
\end{equation}
The goal is now to solve this system for \( X \).  We can blindly solve this system (\href{https://github.com/peeterjoot/mathematica/blob/master/blogit/triangle.nb}{for example using Mathematica}), with the numeric values of the original problem, to find:
\begin{equation}\label{eqn:triangle_area_problem:280}
\begin{aligned}
X &= 630 \\
Y &= 280 \\
Z &= 140,
\end{aligned}
\end{equation}
Therefore, the total area is \( \inv{2} \Abs{\BA \wedge \BB } = X/2 = 315 \).

However, there is more geometry that we can use to simplify things.  In particular
\begin{equation}\label{eqn:triangle_area_problem:320}
\inv{2} \Abs{ \BA \wedge \BC } = \frac{Z}{2} = A_1 + A_2,
\end{equation}
\begin{equation}\label{eqn:triangle_area_problem:340}
\begin{aligned}
4 A_1^2 \lr{ X -2 A_1 - 2 A_2 }^2 &= 4 \lr{ A_1 + A_2 }^2 Y^2 \\
4 A_2^2 \lr{ X -2 A_1 - 2 A_2 }^2 &= 4 \lr{ A_1 + A_2 }^2 \lr{ X - Y -2 A_1 - 2 A_2 }^2 \\
4 A_3^2 \lr{ 2 A_1 + 2 A_2 + Y }^2  &= 4 \lr{ A_1 + A_2 }^2 \lr{ X - Y -2 A_1 - 2 A_2 }^2 \\
4 A_4^2 \lr{ X - Y }^2  &= Y^2 \lr{ X - Y -2 A_1 - 2 A_2 }^2.
\end{aligned}
\end{equation}

We also know that
\begin{equation}\label{eqn:triangle_area_problem:360}
\begin{aligned}
A_1 + A_2 + A_3 &= \inv{2} \Abs{ \BA \wedge \BE } \\
&= \inv{2} \Abs{ \BA \wedge \BC } \frac{X}{Y + Z} \\
&= \inv{2} \frac{Z X}{Y + Z} \\
&= \frac{\lr{A_1 + A_2} X}{Y + 2 A_1 + 2 A_2 },
\end{aligned}
\end{equation}
so
\begin{equation}\label{eqn:triangle_area_problem:380}
Y = - 2 A_1 - 2 A_2 + \frac{ \lr{A_1 + A_2} X }{ A_1 + A_2 + A_3 }.
\end{equation}
We can substitute this into any of the four equations in \( X, Y \) above, and then just solve a quadratic to find twice the area \( X \).

Now, I'll have to watch the video and see how he solved it.

%}
%\EndArticle
\EndNoBibArticle
