%
% Copyright � 2024 Peeter Joot.  All Rights Reserved.
% Licenced as described in the file LICENSE under the root directory of this GIT repository.
%
%{
\input{../latex/blogpost.tex}
\renewcommand{\basename}{distanceBetweenPlanes}
%\renewcommand{\dirname}{notes/phy1520/}
\renewcommand{\dirname}{notes/ece1228-electromagnetic-theory/}
%\newcommand{\dateintitle}{}
%\newcommand{\keywords}{}

\input{../latex/peeter_prologue_print2.tex}

\usepackage{peeters_layout_exercise}
\usepackage{peeters_braket}
\usepackage{peeters_figures}
\usepackage{siunitx}
\usepackage{verbatim}
%\usepackage{mhchem} % \ce{}
%\usepackage{macros_bm} % \bcM
%\usepackage{macros_qed} % \qedmarker
%\usepackage{txfonts} % \ointclockwise

\beginArtNoToc

\generatetitle{Shortest distance between two planes.}
%\chapter{Shortest distance between two planes.}
%\label{chap:distanceBetweenPlanes}

\section{The problem.}
Helping Karl with his linear algebra exam prep, he asked me about this problem
\makeproblem{}{problem:distanceBetweenPlanes:1}{
Find the shortest distance between the two parallel planes, \( P_1 \), and \( P_2 \), with respective equations:
\begin{equation*}
\begin{aligned}
x - y + 2 z &= -3 \\
3 x - 3 y + 6 z &= 1.
\end{aligned}
\end{equation*}
} % problem

\section{A numerical way to tackle the problem.}
A fairly straightforward way to tackle this problem is illustrated in the sketch of \cref{fig:twoPlanes:twoPlanesFig1}.  If we can find a point in the first plane, we can follow the normal to the plane to the next, and compute the length of that connecting vector.
\imageFigure{../figures/blogit/twoPlanesFig1}{Distance between two planes.}{fig:twoPlanes:twoPlanesFig1}{0.3}

For this problem, let
\begin{equation}\label{eqn:distanceBetweenPlanes:20}
\Bn = (1,-1,2),
\end{equation}
and rescale the two plane equations to use the same normal.  That is
\begin{equation}\label{eqn:distanceBetweenPlanes:40}
\begin{aligned}
\Bx_1 \cdot \Bn &= -3 \\
\Bx_2 \cdot \Bn &= \inv{3},
\end{aligned}
\end{equation}
where \( \Bx_1 \) are vectors in the first plane, and \( \Bx_2 \) are vectors in the second plane.  Finding a vector in one of the planes isn't hard.  Suppose, for example, that \( \Bx_0 = (\alpha, \beta, \gamma) \) is a vector in the first plane, then
\begin{equation}\label{eqn:distanceBetweenPlanes:60}
\alpha - \beta + 2 \gamma = -3.
\end{equation}
One solution is \( \alpha = -3, \beta  = 0, \gamma = 0 \), or \( \Bx_0 = (-3, 0, 0) \).  We can follow the normal from that point to the closest point in the second plane by forming
\begin{equation}\label{eqn:distanceBetweenPlanes:80}
\By_0 = \Bx_0 + k \Bn,
\end{equation}
where \( k \) is to be determined.  If \( \By_0 \) is a point in the second plane, we must have
\begin{equation}\label{eqn:distanceBetweenPlanes:100}
\begin{aligned}
\inv{3}
&=
\By_0 \cdot \Bn \\
&=
\lr{ \Bx_0 + k \Bn } \cdot \Bn \\
&=
(-3, 0, 0 ) \cdot (1,-1,2) + k (1,-1,2) \cdot (1,-1,2) \\
&=
-3 + 6 k,
\end{aligned}
\end{equation}
or
\begin{equation}\label{eqn:distanceBetweenPlanes:120}
k = \frac{10}{18} = \frac{5}{9}.
\end{equation}
This means the point in plane two closest to \( \Bx_0 = (-3,0,0) \) is
\begin{equation}\label{eqn:distanceBetweenPlanes:140}
\begin{aligned}
\By_0
&= (-3, 0, 0 ) + \frac{5}{9} (1,-1,2) \\
&= \inv{9} (-27 + 5, -5, 10) \\
&= \inv{9} (-22, -5, 10),
\end{aligned}
\end{equation}
and the vector distance between the planes is
\begin{equation}\label{eqn:distanceBetweenPlanes:160}
\begin{aligned}
\By_0 - \Bx_0
&= \inv{9} (-22, -5, 10) - (-3, 0, 0 ) \\
&= \inv{9} (-22 + 27, -5, 10) \\
&= \inv{9} (5, -5, 10).
\end{aligned}
\end{equation}
This vector's length is \( \sqrt{150}/9 = (5/9) \sqrt{6} \), which is the shortest distance between the planes.

\section{A symbolic approach.}
Generally, we get more clarity if we avoid plugging in numbers until the very end, so let's try a generalization of this problem.
\makeproblem{}{problem:distanceBetweenPlanes:2}{
Find the shortest distance between the two parallel planes, \( P_1 \), and \( P_2 \), with respective equations:
\begin{equation*}
\begin{aligned}
\Bx_1 \cdot \Bn_1 &= d_1 \\
\Bx_2 \cdot \Bn_2 &= d_2.
\end{aligned}
\end{equation*}
} % problem
We can use the same approach, but first, let's rescale the two normals.  Let
\begin{equation}\label{eqn:distanceBetweenPlanes:180}
\Bn_2 = t \Bn_1,
\end{equation}
or
\begin{equation}\label{eqn:distanceBetweenPlanes:200}
\Bn_1 \cdot \Bn_2 = t \Bn_1^2,
\end{equation}
so
\begin{equation}\label{eqn:distanceBetweenPlanes:220}
\Bn_2 = \frac{\Bn_1 \cdot \Bn_2}{\Bn_1^2} \Bn_1,
\end{equation}
which means that our plane equations are
\begin{equation}\label{eqn:distanceBetweenPlanes:240}
\begin{aligned}
\Bx_1 \cdot \Bn_1 &= d_1 \\
\Bx_2 \cdot \Bn_1 &= \frac{\Bn_1^2}{\Bn_1 \cdot \Bn_2} d_2,
\end{aligned}
\end{equation}
We can further streamline our plane equation representation, setting \( \ncap = \Bn_1/\Norm{\Bn_1} \), which gives us
\begin{equation}\label{eqn:distanceBetweenPlanes:260}
\begin{aligned}
\Bx_1 \cdot \ncap &= \frac{d_1}{\Norm{\Bn_1}} \\
\Bx_2 \cdot \ncap &= \frac{d_2}{\ncap \cdot \Bn_2}.
\end{aligned}
\end{equation}

This time, let's assume that we can find a point \( \Bx_0 \) in the first plane, but not actually try to find it.  We can still follow the normal to the second plane from that point
\begin{equation}\label{eqn:distanceBetweenPlanes:280}
\By_0 = \Bx_0 + k \ncap,
\end{equation}
but since we only care about the vector distance between the planes, we seek
\begin{equation}\label{eqn:distanceBetweenPlanes:300}
\By_0 -\Bx_0 = k \ncap.
\end{equation}
Now, the constant \( k \), once we find it, is exactly the distance between the planes that we seek.  Plugging \( \By_0 \) into the \( P_2 \) equation, we find
\begin{equation}\label{eqn:distanceBetweenPlanes:320}
\begin{aligned}
\frac{d_2}{\ncap \cdot \Bn_2}
&=
\lr{ \Bx_0 + k \ncap } \cdot \ncap  \\
&=
\Bx_0 \cdot \ncap + k \\
&=
\frac{d_1}{\Norm{\Bn_1}} + k,
\end{aligned}
\end{equation}
or
\begin{equation}\label{eqn:distanceBetweenPlanes:340}
\boxed{
\Abs{k} = \Norm{\By_0 - \Bx_0} = \Abs{ \frac{d_2}{\ncap \cdot \Bn_2} - \frac{d_1}{\ncap \cdot \Bn_1} }.
}
\end{equation}
If \( \Bn_2 = \Bn_1 = \Bn \), then we have
\begin{equation}\label{eqn:distanceBetweenPlanes:360}
\begin{aligned}
\Norm{\By_0 - \Bx_0} &=
\Abs{
\frac{d_2}{\Bn_1^2/\Norm{\Bn_1}} - \frac{d_1}{\Norm{\Bn_1}}
} \\
&=
\frac{\Abs{d_2 - d_1}}{\Norm{\Bn}},
\end{aligned}
\end{equation}
and if \( \Bn \) is a unit normal, this further reduces to just \( \Abs{d_2 - d_1} \).

Let's try this for the specific problem originally given.  We have \( \Bn_1 = \Bn_2 \), so the distance between the planes is
\begin{equation}\label{eqn:distanceBetweenPlanes:380}
\begin{aligned}
\Norm{\By_0 - \Bx_0}
&= \frac{\Abs{1/3 + 3}}{\sqrt{6}} \\
&= \frac{10}{3 \times 6} \sqrt{6} \\
&= \frac{5}{9} \sqrt{6},
\end{aligned}
\end{equation}
as previously calculated.

%}
%\EndArticle
\EndNoBibArticle
