
\section{Coordinate vectors.}
\subsection{Motivation.}
Having covered the basic operations of vectors in their arrow representation, we are ready to go algebraic on the subject.
In particular, while
it is easy to compute the length of a vector that has an arrow representation, where
one simply lines a ruler of appropriate units along the vector and measures, this graphical procedure is cumbersome when we need to calculate.

Some mathematical baggage from linear algebra is required to do this for general N-dimensional spaces, including
\begin{itemize}
\item coordinates,
\item basis (plural bases),
\item linear dependence and independence,
\item span,
\item dot product, and
\item metric.
\end{itemize}
None of these are hard concepts, but they interfere with the flow of the story, so let's cheat.
We can temporarily avoid the ideas of linear dependence, independence and span, by restricting the story to 2D and 3D spaces that we can describe geometrically.
%When we eventually finish constructing our geometric algebra toolbox, we will have a number of coordinate free methods available to us.
%but we need to understand coordinates to get to that point.
%We also need to understand coordinates, both to read the literature, and in practice.
%Coordinates and non-orthonormal bases are also a good way to introduce non-Euclidean metrics.
\subsection{Basis.}
\makedefinition{Basis (cheat).}{dfn:multivector:180}{
Given a periodic partitioning of a space into repeated parallelopipeds, an ordered set of vectors that lie between the vertices between the edges of one of the cells is called a basis.
} % definition
\index{basis}
The plural of basis is bases.
\index{bases}
A basis for a space subdivided into a parallopiped grid is illustrated with points at all the lattice vertices in
\cref{fig:fbasis:fbasisFig1}.
\imageFigure{../figures/GAelectrodynamics/fbasisFig1}{A basis for a 2D space with a parallopiped lattice.}{fig:fbasis:fbasisFig1}{0.3}
The basis in this example is the ordered set \( \setlr{\Bf_1, \Bf_2} \).
There is no unique choice of basis for a given lattice, as different orderings and directions are possible (
\( \setlr{\pm \Bf_2, \pm \Bf_1} \) or \( \setlr{\pm \Bf_1, \pm \Bf_2} \)).
Alternate bases for this lattice include \( \setlr{\Bf_2, \Bf_1} \), and \( \setlr{-\Bf_1, -\Bf_2} \).
A basis has one and only one vector that lies between the vertices of the smallest cell of the lattice.
%\imageTwoFigures
%{../figures/GAelectrodynamics/fbasisFig1}
%{../figures/GAelectrodynamics/ebasisFig1}
%{Oblique and rectangular two dimensional bases.}{fig:ebasis:ebasisFig1}{scale=0.4}

Of course, the simplest possible lattice is that of a uniform square grid as illustrated in
\cref{fig:ebasis:ebasisFig1}.
\imageFigure{../figures/GAelectrodynamics/ebasisFig1}{A basis for a 2D space with a square lattice.}{fig:ebasis:ebasisFig1}{0.3}
In this example, the pair of vectors \( \setlr{\Be_1, \Be_2}\) is our basis, as they lie along the respective lattice directions.
Alternate bases for this square grid include \( \setlr{\Be_2, \Be_1} \), \( \setlr{-\Be_1, -\Be_2} \), and \( \setlr{\Be_2, -\Be_1} \).
\subsection{Coordinates.}
\makedefinition{Coordinates.}{dfn:multivector:200}{
Given a basis with \( N \) basis vectors \( \setlr{ \Bf_1, \cdots \Bf_N } \), and a vector \( \Ba = \sum_{i = 1}^N a^i \Bf_i \), the coordinates of the vector \( \Ba \) are \((a^1, a^2, \cdots a^N)\), or in matrix (column vector) notation
\begin{equation*}
\begin{bmatrix}
a^1 \\
a^2 \\
\vdots \\
a^N
\end{bmatrix}.
\end{equation*}
} % definition
Superscript is used for the coordinate indexes when the lattice is not composed of square (or cubic) units.
This superscript notation has the disadvantage of introducing an ambiguity, requiring context to determine whether an index or a exponentiation is intended, but will be seen to be worthwhile.

When the lattice is composed of regular cubic cells, where the basis vectors are all of unit length and mutually perpendicular, we can use the more conventional subscript index convention for our coordinates.  In this case, one can also make the engineering identification of a matrix of coordinates as \emph{the vector}.
We choose not to identify the coordinates of a vector as the vector, and consider coordinates to be a specific representation.

We may use the two previous lattice examples to illustrate the idea of coordinates.
For example, consider
\( \Bx = 5 \Bf_1 + 3 \Bf_2 \) as illustrated in
\cref{fig:fbasisSum:fbasisSumFig1}.
\imageFigure{../figures/GAelectrodynamics/fbasisSumFig1}{Decomposition of a particular vector in terms of an oblique set of basis vectors.}{fig:fbasisSum:fbasisSumFig1}{0.3}
The coordinates of this vector with respect the basis \(\setlr{\Bf_1, \Bf_2}\) are \((5,3)\).
Any set of coordinates must be implicitly associated with an underlying basis, and are meaningless without such an association.
For example, the
coordinates of \(\Bx\) with respect the basis \(\setlr{-\Bf_2, \Bf_1}\) would be \((-3,5)\).
As another example, consider the vector \( \By = 3 \Be_1 + 2 \Be_2 \) as illustrated in
\cref{fig:ebasisSum:ebasisSumFig1} for a square lattice.
\imageFigure{../figures/GAelectrodynamics/ebasisSumFig1}{Decomposition of a particular vector on a square lattice.}{fig:ebasisSum:ebasisSumFig1}{0.3}
The coordinates of \( \By \) with respect to the basis \( \setlr{\Be_1, \Be_2}\) are \((3,2)\).
The coordinates of \( \By \) with respect to the an alternate basis \( \setlr{\Be_2, -\Be_1}\) would be \((2,-3)\), that is
\(\By = 2 \Be_2 + (-3)(-\Be_1)\).

%%%The special case of a cubic lattice is so important that we give it a name
%%%\index{standard basis}
%%%\makedefinition{Standard basis.}{dfn:multivector:220}{
%%%Given a unit square(volume) lattice with basis vectors \( \Be_i \) of length 1 (unit vectors), a basis
%%%\( \setlr{\Be_1, \Be_2, \cdots, \Be_N}\) is called a \emph{standard basis}.
%%%} % definition
%%%There are often additional conventions imposed on a standard basis (such as the 3D right hand rule), but we don't care about those for now.
You may ask why we could possibly care to use anything but a cubic lattice with uniform spacing.
Here are a few reasons
\begin{itemize}
\item When we get to vector calculus, we require bases that vary from point to point, depending on the parametization of the space that we are using in our integrals.
Those parameterizations (spherical, cylindrical, toriodal, ...) need not be cubic.
\item There are important applications for non-cubic in solid state physics, as the crystal structures that occur due to molecular bonds are not friendly enough to restrict themselves to cubic configurations.
\item Developing the toolbox for more general bases and coordinates will leave us ready to tackle the non-Euclidean space (spacetime, or ``four-vectors'') that we encounter in special relativity and electromagnetism.
All electromagnetic theory is relativistic, so having the tools to express these ideas is not just academic.
\end{itemize}
\subsection{Scaling.}
Scaling a vector algebraically is so simple that it is almost trivial.  Let
\( \Ba = \sum_{i = 1}^N a^i \Bf_i \), and let \( \alpha \) be any scalar value, then
\begin{equation}\label{eqn:multivector:240}
\alpha \Ba
=
\alpha \sum_{i = 1}^N a^i \Bf_i
=
\sum_{i = 1}^N (\alpha a^i) \Bf_i.
\end{equation}
In the matrix representation of a vector, this is just a scalar matrix product
\begin{equation}\label{eqn:multivector:260}
\alpha
\begin{bmatrix}
a^1 \\
a^2 \\
\vdots \\
a^N
\end{bmatrix}
=
\begin{bmatrix}
\alpha a^1 \\
\alpha a^2 \\
\vdots \\
\alpha a^N
\end{bmatrix}.
\end{equation}
\subsection{Addition and subtraction.}
Like scaling, addition and subtraction are trivial algebraically.  Let
\( \Ba = \sum_{i = 1}^N a^i \Bf_i \) and
\( \Bb = \sum_{i = 1}^N b^i \Bf_i \).  Sums or differences of the two are
\begin{equation}\label{eqn:multivector:280}
\Ba \pm \Bb =
\sum_{i = 1}^N a^i \Bf_i
\pm
\sum_{i = 1}^N b^i \Bf_i
=
\sum_{i = 1}^N (a^i \pm b^i) \Bf_i.
\end{equation}
The matrix equivalent of this sum or difference is
\begin{equation}\label{eqn:multivector:300}
\begin{bmatrix}
a^1 \\
a^2 \\
\vdots \\
a^N
\end{bmatrix}
\pm
\begin{bmatrix}
b^1 \\
b^2 \\
\vdots \\
b^N
\end{bmatrix}
=
\begin{bmatrix}
a^1 \pm b^1 \\
a^2 \pm b^2 \\
\vdots \\
a^N \pm b^N
\end{bmatrix}.
\end{equation}
\subsection{Length.}
