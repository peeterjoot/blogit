%
% Copyright � 2022 Peeter Joot.  All Rights Reserved.
% Licenced as described in the file LICENSE under the root directory of this GIT repository.
%
%{
\input{../latex/blogpost.tex}
\renewcommand{\basename}{gafpInertialTensorQ}
%\renewcommand{\dirname}{notes/phy1520/}
\renewcommand{\dirname}{notes/ece1228-electromagnetic-theory/}
%\newcommand{\dateintitle}{}
%\newcommand{\keywords}{}

\input{../latex/peeter_prologue_print2.tex}

\usepackage{peeters_layout_exercise}
\usepackage{peeters_braket}
\usepackage{peeters_figures}
\usepackage{siunitx}
\usepackage{verbatim}
%\usepackage{mhchem} % \ce{}
%\usepackage{macros_bm} % \bcM
%\usepackage{macros_qed} % \qedmarker
%\usepackage{txfonts} % \ointclockwise

\beginArtNoToc

% https://math.stackexchange.com/questions/4493584/missing-stepsintuition-for-geometric-algebra-manipulations/4494143#4494143
\generatetitle{XXX}
%\chapter{XXX}
%\label{chap:gafpInertialTensorQ}

We can replace the wedge product from the right bivector term with a geometric product
\begin{equation*}
\begin{aligned}
x \wedge \lr{ x \cdot B } &= x \lr{ x \cdot B } - x \cdot \lr{ x \cdot B } \\
                          &= x \lr{ x \cdot B },
\end{aligned}
\end{equation*}
since \( x \) and \( x \cdot B \) are perpendicular ( \( x \cdot B \) is the projection of \( x \) onto \( B \), but rotated 90 degrees in that plane).  As pointed out by mreman, this also follows from the identity \( x \cdot (y \cdot B) = (x \wedge y) \cdot B \).

The dot product in the original expression, can now be written as a grade selection
\begin{equation*}
\begin{aligned}
   A \cdot \lr{ x \wedge \lr{ x \cdot B } } 
   &= \gpgradezero{ A \lr{ x \wedge \lr{ x \cdot B } } } \\
   &= \gpgradezero{ A x \lr{ x \cdot B } } \\
   &= \gpgradezero{ \lr{ A \cdot x + A \wedge x } \lr{ x \cdot B } } \\
   &= \gpgradezero{ \lr{ A \cdot x } \lr{ x \cdot B } } + \gpgradezero{ \lr{ A \wedge x } \lr{ x \cdot B } }.
\end{aligned}
\end{equation*}
The second grade selection operator, is a product of a trivector and a vector, which is a bivector, and has no grade-0 components, so that grade selection is zero.  We are left with
\begin{equation*}
\begin{aligned}
   A \cdot \lr{ x \wedge \lr{ x \cdot B } } 
   &= \gpgradezero{ \lr{ A \cdot x } \lr{ x \cdot B } } \\
   &= \lr{ A \cdot x } \cdot \lr{ x \cdot B } \\
   &= -\lr{ x \cdot A } \cdot \lr{ x \cdot B }.
\end{aligned}
\end{equation*}
This step isn't required to find the next expression in the text, the \( \gpgradezero{ \lr{ A \cdot x } x B } \), but can be used to answer your second question.

Now we apply a reverse operation, noting that \( \lr{M N}^{\dagger} = N^\dagger M^\dagger \), for any multivectors \( M, N \), and that scalars and vectors are their own reverses.  Both \( A \cdot x , x \cdot B \) are vectors, so
\begin{equation*}
\begin{aligned}
   \gpgradezero{ \lr{ A \cdot x } \lr{ x \cdot B } }^\dagger
   &=
   \gpgradezero{ \lr{ x \cdot B } \lr{ A \cdot x } } \\
   &=
   -\lr{ x \cdot B } \cdot \lr{ x \cdot A }.
\end{aligned}
\end{equation*}
This shows that \( A \cdot \lr{ x \wedge \lr{ x \cdot B } }  \) is symmetric in \( A, B \), which provides the end result of (3.126).  We didn't really need the second-last intermediate step \( \gpgradezero{ \lr{ A \cdot x } x B } \), but it's fairly simple to show that from the above, should you want to.

For the second question, from the above, note that 
\begin{equation*}
\begin{aligned}
A \cdot \lr{ x \wedge \lr{ x \cdot A } } &=
   -\lr{ x \cdot A } \cdot \lr{ x \cdot A } \\
   &=
   -\lr{ x \cdot A }^2,
\end{aligned}
\end{equation*}
since \( x \cdot A \) is a vector that, when wedged with itself, is zero.  Substitution of \( A = I a \) provides the transition from the first to second line in the second question.

%}
%\EndArticle
\EndNoBibArticle
