%
% Copyright � 2025 Peeter Joot.  All Rights Reserved.
% Licenced as described in the file LICENSE under the root directory of this GIT repository.
%
%{
\input{../latex/blogpost.tex}
\renewcommand{\basename}{generalizedGaussian}
%\renewcommand{\dirname}{notes/phy1520/}
\renewcommand{\dirname}{notes/ece1228-electromagnetic-theory/}
%\newcommand{\dateintitle}{}
%\newcommand{\keywords}{}

\input{../latex/peeter_prologue_print2.tex}

\usepackage{peeters_layout_exercise}
\usepackage{peeters_braket}
\usepackage{peeters_figures}
\usepackage{siunitx}
\usepackage{verbatim}
%\usepackage{macros_cal} % \LL
%\usepackage{amsthm} % proof
%\usepackage{mhchem} % \ce{}
%\usepackage{macros_bm} % \bcM
%\usepackage{macros_qed} % \qedmarker
%\usepackage{txfonts} % \ointclockwise

\beginArtNoToc

\generatetitle{Generalized Gaussian}
%\chapter{Generalized Gaussian}
%\label{chap:generalizedGaussian}

Here's another problem from \citep{byron1992mca}.  The point is to show that
\begin{equation}\label{eqn:generalizedGaussian:20}
G(x,x',\tau) = \inv{2\pi} \int_{-\infty}^\infty e^{i k\lr{ x - x' } } e^{-k^2 \tau} dk,
\end{equation}
has the value
\begin{equation}\label{eqn:generalizedGaussian:40}
G(x,x',\tau) = \inv{\sqrt{4 \pi \tau} } e^{-\lr{ x - x'}^2/4 \tau },
\end{equation}
not just for real \(\tau\), but also for purely imaginary \(\tau\).

\section{Real case.}
The authors claim the real case is easy, but I don't think the real case is that trivial.  The trivial part is basically just completing the square.  Writing \( x - x ' = \Delta x \), that is
\begin{equation}\label{eqn:generalizedGaussian:60}
\begin{aligned}
-k^2 \tau + i k\lr{ x - x' }
&=
-\tau \lr{ k^2 - i k \Delta x/\tau } \\
&=
-\tau \lr{ \lr{ k - i \Delta x/2\tau }^2 - \lr{ - i \Delta x/2\tau }^2 } \\
&=
-\tau \lr{ \lr{ k - i \Delta x/2\tau }^2 + \lr{ \Delta x/2\tau }^2 }.
\end{aligned}
\end{equation}
So we have
\begin{equation}\label{eqn:generalizedGaussian:80}
G(x,x',\tau) = \inv{2\pi} e^{-(\Delta x/2)^2/\tau} \int_{-\infty}^\infty e^{-\tau \lr{ k - i \Delta x/2\tau }^2 } dk.
\end{equation}
Let's call the integral factor part of this \( I \)
\begin{equation}\label{eqn:generalizedGaussian:100}
I = \int_{-\infty}^\infty e^{-\tau \lr{ k - i \Delta x/2\tau }^2 } dk.
\end{equation}
However, making a change of variables makes this an integral over a complex path
\begin{equation}\label{eqn:generalizedGaussian:120}
I = \int_{-\infty - i \Delta x/2\tau }^{\infty - i \Delta x/2\tau} e^{-\tau k^2 } dk.
\end{equation}
If you are lazy you could say that \( \pm \infty \) adjusted by a constant, even if that constant is imaginary, leaves the integration limits unchanged.  That's clearly true if the constant is real, but I don't think it's that obvious if the constant is imaginary.

To answer that question more exactly, let's consider the integral
\begin{equation}\label{eqn:generalizedGaussian:140}
0 = I + J + K + L = \oint e^{-\tau z^2} dz,
\end{equation}
where the path is illustrated in \cref{fig:RectangularContour:RectangularContourFig1}.
\imageFigure{../figures/blogit/RectangularContourFig1}{Contour for the Gaussian.}{fig:RectangularContour:RectangularContourFig1}{0.2}
Since there are no enclosed poles, we have
\begin{equation}\label{eqn:generalizedGaussian:160}
I = \int_{-\infty}^\infty e^{-\tau k^2 } dk + K + L,
\end{equation}
where
\begin{equation}\label{eqn:generalizedGaussian:180}
\begin{aligned}
K &= \int_{- i \Delta x/2\tau}^0 e^{-\tau z^2} dz \\
L &= \int_0^{- i \Delta x/2\tau} e^{-\tau z^2} dz.
\end{aligned}
\end{equation}
We see now that we see perfect cancellation of \( K \) and \( L \), which justifies the change of variables, and the corresponding integration limits.

We can use the usual trick to evaluate \( I^2 \), to find
\begin{equation}\label{eqn:generalizedGaussian:200}
\begin{aligned}
I^2
&=
\int_{-\infty}^\infty e^{-\tau k^2 } dk
\int_{-\infty}^\infty e^{-\tau m^2 } dm \\
&=
2 \pi \int_0^\infty r e^{-\tau r^2} r dr \\
&=
2 \pi \evalrange{ - \frac{e^{-\tau r^2}}{-2 \tau} }{0}{\infty} \\
&=
\frac{\pi}{\tau}.
\end{aligned}
\end{equation}

So, for real values of \( \tau \) we have
\begin{equation}\label{eqn:generalizedGaussian:220}
G(x,x',\tau) = \inv{2\pi} \sqrt{\frac{\pi}{\tau}}e^{-(\Delta x/2)^2/\tau},
\end{equation}
as expected.

%}
\EndArticle
