%
% Copyright � 2025 Peeter Joot.  All Rights Reserved.
% Licenced as described in the file LICENSE under the root directory of this GIT repository.
%
%{
\input{../latex/blogpost.tex}
\renewcommand{\basename}{gaussianContour}
%\renewcommand{\dirname}{notes/phy1520/}
\renewcommand{\dirname}{notes/ece1228-electromagnetic-theory/}
%\newcommand{\dateintitle}{}
%\newcommand{\keywords}{}

\input{../latex/peeter_prologue_print2.tex}

\usepackage{peeters_layout_exercise}
\usepackage{peeters_braket}
\usepackage{peeters_figures}
\usepackage{siunitx}
\usepackage{verbatim}
%\usepackage{macros_cal} % \LL
%\usepackage{amsthm} % proof
%\usepackage{mhchem} % \ce{}
%\usepackage{macros_bm} % \bcM
%\usepackage{macros_qed} % \qedmarker
%\usepackage{txfonts} % \ointclockwise

\beginArtNoToc

\generatetitle{Gaussian evaluation using contour integration}
%\chapter{Gaussian evaluation using contour integration}
%\label{chap:gaussianContour}

Find
\begin{equation}\label{eqn:gaussianContour:20}
I = \inv{\sqrt{2 \pi}} \int_0^\infty e^{-x^2/8} dx.
\end{equation}
Let \( u = x^2/4 \), so that \( du = x dx/2 \), and
\begin{equation}\label{eqn:gaussianContour:60}
dx = \frac{2}{x} du = \frac{2}{2 \sqrt{u}} du,
\end{equation}
so
\begin{equation}\label{eqn:gaussianContour:40}
2 I = \inv{\sqrt{2 \pi}} \int_{-\infty}^\infty \frac{e^{-u^2/2}}{\sqrt{u}} du.
\end{equation}
Let's now evaluate
\begin{equation}\label{eqn:gaussianContour:80}
2 \sqrt{2 \pi} I = J = \oint_C \frac{e^{-z^2/2}}{\sqrt{z}} dz,
\end{equation}
over \( z \in [-R,R] \), with a semicircular hop around the branch point at the origin, and then back around in a radius \( R \) loop from \( R \) to \( -R \), as sketched in FIXME.
\begin{equation}\label{eqn:gaussianContour:100}
\begin{aligned}
J = - \pi i
\end{aligned}
\end{equation}

%}
%\EndArticle
\EndNoBibArticle
